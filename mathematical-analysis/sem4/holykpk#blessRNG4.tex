\documentclass{article}
\usepackage[utf8]{inputenc}
\usepackage[T2A]{fontenc}
\usepackage[russian]{babel}
\usepackage{amsfonts}
\usepackage{amsmath}
\usepackage{amssymb}
\usepackage{arcs}
\usepackage{fancyhdr}
\usepackage{float}
\usepackage[left=3cm,right=3cm,top=3cm,bottom=3cm]{geometry}
\usepackage{graphicx}
\usepackage{hyperref}
\usepackage{multicol}
\usepackage{stackrel}
\usepackage{xcolor}
\usepackage{epigraph}
\usepackage{tikz}
\usepackage{amsthm}
\usepackage{graphics}
\usepackage{draftwatermark}
\usepackage{ marvosym }
\usepackage{physics}
\usepackage{pdfpages}



\def\letus{%
\mathord{\setbox0=\hbox{$\exists$}%
         \hbox{\kern 0.125\wd0%
               \vbox to \ht0{%
                  \hrule width 0.75\wd0%
                  \vfill%
                  \hrule width 0.75\wd0}%
               \vrule height \ht0%
               \kern 0.125\wd0}%
       }%
        }
\def\dbl{\,\,}
\def\image#1{\includegraphics[width=\linewidth]{static/#1}}
\def\imageh#1{\includegraphics[width=\linewidth, height=0.95\textheight]{static/#1}}
\def\images#1#2{\begin{center}\includegraphics[width=#1\linewidth]{static/#2}\end{center}}

\def\rsh#1{\underset{#1}{\rightrightarrows}}
\def\rshe{\rsh{E}}
\def\eqby#1{\underset{#1}{=}}
\def\rinf{\overline{\mathbb{R}}}
\def\goesto#1{\underset{#1}{\longrightarrow}}
\def\toinf#1{\goesto{#1 \rightarrow \infty}}
\def\ntoinf{\toinf{n}}

\DeclareMathOperator{\sign}{sign}
\DeclareMathOperator{\const}{const}
\DeclareMathOperator{\segm}{Segm}


\newcommand*\lateraleye{%
       \scalebox{0.15}{
    \tikzset{every picture/.style={line width=0.75pt}} 
    \begin{tikzpicture}[x=0.75pt,y=0.75pt,yscale=-1,xscale=1]
    \draw  [line width=1.5]  (300,100.33) .. controls (326,122) and (352,135) .. (378,139.33) .. controls (352,143.67) and (326,156.67) .. (300,178.33) ;
    \draw  [fill={rgb, 255:red, 0; green, 0; blue, 0 }  ,fill opacity=1 ] (308.94,116.33) .. controls (313.87,116.33) and (317.86,125.51) .. (317.85,136.83) .. controls (317.84,148.15) and (313.84,157.33) .. (308.91,157.33) .. controls (303.99,157.32) and (300,148.14) .. (300.01,136.82) .. controls (300.02,125.5) and (304.02,116.32) .. (308.94,116.33) -- cycle ;
    \draw  [draw opacity=0][line width=1.5]  (314.84,166.6) .. controls (311.87,164.64) and (309.14,162.18) .. (306.76,159.24) .. controls (295.12,144.82) and (296.6,124.33) .. (310.07,113.45) .. controls (311.48,112.32) and (312.96,111.33) .. (314.5,110.49) -- (331.14,139.55) -- cycle ; \draw  [line width=1.5]  (314.84,166.6) .. controls (311.87,164.64) and (309.14,162.18) .. (306.76,159.24) .. controls (295.12,144.82) and (296.6,124.33) .. (310.07,113.45) .. controls (311.48,112.32) and (312.96,111.33) .. (314.5,110.49) ;
    \draw  [fill={rgb, 255:red, 255; green, 255; blue, 255 }  ,fill opacity=1 ] (304.43,124.2) .. controls (306.09,124.25) and (307.32,128.01) .. (307.18,132.6) .. controls (307.05,137.19) and (305.59,140.88) .. (303.93,140.83) .. controls (302.27,140.78) and (301.03,137.02) .. (301.17,132.43) .. controls (301.31,127.83) and (302.76,124.15) .. (304.43,124.2) -- cycle ;
    \end{tikzpicture}
    }\,}
    
\def\D{\,\mathrm{d}}

\let\vanillaparagraph\paragraph
\let\vanillasubparagraph\subparagraph
\renewcommand{\paragraph}[1]{\vanillaparagraph{#1}\mbox{}\\}
\renewcommand{\subparagraph}[1]{\vanillasubparagraph{#1}\mbox{}\\}

\graphicspath{{./images/}}

\setlength{\parindent}{0pt}

\setcounter{tocdepth}{4}
\setcounter{secnumdepth}{4}

\SetWatermarkText{$\underset{\text{@imodre @snitron}}{\text{ПРОДАМ ГАРАЖ}}$}
\SetWatermarkScale{2}
\SetWatermarkLightness{0.9}

\begin{document}
\DraftwatermarkOptions{stamp=false}
\begin{titlepage}
    \centering
    \vspace*{\baselineskip}
    \rule{\textwidth}{1.6pt}\vspace*{-\baselineskip}\vspace*{2pt}
    \rule{\textwidth}{0.4pt}\\[\baselineskip]
    {\LARGE СВЯТОЙ КПК\\ [0.3\baselineskip] \#BlessRNG}\\[0.2\baselineskip]
    \rule{\textwidth}{0.4pt}\vspace*{-\baselineskip}\vspace{3.2pt}
    \rule{\textwidth}{1.6pt}\\[\baselineskip]
    \scshape
    Или как не сдохнуть на 4 семе из-за матана \\
    \vspace*{2\baselineskip}
    Разработал \\[\baselineskip]
    {\Large Никита Варламов\quad @snitron}
        \vspace*{2\baselineskip}\par
    Почётный автор \\[\baselineskip]
    {\Large Тимофей Белоусов\quad @imodre}
    \vfill
    v0.0\\
    {\scshape Февраль-??? 2023} \par
\end{titlepage}

Вы в любой момент можете добавить любую недостающую теорему, затехав её и отправив код (фотографии письменного текста запрещены) в телегу любому из указанных авторов, или создав Pull Request в \href{https://github.com/snitron/ct-itmo}{Git-репозиторий конспекта (click)}. Ваше авторство также будет указано, с вашего разрешения.

\newpage

\begin{flushright}
\emph{Ah shit\\
Here we go again!\\
And again...\\
Oh, fuck.}
\end{flushright}


\tableofcontents


\setlength{\parskip}{6pt}%
\newpage
\DraftwatermarkOptions{stamp=false}


\section{Период Палеозойский}
\subsection{Важные определения}

\newpage

\subsection{Определения}
\newpage

\subsection{Важные теоремы}
\newpage

\subsection{Теоремы}

\subsubsection{Теорема об интегрировании положительных рядов}
\textit{Формулировка:}

\begin{itemize}
    \item $(X, \mathfrak{A}, \mu)$ --- пространство с мерой
    \item $u_n: X \rightarrow \rinf, u_n \ge 0$ (при почти всех $x$ ?)
    \item $u_n$ --- измеримы на $E \in \mathfrak{A}$
\end{itemize}

Тогда: 

\[\int_{E} \left(\sum_{n = 1}^{\infty} u_n(x)\right)d\mu(x) = \sum_{n = 1}^{\infty} \left(\int_{E} u_n(x) d\mu(x)\right)\]

\textit{Доказательство:}

Подгоним под теорему Леви 3 (3 семестр). Пусть $S_{N}(x) = \sum_{n = 1}^{N} u_n(x)$ --- последовательность частичных сумм. Очевидно, что эта последовательность --- монотонно неубывающая (так как функции у нас неотрицательные): 

\[0 \le S_{N} \le S_{N + 1} \le S_{N + 2} \le \ldots\]

Тогда, делаем предельный переход (вот тут есть вопрос, почему должен существовать предел, но если подумать: если его не существует, вообще вся эта теорема не имеет смысла (ну бесконечности, чел, смысл их интегрировать)). А так же, измеримость сохраняется, так как у нас исходные функции все были измеримы (ну и по теореме о пределе измеирмых функций): 

\[S_{N}(x) \toinf{N} S(x)\]

Ну и всё, значи, по теореме Леви можем перейти к предельному преходу интегралов: 

\[\int_{E} S_{N}(x) d\mu(x) \toinf{N} \int_{E} S(x) d\mu(x)\]

Левую часть можно расписать по линейности интеграла (там у нас конечное число членов): 

\[\int_{E} S_{N}(x) d\mu(x) = \sum_{n = 1}^{N} \int_{E} u_n(x) d\mu(x)\]

Ну, а раз интграл суммы стремится к интегралу предельной функции, то и сумма интегралов обязана туда стремиться.

\[\sum_{n = 1}^{N} \int_{E} u_n(x) d\mu(x) \toinf{N} \sum_{n = 1}^{\infty} \int_{E} u_n(x) d\mu(x)\]

ч. т. д. 


\textit{Следствие: }

\begin{itemize}
    \item $u_n: X \rightarrow \mathbb{R}$, измеримы на $E \in \mathfrak{A}$
    \item $\sum \int_{E} |u_n(x)| d\mu < +\infty$ (конечна)
\end{itemize}

Тогда $\sum u_n(x)$ --- абсолютно сходящийся при почти всех $x$

\textit{Доказательство: }

Пусть: 

\[S(x) = \int_{n = 1}^{\infty} \left|u_n(x)\right|\]

Тогда, по предыдущей теореме: 

\[\int_{E} S(x) d\mu = \sum_{n = 1}^{\infty} \left(\int_{E} |u_n(x)| d\mu\right) < +\infty\]

Раз интеграл конечен, значит $S(x)$ --- суммируема, а это значит, что $S(x)$ --- почти везде конечна. Ну значит и сходится.

ч. т. д.

\textit{Пример: }

\begin{itemize}
    \item $(x_n)$ --- вещественная последовательность
    \item $\sum a_n$ --- абсолютно сходящийся числовой ряд
\end{itemize}

Тогда функциональный ряд $\sum \frac{a_n}{\sqrt{|x - x_n|}}$  --- абсолютно сходится при почти всех $x$ (в $\mathbb{R}$ по мере Лебега)

\textit{Доказательство: }

Во-первых, можно доказать, что если для $\forall A$ на $[-A, A]$ абсолютно сходится почти везде, то и везде (на $\mathbb{R}$) почти везде сходится (лол). Счётное количество п. в. $\Rightarrow$ п. в. (чтобы количество отрезков было счётным, надо чтобы $A$ были хотя бы рациональными. Кажется, что это не сильная проблема, так как отрезки включают в себя и все вещественные числа на отрезке тоже).

Попробуем подогнать под предыдущую теорему: 

\[\int_{[-A, A]}\frac{|a_n|}{\sqrt{|x - x_n|}} d\lambda = |a_n| \int_{-A}^{A} \frac{dx}{\sqrt{|x - x_n|}} \le\]

Так, стоп. А как мы перешли к определённому интегралу? Оказывается, что так можно делать, на доказано это будет позже (в курсе).

\[\underset{x := x - x_n}{\le} |a_n| \int_{-A - x_n}^{A - x_n} \frac{dx}{\sqrt{|x|}} \le |a_n| \int_{-A}^{A} \frac{dx}{\sqrt{|x|}} \le\]

Почему верен последний переход? Посмотрим на картинке: 

\images{0.5}{sh_pol_r.png}

Ну, по ней очевидно, что мы откусили кусочек поменьше, а добавили побольше. Тогда оценим модуль: 

\[ \le 2 \cdot |a_n| \int_{0}^{A}\frac{dx}{\sqrt{|x|}} = 4 \cdot \sqrt{A} \cdot |a_n|\]

Всё, абсолютный интеграл ограничен, значит сходится (при почти всех $x$).

ч. т. д. 


\newpage


\end{document}

\begin{comment}
    \subsubsection{Теорема X-N}
    \textit{Формулировка:}

    \begin{itemize}
        \item 
    \end{itemize}

    \textit{Доказательство:}
\end{comment}