\documentclass{article}
\usepackage[utf8]{inputenc}
\usepackage[T2A]{fontenc}
\usepackage[russian]{babel}
\usepackage{amsfonts}
\usepackage{amsmath}
\usepackage{amssymb}
\usepackage{arcs}
\usepackage{fancyhdr}
\usepackage{float}
\usepackage[left=3cm,right=3cm,top=3cm,bottom=3cm]{geometry}
\usepackage{graphicx}
\usepackage{hyperref}
\usepackage{multicol}
\usepackage{stackrel}
\usepackage{xcolor}
\usepackage{epigraph}
\usepackage{tikz}
\usepackage{amsthm}
\usepackage{graphics}
\usepackage{draftwatermark}
\usepackage{ marvosym }
\usepackage{physics}
\usepackage{pdfpages}
\usepackage{ulem}
\usepackage{mathtools}

\def\letus{%
\mathord{\setbox0=\hbox{$\exists$}%
         \hbox{\kern 0.125\wd0%
               \vbox to \ht0{%
                  \hrule width 0.75\wd0%
                  \vfill%
                  \hrule width 0.75\wd0}%
               \vrule height \ht0%
               \kern 0.125\wd0}%
       }%
        }
\def\dbl{\,\,}
\def\image#1{\includegraphics[width=\linewidth]{static/#1}}
\def\imageh#1{\includegraphics[width=\linewidth, height=0.95\textheight]{static/#1}}
\def\images#1#2{\begin{center}\includegraphics[width=#1\linewidth]{static/#2}\end{center}}

\def\rsh#1{\underset{#1}{\rightrightarrows}}
\def\rshe{\rsh{E}}
\def\eqby#1{\underset{#1}{=}}
\def\rinf{\overline{\mathbb{R}}}
\def\goesto#1{\underset{#1}{\longrightarrow}}
\def\toinf#1{\goesto{#1 \rightarrow \infty}}
\def\ntoinf{\toinf{n}}
\def\sk#1#2{\langle #1, #2 \rangle}
\def\DD{\mathcal{D}}
\def\Var#1#2{\overset{#2}{\underset{#1}{\text{Var}}}}

\DeclareMathOperator{\sign}{sign}
\DeclareMathOperator{\const}{const}
\DeclareMathOperator{\segm}{Segm}\DeclareMathOperator*{\esssup}{ess\,sup}
\DeclareMathOperator{\supp}{supp}


\newcommand*\lateraleye{%
       \scalebox{0.15}{
    \tikzset{every picture/.style={line width=0.75pt}} 
    \begin{tikzpicture}[x=0.75pt,y=0.75pt,yscale=-1,xscale=1]
    \draw  [line width=1.5]  (300,100.33) .. controls (326,122) and (352,135) .. (378,139.33) .. controls (352,143.67) and (326,156.67) .. (300,178.33) ;
    \draw  [fill={rgb, 255:red, 0; green, 0; blue, 0 }  ,fill opacity=1 ] (308.94,116.33) .. controls (313.87,116.33) and (317.86,125.51) .. (317.85,136.83) .. controls (317.84,148.15) and (313.84,157.33) .. (308.91,157.33) .. controls (303.99,157.32) and (300,148.14) .. (300.01,136.82) .. controls (300.02,125.5) and (304.02,116.32) .. (308.94,116.33) -- cycle ;
    \draw  [draw opacity=0][line width=1.5]  (314.84,166.6) .. controls (311.87,164.64) and (309.14,162.18) .. (306.76,159.24) .. controls (295.12,144.82) and (296.6,124.33) .. (310.07,113.45) .. controls (311.48,112.32) and (312.96,111.33) .. (314.5,110.49) -- (331.14,139.55) -- cycle ; \draw  [line width=1.5]  (314.84,166.6) .. controls (311.87,164.64) and (309.14,162.18) .. (306.76,159.24) .. controls (295.12,144.82) and (296.6,124.33) .. (310.07,113.45) .. controls (311.48,112.32) and (312.96,111.33) .. (314.5,110.49) ;
    \draw  [fill={rgb, 255:red, 255; green, 255; blue, 255 }  ,fill opacity=1 ] (304.43,124.2) .. controls (306.09,124.25) and (307.32,128.01) .. (307.18,132.6) .. controls (307.05,137.19) and (305.59,140.88) .. (303.93,140.83) .. controls (302.27,140.78) and (301.03,137.02) .. (301.17,132.43) .. controls (301.31,127.83) and (302.76,124.15) .. (304.43,124.2) -- cycle ;
    \end{tikzpicture}
    }\,}
    
\def\D{\,\mathrm{d}}

\let\vanillaparagraph\paragraph
\let\vanillasubparagraph\subparagraph
\renewcommand{\paragraph}[1]{\vanillaparagraph{#1}\mbox{}\\}
\renewcommand{\subparagraph}[1]{\vanillasubparagraph{#1}\mbox{}\\}

\graphicspath{{./images/}}

\setlength{\parindent}{0pt}

\setcounter{tocdepth}{4}
\setcounter{secnumdepth}{4}

\SetWatermarkText{$\underset{\text{@imodre @snitron}}{\text{ПРОДАМ ГАРАЖ}}$}
\SetWatermarkScale{2}
\SetWatermarkLightness{0.9}

\begin{document}
\DraftwatermarkOptions{stamp=false}
\begin{titlepage}
    \centering
    \vspace*{\baselineskip}
    \rule{\textwidth}{1.6pt}\vspace*{-\baselineskip}\vspace*{2pt}
    \rule{\textwidth}{0.4pt}\\[\baselineskip]
    {\LARGE СВЯТОЙ КПК\\ [0.3\baselineskip] \#BlessRNG}\\[0.2\baselineskip]
    \rule{\textwidth}{0.4pt}\vspace*{-\baselineskip}\vspace{3.2pt}
    \rule{\textwidth}{1.6pt}\\[\baselineskip]
    \scshape
    Или как не сдохнуть на 4 семе из-за матана \\
    \vspace*{2\baselineskip}
    Разработал \\[\baselineskip]
    {\Large Никита Варламов\quad @snitron}
        \vspace*{2\baselineskip}\par
    Почётный автор \\[\baselineskip]
    {\Large Тимофей Белоусов\quad @imodre}
    \vfill
    v0.0\\
    {\scshape Февраль-??? 2023} \par
\end{titlepage}

Вы в любой момент можете добавить любую недостающую теорему, затехав её и отправив код (фотографии письменного текста запрещены) в телегу любому из указанных авторов, или создав Pull Request в \href{https://github.com/snitron/ct-itmo}{Git-репозиторий конспекта (click)}. Ваше авторство также будет указано, с вашего разрешения.

\newpage

\begin{flushright}
\emph{Ah shit\\
Here we go again!\\
And again...\\
Oh, fuck.}
\end{flushright}


\tableofcontents


\setlength{\parskip}{6pt}%
\newpage
\DraftwatermarkOptions{stamp=false}


\section{Период Палеозойский}
\subsection{Важные определения}

\subsubsection{Пространство $L^p(E,\mu)$}
$1 \le p < +\infty$, $(X, \mathfrak{A}, \mu)$, $E \in \mathfrak{A}$

Тогда $\mathfrak{L}_p(E, \mu) = \{f:$ почти всех $E \rightarrow \mathbb{R} (\mathbb{C}), f$ --- измерима. $\int_{E} |f|^{p} < + \infty\}$

\begin{enumerate}
    \item $\mathfrak{L}_p(E, \mu)$ --- линейное пространство
    \item $f \approx g$, если $f = g$ почти всюду
\end{enumerate}

$L_p := \mathfrak{L}_p /_{\approx}$ --- точки этого пространства. Важно, что эти точки --- классы эквивалентности, как бы не просто функции. Однако, в большинстве случаев удобнее всего работать с каким-то одним представителем класса, ведь, если что, они отличаются только на множестве меры 0.

$[f] = \{g: f \equiv g\}$
$[f_1] + [f_2] = [f_1 + f_2]$

И введём норму $||[f]|| = \left(\int_{E} |f|^{p}\right)^{\frac{1}{p}}$

\subsubsection{Пространство $L^\infty(E,\mu)$}

$\mathfrak{L}^{\infty}(E, \mu) = \{f: $ почти всех $E \rightarrow \mathbb{R} (\mathbb{C})$, измерима, $\esssup |f| < + \infty\}$

$||f||_{\infty} = \esssup f$

$\mathfrak{L}^{\infty} /_\approx = L^{\infty}(E, \mu)$

\subsubsection{Существенный супремум}

$\esssup f = \inf \{a \in \rinf: f \le a $ почти всюду $\}$

$a$ --- существенная верхняя граница функции $f$, если при почти всех $x: \dbl f(x) \le a$

\textit{Свойства: }
\begin{enumerate}
    \item $\esssup f(x) \le \sup f(x)$
    \item при почти всех $x: f(x) \le \esssup f(x)$
    \item $f$ --- суммируемая, $g$ --- измерима: $\esssup |g| < + \infty$
    \[\left|\int_{E} fg\right| \le \esssup |g| \cdot \int_{E} |f|\]

    Доказательство:

    \[\left|\int_{E} fg\right| \le \int_{E} |fg| \le \int_{E} |f| \cdot \esssup |g| = \esssup |g| \int_{E} |f|\]
\end{enumerate}

\newpage

\subsection{Определения}
\subsubsection{Произведение мер}
$(X, \mathfrak{A}, \mu)$, $(Y, \mathfrak{B}, \nu)$ --- пространства с мерой.

\textit{Лемма: }
$\mathcal{A}, \mathcal{B}$ --- полукольца. Тогда $\mathcal{A} \times \mathcal{B} = \{A \times B, A \in \mathcal{A}, B \in \mathcal{B}\}$ --- полукольцо.

Также, множества из $\mathcal{A} \times \mathcal{B}$ являются измеримыми прямоугольниками.

$\mu, \nu$ --- $\sigma$-конечные меры. Тогда стандартное продолжение $m_{0}$ (в смысле теоремы о продолжении меры (?)) с полукольца $\mathfrak{A} \times \mathfrak{B}$, определённой на некоторой $\sigma$-алгебре $\mathfrak{A} \otimes \mathfrak{B}$, и являющееся $\sigma$-конечной полной мерой --- обозначается просто $m$.

И тогда $m$ --- и есть произведение мер $\mu$ и $\nu$ $(\mu \times \nu)$.

\textit{Замечание: }

\[(\mu \times \nu) \times \rho = \mu \times (\nu \times \rho)\]

\subsubsection{Сечения множества}

$X, Y$ --- множества. $C \subset X \times Y$

Тогда: 

\[C_{x} := \{y \in Y: (x, y) \in C\}\]

\[C^{y} := \{x \in X: (x, y) \in C\}\]

--- сечения множества $C$ (1 и 2 рода)

\textit{Замечания: }

\[\left(\bigcup_{\alpha \in A} C_{\alpha}\right)_{x} = \bigcup_{\alpha \in A} \left(C_{\alpha}\right)_{x}\]

\[\left(\bigcap_{\alpha \in A} C_{\alpha}\right)_{x} = \bigcap_{\alpha \in A} \left(C_{\alpha}\right)_{x}\]

\[\left(C \setminus C'\right)_{x} = C_{x} \setminus C'_{x}\]

\subsubsection{Полная мера, сигма-конечная мера}

См. \href{http://gg.gg/holykpksem3}{\color{blue}{конспект прошлого семестра}}

\subsubsection{Образ меры при отображении}

Пусть у нас есть $(X, \mathfrak{A}, \mu)$, $(Y, \mathfrak{B}, \_ )$ --- пространства с мерой, $\Phi: X \rightarrow Y$.

\begin{enumerate}
    \item $\forall \Phi \quad \Phi^{-1}(\mathfrak{B})$ --- $\sigma$-алгебра (это предлагается доказать как упражнение)
    \item Пусть $\Phi$ --- ``измеримо'' $\left(\Phi^{-1}(\mathfrak{B}) \subset \mathfrak{A}\right)$
\end{enumerate}

Для $E \in \mathfrak{B}$ зададим $\nu R := \mu\left(\Phi^{-1}(E)\right) = \int_{\Phi^{-1}(E)} 1 d \mu$

$\nu$ --- образ меры $\mu$ при отображении $\Phi$

\textit{NB: ДОПИСАТЬ НА СЕССИИ, ТУТ ЕЩË ЕСТЬ ДОКАЗАТЕЛЬСТВО, ЧТО ЭТО МЕРА (я на сессии: да вроде оно $+-$ очевидное)}

\subsubsection{Взвешенный образ меры}

$\omega: X \rightarrow \mathbb{R} \ge 0$, измерима на $X$

$B \in \mathfrak{B}, \tilde{\nu}(B) := \int_{\Phi^{-1}(B)} \omega d\mu$ --- тоже мера, это и есть взвешенный образ меры $\mu$ при отображении $\Phi$

\subsubsection{Плотность одной меры по отношению к другой}

$X = Y, \mathfrak{A} = \mathfrak{B}, \Phi = $ id

$\nu b = \int_{B} \omega d \mu$ --- ещё одна мера в $X$

Здесь $\omega$ называется плотностью меры $\nu$ относительно меры $\mu$. И в этом случае:

\[\int_{X}f(x)d\nu(x) = \int f(x) \cdot \omega(x) d\mu(x)\]

\subsubsection{Сферические координаты в $\mathbb{R}^3$}

На основе земных координат, $\varphi \in (0, 2\pi), \dbl \theta \in \left(-\frac{\pi}{2}, \frac{\pi}{2}\right)$

\images{0.3}{sph_3d.jpg}

\[x = \rho \cos \varphi \cos \theta\]
\[y = \rho \cos \varphi \sin \theta\]
\[z = \rho \sin \varphi\]
\[J_{\Phi} = \rho ^ 2 \cos \varphi\]

\subsubsection{Сферические координаты в $\mathbb{R}^m$}
\[\forall i \in [1, m - 2]: \varphi_i \in [0, \pi]\]
\[\varphi_{m - 1} \in [0, 2\pi]\]

Границы верные?

\[x_1 = \rho \cos \varphi_1\]
\[x_2 = \rho \sin \varphi_1 \cos \varphi_2\]
\[x_3 = \rho \sin \varphi_1 \sin \varphi_2 \cos \varphi_3\]
\[\vdots\]
\[x_{m - 1} = \rho \sin \varphi_1 \ldots \sin \varphi_{m - 2} \cos \varphi_{m - 1}\]
\[x_{m} = \rho \sin \varphi_1 \ldots \sin \varphi_{m - 2} \sin \varphi_{m - 1}\]

\[J_{\Phi} = \rho^{m - 1}(\sin \varphi_1)^{m - 2} (\sin \varphi_2)^{m - 3} \ldots (\sin \varphi_{m - 2})^{1}\]

\subsubsection{Условие $L_{loc}$}

\textit{Короче, всё что идёт с ``L'', это так или иначе про суммируемость функции}

$f: X \times \tilde{Y} \rightarrow \rinf, Y \subset \tilde{Y}, a$ --- предельная точка $Y$ в $\tilde{Y}$. 

$f$ удовлетворяет условию $L_{loc}(a): \exists g: X \rightarrow \rinf$ --- суммируемая, $\exists U(a): \forall$ почти всех $x \forall y \in U(a)$:

\[|f(x, y)| \le g(x)\]

\subsubsection{Интегральные неравенства Гельдера и Минковского}

Это просто переиначенные неравенства из 2го семестра на интегралы Лебега, приводятся без доказательств.

$(X, \mathfrak{A}, \mu)$ --- пространство с мерой, $f, g: $ п. в. $E \subset X \rightarrow \mathbb{R}$

\textbf{Неравенство Гельдера}

$p, q > 1, \dbl \frac{1}{p} + \frac{1}{q} = 1$

\[\left(\int_{E} |fg| d\mu \right) \le \left(\int_{E} |f|^p d\mu\right)^{\frac{1}{p}} \cdot \left(\int_{E} |g|^q d\mu\right)^{\frac{1}{q}}\]

\textbf{Неравенство Минковского}

$1 \le p < + \infty$

\[\left(\int_{E} |f + g|^p d\mu \right)^{\frac{1}{p}} \le \left(\int_{E} |f|^p d\mu\right)^{\frac{1}{p}} + \left(\int_{E} |g|^p d\mu\right)^{\frac{1}{p}}\]

\subsubsection{Интеграл комплекснозначной функции}

База базовая: $(X, \mathfrak{A}, \mu), f: X \rightarrow \mathbb{C}$

\[\int_{E}f(z)d\mu = \int_{E}\Re(f(z))d\mu + i\int_{E}\Im(f(z))d\mu\]

Также измеримость и суммируемость следует из соответствующих свойств реальной и мнимой частей функций.

\subsubsection{Фундаментальная последовательность, полное пространство}

Фундаментальная последовательность в $L^p$:

\[\forall \varepsilon > 0 \dbl \exists N \dbl \forall n, m > N: ||f_n - f_m||_p < \varepsilon\]

Любая сходящаяся последовательность является фундаментальной. А если в пространстве выполняется и обратное (т. е. любая фундаментальная последовательность сходится), то оно называется \textbf{полным}.

\subsubsection{Плотное множество}

$A \subset X$ --- нормированное пространство

$A$ --- (всюду) плотное в $X$

\[\forall x \in X \dbl \forall \varepsilon > 0 \quad B(x, \varepsilon) \cap A\text{--- непусто}\]

\subsubsection{Нормальное топологическое пространство}

Нормальное топологическое пространство $X$ --- такое топологическое пространство $X$, в котором выполяются аксиомы:

\begin{enumerate}
    \item $F_0, F_1 \in X$ --- замкнутые, $F_0 \cap F_0 = \varnothing$
    
    Тогда $\exists U(F_0), U(F_1)$ --- открытые. $F_0 \subset U(F_0), \dbl F_1 \subset U(F_1), \dbl U(F_0) \cap U(F_1) = \varnothing$

    \item $\forall x \in X: \{x\}$ --- замкнутое
\end{enumerate}

Например, $\mathbb{R}^m$ является нормальным.

\subsubsection{Непрерывные финитные функции в $\mathbb{R}^m$}

Финитная функция --- функция, равная 0 вне некоторого шара

$C_0(\mathbb{R}^m)$ --- множество непрерывных финитных функций в $\mathbb{R}^m$

\subsubsection{Мера Лебега-Стилтьеса, мера Бореля-Стилтьеса}

\begin{enumerate}
    \item $\mathcal{P}^{1}, g: \mathbb{R} \rightarrow \mathbb{R}, $ возрастает, непрерывно
    \[\mu_g[a, b) := g(b) - g(a)\]

    --- счётно аддитивная мера
    \item $g$ --- возрастает, не обязательно непрерывно
    
    \[\mu_g[a, b) = g(b - 0) - g(a - 0)\]

    --- мера
\end{enumerate}

Запускаем теорему о продолжении, тогда

$\exists \mathfrak{A} \supset \mathcal{P}^{1} \exists$ продолжение $\mu_g \subset \mathcal{P}$ на $\mathfrak{A}$

$\mu_g$ --- полная мера на $\mathfrak{A}$ --- мера Лебега-Стильтьеса

Если рассмотреть $\mu_g$ на борелевском $\mathfrak{B} \rightarrow \rinf$ --- мера Бореля. (тут надо заметить, что если мы запустим на борелевском множестве, мера не обязательно будет полной (?)).


\subsubsection{Функция распределения}
$(X, \mathfrak{A}, \mu)$, $h: X \rightarrow \rinf$, измерима, почти всюду конечна

$\forall t \in \mathbb{R} \quad \mu X(h < t) < +\infty$

Пусть $H(t) = \mu X(h < t)$ --- возрастающая

$H(t)$ --- называется функцией распределения по мере $\mu$

\subsubsection{Остатки определений про $L^p$}

$L^p[0, \tau]$ --- множество $\tau$-периодичных функций $\mathbb{R} \rightarrow \mathbb{R}$

\[||f|| = \int_0^{\tau} |f|^p d\lambda_1 < +\infty\]

\[\forall a : \int_0^{\tau} |f|^p d\lambda_1 = \int_a^{\tau + a} |f|^p d\lambda_1\]

$\tilde{C}[0, \tau]$ --- множество непрерывных $\tau$-периодичных функций

\[f(0) = f(\tau)\]

\[||f|| = \max{|f|}\]

(максимум тут достигается, потому что периодичная функция как-бы строит компакт, так как концы равны (?)). А ещё $\tilde{C}[0, \tau]$ плотно в $L^p[0, \tau]$.

\newpage

\subsection{Важные теоремы}

\subsubsection{Теорема Лебега о мажорированной сходимости для случая сходимости почти везде}
\textit{Формулировка:}

\begin{itemize}
    \item $(X, \mathfrak{A}, \mu)$ --- пространство с мерой
    \item $f_n, f: X \rightarrow \rinf$ --- измеримые
    \item $f_n \rightarrow f$ почти всюду
    \item $\exists g: X \rightarrow \rinf$ --- суммируемая, и $\forall n$ и при почти всех $x \dbl |f_n(x)| \le g(x)$
\end{itemize}

Тогда:

\[\int_{X}|f_n - f| d \mu \dbl \ntoinf \dbl 0\]

И, как очевидное (``уж тем более''):

\[\int_{X} f_n d\mu \dbl \ntoinf \dbl \int_{X} f d\mu\]


\textsc{Дисклеймер:}

Развеем все сомнения насчёт корректности условия (вдруг они у вас были):

\[\left| \int f_n - \int f \right| = \left| \int f_n - f \right| \le \int |f_n - f| \text{ (уж тем более)}\]

А также, наши функции из условия на самом деле даже суммируемые, не просто измеримые. Давайте для каждого $n$ соберём точки, на который $f_n$ не сходится к $f$, сложим (это всё будет множество меры 0) и вычтем, а на остатке сделаем предельный переход:

\[|f_n(x)| \le g(x)\]
\[|f(x)| \le g(x) < +\infty\]

\textit{Доказательство:}

Заведём последовательность $h_n := \sup (|f_n - f|, |f_{n + 1} - f|, |f_{n + 2} - f|, \ldots)$. Она убывает, так как по условию у нас есть сходимость почти везде. Также, можно ограничить её: $0 \le h_n \le 2 g$ (модули больше нуля и по условию все $|f_n| \ge g$). А ещё это просто определение последовательности из верхнего предела:

\[\lim_{n \rightarrow \infty} h_n = \overline{\lim_{n \rightarrow \infty}} |f_n - f| = 0 \text{ (почти везде)}\]

Теперь берём положительную возрастающую последовательность $2g - h_n$ и запускаем теорему Леви (см. 3 семестр, там как раз нужна возрастающая последовательность):

\[\int_{X} (2g - h_n) d\mu \ntoinf \int_{X} 2g d\mu\]

Откуда по линейности первого интеграла следует, что $\int_{X} h_n \ntoinf 0$, ну и добиваем:

\[0 \underset{n \rightarrow \infty}{\longleftarrow} \int_{X} h_n \ge \int_{X} |f_n - f| d \mu\]

ч. т. д. 

\subsubsection{Теорема Лебега о мажорированной сходимости для случая сходимости по мере}
\textit{Формулировка (то же самое, что и выше, только сходится по мере теперь):}

\begin{itemize}
    \item $(X, \mathfrak{A}, \mu)$ --- пространство с мерой
    \item $f_n, f: X \rightarrow \rinf$ --- измеримые
    \item $f_n \underset{\mu}{\Longrightarrow} f$
    \item $\exists g: X \rightarrow \rinf$ --- суммируемая, и $\forall n$ и при почти всех $x \dbl |f_n(x)| \le g(x)$
\end{itemize}

Тогда:

\[\int_{X}|f_n - f| d \mu \dbl \ntoinf \dbl 0\]

\textit{Доказательство:}

Рассмотрим 2 случая.

\textbf{1. $\mu X < + \infty$}

Зафиксируем $\varepsilon > 0$ и сооружаем множества $X_n = X(|f_n - f| \ge \varepsilon)$. Следовательно, $\mu X_n \ntoinf 0$, т.к. есть сходимость по мере. Расписываем:

\[\int_{X} |f_n - f| d\mu = \int_{X_n} |f_n - f| d\mu + \int_{X^c_n} |f_n - f| d\mu \le \underbrace{\int_{X_n} 2g d\mu}_{(1)} + \underbrace{\int_{X^c_n} \varepsilon d\mu}_{(2)}\]

$(1)$ --- оценка разности по условию, и ещё при больших $n$ меньше эпсилона по абсолютной непрерывности интеграла. (2) --- из условия о сходимости по мере выше оцениваем эпсилоном.

\[\le \varepsilon + \varepsilon \cdot \mu X_n^c \le \varepsilon \cdot (1 + \mu X)\]

(оцениваем меру дополнения просто всем пространством)

\textbf{2. $\mu X = \infty$}

Сначала докажем небольшое свойство интеграла по мере:

\[\forall \varepsilon > 0 \dbl \exists A \subset X \text{ измеримое } \mu A < + \infty \quad \int_{X \setminus A} g < \varepsilon\]

Если по-русски, то существует некоторое множество в исходном, на котором в основном концентрируется интеграл, следовательно, на остальном кусочке интеграл крайне мал. И мы можем предъявить такое для сколь угодно малого $\varepsilon$.

Рассмотрим интеграл как супремум ступенчатых функций:

\[\int_{X} g = \sup_{0 \le g_n \le |g|} \int_{X} g_n d\mu\]

Этот супремум значит, что существует какая-то $g_{n_0}$, хорошо ($\varepsilon$) оценивающая нашу функцию:

\[\exists g_{n_0}: \dbl \int_{X} g - g_{n_0} < \varepsilon\]

Давайте возьмём за $A$ носитель функции $g_{n_0}$:

\[A := \supp g_{n_0} = \{x: g_{n_0}(x) \neq 0\}\]

Так как ступенчатая функция есть сумма константы на характеристическую функцию, её интеграл конечен (?). Ну, а на ``хвостиках'' где она равна нулю нам не особо интересно. Таким образом, $\mu A < + \infty$:

\[\int_{X \setminus A} g d\mu = \int_{X \setminus A} g \underbrace{- g_{n_0}}_{\text{так как вне } A \,\, g_{n_0} = 0} \le \int_{X} g - g_{n_0} < \varepsilon\]

Ну и всё, раз доказали, давайте разобьём на два интеграла:

\[\int_{X} |f_n - f| d\mu = \underbrace{\int_{A} |f_n - f| d\mu}_{< \varepsilon \text{ по пункту 1}} - \underbrace{\int_{X \setminus A} |f_n - f| d\mu}_{<2 \varepsilon \text{ по доказанному выше}} \le 3 \varepsilon\]

ч. т. д. 

\subsubsection{Принцип Кавальери}
\textit{Формулировка:}

\begin{itemize}
    \item $(X, \mathfrak{A}, \mu), (Y, \mathfrak{B}, \nu)$ --- пространства с мерой
    \item $\mu, \nu$ --- $\sigma$-конечные, полные меры
    \item $C \in \mathfrak{C}$
    \item $m = \mu \times \nu, \mathfrak{C} = \mathfrak{A} \otimes \mathfrak{B}$
\end{itemize}

Тогда: 

\begin{enumerate}
    \item при почти всех $x \quad C_{x} \in \mathfrak{B}$ 
    \item $x \mapsto \nu C_{X}$ --- измеримо на $X$ (сама функция задана почти везде)
    \item $m C = \int_{X} \nu (C_{x}) d\mu(x)$
\end{enumerate}

Аналогично для сечений $C^{y}$

\textit{Замечания:}
\begin{enumerate}
    \item $C$ --- измеримо $\nRightarrow \forall x C_x$ --- измеримое 
    \item $\forall x \forall y: C_x, C^y$ --- измеримы $\nRightarrow C$ --- измеримо
\end{enumerate}

\textit{Доказательство:}

Введём $D$ --- это множество тех множеств, которые удовлетворяют принципу Кавальери :) . Давайте докажем, что разные типы множеств содержатся в $D$. А потом (внезапно) окажется, что это все множества.

\textbf{1. $G = A \times B$ (измеримые прямоугольники)}

Проверяем здесь и далее попунктно:

\begin{enumerate}
    \item Так как это прямоугольники, $C_x = \begin{cases}
        B, \dbl x \in A\\
        \varnothing, \dbl x \notin A
        \end{cases}$ (очев). Ну, значит, при всех $x: \dbl C_x \in \mathfrak{B}$
    \item Берём в качестве такой функции $\nu(B) \cdot \chi_A(x)$. Она измерима на $X$.
    \item Ну давайте поинтегрируем) $\int_{X} \nu(C_x) d\mu = \int_{X} \nu(B) \cdot \chi_A(x) d\mu = \nu(B) \cdot \mu(A) = m(A \times B)$
\end{enumerate}

\textbf{2. $E_i \in D, E_i$ дизъюнктны, $E = \bigsqcup_{\text{НБЧС}} E_i$. Тогда $E \in D$}

\begin{enumerate}
    \item $E_x = \bigsqcup (E_i)_x$. Обратите внимание, что все множества справа уже лежат в $D$, поэтому они ``измеримы'' (лежат в $\mathfrak{B}$) при почти всех $x$. Ну, значит и объединение их тоже.
    \item Если вы ещё не поняли, мы в этом пункте фактически хотим предоставить функцию вычисления меры сечения по заданному $x$. $\nu E_x = \sum \nu E_{i_x}$. Это сумма измеримых неотрицательных функций, определённых на почти всех $x$ (потому что кусочки уже лежат в $D$).
    \item $\int_{X} \nu(C_x) d\mu = \int_{X} \sum \nu E_{i_x} d\mu$. Тут напрашивается переставить местами сумму и интегрирование, и это можно сделать по теореме об интегрировании положительных рядов!. $\sum \int_{X} \nu E_{i_x} d\mu = \sum mE_i = (*)$ (кусочки уже в $D$, и по счётной аддитивности) $(*) = m E$
\end{enumerate}

\textbf{3. $E_i \in D, \dbl E_i \supset E_{i + 1} \supset \ldots, \dbl \bigcap E_i = E, \dbl mE_1 < +\infty$. Тогда $E \in D$}
\begin{enumerate}
    \item $E_x = \bigcap (E_i)_x$. Аналогично предыдущему.
    \item По теореме о непрерывности меры сверху (условия подходят): $\lim \nu E_{i_x} = \nu E_x$. Ну и тогда, добавляя оговорку о том, что всё это работает на тех $x$, на которых определены функции для кусочков, то и наша функция сопоставления измерима.
    \item $\int_{X} \nu(C_x) d\mu = \int_{X} \lim_{i \rightarrow \infty} \nu E_{i_x} d\mu = $. Замечаем, что все наши функции в пределе положительны (меры) и суммируемы (т. к. $0 \le \nu E_{i_x} \le \nu E_1 < +\infty$ по условию, значит суммируемы). Тогда запускаем теорему Лебега о мажорированной сходимости для случая почти везде (в обратку) и выигрываем! $= \lim_{i \rightarrow \infty} \int_{X} \nu E_{i_x} d\mu = \lim_{i \rightarrow \infty} m E_i = mE$ (последнее тоже по непрерывности меры сверху).
\end{enumerate}

Сделаем небольшое лирическое отступление в прошлое. Как мы помним, у нас есть теорема о продолжении меры, по которой, в частности, строилась и мера Лебега. По одному из её пунктов, меру предлагалось высчитывать, выбирая всё лучше оценивающее покрытие ячейками, и беря по всем таким покрытиям инфимум: $(\mathcal{P} $(п-к.), $\mu_0) \rightarrow (\mathfrak{A} (\sigma-$алг.$), \mu); \quad \mu A = \inf \{\sum \mu P_k, \dbl A \subset \bigcup P_k\}$. Также, если мы рассмотрим конкретно меру Лебега, то измеримое про ней множество можно представить (по теореме о регуляризации?) в виде $A \in \mathfrak{A}, \dbl A = B \setminus C$, где $B$ --- ``борелевское'', а $C$ --- ``меры 0'' (кавычки тут не просто так, ведь мы не задавали никаких топологий и прочего, чтобы их снять. Тут это для общего понимания происходящего). Ну и получается, что если берём за основу ``измеримости'' вот это определение с инфимумом, то $B$ представляется в виде $\bigcap_i \bigcup_j P_{ij}$ (типа взяли всевозможные покрытия и пересекли, получив тем самым наилучшее, что ли). И некоторый остаток меры 0. Однако, не стоит его недооценивать, у нас мера по условию принципа --- полная, а это значит, что ``иерархия'' на этих множествах должна соблюдаться (см. определение полной меры из 3 сем.). Рассматриваем всё это далее!

\textbf{4. $mE = 0$. Тогда $E \in D$}

То же самое: $\dbl mH = 0, \dbl H = \bigcap_i \bigcup_j P_{ij}, \dbl P_{ij} \in \mathfrak{A} \times \mathfrak{B}, \dbl E \subset H$. Заметим, что $H \in D$ по пункту 3.

\begin{enumerate}
    \item $0 = mH = \int_{X} \nu (H_x) d\mu$. Если так случилось, то логично, что $\nu(H_x) = 0$ п. в. $x$. Ну тогда $\nu(E_x) = 0$ при этих $x$, так как $E_x \subset H_x$ по полноте меры. 
    \item Доказано предыдущим пунктом, всё 0.
    \item Как следствие, $\int_{X} \nu (E_x) = 0 = mE$
\end{enumerate}

\textbf{5. $A \in \mathfrak{A} \otimes \mathfrak{B}, mA < + \infty$. Тогда $A \in D$}

Пользуясь лирическим отступлением (и ``обобщённой регулярностью''): $A = B \setminus C, \dbl B = \bigcap_{i} \bigcup_{j} P_{ij} \in D, \dbl mC = 0 \Rightarrow C \in D$

\begin{enumerate}
    \item $mA = mB - mC = mB$, сечения: $A_x = B_x \setminus C_x$ (измеримы при п. в. $x$, т. к. составляющие уже в $D$)
    \item Из общих соображений, $\nu B_x - \nu C_x \le \nu A_x$. С другой стороны, по монотонности ($A \subset B$): $\nu A_x \le \nu B_x$. А т. к. $\nu C_x = 0$ при п. в. $x$, то при тех же $x: \nu A_x = \nu B_x$.
    \item $\int_{X} \nu A_x d\mu = \int_{X} \nu B_x d\mu = $ (оно уже в $D$) $ = mB = mA$ (из начала).
\end{enumerate}

Ну и всё, осталось обощить всё вышеперечисленное и показать, что всё-таки любое множество лежит в нашем классе $D$ (фактически, остались только множества бесконечной меры).

\textbf{6. $A \in \mathfrak{A} \times \mathfrak{B}$ --- любое $ \in D$}

$\mu A = +\infty$. Запускаем $\sigma-$конечность: $X = \bigsqcup X_k, Y = \bigsqcup Y_i$. С другой стороны, $X \times Y = \bigsqcup X_k \times Y_i$. Тогда $A \cap (X_k \times Y_i) \in D$ по пункту 5 (конечная мера), а их дизъюнктное объединение $\bigsqcup A \cap (X_k \times Y_i) \in D$ по пункту 2.

ч. т. д. 

\textit{Следствия:}

\begin{enumerate}
    \item $C \in \mathfrak{A} \otimes \mathfrak{B}, \dbl P_1(C) = \{x \in X: C_x \neq 0\}$ (проекция на $X$) и она измерима на нём, то меру можно считать по ней $mC = \int_{P_1(C)} \nu (C_x) d\mu$. Это очевидно (ну просто проекция удаляет те точки, где сечение и так было равно нулю).
    \item $f: [a, b] \rightarrow \mathbb{R}$. Тогда $\int_a^b f(x) = \int_{[a, b]} f d\lambda_1$
    
    \textit{Доказательство:}

    Достаточно рассмотреть неотрицательную функцию, т. к. оба интеграла аддитивны и можно просто разбить. Тогда, ПГ$(f, [a, b]) = C$ --- измеримое множество (очев). А $C_x = [0, f(x)]$ (см. картинку). Причём, если вспомнить 2й сем, то окажется, что той загадочной площадью $\sigma$, которую мы использовали в рассуждениях, может быть и $\lambda$! Давайте посмотрим поближе: $\lambda(C_x) = \lambda([0, f(x)]) = f(x)$.

    \images{0.3}{kavalieri.jpg}

    $\int_a^b f(x) dx = \lambda_2(\text{ПГ}(f, [a, b])) = $ (по следствию 1 можем считать просто на проекции) $= \int_{[a, b]} \lambda(C_x) d \lambda_1 = \int_{[a, b]} f(x) d\lambda_1$
\end{enumerate}

\subsubsection{Теорема Фубини}
\textit{Формулировка:}

\begin{itemize}
    \item $(X, \mathfrak{A}, \mu), (Y, \mathfrak{B}, \nu)$ --- пространства с мерой
    \item $\mu, \nu$ --- $\sigma$-конечные меры
    \item $m = \mu \times \nu$
    \item $f: X \times Y \rightarrow \rinf$, суммируема на $X \times Y$ по мере $m$
\end{itemize}

Тогда:

\begin{enumerate}
    \item при почти всех $x$ функция $f_{x}$ суммируема на $Y$
    \item $x \mapsto \varphi(x) = \int_{Y} f_{x} d\nu$ --- это суммируемая функция на $X$
    \item \[\int_{X \times Y} f dm= \int_{X} \varphi(x) d \mu(x) = \int_{X} \left( \int_{Y} f(x, y) d  \nu (y)\right) d \mu(x)\]
\end{enumerate}

\textit{Доказательство:}

Теорема-клон Тонелли, и выводится ровно из неё.

\textbf{Подготовка}

$0 \le f_-, f_+ \le |f|$ --- по определению суммируемых функций. Также сразу заметим, что:

\[\underbrace{\int_{X \times Y} f_{-, +} dm}_{(1)} = \underbrace{\int_{X} \left( \int_{Y} f_{-, +} d\nu \right) d\mu}_{(2)} < +\infty \] 

(это всё потому что они измеримы, поэтому применяем Тонелли. Ну и интегралы конечны почти везде в силу суммируемости самой $f$).

$(f_-)_x, (f_+)_x$ --- измеримы по Тонелли. Можно также рассмотреть и 2й пункт:

\[\varphi_- = \int_{Y} (f_-)_x d\nu, \quad \varphi_+ = \int_{Y} (f_+)_x d\nu \]

Эти функции точно так же измеримы по Тонелли \textit{(note для душнил: да, измеримо\textbf{*}, на области определения и почти везде, но кажется, что это уже и так все поняли)}.

По гига-неравенству с интегралами сразу делаем вывод, что $f_-, f_+$ --- суммируемы (интеграл (1) конечен). Но также и $\varphi_-, \varphi_+$ --- суммируемы по интегралу $(2)$.

\textbf{Содержательная часть}

\begin{enumerate}
    \item $f_x = (f_+)_x - (f_-)_x$ --- суммируемая как сумма суммируемых.
    \item $\varphi = \varphi_+ - \varphi_+$ --- аналогично.
    \item $\int_{X \times Y} f dm =  \int_{X \times Y} f_+ dm - \int_{X \times Y} f_- dm$ (по определению). Ну и дальше можно расписать $\int_{X} \int{Y} \ldots$, погруппировав, но всем уже и так всё понятно.
\end{enumerate}

ч. т. д. 

\textit{Следствие (аналогичное принципу Кавальери) [валидно также и для Тонелли]:}

$C \in \mathfrak{A} \otimes \mathfrak{B}, f$ --- суммируемая [измеримая, $\ge 0$]. Если $P_1(C)$ --- измеримо на $X$, тогда:

\[\int_{C} f dm = \int_{P_1(C)} \int_{C_x} fd\nu d\mu\]

\textit{Доказательство:}

Полагаем, что $f$ вне проекции равно 0 и не вносит ничего в результат.


\subsubsection{Теорема о преобразовании меры при диффеоморфизме}
\textit{Формулировка:}

\begin{itemize}
    \item $\Phi: O \subset \mathbb{R}^{m} \rightarrow \mathbb{R}^{m}$, диффеоморфизм
\end{itemize}

Тогда $\forall A \in \mathbb{M}^{m}, A \subset O$

\[\lambda \Phi(a) = \int_{A} |\det \Phi'(x)| dx\]

\textit{Доказательство:}

Введём обозначения: $J_\Phi = \det \Phi'(x), \nu A = \lambda \Phi(A)$. Тогда необходимо проверить, что $J_{\Phi}$ является плотностью $\nu$ относительно $\lambda$:

\[\lambda A \inf_{A} |J_{\Phi}| \le \nu A \le \lambda A\sup_{A} |J_{\Phi}|\]

Сразу скажем, что нам достаточно доказать лишь правую часть. У нас отображение --- диффеоморфизм, поэтому супремум и инфимум как бы ``обратны друг другу'' (в смысле обратности функций):

\[\inf_{A} |\det \Phi'(x)| \le \nu A \dbl | \dbl pow\left(-1\right)\]

\[\nu A \le \frac{1}{\inf_{A} |\det \Phi'(x)|}\]

Но, тк у нас диффеоморфизм, мы можем рассмотреть также $|\det \Phi'^{-1}(x)| = \frac{1}{|\det \Phi'(x)|}$. И логично, что в точке инфимума будет достигаться супремум обратного оператора.

\[\nu A \le\sup_{A} |\det \Phi'(x)|\]

\textbf{1. $A \subset \overline{A} \subset Q$  --- кубическая ячейка}

Доказываем от противного. Пусть предпосылка не выполняется:

\[\nu Q > \lambda Q \sup_{Q} |J_{\Phi}|\]

\images{0.3}{diff_del.jpg}

Давайте выберем такое $c$, чтобы $\nu Q > c \lambda Q > \lambda Q \sup_{Q} |J_{\Phi}|$. Запускаем половинное деление на кубе (покоординатное, $2^m$ частей). Утверждается, что найдётся хоть один кусочек куба, на котором это неравенство выполняется (а если не найдётся, то мы можем посуммировать неравенства, и предпосылка перестанет выполняться). Выбираем этот кусочек, и запускаем деление на нём и так далее до посинения. По теореме Кантора в пересечении их замыканий будет точка $a: \bigcap \overline{Q_i} = a$. И вдруг оказывается, что тогда не выполняется лемма об оценке малых кубов в точке $a$! ($c > \sup_{Q} |J_{\Phi}| = \sup_{\overline{Q}} |J_{\Phi}| > |J_{\Phi}| = |\det \Phi'(a)|$)А вообще-то она должна выполняться. Таким образом --- противоречие, значит неравенство выполняется.

\textbf{2. $A$ --- открытое}

Опять вспоминаем 3й сем. Любое открытое множество можно разбить на дизъюнктное объединение кубических ячеек. $A = \bigsqcup Q_i$. Оценим каждую:

\[\nu Q_i < \lambda Q_i \sup_{Q_i} |J_{\Phi}| \le \lambda Q_i \sup_{A} |J_{\Phi}|\]

Оценили супремумом по всему множеству. Ну и теперь сумммируем всё по $i$:

\[\nu A \le \lambda A \sup_{A} |J_{\Phi}|\]

\textbf{3. $A$ --- измеримое}

Возьмём открытое $G$, такое что $A \subset G$ и запустим по предыдущему пункту:

\[\nu G \le \lambda G \sup_{G} |J_{\Phi}|\]

Берём инифмум по обоим частям, левая достигается по регулярности меры Лебега, а правая --- по Лемме 2 (из леммы об оценке малых кубов).

\[\nu A \le \lambda A \sup_{A} |J_{\Phi}|\]

ч. т. д. 

\subsubsection{Теорема о гладкой замене переменной в интеграле Лебега}
\textit{Формулировка:}

\begin{itemize}
    \item $\Phi: O \subset \mathbb{R}^{m} \rightarrow \mathbb{R}^{m}$, диффеоморфизм
    \item $\Phi(O) = O'$
    \item $f: O' \rightarrow \mathbb{R} \ge 0$ --- измерима
\end{itemize}

Тогда:

\[\int_{O'} fdx = \int_{O} f\left(\Phi(x)\right) \cdot |\det \Phi'(x) d\lambda(x)|\]

\textit{Доказательство:}

Запустим теорему о вычислении интеграла по взвешенному образу меры ($\omega(x) = |\det \Phi'(x)|, \nu(A) = \lambda \Phi(A)$). Тогда искомое --- результат её работы.

ч. т. д.

\textit{Следствие:}

Если $A' \subset O', \Phi(A) = A'$, то:

\[\int_{A'} f = \int_{A} f \circ \Phi |J_{\Phi}| dx\]

\subsubsection{Теорема о непрерывности сдвига}
    \textit{Формулировка:}

    \begin{itemize}
        \item $f: \mathbb{R}^{m} \rightarrow \rinf$
        \item $h \in \mathbb{R}^{m}$
        \item $f_{h}(x) := f(x + h)$
    \end{itemize}

    \begin{enumerate}
        \item $f$ --- равномерно непрерывна в $\mathbb{R}^{m}$
        
        Тогда $||f_h - f||_{\infty} \goesto{h \rightarrow 0} 0$

        \item $1 \le p < +\infty, \dbl f \in L^{p}(\mathbb{R}^{m})$
        
        Тогда $||f_h - f||_{p} \goesto{h \rightarrow 0} 0$

        \item $f \in \tilde{C}[0, \tau]$
        
        Тогда $||f_h - f||_{\infty} \goesto{} 0$

        \item $1 \le p < +\infty, \dbl f \in L^{p}[0, \tau]$
        
        Тогда $||f_h - f||_{p} \goesto{} 0$
    \end{enumerate}

    \textit{Доказательство:}

    \textbf{1. + 3.}

    По определению равномерной непрерывности:

    \[\forall \varepsilon > 0 \dbl \exists \delta > 0 \dbl \forall x, y \dbl |x - y| < \delta : |f(x) - f(y)| < \varepsilon\]

    Заметим, как удачно тут стоит неравенство $|x - y|$, ведь для случая 1 это норма, а для 3 --- модуль. И всё работает! А ещё все эти модули-нормы не совсем обычные. Дело в том, что мы работаем с периодическими функциями, поэтому там близость понимается в смысле: отрезок $[0, 1]$, период 1, точки $1 - \varepsilon$ и $1 + \varepsilon$ лежат на расстоянии $1 - \varepsilon$. А точки $\varepsilon$ и $1 + \varepsilon$ фактически равны друг другу (по периодичности)

    \images{0.5}{rnepr_sdv.jpg}
    
    Проверим это: $f \in \tilde{C}[0, \tau] \Rightarrow f$ --- равномерно непрерывна. Вроде бы очевидно (запускаем теорему Кантора на $[0, \tau]$ и мапим точки в этот отрезок по периодичности). Ну всё, раз у нас для функции заказана равномерная непрерывность, можем проверять объект исследования (специально сделали возможность быть равным эпсилону, это не так важно):

    \[\forall \varepsilon > 0 \dbl \exists \delta > 0 \dbl \forall |h| < \delta: \sup |f(x + h) - f(x)| \le \varepsilon\]

    Ну и всё, просто подставляем из определения равномерной непрерывности неравенство и всё получается (для этого равенство и оставили).

    \textbf{2. и 4.}

    Закажем $g$ из множества финитных (2) и непрерывных периодических (4) функций. Они оба плотны в соответствующих пространствах, а значит мы можем заказать их с хотелкой:

    \[\forall \varepsilon > 0 \dbl ||f - g||_p < \frac{\varepsilon}{3}\]

    Оценим искомое по неравенству треугольника:

    \[||f_h - f||_h \le \underbrace{||f_h - g_h||_p}_{(1)} + \underbrace{||g_h - g||_p}_{(2)} + \underbrace{||f - g||_p}_{(3)} \le (**)\]

    (3) меньше по заказанному. (1):

    \[||f_h - g_h||_p = \left( \int_{(*)} (f(x + h) - g(x + h))^p\right)^{\frac{1}{p}}\]

    (*) --- если это (4.), то там сдвиг ничего не меняет, по определению. Если (2.) --- то $\mathbb{R}^m$, то там тоже бесплатный (почему??? \sout{может быть, по равномерной непрерывности $x \mapsto f(x) - g(x)$, но откуда она тогда берётся?} да нет, всё просто! $t = x + h$, cделали замену, дифференциал не поменялся, на границы просто пофигу видимо, всё равно по всему $\mathbb{R}^m$ интегрируется)

    Осталось доказать лишь (2), отдельно для каждого случая.

    \textbf{4.}

    $g$ --- непрервыная периодическая функция. Оценим:

    \[||g_h - g||_p^p = \int_{[0, \tau]} (g(x + h) - g(x))^p \le \lambda[0, \tau] \cdot \left(\sup_{x \in [0, \tau]} g(x + h) - g(x)\right)^p\]

    Супремум точно меньше всего, что мы закажем, так как мы в самом начале доказали, что $g \in \tilde{C}[0, \tau]$ --- равномерно непрерывна.

    \[||g_h - g||_p \le \tau^{\frac{1}{p}} \cdot \underbrace{\sup_{x \in [0, \tau]} g(x + h) - g(x)}_{\le \frac{\varepsilon}{3\tau^{\frac{1}{p}}}} \le \frac{\varepsilon}{3}\]

    При $|h| < \delta\left(\frac{\varepsilon}{3\tau^{\frac{1}{p}}}\right)$

    \textbf{2.}

    $g$ --- непрерывная финитная функция. Пусть $|h| < 1$, оно всё равно стремится к 0. Раз она финитная, её носитель (множество иксов, на которых она не ноль), помещается в шарик, пусть $B(0, R)$. Заметим, что теперь нам достаточно рассматривать норму на пространстве $L^p(B(0, R + 1), \lambda_m)$ (немножко раздули, чтобы сдвиги поместились), в остальных местах там всё выполнено, константный 0. А ещё заметим, что на $\overline{B(0, R + 1)}$ функция равномерно непрерывна по теореме Кантора (компакт, ёлки-палки!)

    \[||g_h - h||_p \le \left( \lambda_m B(0, R + 1)\right)^{\frac{1}{p}} \cdot \sup_{x \in \overline{B(0, R + 1)}} |g(x + h) - g(x)| \le \frac{\varepsilon}{3}\]

    \textbf{Итого:}

    \[(**) \le \frac{\varepsilon}{3} + \frac{\varepsilon}{3} + \frac{\varepsilon}{3} = \varepsilon\]

    ч. т. д. 

\newpage

\subsection{Теоремы}

\subsubsection{Теорема об интегрировании положительных рядов}
\textit{Формулировка:}

\begin{itemize}
    \item $(X, \mathfrak{A}, \mu)$ --- пространство с мерой
    \item $u_n: X \rightarrow \rinf, u_n \ge 0$ (при почти всех $x$ ?)
    \item $u_n$ --- измеримы на $E \in \mathfrak{A}$
\end{itemize}

Тогда: 

\[\int_{E} \left(\sum_{n = 1}^{\infty} u_n(x)\right)d\mu(x) = \sum_{n = 1}^{\infty} \left(\int_{E} u_n(x) d\mu(x)\right)\]

\textit{Доказательство:}

Подгоним под теорему Леви 3 (3 семестр). Пусть $S_{N}(x) = \sum_{n = 1}^{N} u_n(x)$ --- последовательность частичных сумм. Очевидно, что эта последовательность --- монотонно неубывающая (так как функции у нас неотрицательные): 

\[0 \le S_{N} \le S_{N + 1} \le S_{N + 2} \le \ldots\]

Тогда, делаем предельный переход (вот тут есть вопрос, почему должен существовать предел, но если подумать: если его не существует, вообще вся эта теорема не имеет смысла (ну бесконечности, чел, смысл их интегрировать)). А так же, измеримость сохраняется, так как у нас исходные функции все были измеримы (ну и по теореме о пределе измеримых функций): 

\[S_{N}(x) \toinf{N} S(x)\]

Ну и всё, значит, по теореме Леви можем перейти к предельному переходу интегралов: 

\[\int_{E} S_{N}(x) d\mu(x) \toinf{N} \int_{E} S(x) d\mu(x)\]

Левую часть можно расписать по линейности интеграла (там у нас конечное число членов): 

\[\int_{E} S_{N}(x) d\mu(x) = \sum_{n = 1}^{N} \int_{E} u_n(x) d\mu(x)\]

Ну, а раз интграл суммы стремится к интегралу предельной функции, то и сумма интегралов обязана туда стремиться.

\[\sum_{n = 1}^{N} \int_{E} u_n(x) d\mu(x) \toinf{N} \sum_{n = 1}^{\infty} \int_{E} u_n(x) d\mu(x)\]

ч. т. д. 


\textit{Следствие: }

\begin{itemize}
    \item $u_n: X \rightarrow \mathbb{R}$, измеримы на $E \in \mathfrak{A}$
    \item $\sum \int_{E} |u_n(x)| d\mu < +\infty$ (конечна)
\end{itemize}

Тогда $\sum u_n(x)$ --- абсолютно сходящийся при почти всех $x$

\textit{Доказательство: }

Пусть: 

\[S(x) = \sum_{n = 1}^{\infty} \left|u_n(x)\right|\]

Тогда, по предыдущей теореме: 

\[\int_{E} S(x) d\mu = \sum_{n = 1}^{\infty} \left(\int_{E} |u_n(x)| d\mu\right) < +\infty\]

Раз интеграл конечен, значит $S(x)$ --- суммируема, а это значит, что $S(x)$ --- почти везде конечна. Ну значит и сходится.

ч. т. д.

\textit{Пример: }

\begin{itemize}
    \item $(x_n)$ --- вещественная последовательность
    \item $\sum a_n$ --- абсолютно сходящийся числовой ряд
\end{itemize}

Тогда функциональный ряд $\sum \frac{a_n}{\sqrt{|x - x_n|}}$  --- абсолютно сходится при почти всех $x$ (в $\mathbb{R}$ по мере Лебега)

\textit{Доказательство: }

Во-первых, можно доказать, что если для $\forall A$ на $[-A, A]$ абсолютно сходится почти везде, то и везде (на $\mathbb{R}$) почти везде сходится (лол). Счётное количество п. в. $\Rightarrow$ п. в. (чтобы количество отрезков было счётным, надо чтобы $A$ были хотя бы рациональными. Кажется, что это не сильная проблема, так как отрезки включают в себя и все вещественные числа на отрезке тоже).

Попробуем подогнать под предыдущую теорему: 

\[\int_{[-A, A]}\frac{|a_n|}{\sqrt{|x - x_n|}} d\lambda = |a_n| \int_{-A}^{A} \frac{dx}{\sqrt{|x - x_n|}} \le\]

Так, стоп. А как мы перешли к определённому интегралу? Оказывается, что так можно делать, на доказано это будет позже (в курсе).

\[\underset{x := x - x_n}{\le} |a_n| \int_{-A - x_n}^{A - x_n} \frac{dx}{\sqrt{|x|}} \le |a_n| \int_{-A}^{A} \frac{dx}{\sqrt{|x|}} \le\]

Почему верен последний переход? Посмотрим на картинке: 

\images{0.5}{sh_pol_r.png}

Ну, по ней очевидно, что мы откусили кусочек поменьше, а добавили побольше. Тогда оценим модуль: 

\[ \le 2 \cdot |a_n| \int_{0}^{A}\frac{dx}{\sqrt{|x|}} = 4 \cdot \sqrt{A} \cdot |a_n|\]

Всё, абсолютный интеграл ограничен, значит сходится (при почти всех $x$).

ч. т. д. 

\subsubsection{Абсолютная непрерывность интеграла}
\textit{Формулировка:}

\begin{itemize}
    \item $(X, \mathfrak{A}, \mu)$ --- пространство с мерой
    \item $f: X \rightarrow \rinf$ --- суммируемая
\end{itemize}

Тогда:

\[\forall \varepsilon > 0 \dbl \exists \delta > 0, \quad \forall E\text{--- измеримое} \dbl \mu E < \delta \qquad \left|\int_{E} f d \mu\right| < \varepsilon\]

\textit{Доказательство:}

Для доказательства сего факта нам бы хотелось поисследовать, как на таких множествах ведёт себя функция в зависимости он величины её значений на соответствующих множествах. Давайте заведём множества $X_n$:

\[X_n = X(|f| \ge n)\]

Заметим, что $\ldots \supset X_n \supset X_{n + 1} \supset \ldots$. Причём:

\[\bigcap X_n = X_{\infty} = X(|f| = \infty)\]

А также, ведь по условию наша функция $f$ суммируема, значит она почти везде конечна (а там, где не конечна --- множество меры 0):

\[\mu \left( \bigcap X_n \right) = 0\]

Теперь заведём вспомогательную меру:

\[ \nu(A) = \int_{A} f d\mu\]

И внезапно заметим, что для неё выполняется теорема об непрерывности меры сверху! ($X_0 = X$, так как там у нас условие модуль больший нуля, и интеграл по нему конечен, так как функция суммируема):

\[\nu(X_0) = \int_{X_0 = X} |f| \mu < +\infty\]

Ну а в пересечении, как мы уже выяснили, у нас множество меры ноль (а на нём интеграл тоже нулевой):

\[\nu\left( \bigcap X_n \right) = 0\]

Таким образом, $\nu(X_n) \ntoinf 0$. И это даёт нам право с полной уверенностью сказать, что:

\[\forall \varepsilon > 0 \dbl \exists n_\varepsilon \quad \int_{X_{n_\varepsilon}} |f| d\mu < \frac{\varepsilon}{2}\]

Все приготовления сделаны, давайте оценивать:

\[\forall \varepsilon > 0 \dbl \delta := \frac{\varepsilon}{2 n_{\varepsilon}} \dbl \mu E < \delta \qquad \left|\int_{X_{n_\varepsilon}} f d\mu\right| \le \int_{X_{n_\varepsilon}} |f| d\mu = \int_{E \cap X_{n_\varepsilon}} |f| d\mu + \int_{E \cap X^c_{n_{\varepsilon}}} |f| d\mu\]

Первое слагаемое оценим $X_{n_{\varepsilon}}$, для которого у нас уже есть готовое утверждение выше. А второе оценим мерой, умноженной на $n_\varepsilon$. Так можно сделать, ведь дополнение $X_{n_\varepsilon}$ есть множество точек, на котором функция $< n_{\varepsilon}$

\[\le \frac{\varepsilon}{2} + n_{\varepsilon} \cdot \overbrace{\underbrace{\mu \left( E \cap X^c_{n_{\varepsilon}}\right) \le \mu\left( E \right) < \delta}} \le \frac{\varepsilon}{2} + \frac{\varepsilon}{2} = \varepsilon\]

ч. т. д. 

\textit{Следствие:}
\begin{itemize}
    \item $(e_n) \in \mathfrak{A}$ --- последовательность (?) множеств\
    \item $\mu e_n \ntoinf 0$
    \item $f$ --- суммируемая на $X$
\end{itemize}

Тогда:

\[\int_{e_n} f d \mu \ntoinf 0\]

\textit{Доказательство:}


Очевидно следует из теоремы, ну камон)

\subsubsection{Теорема о произведении мер}
\textit{Формулировка:}

\begin{itemize}
    \item $(X, \mathfrak{A}, \mu)$, $(Y, \mathfrak{B}, \nu)$ --- пространства с мерой (полукольца (?))
    \item Зададим $m_0(A \times B) = \mu A \cdot \nu B$
\end{itemize}

Тогда:

\begin{enumerate}
    \item $m_0$ --- мера на $\mathfrak{A} \times \mathfrak{B}$
    \item $\mu, \nu$ --- $\sigma$-конечные меры $\Longrightarrow$ $m_0$ --- $\sigma$-конечная
\end{enumerate}

\textit{Доказательство:}

\textbf{1.}

Давайте рассмотрим какой-то $P = \bigsqcup P_k$ --- измеримые прямоугольники. Чтобы доказать, что это действительно мера на $\mathfrak{A} \times \mathfrak{B}$, необходимо доказать счётную аддитивность: $m_0(P) \underset{?}{=} \sum m_0(P_k)$

Верно, что $P = A \times B, P_k = A_k \times B_k$ (наше множество есть результат перемножение множеств из каждого пространства). Также из этого следует, что:

\[\chi_P = \sum \chi_{P_k}\]
\[\chi_A(x)\chi_B(y) = \sum \chi_{A_k}(x)\chi_{B_k}(y)\]

Поинтегрируем это по $Y$!

\[\chi_A(x) \nu(B) = \sum \chi_{A_k}(x)\nu(B_k)\]

А теперь по $X$!

\[\mu(A)\nu(B) = \sum \mu(A_k) \nu(B_k)\]

Всё проверили, это действительно мера.

\textbf{2.}

По сигма-конечности исходных мер, мы можем разбить исходные пространства на счётное объединение множеств, имеющих конечную меру.

\[X = \bigcup X_k, \dbl \mu X_k < +\infty\]
\[Y = \bigcup Y_k, \dbl \nu Y_n < +\infty\]

Ну и тогда мера перемножения двух этих множеств будет просто результатом перемножения нескольких конечных чисел и их сумма, что, очевидно, конечно:

\[X \times Y = \bigcup_{(i, j)} X_i \times Y_j\]
\[m_0(X \times Y) = \sum_{(i, j)} \mu(X_i) \cdot \nu(Y_j)\]

ч. т. д. 

\subsubsection{Теорема Тонелли}
\textit{Формулировка:}

$f_x(y) = f^y(x) = f(x, y)$ --- новая нотация для функций с фиксированным аргументом.

\begin{itemize}
    \item $(X, \mathfrak{A}, \mu), (Y, \mathfrak{B}, \nu)$ --- пространства с мерой
    \item $\mu, \nu$ --- $\sigma$-конечные меры
    \item $m = \mu \times \nu$
    \item $f: X \times Y \rightarrow \rinf \ge 0$, измерима относительно $\mathfrak{A} \otimes \mathfrak{B}$
\end{itemize}

Тогда:

\begin{enumerate}
    \item при почти всех $x$ функция$f_{x}$ измерима на $Y$
    \item $x \mapsto \varphi(x) = \int_{Y} f_{x} d\nu$ --- это измеримая функция на $X$
    \item \[\int_{X \times Y} f dm= \int_{X} \varphi(x) d \mu(x) = \int_{X} \left( \int_{Y} f(x, y) d  \nu (y)\right) d \mu(x)\]
\end{enumerate}

Всё то же самое валидно и для $y$.

\textit{Доказательство:}

будет принципом ``конструктора''. Соберём измеримую функцию из кусочков.

\textbf{1. $C \in \mathfrak{A} \otimes \mathfrak{B}, f = \chi_C$}

То есть, сначала рассмотрим функцию-``ступеньку''. 

\begin{enumerate}
    \item $f_x = \chi_{C_x}$ --- она измерима тогда, когда измеримо $C_x$. А оно измеримо по принципу Кавальери!
    \item $\int_{Y} f_x d\nu = \int_{Y} \chi_{C_x} d\nu = \nu(C_x)$ --- измеримо по принципу Кавальери!
    \item $\int_{X} \varphi(x) d\mu = \int_{X} \nu(C_x) d\mu = mC$ (по принципу Кавальери). Тогда в обратную сторону $= \int_{X \times Y} \chi_{C} dm = \int_{X \times Y} f dm$
\end{enumerate}

\textbf{2. $f = \sum_{i = 1}^{n} c_i \chi_i \ge 0$ --- ступенчатая}

\begin{enumerate}
    \item $f_x = \sum c_i (\chi_{C_I})_x$ --- аналогично предыдущему пункту, сумма измеримых почти везде.
    \item $\int_{Y} f_x d\nu = \int_{Y} \sum c_i (\chi_{C_i})_x d\nu = \sum c_i \int_{Y} (\chi_{C_i})_x d\nu$. Ну и собственно говоря, у нас конечная сумма почти везде измеримых функций. Всё хорошо.
    \item $\int_{X \times Y} f dm = \sum c_i \int_{X \times Y} \chi_{C_i} dm=$ вот тут просто раскрываем по пункту для ``ступеньки'' и заносим сумму внутрь $=\sum c_i \int_{X} \left(\int_{Y} \chi_{C_i} d\nu\right) d\mu = \int_{X} \sum c_i \left( \int_{Y} \chi_{C_i} d\nu \right) d\mu = \int_{X} \int_{Y} \left( \sum c_i \chi_{C_i}\right) d\nu d\mu$
\end{enumerate}

\textbf{3. $f \ge 0$ --- измеримая}

Идея: аппроксимация + теорема Леви $\times \, 2$. \textit{(в этом разделе постоянно используется такой приём --- прим. авт.)}

Запускаем теорему о характеризации измеримых функций ступенчатыми (?), $f = \lim_{n \rightarrow \infty} g_n, g_n$ --- ступенчатые, возрастающие.

\begin{enumerate}
    \item $f_x = \lim_{n \rightarrow \infty} (g_n)_x$ --- измерима как предел измеримых функций (3 сем).
    \item $\int_{Y} f_x d\nu \underset{\infty \leftarrow n}{\longleftarrow} \int_{Y} (g_n)_x d\nu$ (по теореме Леви). Предел измеримых.
    \item Обозначим интеграл каждой ступенчатой функции как $\varphi_n(x) = \int_{Y} g_n d\nu$. Так вот, оказывается $\varphi_n(x) \le \varphi_{n + 1}(x) \le \varphi_{n + 2}(x)$ (так как там подынтегральные функции возрастающие, все дела), и при этом $\varphi_n(x) \ntoinf \varphi(x)$ (предыдущий пункт). Тогда давайте просто $\int_{X} \varphi(x) d\mu = \lim_{n \rightarrow \infty} \int_{X} \varphi_n(x) d\mu$ по теореме Леви (типа в обратную сторону). А ещё, зная что в основе $\varphi_n$ лежит ступенчатая функция, мы понимаем, что для неё уже выполняется наша теорема, таким образом применив пункты 2 и 3 мы можем перейти к равенству $= \lim \int_{X \times Y} g_n dm = $ и опять по Леви $ = \int_{X \times Y} fdm$
\end{enumerate}

ч. т. д. 

\subsubsection{Формула для бета-функции}
\textit{Формулировка:}

Бета-функция задаётся следующим образом: 

\[B(s, t) = \int_{0}^{1}x^{s - 1}(1 - x)^{t - 1}dx, \quad s, t > 0\]

Тогда:

\[B(s, t) = \frac{\Gamma(s)\Gamma(t)}{\Gamma(s + t)}\]

\textit{Доказательство:}

Рассмотрим:

\[\Gamma(s)\Gamma(t) = \int_{0}^{\infty}x^{s - 1}e^{-x}dx \cdot \int_{0}^{\infty} y^{t - 1}e^{-y} dy = \]

Заметим, что второй интеграл есть ничто иное, как константа! Внесём его внутрь:

\[= \int_{0}^{\infty} x^{s - 1}e^{-x} \left(\int_{0}^{\infty} y^{t - 1}e^{-y}dy\right)dx = \int_{0}^{\infty}  \left(\int_{0}^{\infty} x^{s - 1} y^{t - 1}e^{-(x + y)}dy\right)dx = \]

Заменим $y = u - x$:

\[= \int_{0}^{\infty} \left(\int_{x}^{\infty}x^{s - 1}(u - x)^{t - 1}e^{-u}du\right)dx = \]

А теперь финт ушами! По теореме Тонелли, этот повторный интеграл является двойным интегралом по некоторой области $C$:

\images{0.3}{betta.jpg}

Так давайте просто поменяем пределы интегрирования: 

\[= \int_{0}^{\infty} \left(\int_{0}^{u}x^{s - 1}(u - x)^{t - 1}e^{-u}dx\right)du = \]

И ещё раз заменим: $x = uv, \dbl dx = udv$ ($u$ типа как константа, пределы интегрирования тоже поменялись!)

\[= \int_{0}^{\infty} \left(\int_{0}^{1}(uv)^{s - 1}(u - uv)^{t - 1}e^{-u}dv\right)udu = \int_{0}^{\infty} \left(\int_{0}^{1}u^{s - 1}v^{s - 1}u^{t - 1}(1 - v)^{t - 1}e^{-u}dv\right)udu = \]

\[\int_{0}^{\infty}u^{s + t - 1}e^{-u} du \cdot \int_{0}^{1}v^{s - 1}(1 - v)^{t - 1}dv = \Gamma(s + t)B(s, t)\]

ч. т. д.

\subsubsection{Объем шара в $\mathbb{R}^m$}
\textit{Формулировка:}

\begin{itemize}
    \item $B(0, R) = \{x \in \mathbb{R}^{m}: x_1^{2} + x_2^{2} + \ldots + x_m^{2} \le R^{2}\}$
    \item $\alpha_{m} \ \lambda_{m}(B(0, 1))$
\end{itemize}

Тогда: 

\[\mu\left(B(0, R)\right) = \alpha_m R^{m}\]

\textit{Доказательство:}

Почему вылез радиус в степени $m$ --- это при линейном растяжении шарика $B(0, 1)$ просто вылез множитель (по прошлому сему (?)). Поэтому достаточно рассмотреть только этот базированный шар единичного радиуса. Будем же наконец искать его объём, интегрируя!

\[\alpha_m = \int_{-1}^{1} \lambda_{m - 1} \left(B(0, 1)_{x_1}\right) dx_1 = \]

А почему так? Да очень просто. Дело в том, что сечение шара размерности $m$ есть подпространство размерности $m - 1$, а именно --- шар радиуса $\sqrt{1 - x_1^2}$.

\images{0.5}{objom.jpg}

\[= \int_{-1}^{1} \alpha_{m - 1} (1 - x_1^2)^{\frac{m - 1}{2}} dx_1 = \]

Делаем замену $x_1^2 = x, \dbl dx_1 = \frac{dx}{2\sqrt{x}}$:

\[= \frac{\alpha_{m - 1}}{2} \int_{-1}^{1} x^{\frac{1}{2}}(1 - x)^{\frac{m - 1}{2}}dx = \alpha_{m - 1} \cdot B\left(\frac{1}{2}, \frac{m + 1}{2}\right) = \alpha_{m - 1}\frac{\Gamma\left(\frac{1}{2}\right)\Gamma\left(\frac{m + 1}{2}\right)}{\Gamma\left(\frac{m}{2} + 1\right)}\]

Двойка из знаменателя пропала из-за того, что подинтергальная функция чётна, значит, изначальный интеграл можно разбить на два на промежутках $(-1, 0)$ и $(0, 1)$ и они будут равны, и равны бета-функции. Ну и всё, двойка сократилась. Гораздо интереснее, что же там будет, если мы будем раскрывать ``альфы'' до талого. Сразу заметим, что $\alpha_1 = 2$ (ну просто длина промежутка $(-1, 1)$). Посмотрим (пары, эквивалентные ``подчёркнутым'' сократятся, и так далее со сдвигом на один через один, лол):

\[\alpha_{m} = \frac{\Gamma\left(\frac{1}{2}\right)\uwave{\Gamma\left(\frac{m + 1}{2}\right)}}{\Gamma\left(\frac{m}{2} + 1\right)} \cdot \frac{\Gamma\left(\frac{1}{2}\right)\Gamma\left(\frac{m}{2}\right)}{\uwave{\Gamma\left(\frac{m - 1}{2} + 1\right)}} \cdot \frac{\Gamma\left(\frac{1}{2}\right)\Gamma\left(\frac{m - 1}{2}\right)}{\Gamma\left(\frac{m - 2}{2} + 1\right)} \cdot \ldots \cdot 2 = \]

Вспоминаем ``факториальность'' гамма-функции $\Gamma(n + 1) = n\Gamma(n)$ и формулу из темы про бесконечные произведения $\Gamma(x)\Gamma(1 - x) = \frac{\pi}{\sin \pi x}$:

\[= 2\frac{\Gamma\left(\frac{1}{2}\right)^{m - 1}\Gamma\left(\frac{3}{2}\right)}{\Gamma\left(\frac{m}{2} + 1\right)} = 2\frac{\Gamma\left(\frac{1}{2}\right)^{m - 1}\cdot \frac{1}{2} \cdot\Gamma\left(\frac{1}{2}\right)}{\Gamma\left(\frac{m}{2} + 1\right)} = \frac{\Gamma\left(\frac{1}{2}\right)^{m}}{\Gamma\left(\frac{m}{2} + 1\right)} = \frac{\left(\frac{\pi}{\sin \frac{\pi}{2}}\right)^{\frac{m}{2}}}{\Gamma\left(\frac{m}{2} + 1\right)} = \frac{\pi^{\frac{m}{2}}}{\Gamma\left(\frac{m}{2} + 1\right)}\]

(можно прогнать ещё для первых размерностей 2, 3)

ч. т. д.

\subsubsection{Теорема Фату. Следствия}
\textit{Формулировка:}

\begin{itemize}
    \item $(X, \mathfrak{A}, \mu)$ --- пространство с мерой
    \item $f_n \ge 0$ --- измерима
    \item $f_n \rightarrow f$ почти везде
    \item Если $\exists C > 0\ \dbl \forall n \int_{X} f_n d\mu \le C$  
\end{itemize}

Тогда:

\[\int_{X} f d\mu \le C\]

(тут, вообще говоря, не предполагается, что интегрально функции сходятся)

\textit{Доказательство:}

Заведём $g_n = \inf \{f_n, f_{n + 1}, f_{n + 2}, \ldots\}$ (должно уже на что-то намекать). Очевидно, что эта последовательность возрастающая, так как у нас есть сходимость почти везде изначально. Также $\forall n : 0 \le g_n \le f_n$

Очевидно, что $g_n \le f_n$, интегрируем!

\[\int_{X} g_n \le \int_{X} f_n \le C \quad (*)\]

С другой стороны, так как $\lim_{n \rightarrow \infty} g_n$ суть есть $\underline{\lim} f_n = f$ (ну раз у нас есть сходимость, то и нижний предел сходится к $f$). Тогда по теореме Леви:

\[\lim_{n \rightarrow \infty} \int_{X} g_n = \int_{X} f \le C\]

ч. т. д.

\textit{Следствие: }

То же самое, только меняем сходимость почти везде на: 

\begin{itemize}
    \item $f_n, f \ge 0$, измеримы, почти везде конечны
    \item $f_n \underset{\mu}{\Longrightarrow} f$
\end{itemize}

\textit{Доказательство: }

Запускаем теорему Рисса, выбираем сходящуюся подпоследовательность и доказательство сработает.

\textit{Следствие: }

\begin{itemize}
    \item $f_n \ge 0$, измеримы
\end{itemize}

Тогда:

\[\int_{X} \underline{\lim}f_n \le \underline{\lim}\int_{X} f_n\]


\textit{Доказательство: }

Давайте введём $g = \underline{\lim} f_n$, ($g = \lim g_n$ из теоремы). Заметим, что $g$ измерима по теореме об измеримости предела. Тогда, по монотонности интеграла:

\[\int_{X} g_n \le \int_{X} f_n\]

Ну тогда запускаем теорему Леви ($g_n$ возрастают и стремятся к $g$):

\[\int_{X} g = \lim \int_{X} g_n \underset{(3)}{=}  \underline{\lim} \int_{X} g_n \le \underline{\lim} \int_{X} f_n\]

(3) --- раз уж у нас есть предел, то есть и нижний, и он равен основному. Ну и оцениваем подинтегральные выражения и вспоминаем, что левая часть равна $\int g = \int \underline{\lim} f_n$. Получилось!

ч. т. д. 

\subsubsection{Теорема о вычислении интеграла по взвешенному образу меры}
\textit{Формулировка:}

\begin{itemize}
    \item $(X, \mathfrak{A}, \mu), (Y, \mathfrak{B}, \_)$ --- пространства с мерой
    \item $\omega: X \rightarrow \rinf \ge 0$ --- измеримо
    \item $\Phi: X \rightarrow Y$ --- ``измеримое''
    \item $\nu$ --- взвешенный образ $\mu$ (с весом $\omega$)
\end{itemize}

Тогда для $\forall f: Y \rightarrow \rinf \ge 0$ --- измеримых:
\begin{enumerate}
    \item $f \circ \Phi$ --- измеримо (относительно $\mathfrak{A}$)
    \item $\int_{Y} f d\nu = \int_{X} f(\Phi(x))\cdot\omega(x) d\mu(x)$
\end{enumerate}
\textit{Доказательство:}

\textbf{1.}

Ну тут всё достаточно просто, нам дано, что $f$ --- измерима относительно $\mathfrak{B}$. Выводим измеримость через данное:

\[X(f \circ \Phi < a) = \Phi^{-1}(Y(f < a)) \in \mathfrak{A}, \quad (Y(f < a) \in \mathfrak{B})\]

Ну типа, мы перегоняем каждую точку из пространства, в котором нам известна измеримость, в новое. Причём, важно что прообраз этих множеств точно лежит в $\mathfrak{A}$ (по ``измеримости'' отображения $\Phi$)

\textbf{2. ``Зоологическая теорема'' }

Запускаем классическое ``ступенчатое'' доказательство.

\textbf{2.1 $B \in \mathfrak{B}, f = \chi_{B}$ -- ступенька}

По условию: $f \circ \Phi(x) = \chi_{B}(\Phi(x))$. Вообразим это в голове, и поймём, что это характеристическая функция образа $B_k: = \chi_{\Phi^{-1}(B)}(x)$. С другой стороны, мы можем поинтегрировать функцию по $Y:$

\[\int_{Y} f d\nu = \nu(B) = \]

$\nu$ --- взвешенная мера (по определению). Распишем:

\[= \int_{\Phi^{-1}(B)} \omega d\mu = \int_{X} \chi_{\Phi^{-1}(B)} \omega d\mu = \]

Мы расширили интеграл, но добавили характеристическую функцию, чтобы занулить его вне искомой области. Ну и теперь подинтегральная функция просто и есть $f$:

\[ = \int_{X} f \circ \Phi(x) \omega(x) d\mu\]

\textbf{2. $f = \sum \alpha_k \chi_{B_k}(x)$ --- ступенчатая}

По линейности интеграла всё работает.

\[\int_{Y} \sum \alpha_k \chi_{B_k} d\nu = \sum \alpha_k \int_{Y} \chi_{B_k} d\nu = \sum \alpha_k \nu(B_k) =\]

\[= \sum \alpha_k \int_{X} (f_{B_k} \circ \Phi)(x) \omega(x) d\mu = \int_{X} \sum \alpha_k = \ldots\]

Ну короче, всё хорошо.

\textbf{3. $f \ge 0$ --- измеримая}

$g_n \ntoinf f, g_n \ge 0$ --- ступенчатые. Запускаем теорему Леви и всё получается.

ч. т. д. 

\textit{Следствие:}

Вместо измеримости $\ge 0$ можно взять и суммируемость.

\textit{Доказательство:}

$|f|$ подходит по условию теоремы.

$|f|$ --- суммируема относительно $\nu \Leftrightarrow f \circ \Phi$ суммируема относительно $\mu$ (почему?). ``Тогда с задачей срезок не будет никаких проблем''

\subsubsection{Критерий плотности}

Тут мы резко свернули с абстрактных рельс на $\mathfrak{A} = \mathfrak{B}, \Phi = \text{id}$

\textit{Формулировка:}

\begin{itemize}
    \item $(X, \mathfrak{A}, \mu)$ --- пространство с мерой
    \item $\nu$ --- ещё одна мера на $\mathfrak{A}$
    \item $\omega: X \rightarrow \rinf \ge 0$, измеримо 
\end{itemize}

Тогда эквивалентно:

\begin{enumerate}
    \item $\omega$ --- плотность $\nu$ отностительно $\mu$
    \item $\forall A \in \mathfrak{A} \quad \inf_{A} \omega \cdot \mu A \le \nu A \le \sup_{A} \omega \cdot \mu A$
\end{enumerate}

\textit{Доказательство:}

\textbf{$1 \Rightarrow 2$}

Очевидно (расписать по определению).

\textbf{$2 \Rightarrow 1$}

Проще рассматривать множество на ненулевых $\omega$. Для начала, на нулевых всё выполняется: $B = X(\omega = 0)$.

\[\nu B = 0 = \int_{B} 0 d\mu\]

Ещё надо бы показать, что если множесто оказалось на перечении ``нулевого'' и ``пложительного'' веса, то оно не испортит нам оценку. 

\images{0.5}{kr_pl_1.jpg}

$A$ --- синее множество. Тогда $A \cap B$ --- часть, где вес равен нулю, а $A \setminus B$ --- где положительный. Посмотрим методом пристального взгляда на неравенства:

\[\nu A \le \sup \omega \mu A\]
\[\nu A  \setminus B + \underbrace{\nu A \cap B}_{=0} \le \sup \omega \cdot ( \mu (A \setminus B) + \mu (A \cap B))\]

Как видно, мы только усилили неравенство (во втором случае мы проверяем только часть множества).

Зафиксируем $q \in (0, 1)$. Рассмотрим $A_j = A(q^j < \omega < q^{j - 1}), \dbl j \in \mathbb{Z}$. Так как степень пробегает целые числа, то такое замощение покрывает всю положительную ось $\mathbb{R}$ ($A = \bigsqcup A_j$).

\images{0.4}{kr_pl_2.jpg}

\[q^j \mu A_j \underbrace{\le}_{1} \nu A_j \underbrace{\le}_{2} q^{j - 1} \mu A_j\]

\[q^j \mu A_j \underbrace{\le}_{3} \int_{A_j} \omega d\mu \underbrace{\le}_{4} q^{j - 1} \mu A_j\]

Откуда взялись эти неравенства? Ну, первое просто напросто вытекает из предпосылки, и т. к. мы ограничили множество, очевидно, какие у него инфимум и супремум. А второе --- просто расписали взвешенную меру. Записываем \textit{очень} длинное оценочное неравенство:

\[q\int_{A} \omega d\mu = q\sum \int_{A_j} \omega d\mu \underbrace{\le}_{4} q \sum q^{j - 1} \mu A_j = \sum q^j \mu A_j \le\]
\[\underbrace{\le}_{1} \sum \nu A_j \underbrace{\le}_{2} \sum q^{j - 1} \mu A_j = q^{-1} \sum q^j \mu A_j \le \]

\[\underbrace{\le}_{3} q^{-1} \sum \int_{A_j} \omega d\mu = q^{-1} \int_{A} \omega d \mu\]

Таким образом, мы окольцевали:

\[q\int_{A} \omega d\mu \le \sum \nu A_j = \nu A \le q^{-1} \int_{A} \omega d \mu\]

Устремляем $q \rightarrow 1$ и получаем искомое.

ч. т. д. 

\subsubsection{Лемма о единственности плотности}
\textit{Формулировка:}

\begin{itemize}
    \item $f, g$ --- суммируемы на $X$
    \item $\forall A$ --- измеримое, $\int_{A} f = \int_{A} g$
\end{itemize}

Тогда $f = g$ почти везде

\textit{Доказательство:}

Для удобства будем расматривать $h := f - g$. Тогда по условию теоремы $\forall A: \int_{A} |h| = 0$

Заведём $X_+ := X(h \ge 0), X_- := X(h < 0)$. Очевидно, что $X = X_+ \sqcup X_-$.

$\int_{X_+} |h| = \int_{X_+} h = 0$ (как и по любому измеримому множеству), $\int_{X_-} |h| = -\int_{X_-} h = 0$

Ну и значит и по всему пространству: $\int_{X} |h| = \int_{X_+} |h| + \int_{X_-} |h| = 0 - 0 = 0$. Получается, что $h = 0$ почти везде.

ч. т. д.

\textit{Следствие: }

Плотность меры определяется однозначно с точностью до изменения на множестве меры 0.

\subsubsection{Лемма об оценке мер образов малых кубов}
\textit{Формулировка:}

\begin{itemize}
    \item $\Phi: O \subset \mathbb{R}^{m} \rightarrow \mathbb{R}^{m}$
    \item $\Phi \in C^{1}$
    \item $a \in O$
    \item Пусть $c > |\det \Phi'(a)| \neq 0$
\end{itemize}

Тогда $\exists \delta > 0 \dbl \forall$ Куб $Q \subset B(a, \delta)$, $a \in Q$ (кубик задевает за точку)

\[\lambda \cdot \Phi(Q) < c \cdot \lambda Q\]

\textit{Доказательство:}

Пусть $L = \Phi'(a)$ (и оно ещё и обратимое, выводится из условия). Так как по условию $\Phi \in C^{-1}$, вблизи точки $a$ она представляется как:

\[\Phi(x) = \Phi(a) + L (x - a) + o(x - a)\]

Преобразуем (там мы применили обратный оператор $L^{-1}$ к ``о''-шке, но ничего страшного, она осталась ``о''-шкой):

\[\underbrace{a + L^{-1}(\Phi(x) - \Phi(a))}_{\Psi(x)} = x + o(x - a)\]

$\Psi(x)$ представляет из себя сдвинутый (на $a$ и $\Phi(a)$) $\Phi$ под действием обратного отображения $L$. И вот так получается, что он мапит иксы почти в себя самих. Давайте посмотрим на определение ``о''-маленького (эпсилон немного отнормирован):

\[\forall \varepsilon > 0 \dbl \exists \text{ шар } B_{\varepsilon}(a) \dbl \forall x \in B_{\varepsilon}(a): |\Psi(x) - x| < \frac{\varepsilon}{\sqrt{m}}|x - a|\]

Пусть у нас есть куб $Q$ внутри этого шара со стороной $h$: $Q \subset B_{\varepsilon}(a)$. Тогда $\forall x \in Q: |x - a| < \sqrt{m}h$. (точка $a$ лежит внутри куба, оценили диагональю). Применяя выкладку из ``о''-маленького, получаем:

\[|\Psi(x) - x| < \varepsilon h\]

Теперь неочевидное: оценим для произвольных $x, y \in Q$ насколько близко они лежат друг к другу. Оценка на покоординатные функции по неравенству треугольника:

\[|\Psi_i(x) - \Psi_i(y)| = \underbrace{|\Psi_i(x) - x_i|}_{(1)} + \underbrace{|\Psi_i(y) - y_i|}_{(2)} + \underbrace{|x_i - y_i|}_{(3)} \le\]

Сразу скажем, что покоординатные функции можно оценить сверху нормами на полную функцию (по последним достижениям), а две точки внутри куба уж точно лежат не более чем на $h$ друг от друга. Таким образом:

\[\le \varepsilon h + \varepsilon h + h = (1 + 2\varepsilon) h \]

Ну и всё, тогда $\Psi(Q)$ лежит внутри куба со стороной $(1 + 2\varepsilon)h$ (ну, мы же только что узнали, насколько сильно развозит точки при отображении). Оценим меру (мера $Q$ тривиальна):

\[\lambda\Psi(Q) \le (1 + 2\varepsilon)^mh^m = (1 + 2\varepsilon)^m\lambda Q\]

А как мы обсудили ранее, $\Psi$ отличается от $\Phi$ только линейным отображением. Вспоминая прошлый сем, нам известно, что тогда мера Лебега множества после линейного отображения обязана домножиться на определитель оператора:

\[\lambda\Phi(Q) = |\det L|\lambda \Psi(Q) \le |\det L|(1 + 2\varepsilon)^m \lambda Q\]

Последний штрих, так как нам дали $c$, подберём такой $\varepsilon$, чтобы $|\det L|(1 + 2\varepsilon)^m < c$ (у нас есть такая возможность, так как по условию коэффициент $c$ больше определителя). А в качестве $\delta$ выберем радиус $B_{\varepsilon}(a)$.

ч. т. д.

\textit{Лемма 2 (без доказательства):}

Пусть $f: O \subset \mathbb{R}^m \rightarrow \mathbb{R}$ --- непрерывна, $A \subset O$ --- измеримое множество. Тогда:

\[\inf_{G \subset O\text{ --- открытые}, A \subset G} \left(\lambda G \cdot \sup_{G}f\right) = \lambda A \cdot \sup_{A} f  \]

\subsubsection{Предельный переход под знаком интеграла при наличии равномерной сходимости или $L_{loc}$}
\textit{Формулировка:}

\begin{itemize}
    \item $f: X \times \tilde{Y} \rightarrow \rinf$
    \item $X$ --- пространство с мерой, $\mu X < + \rinf$
    \item $\tilde{Y}$ --- метризуемое топологическое пространство
    \item $Y \subset \tilde{Y}$
    \item $a \in \tilde{Y}$ --- предельная точка  $Y$
    \item $\forall y \in Y \quad x \mapsto f(x, y)$ --- суммируема на $X$
    \item Пусть $f(x, y) \rsh{y \rightarrow a} \varphi(x)$
\end{itemize}

Тогда $\varphi$ --- суммируема на $X$ и 

\[\lim_{y \rightarrow a} \int_{X} f(x, y) d \mu(x) = \int_{X} \varphi(x) d \mu(x)\]
 
\textit{Доказательство:}

Сначала идёт часть для равномерной сходимости. Разберёмся с ней экстравагантно: запускаем $y_n \rightarrow a$ по Гейне! Далее, по условию равномерной сходимости, для (любого) $\varepsilon = 1$ (при больших $n$):

\[\forall x \dbl |f(x, y_n) - \varphi(x)| < 1\]

Ну ведь действительно, при больших $n \dbl y_n \rightarrow a$, а как раз при стремлении $y$ к $a$ у нас достигается равномерная сходимость. Из этого мы выводим суммируемость $\varphi(x)$ (меньше суммируемой $f$, неравество треугольника (?)):

\[|\varphi(x)| < |f(x, y_n)| + 1\]

А это значит, что мы имеем право ставить $\varphi$ под знак интеграла. Теперь посмотрим и оценим разность интегралов (последний переход --- мы оцениваем интеграл наибольшим значением на меру всего множества $X$):

\[\left| \int_{X} f(x, y_n) - \int_{X} \varphi(x)\right| \le \int_{X} |f(x, y_n) - \varphi(x)| \le \cdot \sup_{X} |f(x, y_n) - \varphi(x)| \cdot \mu X \ntoinf 0\]

Заметьте, что сыграла конечность меры $X$ (супремум стремится к нулю, очевидно, по равномерной сходимости).

ч. т. д. 

\textit{Формулировка для $L_{loc}$:}

$f, g$ из определения $L_{loc}$, $f \in L_{loc}(a)$ с данной $g$. $\lim_{y \rightarrow a} f(x, y) = \varphi(x)$ при почти всех $x$.

Тогда $\varphi$ суммируемая на $X$ и:

\[\lim_{y \rightarrow a} \int_{X} f(x, y) = \int_{X} \varphi(x)\]

\textit{Доказательство:}

Запускаем Гейне $y_n \rightarrow a$. Раз при почти всех $x$ условие выполняется, то на этой последовательности $y_n$ запускаем теорему Лебега о мажорированной сходимости (мажоранта --- $g$ из условия $L_{loc}$) и получаем искомое.

ч. т. д.

\textit{Следствие:}

$f: X \times Y \rightarrow \rinf$, $\forall y: x \mapsto f(x, y)$ --- измерима на $X$.

$a \in Y$, при п. в $x: f(x, y)$ непрерывна в $y = a$. Пусть $f \in L_{loc}(a)$. Тогда:


\[J(y) = \int_{X} f(x, y) d\mu(x) \text{ --- непрерывна в точке } a\]

\textit{Доказательство: }

Берём $\varphi(x) = f(x, a)$ и проверяем непрерывность теоремой:

\[J(y) = \int_{X} f(x, y) d\mu \goesto{y \rightarrow a} \int_{X} f(x, a) d\mu = J(a)\]

ч. т. д. 

\subsubsection{Предельный переход по параметру в несобственном интеграле}
\textit{Формулировка:}

\begin{itemize}
    \item $f: \langle a, b \rangle \times Y \rightarrow \rinf$
    \item $Y \subset \tilde{Y}$ --- метризуемое
    \item $y_0 \in \tilde{Y}$ --- предельная точка $Y$
\end{itemize}

\begin{enumerate}
    \item при почти всех $x \exists f_0(x) = \lim_{y \rightarrow y_0} f(x, y)$
    \item $\forall t \in (a, b) \dbl \forall f(x, y_0), f(x, y)$ --- суммируемые по $x$ на $(a, t)$ и $\int_a^{t} f(x, y) dx \goesto{y \rightarrow y_0} \int_a^{t} f_0(x) dx$
    \item $J(y) = \int_a^{\rightarrow b} f(x, y)$ --- равномерно сходящаяся при $y \in Y$
\end{enumerate}

Тогда $\int_a^{\rightarrow b} f_0(x) dx$ --- существует (как несобственный)

\textit{Доказательство:}

КПК в данном месте подарил нам ``ложное напоминание'' (на самом деле это ``Теорема о перестановке двух предельных переходов'' из 3го семестра, записанная в более общем виде). Приведём её и подгоним под неё:

\begin{itemize}
    \item $F: T \times Y \rightarrow \rinf$
    \item $Y \subset \tilde{Y}, T \subset \tilde{T}$ --- метризуемые т. п. (ну или метрические)
    \item $y_0 \in \tilde{Y}, t_0 \in \tilde{T}$ --- предельныя точка
    \item $\forall t \in T \dbl \exists$ конечный $L(t) = \lim_{y \rightarrow y_0} F(t, y)$
    \item $\forall y \in Y \dbl \exists$ конечный $J(y) = \lim_{t \rightarrow t_0} F(t, y)$
    \item Хотя бы один из этих пределов равномерный
\end{itemize}

Тогда $\exists$ конечный $\lim_{t \rightarrow t_0} L(t) = \lim_{y \rightarrow y_0} J(y)$

Ну и всё, теперь пишем ``словарик'' и запускаем воспоминание:

$T = \langle a, b\rangle, \dbl \tilde{T} = \rinf, \dbl t_0 = b, \dbl F(t, y) = \int_a^t f(x, y) dx, \dbl L(t) = \int_a^t f_0(x) dx, \dbl J(y)$ --- равномерно сходится.

ч. т. д. 

\textit{Следствие: }

Добавляем условия:

\begin{itemize}
    \item вместо условия (1): при почти всех $x: f(x, y)$ --- непрерывна в $y_0$
    \item $f_0(x) := f(x, y_0)$
\end{itemize}

Тогда $J(y)$ --- непрерывна в $y_0$ (последнее заключение теоремы будет ровно этим).

\textit{Доказательство:}

Видимо, там $+-$ то же самое, что и в прошлой теореме. Приведено не было.

\subsubsection{Теорема об интегрировании по параметру в несобственном интеграле}
\textit{Формулировка:}

\begin{itemize}
    \item $f: (a, b) \times Y \rightarrow \rinf$ --- ``допустима'', т. е. $f$ суммируема по мере $\lambda_1 \times \mu$ для $\forall t \in (a, b)$ на $(a, t) \times Y$ 
    \item $\mu Y < +\infty$
    \item $J(y) = \int_a^{\rightarrow b} f(x, y) dx$ --- равномерно сходится на $Y$
    \item $I(x) := \int_{Y} f(x, y) d\mu$
\end{itemize}

Тогда:

\begin{enumerate}
    \item $J(y)$ --- суммируема на $Y$
    \item $\int_a^{\rightarrow b} I(x) dx$ --- сходится
    
    \[\int_{Y} \left( \int_a^{\rightarrow b} f(x, y) dx\right) d\mu = \int_a^{\rightarrow b} \left( \int_{Y} f(x, y) d\mu\right) dx\]

    \[\int_{Y} J(y) d\mu = \int_a^{\rightarrow b} I(x) dx\]
\end{enumerate}

\textit{Доказательство:}

\textbf{1. Суммируемость $J(y)$}

Введём $J_t(y) = \int_a^t f(x, y) dx$. По теореме Фубини (внезапно) (под теорему Фубини подходит, так как функция суммируема по условию (``допустимость'')) эта функция суммируема по $y$! (первый пункт). Зашибись. Тогда вспоминаем про равномерную сходимость: $J_t \rsh{t \rightarrow b} J$. По определению, для $\varepsilon = 1$ при любых $y$ будет выполняться:

\[|J_t(y) - J(y)| = \left| \int_t^{\rightarrow b} f(x, y) dx\right| < 1\]

Получается, $J_t(y) - J(y)$ --- суммируема по $y$ (потому что при любых $y$ она ограничена, а $Y$ конечной меры по условию).

Ну а раз всё это случилось, то и $J(y)$ --- суммируемая.

\textbf{2. Сходимость}

Заметим, что $I_t(x)$ есть результат применения теоремы Фубини к $J_t(y)$! А так как она тут прекрасно применятся, расписываем повторный интеграл:

\[\int_a^t I(x) dx = \int_{Y} \int_a^t f(x, y) dx d\mu = \int_{Y} \left(\int_a^{\rightarrow b} f dx - \int_t^{\rightarrow b} fdx \right) d\mu = \]

\[= \int_{Y} J(y) d\mu - \int_{Y} \int_t^{\rightarrow b} f dx d\mu\]

Отсюда получаем оценку:

\[\left| \int_a^t I(x) dx - \int_{Y} J(y) d\mu\right| \le \left| \int_{Y} \int_t^{\rightarrow b} f dx d\mu\right| \le \mu Y \cdot \sup_{Y} \left| \int_t^{\rightarrow b}f dx\right| \goesto{t \rightarrow b - 0} 0 \]

$Y$ конечной меры, оцениваем как обычно, и главное понять, что супремум --- ровно из определения равномерной схоимости! Супремум хвостиков несобственного интеграла стремится к нулю!

ч. т. д.

\subsubsection{Правило Лейбница для несобственного интеграла}
\textit{Формулировка:}

\begin{itemize}
    \item $f: [a, b) \times \langle c, d \rangle \rightarrow \mathbb{R}$ --- непрерывная
    \item $\forall y \in \langle c, d \rangle \dbl \exists \dbl J(y) = \int_a^{\rightarrow b} f(x, y) dx$
    \item $\forall x \in [a, b), \dbl \forall y \in \langle c, d \rangle \dbl \exists \dbl f'_y(x, y)$ непрерывная на $(a, b) \times \langle c, d \rangle$
    \item $I(y) = \int_a^{\rightarrow b} f'_y(x, y) dx$ --- равномерно сходится на $Y$
\end{itemize}

Тогда $J(y) \in C^1$ и $J'(y) = I(y)$, то есть:

\[\left(\int_a^{\rightarrow b} f(x, y) dx\right)'_y = \int_a^{\rightarrow b} f'_y(x, y) dx\]

\textit{Доказательство:}

Ну, сразу, $I(y)$ непрерывен по следствиюи из теоремы о предельном переходе в несобственном интеграле. Рассмотрим $s_0, s_1 \in \langle c, d \rangle$ и поинтегрируем по ним:

\[\int_{s_0}^{s_1} I(y) dy = \int_{s_0}^{s_1} \int_a^{\rightarrow b} f'_y(x, y) dx dy = \]

Заметим, что это ровно предыдущая теорема! Так поменяем пределы интегрирования и посчитаем:

\[= \int_a^{\rightarrow b} \int_{s_0}^{s_1} f'_y(x, y) dy dx = \int_{a}^{\rightarrow b} f(x, s_1) - f(x, s_0) dx = J(s_1) - J(s_0)\]

Вспоминаем теорему Барроу. Если мы фиксируем $s_0$, то у нас фактически получается интеграл с переменным верхним пределом, и, таким образом, $J(y)$ дифференцируема и суть есть первообразная $I(y)$ (аж формула Ньютона-Лейбница вылезла + там всё работает из-за непрерывности на аккуратно подрезанном нами прямоугольнике $[a, b) \times \langle c, d \rangle$).

ч. т. д. 


\subsubsection{Правило Лейбница дифференцирования интеграла по параметру}
\textit{Формулировка:}

\begin{itemize}
    \item $Y$ --- промежуток $\subset \mathbb{R}$
    \item $f: X \times Y \rightarrow \mathbb{R}$
    \item $\forall \quad f(x, y)$ --- суммируемая функция от $x$
    \item При почти всех $x \dbl \forall y \exists f'_y(x, y)$
    \item $f'_y$ --- удовлетворяет условию $L_{loc}(y_0)$
\end{itemize}

Тогда:

\begin{itemize}
    \item $J(y) = \int_{X} f(x, y) d\mu(x)$ --- дифференцируема в $y_0$
    \item $J'(y_0) = \int_{X} f'_y(x, y) d \mu(x)$
\end{itemize}

\textit{Доказательство:}

Заметим, что первые условия --- просто ``проверка на дурака'', чтобы не было желания засунуть под знак интеграла плохую функцию. Рассмотрим:

\[F(x, h) := \frac{f(x, y_0 + h) - f(x, y_0)}{h} \quad (*)\]

С другой стороны:

\[\frac{J(y_0 + h) - J(y_0)}{h} = \int_{X} F(x, h)\]

А что вообще произошло-то? Мы просто оп определению записали то, что хотим получить и доказать. Ещё раз, нам очень хочетнся, чтобы последний интеграл (который есть производная всей функции по параметру) стремился к интералу по производной:

\[\int_{X} F(x, h) \goesto{?, h \rightarrow 0} \int_{X} f'_y(x, y_0)\]

Чтобы иметь возможность сделать такой предельный переход, надо иметь какие-то на это основания. Давайте проверим, а не принадлежит ли $F \in L_{loc}(h = 0)$? Для этого надо придумать оценку на неё. А (*) вам ничего не напоминает? Конечно, это же теорема Лагранжа! Тогда наша функция суть есть производная в какой-то средней точке:

\[|F(x, h)| = |f'_y(x, y_0 + \theta h)| \le g(x)\]

$h$ стремится к нулю, а в окрестности $y_0$ для $f'_y$ выполняется условие $L_{loc}(y_0)$ (по условию теоремы), вследствие чего $F$ тоже имеет суммируемую мажоранту. Ну и теперь, раз мы доказали, значит $F$ суммируема при $h \rightarrow 0$ (???) и предельный переход законен.

ч. т. д.

\subsubsection{Теорема о вложении пространств $L^p$}
\textit{Формулировка:}

\begin{itemize}
    \item $\mu E < + \infty, 1 \le s < r \le +\infty$
\end{itemize}

Тогда:

\begin{enumerate}
    \item $L_r(E, \mu) \subset L_s(E, \mu)$
    \item $||f||_s \le \left(\mu E\right)^{\frac{1}{s} - \frac{1}{r}} \cdot ||f||_{r}$
\end{enumerate}

\textit{Доказательство:}

То есть, пространства становятся всё уже по мере роста параметра $p$. Заметим, что 1 следует из 2го (по определению у нас пространства задаются через (так или иначе) через норму (интеграл в степени)), и тут мы показали, что если $||f||_r$ конечен, то $s$ уж точно конечен, а наоборот может быть и неправда.

Для случая $r = \infty$ всё очевидно (оцениваем сверху интеграл мерой множества на сущ. супремум, по его свойствам всё хорошо (и, что самое главное, по здравому смыслу тоже)):

\[||f||_s = \left(\int_{E} |f|^s d\mu\right)^{\frac{1}{s}} \le \left(\mu E\right)^{\frac{1}{s} - \frac{1}{r}} \cdot \esssup f \le (\mu E)^{\frac{1}{s}} \esssup f\]

Тогда остался случай $r < + \infty$. Распишем норму $f$ в $L_s$ в степени $s$:

\[||f||_s^s = \int_{E} |f|^s d\mu = \int_{E} |f|^s \cdot 1 d\mu\]

Приготовим $p = \frac{r}{s}$ и $q = p' = \frac{r}{r - s}$ и запустим неравенство Гёльдера! 

\[ \int_{E} |f|^s \cdot 1 d\mu \le \left(\int_{E} |f|^{s \cdot \frac{r}{s}} d\mu\right)^{\frac{s}{r}} \cdot \left(\int_{E} 1 d\mu\right)^{\frac{r - s}{r}} = \left(\int_{E} |f|^{r} d\mu\right)^{\frac{s}{r}} \cdot \left(\mu E\right)^{\frac{r - s}{r}}\]

Возведём обе части неравенства в степень $\frac{1}{s}$:

\[\left( \int_{E} |f|^s d\mu\right)^{\frac{1}{s}} \le  \left(\int_{E} |f|^{r} d\mu\right)^{\frac{1}{r}} \cdot \left(\mu E\right)^{\frac{1}{s} - \frac{1}{r}}\]

Приглядимся, и увидим, что это то, что мы искали!

\[||f||_s \le \left(\mu E\right)^{\frac{1}{s} - \frac{1}{r}} \cdot ||f||_r\]

ч. т. д. 

\textit{Следствие:}

Если в условиях теоремы $f_n \goesto{L_r} f$, то и $f_n \goesto{L_s} f$.

\textit{Доказательство:}

\[||f_n - f||_s \le \left(\mu E\right)^{\frac{1}{s} - \frac{1}{r}} \cdot ||f_n - f||_r \ntoinf 0\]

ч. т. д. 

\subsubsection{Теорема о сходимости в $L^p$ и по мере}
\textit{Формулировка:}

$1 \le p < +\infty \quad f_n \in L_p(E, \mu)$: 

\begin{enumerate}
    \item $f \in L_p \quad f_n \goesto{L_p} f$, тогда $f_n \underset{\mu}{\Longrightarrow} f$
    \item $f_n \underset{\mu}{\Longrightarrow} f$ [либо $f_n \rightarrow f$ почти всюду], $|f_n| \le g$ почти всюду, при всех $n$, где $g \in L^p$. Тогда $f_n \goesto{L_p} f$
\end{enumerate}

\textit{Доказательство:}

\textbf{1. Сходимость по мере}

Классический алгоритм: $E_n(\varepsilon) = E(|f_n - f| \ge \varepsilon)$. Понаблюдаем за мерой $n$-го такого множества:

\[\mu E_n(\varepsilon) = \int_{E_n} 1 d\mu \le \int_{E_n} \frac{|f_n - f|}{\varepsilon} d\mu \le \int_{E_n} \frac{|f_n - f|^p}{\varepsilon^p} d\mu = \frac{1}{\varepsilon^p} \int_{E_n} |f_n - f|^p d\mu = \frac{||f_n - f||_p^p}{\varepsilon^n} \ntoinf 0\]

Ну, собственно говоря, просто череда оценок, в конце которой делаем предельный переход и получаем, что по определению сходимость по мере есть.

\textbf{2. Сходимость в $L^p$}

Из сходимости по мере достаём сходящуюся подпоследовательность (по теореме Рисса) $f_{n_k} \rightarrow f$ (ну или если у нас именно такое условие, то просто берём сходящуюся почти везде последовательность). Отсюда получаем принадлежность $f \in L^p$: делаем предельный переход в неравенстве из условия $|f_n| \le g$ при п. в. $x$, причём $g \in L^p$. Ну супер, получаем положительную (даже лучше чем суммируюмую (ведь просто суммирумость есть $L^1$)) мажоранту:

\[\forall n: |f_n - f| \le 2g\]

И запускаем теорему Лебега о мажорированной сходимости:

\[||f_n - f||_p^p = \int_{E} |f - f_n|^p d\mu \ntoinf 0 \]

ч. т. д. 

\subsubsection{Полнота $L^p$}
\textit{Формулировка:}

$L^p(E, \mu)$ ---- полное ($1 \le p < + \infty$)

\textit{Доказательство:}

\textit{Основная идея: соорудить кандидата на предел и свести фундаментальную последовательность к нему, как в доказательстве полноты непрерывных функций на компакте в прошлом семестре.}

\[\forall \varepsilon > 0 \dbl \exists N \dbl \forall n, m > N: ||f_n - f_m||_p < \varepsilon\]

Вот у нас есть такая фундаментальная последовательность. И мы можем заказывать себе коэффициенты! Давайте попробуем:

\[\exists n_1: \forall n_k > n_1 \dbl ||f_{n_1} - f_{n_k}||_p < \frac{1}{2}\]

Получилось) Продолжаем:

\[\exists n_2: \forall n_k > n_2 \dbl ||f_{n_2} - f_{n_k}||_p < \frac{1}{4}\]
\[\ldots\]
\[\exists n_m: \forall n_k > n_m \dbl ||f_{n_m} - f_{n_k}||_p < \frac{1}{2^m}\]

Соорудим функциональный ряд $S(x) := \sum_{k = 1}^{\infty} |f_{n_{k + 1}} - f_{n_k}|, S(x) \in [0, +\infty]$ (тут уже просто модуль, не перепутайте) и посмотрим на его частичные суммы $S_n$. Его норма ограничена (по неравенству треугольника):

\[||S_n||_p \le \sum_{k = 1}^{n} ||f_{n_{k + 1}} - f_{n_k}||_p \le 1\]

Требуется сразу несколько комментариев. Во-первых, по-хорошему мы должны были брать норму от модуля разности, но это немного бессмысленно, поэтому просто пишем норму. Ну и оценка на эти нормы у нас была заказана выше, поэтому конечных сумма обратных степеней двойки действительно меньше 1. Окей, также, очевидно, что частичные суммы сходятся к самому ряду. А ещё, если взглянуть с другой стороны:

\[||S_n||_p^p = \int_{E} |S_n|^p \le 1\]

Тут посмотрели на норму в степени, чтобы избавиться от дробной. Ну а единичке без разницы, в какую степень её возводят. Так вот, теперь можно применить теорему Фату!

\[\int_{E} |S|^p \le 1\]

Получается, что интеграл нашего ряда ограничен! Значит, сама функция --- суммируемая, супер, а значит --- она почти везде конечна. Внимание, появился кандидат на предельную функцию!

\[f(x) := f_{n_1} + S(x) = f_{n_1} + \sum_{k = 1}^{\infty} (f_{n_{k + 1}} - f_{n_k})\]

Посмторим на $f_n$:

\[f_{n_N} = f_{n_1} + \sum_{k = 1}^{N} (f_{n_{k + 1}} - f_{n_k}) = f_{n_{N + 1}}\]

У нас есть сходимость ряда $S(x)$ почти везде, из-за чего можно сказать, что $f_n \rightarrow f$ по сходимости изначального ряда (вроде бы очевидно). Ну вот, у нас есть тепрь обычная сходимость, а надо притянуть её за уши к сходимости в $L^p$. Запишем ещё раз определение фундаментальной последовательности для нашего случая:

\[\forall \varepsilon > 0 \dbl \exists N \dbl \forall n, m > N: ||f_n - f_m||_p < \varepsilon\]

Возьмём $m = n_k$ и возведём неравенство в степень $p$:

\[||f_n - f_{n_k}||_p^p = \int_{E} |f_n - f_{n_k}|^p < \varepsilon^p\]

Опять применяем теорему Фату и получаем искомое (т. к. у нас есть сходимость $f_{n_k} \rightarrow f$):

\[\int_{E} |f_n - f|^p < \varepsilon^p\]

Ура, всё сошлось и в $L^p$!

ч. т. д. 

\subsubsection{Плотность в $L^p$ множества ступенчатых функций}
\textit{Формулировка:}

\begin{itemize}
    \item $(X, \mathfrak{A}, \mu), 1 \le p \le +\infty$
\end{itemize}

Тогда множество ступенчатых функций плотно в $L_p(X, \mu)$

\textit{Доказательство:}

\textbf{1. $r = \infty$}

Давайте изменим нашу функцию нулями так, чтобы $|f| \le ||f||_{\infty} = \esssup |f|$ п. в. $x$. Заметим, что изменения эти будут на множестве меры 0! (по определению и свойствам существенного супремума). (чтобы корректно определять ступенчатую функцию для $f$, веди наша функция на множестве меры 0 может улетать в бесконечность, и такую нормально ступеньками не аппроксимируешь. Поэтому мы сразу избавляемся от такого).

Тогда (сдувая пыль) по теореме из 3го семестра о характеризации измеримых функций ступенчатыми, а точнее по её следствию, у нас существуют $\varphi \rshe f_-, \psi \rshe f_+$, и их разность $\psi - \varphi \rshe f$. По определению равномерной сходимости:

\[\forall \varepsilon > 0 \dbl \exists N \dbl \forall n > N \dbl \forall x \in E: \sup |f - (\psi - \varphi)| < \varepsilon\]

\textbf{2. $r < +\infty$}

Пусть $f \ge 0$. Тогда по той же теореме существует $g_n \ntoinf f, g_n$ --- ступенчатые.

\[||g_n - f||_p^p = \int_{E} |g_n - f| ^p d\mu \ntoinf 0\]

По теореме Лебега о мажорируемой сходимости (мажорантой тут выступает $f^p: |g_n - f|^p < f^p$). Для поддержки функций разного знака надо повторить выкладки из пункта 1 и всё получится.

ч. т. д. 

\subsubsection{Лемма Урысона}
\textit{Формулировка:}

\begin{itemize}
    \item $X$ --- нормальное топологическое пространство (например, $\mathbb{R}^m$)
    \item $F_0, F_1 \subset X$ --- замкнутое
    \item $F_0 \cap F_1 = \varnothing$
\end{itemize}

Тогда: $f: X \rightarrow \mathbb{R}, \quad 0 \le f \le 1$ --- непрерывное

$f|_{F_0} \equiv 0$, $f|_{F_1} \equiv 1$

\textit{Доказательство:}

Для начала переиначим первую аксиому нормальности: 

$\forall F$ --- замкнутое, $\forall G$ --- открытое, $F \subset G$, $\exists U(F)$ --- открытое: 

\[F \subset U(F) \subset \overline{U(F)} \subset G\]

\images{0.5}{uryhson.jpg}

\textit{на картинке $F := F_1$}

Ну, то есть, мы всегда можем поместить замкнутое множество внутрь откртой окрестности так, что даже если мы её замкнём, она всё ещё будет лежать внутри открытого множества. Ну или сузить окрестность (в случае, если это дополнение пространства).

Запустим эту тему на наших входных данных: $F := F_0, G := F^c_1 =: G_1$. Тогда:

\[\exists U(F): F \subset \underbrace{U(F)}_{G_0} \subset \underbrace{\overline{U(F)}}_{\overline{G_0}} \subset G_1\]
\[\exists U(F): \overline{G_0} \subset \underbrace{U(\overline{G_0})}_{G_{\frac{1}{2}}} \subset \underbrace{\overline{U(\overline{G_0})}}_{\overline{G_{\frac{1}{2}}}} \subset G_1\]
\[\exists U(F): \overline{G_{\frac{1}{2}}} \subset \underbrace{U(\overline{G_{\frac{1}{2}}})}_{G_{\frac{3}{4}}} \subset \underbrace{\overline{U(\overline{G_{\frac{1}{2}}})}}_{\overline{G_{\frac{3}{4}}}} \subset G_1\]

Ну и вот таким алгоритмом мы строим $G_q$ для каждого двоично-рационального $q \in (0, 1)$ (далее в повестровании $q, q_1$ --- двоично рационльные числа), причём $\forall \alpha < \beta: G_{\alpha} \subset \overline{G_{\alpha}} \subset G_{\beta}$ (по построению). Тогда утверждается, что:

\[f(x) := \inf \{q: x \in G_q\}\]

Ну, то что на $F_0$ у нас функция равна 0 вроде бы очевидно, т. к. тогда у нас $x$ будут сожержаться в $F_0 \subset G_0$. А если $x \in F_1$, то точка не содержится вообще ни в одном $G_q$, то инфимум просто выдаст 1. Осталось лишь доказать непрерывность. А она доказывается по топологическому определению непрерывности:

\[f \text{--- непрерывна} \Leftrightarrow \forall a, b:  f^{-1}(a, b) \text{--- открыто}\]

На самом деле, достаточно доказать что: 

\[f^{-1}(a, b) = \underbrace{f^{-1}(-\infty, b)}_{(1)} \setminus \underbrace{f^{-1}(-\infty, a]}_{(2)}\]

\begin{enumerate}
    \item $\forall s: f^{-1}(-\infty, s)$ --- открытое
    
    $f^{-1}(-\infty, s) = \bigcup_{q < s} G_q$ (объединение открытых открыто)

    \textit{Доказательство:}
    
    \textbf{a. $\subset$}
    
    $f(x) \le q < s$ (ну просто отправляем в какое-то множество до $s$), очев.

    \textbf{b. $\supset$}

    $f(x) = s_0 < s$. Давайте вобьём между ними двоично рациональное $q$, и у нас всё получится (т. к. множества упорядочены по включению, показывали выше) $f(x) = s_0 < q < s$.

    \item $\forall s: f^{-1}(-\infty, s]$ --- замкнутое
    
    $f^{-1}(-\infty, s) = \bigcap_{q > s} G_q = \bigcap_{q > s} \overline{G_q}$ (пересечение замкнутых замкнуто)

    \textit{Доказательство (последнего перехода):}
    
    \textbf{a. $\subset$}
    
    Очевидно, множества лежат внутри замыканий.

    \textbf{b. $\supset$}

    Вот у нас есть $q > s$. Давайте возьмём $r$, такое что $q > r > s$. Тогда $\bigcap G_q \supset \bigcap \overline{G_r}$ (некоторых). То есть, мы помапили $q$ в $r$, возможно где-то там выпал один и тот же $r$, и их пересечение (т. к. ещё множества стали поменьше), лежат в пересечении без замыкания. Ну а потом $\supset \overline{G_r}$ (всех, пересекли все множества $r$, и попасть туда стало сложнее).
\end{enumerate}

ч. т. д. 

\subsubsection{Плотность в $L^p$ непрерывных финитных функций}
\textit{Формулировка:}

\begin{itemize}
    \item $(\mathbb{R}^{m}, \mathfrak{M}^{m}, \lambda_m)$
\end{itemize}

Тогда $C_0(\mathbb{R}^{m})$ плотно в $L^{p}(\mathbb{R}^{m}, \lambda_m)$

\textit{Доказательство:}

Первое, что приходит в голову, это приблизить ступенчатую функцию финитными, и тогда у нас всё получится. Однако, возникает проблема, что некоторые ступеньки в ступенчатой функции могут быть заданы на неораниченном множестве, что противоречит определению финитной функции --- она должна принимать ненулевые значения только внутри некоторого, вполне конечного, шара. Поэтому с этим необходимо разобраться отдельно (придумать такую систему, в которой мы сможем прииближать даже неограниченные ступеньки финитными функциями). Займёмся этим:

\[f \in L^p, \dbl \forall \varepsilon > 0 \dbl \exists B_1 \text{ --- шар }: ||f - f \cdot \chi_{B_1}||_p < \varepsilon\]

Почему это выполняется? Заведём меру $\mu E = \int_{E} |f|^p$ и применим для неё теорему о непрерывности меры снизу:

\[\mu B(0, r) \goesto{r \rightarrow \infty} \mu(\mathbb{R}^m) = \int_{\mathbb{R}^m} |f|^p = ||f||_p^p\]

Взяли систему ``возрастающих'' множеств, и в пределе оно дало меру объединения. А мера объединения оказалось нормой! Так, тогда давайте проверять наше неведомо условие:

\[||f - f \cdot \chi_{B_1}||_p^p = \int_{\mathbb{R}^m} |f - f \cdot \chi_{B_1}| = \int_{\mathbb{R}^m \setminus B_1} |f|^p = \mu(\mathbb{R}^m) - \mu(B_1)\]

Очев, что мы можем за бесплатно сузить интеграл и расписать его в терминах нашей экспериментальной меры. А эта разница, как мы выяснили только что, стремится к нулю при больших $r$. Таким образом, мы умеем хорошо приближать функцию из $L^p$ ступенчатой, ступеньки которой ограничены и лежат внутри шара (а то, что не помещается, принебрежимо мало). Тут написано ровно это. Тогда, для $f$ существует $f_1$ --- ступенчатая, такая что:

\[\forall \varepsilon: ||f - f_1||_p < \varepsilon\]

По плотности ступенчатых функций. Ну и тогда, по новейшим достижениям, существует и шар $B$:

\[||f - f_1 \cdot \chi_{B}||_p < \varepsilon\]

Причём:

\[f_1 \cdot \chi_{B} = \sum \alpha_k \chi_{E_k}\]

Заметьте, что все $E_k$ уже лежат внутри шара $B$, и они ограничены (ну, так получается по построению). Ура, теперь можно приходить к самой приятной части --- применению тяжёлой артиллерии (в этом месте передаём привет конспекту Jovvik'а). Будем приближать финитной функцией каждую $E_k$. Очень хочется запустить лемму Урысона (иначе зачем она была вообще нужна?!), но для этого нам нужны два непересекающихся замкнутых множества. Вспоминаем о регулярности меры Лебега, и получаем, что существуют $F_k \subset E_k \subset G_k$, $F_k$ --- замкнутое, $G_k$ --- открытое, да ещё и с неплохими оценками:

\[\lambda (E_k \setminus F_k) < \varepsilon\]
\[\lambda (G_k \setminus E_k) < \varepsilon\]

$F_0 = G_k^c, F_1 = F_k$ --- окружаем наше $E_k$ и запускаем лемму Урысона. Получается, нам выдали $f$, которая на почти всём $E_k$ даёт 1, а на всём остальном --- 0. А что такое ``почти''? Посмотрим поближе:

\images{0.3}{pl_fin.jpg}

Заметим, что не умаляя общности, мы можем выбрать эти множества так, чтобы выполялось $F_k \subset E_k \subset G_k \subset B$ (ну там типа подрезать $G_k$ немного если что, не страшно, $E_k \subset B$ изначально). Ещё, взглянув на картинку, становится видно, что почти на всём $E_k: f \equiv 1$, на дополнении 0, а вот на разности $G_k \setminus F_k$ не ноль. Оценим это дело единицей (разность двух чисел под модулем, наверное по-хорошему надо двойкой оценивать, но там дальше удобнее единицей, сильно ничего не поменяется, можно будет больше эпсилонов взять).

\[||f - \chi_{E_k}||_p^p = \int_{B} |f - \chi_{E_k}|^p \le \int_{G_k \setminus F_k} 1^p = \lambda G_k \setminus F_k = 2 \varepsilon\]

По оценкам сверху.

ч. т. д. 

\subsubsection{Интегрирование по мере Бореля--Стильтьеса, порожденной функцией распределения (с леммой)}

\textit{Факт (в который мы верим)}:

$h$ --- измерима $\Rightarrow \dbl \forall B \in \mathfrak{B}, \dbl D \subset \mathbb{R}, \dbl h^{-1}(B)$ --- измеримо в $X$.

\textit{Формулировка (лемма):}

\begin{itemize}
    \item $(X, \mathfrak{A}, \mu)$
    \item $h: X \rightarrow \rinf$ --- измерима, почти везде конечна
    \item $\mu_H$ --- мера Бореля-Стилетьса
    \item $\nu = h(\mu)$ --- образ меры $\mu$ под действием отображения $h$
\end{itemize}

Тогда $\mu_H$ совпадает с $\nu$.

\textit{Доказательство (леммы):}

Проверим на полукольце ячеек $\mathcal{P}^1$, а потом запустим теорему о продолжении. По определению, у нас функция $H(x)$ задана как $X(h < a) \forall a \in \rinf$. То есть, для неё выполняется теорема о непрерывности меры снизу, а значит $\forall s: H(s - 0) = H(s)$. $\forall [a, b) \in \mathcal{P}^1$:

\[\mu_H[a, b) = H(b) - H(a) = \mu X(h < b) - \mu X(h < a) = \mu X(a \le h < b) = \]

Заметим, что $\nu(A) = \mu(h^{-1}(A))$ по определению, $\nu [a, b) = \mu\left(h^{-1} [a, b)\right)$. Наконец, $X(a \le h < b) = h^{-1}[a, b)$ (ну, просто мы смотрим на множество точек, при которых $a \le h < b$. Ну, это оно и есть).

\[ = \mu\left(h^{-1} [a, b)\right) = \nu [a, b)\]

Таким образом, на полукольце ячеек они действительно совпадают. Теперь запускаем теорему о продолжении, на мере Бореля-Стильтьеса всё получается по определению, а вот $h(\mu)$ всё получается просто по единственности меры (?). Дело в том, что она задана на борелевской $\sigma$-алгебре (в каком-то смысле минимальной), и тогда теорема о продолжении просто вернёт саму $h(\mu)$.

ч. т. д. 

\textit{Формулировка:}

\begin{itemize}
    \item $f: \mathbb{R} \rightarrow \rinf \ge 0$, измерима по Борелю
    \item $h: X \rightarrow \rinf$, измерима, почти везде конечна
    \item $H$ --- функция распеределения
    \item $\mu_{H}$ --- мера Бореля-Стильетса
\end{itemize}

Тогда:

\[\int_{X} f\left(h(x)\right)d\mu(x) = \int_{\mathbb{R}} f(t)d\mu_{H}(t)\]

\textit{Доказательство:}

Просто применяем теорему о вычислении интеграла по взвешенному образу меры: $Y = \mathbb{R}, \dbl \mathfrak{B}, h(\mu)$. $\Phi = h, \dbl \omega = 1$.

ч. т. д. 

\textit{Замечание: }

Если $f$ меняет знак, то $f$ --- $\mu_H$-суммируемая (или $f \circ h \dbl \mu$-суммируемая), то тоже имеет место такая теорема.

\subsubsection{Теорема об интегрировании по частям}

\textit{Замечания (их надо доказывать?):}

\begin{enumerate}
    \item $g$ --- возврастающая, $g \in C^1$
    
    \[\mu_g(A) = \int_{A} g'(t) dt\]

    \item $f: \langle a, b \rangle \rightarrow \mathbb{R}$ --- абсолютно-непрерывная функция, если $\exists g$ --- суммируемая (может быть локально суммируемая):
    
    \[\forall p, q \in \langle A, B \rangle: f(q) - f(p) = \int_p^q g(t) dt\]

    \item В случае $g: \mathbb{R} \rightarrow \mathbb{R}$, монотонная и возрастающая:
    
    \[d \mu_g(t) \leftrightarrow dg(t)\]

    \item Если $G \ge 0$, Тогда $F$ --- непрерывная, возрастающая, и тогда при почти всех $x \dbl \exists \dbl F'(x) = g(x)$
\end{enumerate}

\textit{Формулировка: }

\begin{itemize}
    \item $g: [a, b] \rightarrow \mathbb{R}$, возрастающая
    \item $f$ --- абсолютно непрерывная функция ($C^{1}$) на $[a, b]$
    \item $\mu_{H}$ --- мера Лебега-Стильетса (так как для теоремы Фубини нужна полнота меры)
\end{itemize}

Тогда:

\[\int_{[a, b)} f(x) dg(x) = fg|_{a}^{b} - \int_{[a, b]} f'(x)g(x)dx\]

\textit{Доказательство:}

\textbf{1. $f(a) = g(b) = 0$}

По абсолютной непрерывности $f$: 

\[f(x) - f(a) = f(x) - 0 = \int_a^x f'(t)dt = f(x)\]

Запускаем теорему Фубини на $\mu_g \times \lambda$:

\[\int_{[a, b)} f(x) dg(x) = \int_{[a, b)} \left( \int_a^x f'(t) d \lambda_1(t)\right) dg(x) = \]

Глядим на график:

\images{0.5}{po_chast.jpg}

И меняем границы интегрирования, сохраняя единообразие интеграла:

\[ = \int_a^b \left( \int_{[t, b)} f'(t) dg(x)\right) d\lambda_1(t) = \int_a^b f'(t) \left( \int_{[t, b)} dg(x)\right) d \lambda_1(t) = - \int_{[a, b]} f'(t) g(t) d\lambda_1(t)\]

Минус вылез при подстановке $[t, b): g(b) - g(t) = 0 - g(t) = -g(t)$

\textbf{2. Общий случай}

Отказываемся от нулей и просто интегрируем:

\[\tilde{g}(x) = g(x) - g(b), d\tilde{g}(x) = dg(x)\]

\[\int_{[a, b)} f(x) - f(a) d\tilde{g} = \left(f(x) - f(a)\right)\left(g(x) - g(b)\right)|_a^b \underbrace{-\int_{[a, b]} f'(x) \left(g(x) - g(b)\right) dx}_{(1)} \]

\[\int_{[a, b)} f(x) d\tilde{g} \uwave{- f(a)\left(g(a) - g(b)\right)} = f(x)g(x)|_a^b \uwave{- f(a)\left(g(a) - g(b)\right)} \underline{- f(x)g(b)|_a^b} - \int_{[a, b)} f'(x)g(x)dx \underline{ +g(b) f(x)|_a^b} \]

Подчёркнутое сокращается, и всё получается.

ч. т. д. 

\subsubsection{Лемма о  ``почти признаке Дирихле''}
\textit{Формулировка:}

\begin{itemize}
    \item $-\inf < a < b \le +\inf$
    \item $f$ --- ``доп.'' на $[a, b)$ ($\forall A \in (a, b) f$ --- суммируема на $(a, A)$)
    \item $g(x)$ --- монотонно стремится к 0 при $x \rightarrow b - 0$
    \item Пусть функция $F(t) = \int_{a}^{t} fdx$ --- ограничена
\end{itemize}

Тогда:

\[\int_{a}^{\rightarrow b} fg dx\] --- сходится

\[\left|\int_{a}^{\rightarrow b} fg dx\right| \le |g(a)| \cdot \sup_{t \in (a, b)} \left|\int_{t}^{\rightarrow b} f dx\right|\]

\textit{Доказательство:}

По теореме Барроу (?):

\[F'(t) = f(t)\]

Зафиксируем $a < t < b$, запишем $\int F(t)g(t)$ по частям, поменяв крайние слагаемые местами:

\[\underbrace{\int_a^t g(x) f(x)dx}_{\int g dF(x)} = F(x)g(x)|_a^t - \int_a^t F(x) dg(x)\]

Заметим, что $\mu_g$ --- конечна (так как $g$ --- монотонно убывает, и максимальное значение принимает в точке $a$),  а $F(t)$ --- ограничена. Получается, что $F(t)$ суммируема по мере $\mu_g$. То же самое, только по-другому:

\[\exists \text{ конечный }\lim_{t \rightarrow b} \int_{a}^{t}F(x)dg(x)\]

С другой стороны, опять же по ограниченности $F(t)$ и монотонному убыванию $g(t) \rightarrow 0$:

\[\lim_{t \rightarrow b} F(t)g(t) = 0\]

Делаем предельный переход в исходном равенстве (интегрировали по частям), и получаем:

\[\int_a^{\rightarrow b} f(x)g(x) dx = -\int_a^{\rightarrow b} F(x)dg(x)\]

Заметим, что интеграл справа на самом деле собственный (доказали выше, из этого, кстати, следует сходимость исходного интеграла (+ из суммируемости $F$)). Ну и осталось соорудить оценку:

\[\left|\int_a^{\rightarrow b} f(x)g(x) dx\right| = \left|\int_a^{\rightarrow b} F(x)dg(x)\right| \le |g(a)| \cdot \sup_{t \in (a, b)}|F(t)| = |g(a)| \cdot \sup_{t \in (a, b)} \left|\int_a^t f(x)\right| dx\]

$g(a) = \max_{[a, b) g(x)}$ --- максимальное значение, ну и стандартная оценка интеграла супремумом.

ч. т. д. 

\subsubsection{Следствие о ``почти признаке Абеля''}
\textit{Формулировка:}

\begin{itemize}
    \item $\int_{a}^{\rightarrow b} f dx$ --- сходится, $g$ --- монотонна и ограничена на $[a, b)$
\end{itemize}

Тогда: $\int_{a}^{\rightarrow b} fg dx$ --- сходится, и к тому же: 

\[\left|\int_{a}^{\rightarrow b} fg\right| \le 5 \cdot \sup_{(a, b)} \left|g(t)\right| \cdot \sup_{[a, b)} \left|\int_{t}^{\rightarrow b} f(x) dx \right|\]

\textit{Доказательство:}

Ну, тут что-то более менее классическое:

\[\lim_{x \rightarrow b} g(x)= L \text{ (монотонна ограничена же)}\]

Тогда:

\[\int_a^{\rightarrow b} fg dx = \underbrace{\int_a^{\rightarrow b} f(g - L) dx}_{(1)} + L \underbrace{\int_a^{\rightarrow b} fdx}_{(2)}\]

(1) сходится по Дирихле, (2) --- по условию. Сходимость есть, осталась оценка ($III + IV$ по Дирихле опять же): 

\[\left| \int_a^{\rightarrow b} fg dx \right| \le \underbrace{|L|}_{I} \cdot \underbrace{\left| \int_a^{\rightarrow b} f dx\right|}_{II} + \underbrace{|g(a) - L|}_{III} \cdot \underbrace{\sup_{t \in [a, b)} \left| \int_a^{\rightarrow b} f dx\right|}_{IV} \le (*)\]

Немного глина:

\[(II) = \left| \int_a^{\rightarrow b} f\right| \le \sup_{t \in [a, b)} \left| \int_t^{\rightarrow b} f dx\right|\]

Почему? Ну, типа мы взяли супремум, он выбирает \underline{наибольший} (в каком-то смысле) элемент. У нас 2 варианта: либо супремум реализуется в $t = a$ (тогда это правда), либо же в другой точке (тогда это тоже правда). Подобным образом оценим $IV$:

\[\left| \int_a^t f dx\right| = \left| \int_a^{\rightarrow b} fdx - \int_t^{\rightarrow b} f dx\right| \le 2 \sup_{t \in [a, b)} \left| \int_t^{\rightarrow b} fdx \right|\]

Ну и остатки:

\[I = |L| \le \sup_{(a, b)} |g(x)|\]

\[III = |g(a) - L| \le 2 \sup_{(a, b)} |g(x)|\]

Складываем:

\[(*) \le \sup_{(a, b)} |g(x)| \cdot \sup_{t \in [a, b)} \left| \int_t^{\rightarrow b} f dx\right| + 2 \cdot \sup_{(a, b)} |g(x)| \cdot 2 \cdot \sup_{t \in [a, b)} \left| \int_t^{\rightarrow b} f dx\right| = 5 \cdot \sup_{(a, b)} |g(x)| \cdot \sup_{t \in [a, b)} \left| \int_t^{\rightarrow b} f dx\right|\]

ч. т. д. 

\subsubsection{Признак Абеля равномерной сходимости интеграла}
\textit{Формулировка:}

\begin{itemize}
    \item $f, g: [a, b) \times Y \rightarrow \mathbb{R}$
    \item $\int_{a}^{\rightarrow b} f(x, y) dx$ --- равномерно сходящийся на $Y$
    \item $g(x, y)$ --- ограничена на $[a, b) \times Y$ и монотонна
\end{itemize}

Тогда:

\[\int_{a}^{\rightarrow b} f(x, y) g(x, y) dx\]

--- равномерно сходящийся на $Y$.

\textit{Доказательство:}

Удивительно, но почти всё уже доказано! В частности, поточечная сходимость просто напрямую следует из предыдущей леммы! Посмотрим на определение равномерной сходимости (по хвостикам):

\[\forall \varepsilon > 0 \dbl \exists U(b) \dbl \forall t \in U(b) \dbl \forall y \in Y: \left| \int_t^{\rightarrow b} fg dx\right| < \varepsilon\]

Ну чтож, давайте применим нашу оценку на хвостик для какого-то $t_0 \int U(b)$:

\[\left| \int_{t_0}^{\rightarrow b} fg dx\right| \le \underbrace{5 \cdot \sup_{(a, b)}g(x)}_{\const} \cdot \underbrace{\sup_{t \in[t_0, b)} \left| \int_t^{\rightarrow b} f dx\right|}_{(1)}\]

Заметьте, что у нас есть равномерная сходимость для $(1)$! Так давайте подберём такую окрестность $U(b)$, чтобы в ней:

\[\forall y \in Y : \sup_{t \in[t_0, b)} \left| \int_t^{\rightarrow b} f dx\right| < \frac{\varepsilon}{5 \cdot \sup_{(a, b)} |g(x)|}\]

По многолетним традициям китайского письма, всё получается!

ч. т. д. 
\newpage

\section{Период Мезозойский}
\subsection{Важные определения}


\subsubsection{Потенциал, потенциальное векторное поле}
$O$ --- область (открытое + связное)

\[V, f: O \subset \mathbb{R}^m \rightarrow \mathbb{R}\]

$f \in C^1(O)$ является потенциалом (потенциального) векторного поля $V$, если:

\[\forall t \in O: \grad f(t) = V(t)\]

Загадка от КПК! Если $f_1, f_2$ --- потенциалы $V$, то $f_1 - f_2 = \const$ (ну доказательство вроде очев, типа, мы же берём градиенты от $f_i$, и оба этих градиента равны $V$. При дифференцировании константа уничтожается и всё получается)

\subsubsection{Поверхностный интеграл первого рода}

\begin{itemize}
    \item $M$ --- простое гладкое двумерное многообразие в $\mathbb{R}^3$ с параметризацией $\Phi$
    \item $f: M \rightarrow \mathbb{R}$ --- суммируема по мере $S$
\end{itemize}

Тогда $\iint_M f dS$ --- называется интегралом I рода функции $f$ по поверхности $M$. Обозначается как:

\[\iint_M f(x, y, z) dS \]

Причём, так как мы уже определили меру $S$, по теореме о взвешенном образе меры:

\[\iint_A f(x, y, z) dS = \iint_{\Phi^{-1}(A)} f \circ \Phi |[\Phi'_u, \Phi'_v]| du dv = \]
\[= \iint_{\Phi^{-1}(A)} f \circ \Phi \sqrt{EG - F^2} du dv\]

\[E = \Phi'^2_{1u} + \Phi'^2_{2u} + \Phi'^2_{3u}\]
\[G = \Phi'^2_{1v} + \Phi^2_{2v} + \Phi'^2_{3v}\]
\[F = \Phi'_{1u}\Phi'_{1v} + \Phi'_{2u}\Phi'_{2v} + \Phi'_{3u}\Phi'_{3v}\]

Ну либо можно через миноры расписать, векторно перемножив.

\subsubsection{Интеграл II рода}

\begin{itemize}
    \item $M$ --- простое гладкое двумерное многообразие в $\mathbb{R}^3$ с параметризацией $\Phi$
    \item $f: M \rightarrow \mathbb{R}^3$ --- непрерывная
    \item $n_0$ --- сторона $M$
\end{itemize}

Тогда:

\[\int_{M} \sk{F(x)}{n_0(x)} dS = (*)\]

--- поверхностный интеграл II рода (он же --- поток векторного поля через поверхность $M$, на лекции был показательный пример про кита и китовый ус,  рекомендую!)

Обозначения:

\[F = F(x, y, z) = (P(x, y, z), \quad Q(x, y, z), \quad R(x, y, z))\]

\[(*) = \iint_{O} P \begin{vmatrix}
    y'_u & y'_v \\
    z'_u & z'_v
 \end{vmatrix} + Q\begin{vmatrix}
    z'_u & z'_v \\
    x'_u & x'_v
 \end{vmatrix} + R \begin{vmatrix}
    x'_u & x'_v \\
    y'_u & y'_v
\end{vmatrix} dudv\]

Обозначается как:

\[\iint_{M} P dydx + Q dzdx + R dzdx\]
\subsubsection{Ротор, дивергенция векторного поля}

\[\textbf{rot } W = (R'_y - Q'_z, P'_z - R'_x, Q'_x - P'_y)\] --- ротор векторного поля (ещё одно векторное поле). В нашей литературе ещё называют вихрем.

\[\textbf{div } W = (P'_x, Q'_y, R'_z)\]

--- дивергенция векторного поля.

\newpage

\subsection{Определения}

\subsubsection{Кусочно-гладкий путь}

Просто путь: $\gamma: [a, b] \rightarrow \mathbb{R}^m$. Ну и кусочная гладкость означает, что существует конечное число точек разрыва 1го рода (те пределы односторонние существуют, но не равны) и если по ним разбить, то кусочки будут гладкими:

\[\exists \dbl \{t_i\}_{i = 1}^n, \dbl t_0 = a, \dbl t_n = b: \gamma(x)|_{[t_i - t_{i - 1}]} \text{ --- гладкий }\]

\subsubsection{Векторное поле}

$V: E \subset \mathbb{R}^m \rightarrow \mathbb{R}^m$ --- непрерывное отображение. $V$ --- \textbf{векторное поле}.

В каждой точке $E$ оно как бы задаёт вектор, который в этой точке находится. Широко применяется в физике, там кучу всего можно охарактеризовать векторным полем. Например, вы ложкой зачерпнули сметану, и теперь она с неё стекает. Каждой сточке сметаны можно сопоставить вектор скорости, с которой он стекает и таким образом что-то моделировать.


\subsubsection{Интеграл векторного поля по кусочно-гладкому пути}

$\gamma$ --- кусочно-гладкий, $V$ --- непрерывно, $V = (V_1, V_2, \ldots, V_m)$

\[I(V, \gamma) = \int_a^b\langle V\left(\gamma(t)\right), \gamma'(t)\rangle dt = \]

\[\int_a^b V_1\left(\gamma(t)\right)\gamma'_1(t) + V_2\left(\gamma(t)\right)\gamma'_2(t) + \ldots + V_m\left(\gamma(t)\right)\gamma'_m(t) dt =\]

Делаем замену: $x = \gamma(t)$; $x_1 = \gamma_1(t), x_2 = \gamma_2(t)$; $dx_m = \gamma'_m(t) dt$

\[\int_{\gamma} V_1 dx_1 + V_2 dx_2 + \ldots + D_m dm\]

\textit{Свойства:}

\begin{enumerate}
    \item Линейность по полю:
    
    \[I(\alpha V + \beta W, \gamma) = \alpha I(V, \gamma) + \beta I(W, \gamma)\]

    Доказывается очевидно внутри по свойствам скалярного произведения.

    \item Аддитивность по дроблению:
    
    $\gamma: [a, b] \rightarrow O, \dbl c \in (a, b), \dbl \gamma_1 = \gamma|_{[a, c]}, \dbl \gamma_2 = \gamma|_{[c, b]}$:

    \[I(V, \gamma_1) + I(V, \gamma_2) = I(V, \gamma)\]

    Очевидно по линейности интеграла (ну там пошаманить в скалярных произведениях)

    \item Замена параметра:
    
    \begin{itemize}
        \item $\gamma: [a, b] \rightarrow \mathbb{R}^m$ --- гладкий
        \item $\varphi: [p, q] \rightarrow [a, b]$ --- гладкий
        \item $\varphi(p) = a, \dbl \varphi(q) = b$
        \item $\tilde{\gamma} = \gamma \circ \varphi: [p, q] \rightarrow \mathbb{R}^m$
    \end{itemize}

    \[I(V, \gamma) = I(V, \tilde{\gamma})\]

    Доказательство (заменой просто):

    \[I(V, \gamma) = \int_{\gamma} \sk{V(\gamma(t))}{\gamma'(t)} dt \underset{t = \varphi(s)}{=} \int_{\tilde{\gamma}} \sk{V(\tilde{\gamma}(s))}{\gamma'(\varphi(s))} \varphi'(s)ds = \] 
    
    \[ = \int_{\tilde{\gamma}} \sk{V(\tilde{\gamma}(s))}{\gamma'(\varphi(s))\varphi'(s)}ds = \int_{\tilde{\gamma}} \sk{V(\tilde{\gamma}(s))}{\tilde{\gamma}'(s)}ds\]

    Ну и ещё надо вспомнить следствие о 2х параметризациях из прошлого семестра, которое говорит что как раз существует диффеоморфизм, перевозящий точки из $[p, q]$ в $[a, b]$ ($\varphi$ --- ?)

    \item Объединение носителей
    
    $\gamma_1: [a, b] \rightarrow \mathbb{R}^m, \gamma_2: [c, d] \rightarrow \mathbb{R}^m, \dbl \gamma_1(b) = \gamma_2(c)$

    Тогда $\gamma = \gamma_2\gamma_1: [a, b + (d - c)] \rightarrow \mathbb{R}^m$ (обозначение).

    \[\gamma(t) = \begin{cases}
        \gamma_1(t), & t \in [a, b]\\
        \gamma_2(t + c - b), & t \in [b + c, b + (d - c)]
    \end{cases}\]

    \[I(V, \gamma_2\gamma_1) = I(V, \gamma_1) + I(V, \gamma_2)\]

    \item Противоположный путь
    
    $\gamma: [a, b] \rightarrow \mathbb{R}^m$

    Тогда $\gamma^-: [a, b] \rightarrow \mathbb{R}^m = \gamma(a + b - t)$

    \[I(V, \gamma) = -I(V, \gamma^-)\]

    \item Оценка интеграла по пути:
    
    $L = \gamma([a, b])$ --- носитель, $l(\gamma)$ --- длина пути $\gamma$

    \[\left|I(V, \gamma)\right| \le l(\gamma) \cdot \sup_{x \in L} |V(x)|\]

    Доказательство по КБШ и теореме Вейерштрасса (вы удивитесь, 1й сем, максимум на компакте достигается):

    \[\left|I(V, \gamma)\right| \le \int_{\gamma} |\sk{V(\gamma(t))}{\gamma'(t)}| dt \le \int_{\gamma} ||V(\gamma(t))|| \cdot ||\gamma'(t)|| dt \le \]
    \[\le \max_{t \in L} |V(t)| \cdot \int_{\gamma} ||\gamma'(t)|| dt\]
\end{enumerate}

\subsubsection{Криволинейный интеграл, не зависящий от пути в области $O$}

Если $V: O \subset \mathbb{R}^m \rightarrow \mathbb{R}^m$, $\forall \dbl \gamma_1, \gamma_2$ --- кусочно-гладкие пути из $A$ в $B$:

\[I(V, \gamma_1) = I(V, \gamma_2)\]

--- то такой интеграл не зависит от пути в области $O$

\subsubsection{Локально потенциальное векторное поле}

$V: O \subset \mathbb{R}^m \rightarrow \mathbb{R}^m$ --- локально потенциальное векторное поле, если $\forall a \in O \dbl \exists r_a: V $ на $B(a, r_a)$ --- потенциальное векторное поле.

\subsubsection{Лемма о гусенице}

\begin{itemize}
    \item $\gamma : [a, b] \rightarrow \mathbb{R}^m$ --- непрерывный путь
\end{itemize}

Тогда существует дробление $a = t_0 < t_1 < \ldots < t_k < \ldots < t_n = b$ такое, что для каждого $k$ существует шар $B_k \in O, \gamma([t_{k - 1}, t_k]) \subset B_k$

При этом, можно требовать, что выполняется что-то из:

\begin{enumerate}
    \item Все шары $B_k$ имеют радиус меньше $\varepsilon$
    \item $V$ локально потенциально на каждом из шаров ($B(a, r_a)$ выбраны так, что помещаются внутри гусеничных шаров)
\end{enumerate}

\images{0.5}{gsntsa.jpg}

На лекции ещё был набросок доказательства, но там кмк что-то душно-сложное...

\subsubsection{Похожие пути}

$\gamma, \tilde{\gamma}: [a, b] \rightarrow \mathbb{R}^m$ называются похожими, если у них имеется общая гусеница.

\textit{Замечание:}

Любой путь похож на ломаную (что очень удобно).

\images{0.5}{pohozh_puti.jpg}

\subsubsection{Интеграл локально-потенциального векторного поля $V$ по непрерывному пути $\gamma$}

Фиксируем $\delta$ из леммы о похожести путей, близких данному и строим кусочно-гладкий путь $\tilde{\gamma}$, и считаем интеграл на нём: $I(V, \tilde{\gamma})$.

\subsubsection{Гомотопия путей (связанная и петельная)}

Гомотопия $\Gamma(t, u)$ --- непрерывное отображение вида $[a, b] \times [0, 1] \rightarrow O$, которое как бы объединяет два пути и непрерывно перегоняет один в другой. Параметр в данном случае обозначает ``время'' (уровень перегонки). Для двух путей $\gamma_0, \gamma_1: [a, b] \rightarrow O$ (для которых и строится гомотопия) выполняется:

\[\forall t \in [a, b]: \Gamma(t, 0) = \gamma_0(t)\]
\[\forall t \in [a, b]: \Gamma(t, 1) = \gamma_1(t)\]

Причём возможно наложение дополнительных условий на гомотопию, например:

\begin{enumerate}
    \item \textbf{Связанная гомотопия}
    
    \[\gamma_0(a) = \gamma_1(a)\]
    \[\gamma_0(b) = \gamma_1(b)\]

    \[\forall s \in [0, 1] \quad \Gamma(a, s) = \gamma_0(a), \dbl \Gamma(b, s) = \gamma_1(b)\]

    \images{0.4}{gomotop_sv.jpg}

    (направлены они в одну сторону, само собой)

    \item \textbf{Петельная гомотопия}
    \[\gamma_0(a) = \gamma_0(b)\]
    \[\gamma_1(a) = \gamma_1(b)\]

    \[\forall s \in [0, 1] \quad \Gamma(a, s) = \Gamma(b, s)\]

    \images{0.3}{gomotop_pet.jpg}
\end{enumerate}

\subsubsection{Односвязная область}

$O \subset \mathbb{R}^m$ --- область

Если $\forall \gamma: [a, b] \rightarrow O$ путь --- петля (замкнутая) --- гомотопен (петельно) постоянному пути, то такая область $O$ называется \textbf{односвязной}. (ну типа любую петлю можно стянуть в точку).

\subsubsection{Измеримое множество на простом гладком двумерном многообразии в $\mathbb{R}^3$}

Повторение:

\begin{itemize}
    \item $M \subset \mathbb{R}^3$ --- простое гладкое двумерное многообразие в $\mathbb{R}^3$
    \item $\Phi: O \subset \mathbb{R}^2 \rightarrow \mathbb{R}^3 \in C^1$ --- гомеоморфизм (параметризация, непрерывна в обе стороны)
    \item $\rank \Phi'(x) = 2$ во всех точках $x \in O$
\end{itemize}

Множество $E \subset M$ называется измеримым по Лебегу в $M$, если $\Phi^{-1}(E) \in \mathfrak{M}^2$ (сигма-алгебра измеримых множеств)

\subsubsection{Мера Лебега на на простом гладком двумерном многообразии в $\mathbb{R}^3$}

Досооружаем сигма-алгебру:

$\mathfrak{A}_M := \{ E \subset M, E \text{ --- измеримое}\}$

Ну и тогда мера задаётся как:

\[\Phi'_u = \begin{pmatrix}
    \Phi'_{u_1}\\
    \Phi'_{u_2}\\
    \Phi'_{u_3}
\end{pmatrix}, \dbl \Phi'_{v} = \begin{pmatrix}
    \Phi'_{v_1}\\
    \Phi'_{v_2}\\
    \Phi'_{v_3}
\end{pmatrix}\]

\[S(E) = \int_{\Phi^{-1}(E)} [\Phi'_u, \Phi'_v]du dv\]

(скобочки --- это векторное произведение)

\subsubsection{Кусочно-гладкая поверхность в $\mathbb{R}^3$}

$M \subset \mathbb{R}^3$ --- кусочно-гладкое двумерное многообразие, если $M$ объединение конечного числа:

\begin{enumerate}
    \item простых гладких двумерных многообразий в $\mathbb{R}^3$ ($M_i$)
    \item кусочно-гладких кривых ($L_j$)
    \item конечного числа точек ($P_k$)
\end{enumerate}

\[S(E) = \sum_i S(E \cap M_i)\]

\subsubsection{Сторона поверхности}

Сторона поверхности --- непрерывное поле единичных нормалей для этой поверхности.

Бывают \textbf{односторонние} поверхности (например, лента Мёбиуса), на которых невозможно задать сторону ( попробуйте) )

На лекции был пример про собачек, рекомендую к просмотру! (кратко: две собачки в $\mathbb{R}^2$, смотрящие друг на друга, никак не смогут поменять своё направление, не выходя в $\mathbb{R}^3$)

\subsubsection{Задание стороны поверхности с помощью касательных реперов}

Касательные реперы --- это \underline{упорядоченный набор} из двух (в $\mathbb{R}^3$) ЛНЗ касательных векторов, обладающих таким свойством, что наименьший угол между ними всегда направлен от вектора 1 к вектору 2.

Ну и тогда берём их векторное произведение (оно никода не вырождается в силу ЛНЗ) в каждой точке, и получаем сторону поверхности.

\images{0.3}{kas_repe.jpg}

\subsubsection{Ориентация контура, согласованная со стороной поверхности}

Ориентации $M$ (простое гладкое двумерное многообразие в $\mathbb{R}^3$) и $\partial M$ согласованы, если выполнено что-то из:

\begin{enumerate}
    \item Контур $\partial M$ с его ориентацией задаёт нужную ориентацию на $M$ 
    \item Строим в рандомной точке кривой касательное пространство. Из него выбираем вектор $\tau$, сонаправленный ориентации кривой. И вектор нормали $n$ (направленный от кривой (?)). И тогда репер, составленный из этих векторов будет задавать ориентацию.
    \item $[n, \tau]$ --- ориентирующая нормаль
\end{enumerate}

\images{0.35}{sogl_orients.jpg}

\subsubsection{Соленоидальное векторное поле}

Векторное поле $V$ называется соленоидальным, если существует векторное поле $B$ (векторный потенциал), такой что:

\[\textbf{rot } B = V\]

\newpage

\subsection{Важные теоремы}

\subsubsection{Обобщённая формула Ньютона-Лейбница}
\textit{Формулировка:}

\begin{itemize}
    \item $V: O \subset \mathbb{R}^m \rightarrow \mathbb{R}^m$ --- потенциальное векторное поле
    \item $\gamma: [a, b] \rightarrow O$ --- путь (кусочно-гладкий?)
    \item $\gamma(a) = A, \dbl \gamma(b) = B$
    \item $f$ --- потенциал
\end{itemize}


Тогда:

\[\int_{\gamma} Vdx_1 + Vdx_2 + \ldots + Vdx_m = f(B) - f(A)\]

\textit{Доказательство:}

\textbf{1. $\gamma$ --- гладкий}

Посмотрим на вектор значений векторного поля $V$ при фиксированной точке $\forall t \in [a, b]: \gamma(t)$:

\[\left(V_1(\gamma_1(t), \gamma_2(t), \ldots, \gamma_m(t)), \dbl \ldots, V_m(\gamma(t))\right) = (f'_{x_1}(\gamma_1(t), \gamma_2(t), \ldots, \gamma_m(t)), \dbl \ldots, f'_{x_m}(\gamma(t)))\]

И если раскрыть криволинейный интеграл по определению, то получится, что:

\[\int_a^b f'_{x_1}(\gamma(t))\gamma'_1(t) + \ldots + f'_{x_m}(\gamma(t))\gamma'_m(t)dt = \int_a^b \left( f(\gamma_1(t), \gamma_2(t), \ldots, \gamma_m(t))\right)'_t dt = f(\gamma(b)) - f(\gamma(a)) = f(B) - f(A)\]

Мы типа взяли производную от всей функции $f$, типа дифференциал, поэтому у нас получилась сумма частных производных, которые, в свою очередь, продифференцировались по правилу для композиции и у нас у каждого слагаемого вылез $\gamma'_i(t)$. Ну всё, для гладких разобрались.

\textbf{2.}

Проблемы могут возникать, если у нас путь составлен из нескольких путей, и тогда в точках склейки может не быть дифференцируемости (но зато будут односторонние производные, поэтому непрерывность будет). Так давайте поинтегрируем на кусочках! $\exists$ дробление:

\[a = t_0 < t_1 < \ldots < t_k = b: \gamma|_{[t_i, t_{i + 1}]} \text{ --- гладкий}\]

\[I(V, \gamma) = \sum_{i = 1}^{k} \int_{\gamma|_{t_{i - 1}, t_i}} \sk{V(\gamma(s))}{\gamma'(s)}ds = \sum_{i = 1}^{k} (f(\gamma(t_{k}) - \gamma(t_{k - 1}))) = f(\gamma(b)) - f(\gamma(a)) = f(B) - f(A)\]

ч. т. д. 

\subsubsection{Необходимое условие потенциальности гладкого поля. Лемма Пуанкаре}
\textit{Формулировка (необходимое условие потенциальности):}

\begin{itemize}
    \item $V: O \subset \mathbb{R}^m \rightarrow \mathbb{R}^m$ --- гладкое потенциальное векторное поле
\end{itemize}

Тогда:

\[\forall i, j \in [1, m]: \frac{\partial V_i}{\partial x_j} = \frac{\partial V_j}{\partial x_i}\]

\textit{Доказательство:}

Раз поле потенциально, значит есть потенциал $f$. По определению для каждого $k$:

\[\frac{\partial f}{\partial x_k} = V_k\]

Ну и всё, выражаем тогда:

\[\frac{\partial V_i}{\partial x_j} = \frac{\partial f^2}{\partial x_i \partial x_j}\]

\[\frac{\partial V_j}{\partial x_i} = \frac{\partial f^2}{\partial x_j \partial x_i}\]

Ну а раз у нас всё непрерывное, то можно переставлять местами порядок дифференцирования (2 сем).

ч. т. д. 

\textit{Формулировка (лемма Пуанкаре):}

\begin{itemize}
    \item $V: O \subset \mathbb{R}^m \rightarrow \mathbb{R}^m$ --- гладкое потенциальное векторное полное
    \item $O$ --- выпуклая область
    \item выполняется необходимое условие потенциальности
\end{itemize}

Тогда $V$ потенциально на $O$

\textit{Доказательство: }

Вспоминаем доказательство ``Характеризации потенциальных векторных полей в терминах интегралов''. Там мы фиксировали $A \in O$ и брали путь от каждой точки $x$ до $A$. Здесь мы шагаем ещё шире: мы зададим через такой путь потенциал!

\[\forall x \in O : \gamma_x(t) = A + t(x - A), \quad t \in [0, 1]\]

\[\gamma'_x(t) = x - A\]

Ну и тогда:

\[f(x) = \int_{\gamma_x} \sk{V(\gamma(t))}{\gamma'(t)}dt = \int_0^1 \sum_{k = 1}^m \underbrace{V_k(A + t(x - A))}_{(1)}\cdot\underbrace{(x_k - A_k)}_{(2)} dt\]

Осталось лишь проверить, что это действительно потенциал (применяем правило Лейбница! Тут у нас, получается, параметром является $x_i$, по нему и дифференцируем. То есть, мы фактически применяем правило Лейбница из прошлого семестра!) Сразу заметим 2 вещи, вот мы сейчас будем дифференцировать по $x_i$, выражение $(2)$ при $k = i$ продифференцируется и даст 1, и мы получим отдельно стоящее слагаемое (1) (дифференцирование произведения). Ну а в остальных случаях всё (в смысле сумма из (1)) будет дифференцироваться как обычно, т. к. там внутри векторного объекта везде есть $i$-я координата:

\[\frac{\partial f}{\partial x_i} = \int_0^1 V_i(A + t(x - A)) + \sum_{k = 1}^m (V_k)'_i(A + t(x - A)) t\cdot (x_k - A_k )dt = \]

Замечаем, что можем в сумме поменять индексы производной и суммирования, так как выполнено необходимое условие потенциальности:

\[ = \int_0^1 V_i(A + t(x - A)) + \sum_{k = 1}^m (V_i)'_k(A + t(x - A)) t\cdot (x_k - A_k )dt =\]

Следующее доказывается методом пристального взгляда на предыдущее:

\[ = \int_0^1 \left(tV_i(A + t(x - A))\right)'_t dt = tV_i(A + t(x - A))|_0^1 = V_i(x)\]

Вот и доказалось.

ч. т. д. 

\textit{Замечание:}

Вместо выпуклой области можно использовать ``звёздную'', это такая клякса, в которой каждая точка доступна по отрезку из центра (внутри кляксы).

\textit{Следствие:}

Всё то же самое, что и в самой лемме, но $O$ --- любая область, и тогда $V$ --- локально-потенциально.

\textit{Доказательство} вроде очевидно, выбираем шарики и на них запускаем лемму.


\subsubsection{Теорема Пуанкаре для односвязной области}
\textit{Формулировка:}

\begin{itemize}
    \item $V: O \subset \mathbb{R}^m \rightarrow \mathbb{R}^m$ --- локально-потенциальное векторное поле
    \item $O$ --- односвязная область 
\end{itemize}

Тогда $V$ --- потенциально

\textit{Доказательство:}

Рассмотрим любой замкнутный путь (петлю) $\gamma$. Утверждается, что она гомотопна (петельно) любому постоянному пути $\tilde{\gamma}$ --- по односвязности области. Тогда, можно запустить теорему о равенстве интегралов по гомотопным путям (но там были необходимы \textbf{связанно гомотопные} пути, а у нас они гомотопны петельно. Или пофигу? \textit{Ответ КПК: именно! Мы доказывали всё для связанных, а тут применили для петельных. Так и задумано, произошла подмена. Типа для петельных в теоремах примерно то же самое, поэтому мы для них не доказывали.}). Заметим, что $\tilde{\gamma}'(t) = 0$, так как у постоянного пути скорость нулевая:

\[\int_{\gamma} = \int_{\tilde{\gamma}} \sk{V(\tilde{\gamma}(t))}{\tilde{\gamma}'(t)}dt = 0\]

Ну и получается, что у нас интеграл любой петли равен 0, что по теореме о характеризации векторных полей в терминах интегралов значит, что $V$ --- потенциально.

ч. т. д. 

\textit{Следствие:}

Лемма Пуанкаре верна для любой односвязной области.

\subsubsection{Формула Стокса}
\textit{Формулировка:}

\begin{itemize}
    \item $\Omega$ --- простое двумерное гладкое многообразие в $\mathbb{R}^3$
    \item $\partial \Omega$ --- кусочно-гладкая кривая
    \item На них заданы согласованные ориентации
    \item $\Phi: G \subset \mathbb{R}^2 \rightarrow \mathbb{R}^3 \in C^2$ --- параметризация
    \item $(P, Q, R)$ --- гладкое $C^1$ векторное поле в $U(\Omega)$.
\end{itemize}

Тогда:

\[\int_{\partial \Omega} P dx + Q dy + R dz= \iint_{\Omega} (R'_y - Q'_z)dydz + (P'_z - R'_x)dxdz + (Q'_x -P'_y)dxdy\]

\textit{Доказательство:}

Опять докажем только для $P$:

\[\int_{\partial \Omega} P dx = \iint_{\Omega} P'_z dxdz - P'_ydxdy\]

Смотрим на картинку: у нас есть некоторое $G$, в котором задана параметризация многообразия. Давайте параметризуем нашу кривую $\partial \Omega$ в ней. То есть, $L$ --- прообраз границы $\Omega$. И параметризуем эту кривую через $t$. Получается, $L$ задаётся как $u(t)$ и $v(t)$, а $\partial \Omega$ в свою очередь $x(u(t), v(t)), \dbl y(u(t), v(t)), \dbl z(u(t), v(t))$. Ну что же, давайте подставим:


\images{0.3}{stocks.jpg}

\[\int_{\partial \Omega} Pdx = \int_{L} P(x(u(t), v(t)), \ldots)(x'_udu + x'_vdv) dt= \int_{L} Px'_udu + Px'_vdv = \]

Оформили подстановку и пересчитали дифференциал по правилу дифференцирования ФНП. Замечаем, что можем применить формулу Грина и перейти к двойному интегралу! (тут должна быть звёздочка, так как надо учитывать ориентацию. Доказательство того, что всё хорошо заключается в яростном махании руками: надо нарисовать касательный репер на границе $G$ и посмотреть, как они переезжают в $\Omega$ при отображении).

\[ = \iint_{\Omega} (Px'_v)'_u - (Px'_v)'_v dudv = \iint_{G} (\uwave{P'_xx'_u} + P'_yy'_u+ P'_zz'_u)x'_v + \uwave{P x''_{vu}} - (\uwave{P'_xx'_v }+ P'_yy'_v+ P'_zz'_v)x'_u - \uwave{P x''_{uv}} du dv = \]

Выделенное сокращается, остальное запихивается в определители:

\[ = \iint_{G} P'_yy'_ux'_v+ P'_zz'_ux'_v -P'_yy'_vx'_u -P'_zz'_vx'_u du =\]

\[ = \iint_{G} P'_z \begin{vmatrix}
    z'_u & z'_v \\
    x'_u & x'_v
\end{vmatrix}  -P'_y \begin{vmatrix}
    x'_u & x'_v \\
    y'_u & y'_v
\end{vmatrix} du dv =\]

\[ = \iint_{\Omega} P'_z dzdx - P'_ydxdy\]

ч. т. д. 


\subsubsection{Формула Гаусса--Остроградского}
\textit{Формулировка:}

\begin{itemize}
    \item $\partial G$ --- кусочно-гладкая кривая в $\mathbb{R}^2$
    \item $V$ --- ``хорошее'' множество, для которого для каждой пары $(x, y), \dbl (y, z), \dbl (x, z)$ заданы функции $f_1, f_2: U(G) \rightarrow \mathbb{R} \in C^1$ (пример для $(x, y)$):
    \[V = \{(x, y, z): \forall x, y \in G \subset \mathbb{R}^2: f_1(x, y) \le z \le f_2(x, y)\}\]
    \item На $\partial V$ задана ориентация полем внешних нормалей
    \item $(P, Q, R): U(V) \rightarrow \mathbb{R}^3$ --- гладкое $C^1$ векторное поле.
\end{itemize}

Тогда:

\[\iint_{\partial V} Pdydz + Qdxdz + Rdxdy = \iiint_{V} (P'_x + Q'_y + R'_z)dxdydz\]

\textit{Доказательство:}

Опять проверяем только одну часть: $R$.

\[\iint_{\partial V} R dxdy = \iiint_{V} R'_zdxdydz\]


Опять проверяем обе части. Справа:

\[\iiint_{V} R'_z dxdydz = \iint_{G} dxdy\int_{f_1(x, y)}^{f_2(x, y)} R'_z(x, y, z) dz = \iint_{G} R(x, y, f_2(x, y)) - R(x, y, f_1(x, y)) dxdy\]

Слева:

\[\iint_{\partial V} R dxdy = \iint_{\text{низ}} + \underbrace{\iint_{\text{вертикальная часть}}}_{0} +  \iint_{\text{верх}} = \iint_{G} R(x, y, f_2(x, y)) dxdy - \iint_{G} R(x, y, f_1(x, y)) dxdy\]

``Низ'' и ``верх'' --- это как раз наши ограничивающие функции $f_1$ и $f_2$. Причём у одного из интегралов поменялся знак --- у него поле нормалей смотрело в противоположную сторону. Ну, и вертикальная часть --- множество меры 0 (ну или перпендикулярно нормалям, не принципиально).

ч. т. д.

\subsection{Теоремы}

\subsubsection{Характеризация потенциальных векторных полей в терминах интегралов}
\textit{Формулировка:}

$V: O \subset \mathbb{R}^m \rightarrow \mathbb{R}^m$ --- векторное полное

Тогда следующее эквивалентно:

\begin{enumerate}
    \item $V$ --- потенциальное векторное полное
    \item интеграл по $V$ не зависит от пути (см. определение)
    \item $\forall \gamma : [a, b] \rightarrow O$ --- кусочно гладкий $, \dbl \gamma(a) = \gamma(b) $ (замкнутый) $ \quad I(V, \gamma) = 0$
\end{enumerate}

\textit{Доказательство:}

\textbf{1 $\Rightarrow$ 2}

Проверим по обобщённой формуле Ньютона-Лейбница:

\[\forall \gamma_1, \gamma_2 \text{ от } A \text{ до } B: I(V, \gamma_1) - I(V, \gamma_2) = f(B) - f(A) - (f(B) - f(A)) = 0\]

\textbf{2 $\Rightarrow$ 3}

Сочиним 2 кусочно-гладких пути $\gamma_1, \gamma_2: [a, b] \rightarrow \mathbb{R}^m, \dbl \gamma_1(a) = \gamma_2(a) = A, \dbl \gamma_1(b) = \gamma_2(b) = B$. $I(V, \gamma_1) = I(V, \gamma_2)$ по условию $2$. Ну и всё, замкнём его, взяв пути $\gamma_1$ и $(\gamma_2)^-$:

\[I(V, \gamma_1) + I(V, (\gamma_2)^-) = I(V, \gamma_1) - I(V, (\gamma_2)) = 0\]

\textbf{3 $\Rightarrow$ 2}

\images{0.3}{harakt_pot_vp_1.jpg}

Почти то же самое, выбираем два пути от $A$ до $B$, один прогоняем как обратный, и получаем петлю. Интеграл петли по условию ноль, ну значит интегралы двух составляющих путей ($\gamma_1$ и $\gamma_2$ на картинке (второй путь тоже от $A$ до $B$, стелочка для его противоположного)) равны.

\[I(V, \gamma_1) + I(V, (\gamma_2)^-) = 0 \Leftrightarrow I(V, \gamma_1) = I(V, \gamma_2)\]

\textbf{2 $\Rightarrow$ 1}

Самое неочевидное. Ищем потенциал. Фиксируем $A \in O$ и для каждого $x \in O$ выберем какой-то путь $\gamma_x$ от $x$ до $A$. И утверждаем, что $f$ --- потенциал:

\[f(x) = \int_{\gamma_x} Vdx_1 + \ldots + Vdx_m\]

Конечно, никто нам просто так на слово не поверит, поэтому давайте проверим это утверждение. Вообще, достаточно проверить для одной координаты (не умаляя общности) условие:

\[\forall x \in O : \frac{\partial f}{\partial x_1} = V_1(x)\]

Посмотрим на картинку:

\images{0.5}{harakt_pot_vp_2.jpg}

Зафиксировали путь от $A$ до $x: \gamma_x = \gamma_1$ (на картинке). Дальше нас просят производную по первой координате, значит, мы должны предложить какой-то путь, который смещает точку $x$ вдоль 1й координаты. Да вот же он! $\gamma_2(t) = x + the_1$, где $t \in (0, 1)$. Тогда можно будет посчитать производную ``школьным'' способом (сначала посчитаем числитель, берём значение в двух точках, он суть есть криволинейный интеграл):

\[f(x + he_1) - f(x) = \int_{\gamma_2\gamma_x} - \int_{\gamma_x} = \int_{\gamma_2} = (*)\]

Приглядимся, а кто вообще этот путь $\gamma_2(t)$ в векторном смысле?

\[\gamma_2(t) = \begin{pmatrix}
    x_1 + th\\
    x_2\\
    \vdots \\
    x_m\\
\end{pmatrix}, \quad \gamma'_2(t) = \begin{pmatrix}
    h\\
    0\\
    \vdots\\
    0
\end{pmatrix}\]

$e_1$ пропал, т. к. он там и так на вктор домножается за кадром (?)

\[(*) = \int_{0}^{1} \underbrace{V_1(x_1 + th, x_2, \ldots, x_m) \cdot h}_{\Phi(t)}dt =\]

А вот сейчас внимательно! Существует такая теорема о среднем (мы вроде бы доказывали её во втором семестре), которая гласит о том, что определённый интеграл равен значению функции в какой-то средней точке, помноженной на длину промежутка. А ещё у нас самый обычный интеграл, зависит только от первой координаты (только там есть $t$)! Поэтому интегрируем, применяя теорему о среднем:

\[ = V_1(x_1 + ch, x_2, \ldots, x_m) \cdot h \cdot (1 - 0), \quad c \in [0, 1]\]

Ну и всё, собираем определение производной:

\[\frac{f(x_1 + he_1) - f(x)}{h} = V_1(x_1 + ch, x_2, \ldots, x_m)\]

$h \rightarrow 0$:

\[\frac{\partial f}{\partial x_1} = V_1(x_1, x_2, \ldots, x_m) = V_1(x)\]

ч. т. д. 

\subsubsection{Лемма о равенстве интегралов по похожим путям}
\textit{Формулировка:}

\begin{itemize}
    \item $V: O \subset \mathbb{R}^m \rightarrow \mathbb{R}^m$ --- векторное поле
    \item $\gamma, \tilde{\gamma}: [a, b] \rightarrow O$ --- похожие кусочно-гладкие пути, причём $\gamma(a) = \tilde{\gamma}(a) = A, \dbl \gamma(b) = \tilde{\gamma}(b) = B$
\end{itemize}

Тогда:

\[I(V, \gamma) = I(V, \tilde{\gamma})\]

\textit{Доказательство:}

Возьмём их общую гусеницу с $V$ ($a = t_0 < t_1 < \ldots < t_n = b$, на каждом шарике свой потенциал $f_i$. Причём, как мы знаем из загадки КПК два потенциала отличаются максимум на константу. Ну так вот, на пересечении шаров у нас как раз образуются два потенциала. И мы констанктой можем подогнать их друг к другу, чтобы $\forall i: f(\gamma(t_i)) = f_{i + 1}(\gamma(t_i))$).

\images{0.5}{lemm_pohoshie_putiii.jpg}

Расписываем по дроблению, у нас всё кусочно-гладкое, все дела (видимо, дробление должно согласовываться ещё и с этим (?)):

\[\int_{\gamma} \sk{V(\gamma(t))}{\gamma'(t)} dt = \sum_{k = 1}^m \int_{t_{k - 1}}^{t_k} \sk{V(\gamma(t))}{\gamma'(t)} dt = \sum_{k = 1}^m f_i(\gamma(t_k)) - f_i(\gamma(t_{k - 1})) = \] 

По обобщённой формуле Ньютона-Лейбница. Сумма получается телескопическая, так как соседние потенциалы мы подогнали и в итоге получается :

\[= f_n(\gamma(t_n)) - f_0(\gamma(t_0)) = f_n(B) - f_0(A)\]

Для пути $\tilde{\gamma}$ аналогично, учитывая, что потенциалы на границах мы подогнали:

\[= f_n(\tilde{\gamma}(t_n)) - f_0(\tilde{\gamma}(t_0)) = f_n(B) - f_0(A)\]

ч. т. д. 

\textit{Замечание:}

Вместо нашего условия о равенстве начал и концов (c), можно было использовать $\gamma(a) = \gamma(b)$, $\tilde{\gamma}(a) = \tilde{\gamma}(b)$

\subsubsection{Лемма о похожести путей, близких к данному}
\textit{Формулировка:}

\begin{itemize}
    \item $V: O \subset \mathbb{R}^m \rightarrow \mathbb{R}^m$ --- локально-потенциальное векторное поле
    \item $\gamma: [a, b] \rightarrow O$ --- непрерывный путь
\end{itemize}

Тогда $\exists \delta > 0$: если $\tilde{\gamma}, \tilde{\tilde{\gamma}}: [a, b] \rightarrow O$ таковы, что $\forall t \in [a, b]$:

\[|\gamma(t) - \tilde{\gamma}(t)| < \delta\]
\[|\gamma(t) - \tilde{\tilde{\gamma}}(t)| < \delta\]

То пути $\gamma, \tilde{\gamma}, \tilde{\tilde{\gamma}}$ похожи в смысле поля $V$.

\textit{Доказательство:}

Зафиксируем гусеницу для $\gamma$. Утверждается, что найдётся такое $\delta_k > 0$ для каждого шара, что весь кусок путь $\gamma$ лежит в ``обмотке'' ($\delta_k$-окрестности) внутри шара гусеницы. Почему такое возможно? Давайте посмотрим на картинку:

\images{0.35}{lemm_blz_dann.jpg}

Зелёная линия --- это путь $\gamma$. Далее фиксируем какой-нибудь шар $k$ и рассматриваем поведение пути внутри него. Нам хочется, чтобы существовала $\delta_k$-окрестность этого участка пути, она схематично показана красной ``обмоткой'' (как на проводе). Рассмотрим точку $u$ на пути и найдём следующую величину ($S_k$ --- сфера (ну вроде бы да, дальше мы это расстояние мы будем использовать для шара, значит за границу $B_k$ не вылезем)):

\[r_k = \inf_{u \in \gamma([t_{k - 1}, t_k])} \rho(u, s), \quad s \in S_k\]

Фактически, это просто минимальное расстояние от пути до границы шара (оно всегда положительно, путь лежит внутри сферы по лемме о гусенице), оно бы нам как раз подошло в качестве $\delta_k$, ведь тогда мы сможем в каждой точке соорудить шарик этого радиуса и получить заветную ``обмотку''. И получается так, что это расстояние достигается по теореме Вейерштрасса ($\gamma([a, b])$ --- компакт (компакт под действием непрерывного отображения), $S_k$ --- компакт и $\gamma$ --- непрерывно)!

Ура, тогда пробегаем так по всем $n$ шарикам и получаем $\delta$:

\[\delta := \min \{ \delta_1, \delta_2, \ldots, \delta_n \} \]

Получается, что мы требуем от кандидатов на похожесть лежать в одной ``обмотке'' с данным путём, что гораздо более сильное условие, чем даже в одном шаре.

ч. т. д. 

\subsubsection{Равенство интегралов по гомотопным путям}
\textit{Формулировка:}

\begin{itemize}
    \item $V: O \subset \mathbb{R}^m \rightarrow \mathbb{R}^m$ --- локально-потенциальное векторное поле
    \item $\gamma_0, \gamma_1: [a, b] \rightarrow O$ --- связанно-гомотопные пути
\end{itemize}

Тогда $I(V, \gamma_0) = I(V, \gamma_1)$

\textit{Доказательство:}

Для начала зафиксируем путь $\gamma_u(t) = \Gamma(t, u)$ (гомотопия $\gamma_0$ и $\gamma_1$) для каждого момента времени $u$. А ещё заметим, что $\Gamma$ --- равнормено непрерывна (задана на $[a, b] \times [0, 1]$ --- компакт и непрерывна, теорема Кантора). Распишем это (в конце с ``китайским'' акцентом):

\[\forall \delta > 0 \dbl \exists \sigma > 0 \dbl \forall t, \tilde{t} \in [a, b] \dbl |t - \tilde{t}| < \sigma \dbl \forall u, \tilde{u} \in [0, 1] \dbl |u - \tilde{u}| < \sigma \quad |\Gamma(t, u) - \Gamma(\tilde{t}, \tilde{u})| < \frac{\delta}{2}\]

А как же считать интеграл? Заведём функцию:

\[\Phi(u) = I(V, \gamma_u)\]

Нам хочется показать, что она локально-постоянна:

\[\forall u_0 \in [0, 1] \dbl \exists V(u_0) \dbl \forall u \in V(u_0) \quad \Phi(u) = \Phi(u_0)\]

Если мы это покажем, то тогда по компактности отрезка $[0, 1]$ можно будет сказать, что для каждого покрытия этого отрезка открытыми множествами существует конечное подпокрытие (определение компакта), и вот в каждом кусочке этого покрытия будет лежать наша локально-постоянная окрестность, в т. ч. на пересечениях, и таким образом вся функция на отрезке будет одинаковой. Вот что-то в таком духе. Картинка для визулизации процесса:

\images{0.3}{rav_gomotop_put.jpg}

Давайте доказывать. Мы фиксируем $u_0$, выбераем $\delta$ из прошлой леммы для $\gamma_{u_0}$, по нему вычисляем $\sigma$ из равномерной непрерывности $\Gamma$. Тогда для $\forall t \in [a, b]$ и $|u - u_0| < \sigma$:

\[|\gamma_{u_0}(t) - \gamma_{u}(t)| = |\Gamma(t, u_0) - \Gamma(t, u)| < \frac{\delta}{2}\]

Есть 2 новости. Хорошая состоит в том, что пути $\gamma_{u}$ и $\gamma_{u_0}$ оказались похожи! Плохая в том, что эти пути непрерывные, а для запуска леммы о равенстве интегралов по похожиму путям нам нужны кусочно-гладкие. Да не проблема, давайте построим кусочно-гладкие пути $\tilde{\gamma}_{u}$ и $\tilde{\gamma}_{u_0}$ отстоящие от своих непрерывных собратьев на $\frac{\delta}{4}$:

\[|\gamma_{u} - \tilde{\gamma}_{u}| < \frac{\delta}{4}\]
\[|\gamma_{u_0} - \tilde{\gamma}_{u_0}| < \frac{\delta}{4}\]

Тогда получается, что по всей цепочке $|\tilde{\gamma}_{u} - \tilde{\gamma}_{u_0}| < \delta$ ! (ну типа сравниваем $\tilde{\gamma}_{u}$ с $\gamma_{u}$ --- не больше $\frac{\delta}{4}$, потом $\gamma_{u}$ с $\gamma_{u_{0}}$ --- не больше $\frac{\delta}{2}$, ну и наконец $\tilde{\gamma}_{u}$ с $\gamma_{u}$ --- тоже не больше $\frac{\delta}{4}$ --- в сумме не больше $\delta$). Значит, по лемме кусочно-гладкие пути тоже похожи, и их интегралы ($\Phi$) равны. Таким образом, $\Phi$ --- локально-постоянно и всё супер!

ч. т. д


\subsubsection{Теорема о веревочке}
\textit{Формулировка:}

Правда ли, что $\mathbb{R}^2 \setminus (0, 0)$ --- не односвязно? 

Ну или, что эквивалентно, петля $\gamma: [0, 2\pi] \rightarrow O$, $t \mapsto (\cos t, \sin t)$ --- нестягиваемая.

\images{0.5}{lemm_o_ver.jpg}

(ниточку стащить с гвоздика нельзя, не разрывая)

\textit{Доказательство:}

Предлагаем такое вектороное поле:

\[V = \left( \frac{-y}{x^2 + y^2}, \frac{x}{x^2 + y^2}\right)\]

Проверим локальную потенциальность:

\[\frac{\partial V_1}{\partial y} = \frac{-(x^2 + y^2) + y(2y)}{(x^2 + y^2)^2} = \frac{y^2 - x^2}{(x^2 + y^2)^2}\]

\[\frac{\partial V_2}{\partial x} = \frac{(x^2 + y^2) - x(2x)}{(x^2 + y^2)^2} = \frac{y^2 - x^2}{(x^2 + y^2)^2}\]

Выполнилось (+ лемма Пуанкаре (?)). Тогда мы ожидаем, что интеграл по петле будет равено нулю. Ну что ж, давайте проверим:

\[\int_{\gamma} V_1 dx + V_2 dy = \int_{0}^{2\pi} \frac{-\sin t}{\cos^2 t + \sin^2 t} \cdot (- \sin t) + \frac{\cos t}{\cos^2 t + \sin^2 t} \cdot (\cos t) dt = \]

\[= \int_0^{2\pi} 1 dt = 2\pi\]

Не ноль, значит не односвязно. 

ч. т. д. 

\subsubsection{Формула Грина}
\textit{Формулировка:}

\begin{itemize}
    \item $D$ --- ``хорошая'' фигура (для которой существуют ограничивающие функции по каждой координате, см. картинку) с кусочно-гладкой границей. Направление границы ($\partial$) выбираем против часовой стрелки (т. к. это нормальное направление, если смотреть сверху на стандартную систму координат)
    
    \[\exists f_1, f_2: D = \{(x, y): \forall x \in [a, b]: f_1(x) \le y \le f_2(x)\}\]

    (по $y$ аналогично)

    \images{0.5}{green.jpg}

    Утверждается, что более-менее любую фигуру можно нарезать на хорошие и поинтегрировать по частям.

    \item $P, Q: U(D) \rightarrow \mathbb{R} \in C^1$
\end{itemize}

Тогда:

\[\int_{\partial D} Pdx + Qdy = \iint_{D} \left(\frac{\partial Q}{\partial x} - \frac{\partial P}{\partial y}\right) dxdy\]

\textit{Доказательство:}

Достаточно доказать лишь часть про $P$ (типа разбиваем на 2 векторных поля $(P, 0), (0, Q)$, а потом обратно складываем, в другом случае всё аналогично):

\[\int_{\partial D} Pdx = \iint_{D} -\frac{\partial P}{\partial y} dxdy\]

Ну, давайте проверять. Введём параметризацию путями, как на картинке. Правая часть:

\[\iint_{D} -\frac{\partial P}{\partial y} dxdy = -\int_{a}^{b} dx \int_{f_1(x)}^{f_2(x)} P'_y(x, y) dy = -\int_{a}^{b} P(x, f_2(x)) - P(x, f_1(x))dx\]

Левая (интегралы по вертикальным путям зануляются (интегрирование по множествам меры 0)):

\[\int_{\partial D} P dx = \int_{\gamma_1} + \underbrace{\int_{\gamma_2}}_{0} + \int_{\gamma_3} + \underbrace{\int_{\gamma_4}}_{0} = \int_{a}^{b} P(x, f_1(x)) dx + \int_{b}^{a} P(x, f_2(x))dx =\]

Замечаем, что второй интеграл неправильно ориентирован и меняем:

\[ = \int_{a}^{b} P(x, f_1(x)) dx - \int_{a}^{b} P(x, f_2(x))dx = -\int_{a}^{b} P(x, f_2(x)) - P(x, f_1(x)) dx \]

Получилось!

ч. т. д. 

\textit{Замечание: }

Вообще, эта и следующие теоремы есть следствие более общей теоремы Стокса о дифференциальных формах. Кому интересно --- приглашаю посмотреть лекцию!

\subsubsection{Соленоидальность бездивиргентного векторного поля}
\textit{Формулировка:}

\begin{itemize}
    \item $K \subset \mathbb{R}^3$ --- куб
    \item $A \in C^1$ --- векторное поле
\end{itemize}

Тогда $A$ --- соленоидально $\Leftrightarrow \textbf{div} A = 0$

\textit{Доказательство:}

$\Rightarrow$

Очевидно, тогда существует $: \textbf{rot } B = A$. Рассмотрим:

\[\textbf{div }A = \textbf{div rot }B = \textbf{div } (R'_y - Q'_z, P'_z - R'_x, Q'_x - P'_y) = R''_{yx} - Q''_{zx} + P''_{zy} - R''_{xy} + Q''_{xz} - P''_{yz} = 0\]

$\Leftarrow$ 

Давайте напишем систему уравнений на $A$ (мы хотим найти для него некоторый векторный потенциал $B = (P, Q, R$)):

\[R'_y - Q'_z = A_1\]
\[P'_z - R'_x = A_2\]
\[Q'_x - P'_y = A_3\]

И запускаем программу ``наглость --- второе счастье!''. Пусть $R \equiv 0$. Тогда уравнения упрощаются:

\[- Q'_z = A_1\]
\[P'_z= A_2\]
\[Q'_x - P'_y = A_3\]

Зафиксируем точку $(x_0, y_0, z_0) \in K$, и соорудим решения первых двух уравнений в виде интегралов с переменным верхним пределом:

\[P = \int_{z_0}^{z} A_2(x, y, \xi) d\xi\]
\[Q = - \int_{z_0}^{z} A_1(x, y, \xi) d\xi + C(x, y)\]

Причём для $A_1$ прибавим ``константу'', которая исчезает при дифференцировании по $z$.

Тогда:

\[\textbf{div } A = (A_1)'_x + (A_2)'_y + (A_3)'_z = 0\]
\[ (A_3)'_z  = - (A_1)'_x - (A_2)'_y\]

Взятие производных от интегралов обеспечено Правилом Лейбница из прошлого семестра, всё непрерывно:

\[\underbrace{\int_{z_0}^{z} -(A_1)'_x(x, y, \xi) - (A_2)'_y(x, y, \xi) d\xi}_{(A_3)'_z} + C'_x(x, y) = A_3\]
\[A_3(x, y, z) - A_3(x, y, z_0) + C'_x(x, y) = A_3(x, y, z)\]

\[C'_x(x, y) = A_3(x, y, z_0)\]
\[C(x, y) = \int_{y_0}^{y} A_3(t, y, z_0) dt\]

Таким образом, соорудили $B$, значит, соленоидально.

ч. т. д. 

\newpage

\section{Период Кайнозойский}
\subsection{Важные определения}

\subsubsection{Гильбертово пространство}

$\mathcal{H}$ --- линейное пространство, в котором задано скалярное произведение и соответствующая норма. Если $\mathcal{H}$ --- полное, то оно называется гильбертовым.

Примеры: $\mathbb{R}^m, \mathbb{C}^m, L^2(X, \mu)$. А ещё $l^2(X, \mu)$ --- пространство счётных последовательностей: $(x_1, x_2, \ldots) = x$, $\mu$ --- счётная мера (на $\mathbb{N}$, просто считает количество элементов в множестве).

\subsubsection{Ортонормированная система, примеры}

${e_k}$ --- О. С. , тогда ${\frac{e_k}{|| e_k ||}}$ --- ортонормированная система.

Примеры: 

\begin{enumerate}
    \item $l^2 \quad e_k = (0, \ldots, 0, 1, 0, \ldots)$
    \item $L^2[0, 2\pi] \quad \{1, \cos t, \sin t, \cos 2t, \sin 2t, \cos 3t, \sin 3t, ldots\}$
    \item $\left(\frac{e^{ikt}}{\sqrt{2\pi}}\right)_{k \in \mathbb{Z}}$
\end{enumerate}

\subsubsection{Свертка}

Умножение на пространстве функций!

\[\forall f, K \in L^1[-\pi, \pi]: (f * K)(x) = \int_{-\pi}^{\pi} f(x - t)K(t) dt\]

--- \textbf{свёртка} функций $f, K$ в точке $x$.

\subsubsection{Аппроксимативная единица}

$D \subset \mathbb{R}, \dbl h_0$ --- предельная точка $D$, $E_\delta = [-\pi, \pi] \setminus [-\delta, \delta]$

Семейство функций $(K_h)_{h \in D}$ называется \textbf{аппроксимативной единицей}, если выполняются аксиомы

\begin{enumerate}
    \item \[\forall h \in D, \dbl K_h \in L^1[-\pi, \pi]: \int_{-\pi}^{\pi} K_h = 1\]
    \item $L^1$-нормы ограничены в совокупности
    
    \[\exists M > 0 \dbl \forall h \in D : ||K_h||_1 \le M \]
    \item \[ \forall \delta \in (0, \pi) : \int_{E_{\delta}} |K_h| dx \goesto{h \rightarrow h_0} 0 \]
\end{enumerate}

\textit{Замечания:}

\begin{enumerate}
    \item Если $K_h \ge 0$, то АЕ1 $\Rightarrow$ АЕ2
    \item Если $K_h$ --- (усиленная) АЕ, то $\frac{|K_h|}{||K_h||_1}$ --- тоже (усиленная) АЕ. (доказывается вроде тоже очев, АЕ1 $\Rightarrow ||K_h||_1 \ge 1$)
\end{enumerate}
\newpage

\subsection{Определения}
\subsubsection{Ортогональный ряд}

Ряд $\sum a_k$ --- ортогональный, если $\forall k, l a_k \perp a_l$

\subsubsection{Сходящийся ряд в гильбертовом пространстве}

$\sum a_n, a_n \in \mathfrak{H}$

$S_N := \sum_{1 \le n \le N} a_n$, если $\exists S \in \mathfrak{H}: S_N \goesto{\mathfrak{H}} S$

Такой ряд называется сходящимся.

\subsubsection{Ортогональная система (семейство) векторов}

${e_k} \subset \mathcal{H}$ --- ортогональная система, если:

\begin{enumerate}
    \item $k \neq j \dbl e_k \perp e_j$
    \item $\forall k \dbl e_k \neq 0$
\end{enumerate}

\subsubsection{Коэффициенты Фурье}

\[c_k(x) = \frac{\langle x, e_k \rangle}{||e_k||^2}\]

--- коэффициент Фурье вектора $x$ по О. С. $e_k$

\subsubsection{Ряд Фурье в Гильбертовом пространстве}

\[\sum_{k = 1}^{\infty} c_k(x) \cdot e_k\]
--- ряд Фурье вектора $x$ по О. С. $\{e_k\}$

\subsubsection{Базис, полная, замкнутая ОС}

$\{e_k\}$ --- ОС

Если $\forall x \in \mathcal{H}$, $x$ раскладывается по $\{e_k\}$ 

\[x = \sum_{k = 1}^{\infty} c_k e_k\] 

то $\{e_k\}$ --- \textbf{базис}.

Если же в ОС нечего добавить, то есть $x = \sum_{k = 1}^{\infty} c_k e_k + z$, такой что $\forall k : z \perp e_k$, то $z = 0$, то такая ОС называется \textbf{полной}.

Если в ОС выполняется уравнение замкнутости (равенство Парсеваля):

\[\forall x \in \mathcal{H} : \sum_{k = 1}^{\infty} |c_k|^2 \cdot ||e_k||^2 = ||x||^2\]

то она называется \textbf{замкнутой}.


\subsubsection{Тригонометрический ряд}

Тригонометрический полином (+ комплексная форма):

\[T_n(t) = \frac{a_0}{2} + \sum_{k = 1}^{n} a_k\cos kt + b_k \sin kt\]

\[T_n(t) = \sum_{k = -n}^{n} c_k e^{-ikt}\]

Тригонометрический ряд (+ комплексная форма):

\[\frac{a_0}{2} + \sum_{k = 1}^{\infty} a_k\cos kt + b_k \sin kt\]

\[\sum_{k \in \mathbb{Z}} c_k e^{-ikt}\]

Причём сходимость в комплексном случае обсуждается в смысле частичных сумм $T_n$, а не деленем ряда на положительную и отрицательную часть по индексам.

Комплексный вариант получается подстановкой формул:

\[\cos kt = \frac{e^{ikt} + e^{-ikt}}{2} \quad \sin t = \frac{e^{ikt} - e^{-ikt}}{2i}\]

\subsubsection{Коэффициенты Фурье функции}

\begin{enumerate}
    \item \[a_k(f) = \frac{1}{\pi} \int_{-\pi}^{\pi} f(t) \cos kt dt, \quad k = 0, 1, 2, \ldots\]
    \item \[b_k(f) = \frac{1}{\pi} \int_{-\pi}^{\pi} f(t) \sin kt dt, \quad k = 1, 2, \ldots\]
    \item \[c_k(f) = \frac{1}{2\pi} \int_{-\pi}^{\pi} f(t) e^{-ikt} dt\]
\end{enumerate}

--- коэффициенты Фурье функции $f$

\[\frac{a_0}{2} + \sum_{k = 1}^{\infty} a_k(f) \cos kt + b_k(f) \sin kt = \sum_{k \in \mathbb{Z}} c_k(f) e^{-ikt}\]

--- (тригонометрический) ряд Фурье (вещественный или комплексный)

\textit{Замечания:}

\begin{enumerate}
    \item В $L^1[0, 2\pi]$ всё тоже работает, просто сдвигаем на $\pi$ влево (?)
    \item Если $f \in L^1[-\pi, \pi]$ --- чётная, то $\forall k : b_k(f) = 0$
    
    А если нечётная, то $\forall k : a_k(f) = 0$ (по определению коффициентов и свойств чётной/нечётной функции)

    \item $f \in L^1[0, \pi]$
    
    \[f \sim \frac{a_0}{2} + \sum_{k = 1}^{\infty} a_k \cos kt\]
    \[f \sim \sum_{k = 1}^{\infty} b_k \sin kt\]

    --- это функции чётного и нечётного продолжения $f$ (?)

    \item \[A_k(f, x) = \begin{dcases}
        \frac{a_0(f)}{2}, & k = 0 \\
        a_k(f) \cos kt + b_k(f) \sin kt, & k \in \mathbb{N}
    \end{dcases}\]

    Тогда:

    \[A_k(f, x) = \begin{dcases}
        \frac{1}{2\pi} \int_{-\pi}^{\pi} f(x + t) dt, & k = 0 \\
        \frac{1}{\pi} \int_{-\pi}^{\pi} f(x + t) \cos kt dt, & k \in \mathbb{N}
    \end{dcases}\]

    (доказывается раскрытием формул, там вылазит косинус суммы и замена переменной в конце (с изменением границ интегрирования всё ОК, ведь функция периодичная))

    \item 
    
    \[|a_k(f)|, |b_k(f)|, |2c_k(f)| < \frac{1}{\pi}||f||_1\]

    \[A(f, t) \le \begin{dcases}
        \frac{1}{2\pi}||f||_1, & k = 0 \\
        \frac{1}{\pi} ||f||_1, & k \in \mathbb{N}
    \end{dcases}\]

    (доказывается очевидной оценкой интеграла сверху, тригонометрические функции сверху единицей)

    \item Историческая справка:
    \begin{enumerate}
        \item $f \in \tilde{C}[-\pi, \pi]$, такая что $\exists x_0$ что ряд Фурье в ней расходится (?)
        \item Колмогоров придумал $f \in L^1$, в каждой точке ряд Фурье расходится
        \item Карлесон: $f \in L^2$ --- ряд Фурье сходится почти везде
        \item Хант: $f \in L^p, p > 1$ --- ряд Фурье сходится почти везде
    \end{enumerate}
\end{enumerate}

\subsubsection{Модуль непрерывности}

\[\omega(f, h) := \sup_{x, y \in E,  |x - y| < h} |f(x) - f(y)|\]

--- модуль непрерывности. Причём если:

\[\omega(f, h) \goesto{h \rightarrow 0} 0\]

--- тогда $f$ --- равномерно непрерывная (просто по определению)

\subsubsection{Класс Липшица с константой M и показателем альфа}

\[E = \langle a, b \rangle\]

\[\text{Lip}_M^\alpha(E) = \{f: E \rightarrow \mathbb{R}: \forall x, y \in E \quad |f(x) - f(y)| < M|x - y|^\alpha\}\] 


\textit{Замечания:}

\begin{enumerate}
    \item $f$ --- дифференцируема и $|f'| \le M \dbl \Rightarrow f \in$ Lip$_M^1$.
    
    (очевидно по теореме... Лагранжа!)

    \item $f \in $ Lip$_M^\alpha$, тогда $\omega(f, h) \le M |h|^\alpha$ (очевидно)
    \item $f \in $ Lip$_M^\alpha$, $\alpha > 1$, тогда $f = \const$ (загадка, без комментариев)
\end{enumerate}

\subsubsection{Ядро Дирихле, ядро Фейера}

\[\DD_n(t) = \frac{1}{\pi}\left(\frac{1}{2}t + \sum_{k = 1}^{n} \cos kt\right)\]
--- ядро Дирихле

\[\Phi_n(t) = \frac{\DD_0 + \DD_1 + \ldots + \DD_n}{n + 1}\]

-- ядро Фейера

Мини-лемма про свойства:

\begin{enumerate}
    \item  \[\DD_n = \frac{\sin \left(n + \frac{1}{2}\right)t}{2\pi \sin \frac{t}{2}}\]
    \item  \[\Phi_n(t) = \frac{1}{2\pi(n + 1)}\cdot \frac{\sin^2\left(\frac{n + 1}{2}\right)t}{\sin^2 \frac{t}{2}}\]
    \item $\DD_n, \Phi_n$ --- чётные, $\Phi_n \ge 0, \int_{-\pi}^\pi \DD_n = \int_{-\pi}^\pi \Phi_n = 1$
    \item $f \in L^1[-\pi. \pi] \Rightarrow S_n(f, x) = \int_{-\pi}^{\pi} f(x + t) \DD_n(t) dt$
\end{enumerate}

\textit{Доказательство:}

\begin{enumerate}
    \item Трюк из доказательства интеграла Дирихле: приравниваем две формулы, отсекаем общие части, а потом оригинальное ядро (где сумма косинусов) домножаем на $\sin \frac{t}{2}$. (расписывать у меня сил не хватит, сорре :( Но там вроде просто всё))
    \item См. пункт 1
    \item Очевидно. Чётные, потому что состоят из чётных косинусов. Интеграл тривиально получается подстановкой.
    \item Юзаем $A_k(f, x) = \frac{1}{\pi} \int_{-\pi}^{\pi} f(x + t) \cos kt dt$, в сумме оно и получается.
\end{enumerate}

ч. т. д. 

\subsubsection{Усиленная аппроксимативная единица}

Аксиома 3':

\[K_h \in L^{\infty}[-\pi, \pi] \quad \esssup_{E_\delta} K_h \goesto{h \rightarrow h_0} 0\]

\textit{Замечания: }

\begin{enumerate}
    \item АЕ3' $\Rightarrow$ АЕ3 (вроде очев)
    \item Система функций, которая удовлетворяет АЕ1, АЕ2 и АЕ3', называется \textbf{усиленной аппроксимативной единицей}
\end{enumerate}

\subsubsection{Метод суммирования средними арифметическими}

\[S_n = \sum_{k = 1}^{n} a_k\]
\[\sigma_n = \frac{S_0 + S_1 + \ldots + S_n}{n + 1}\]

Если $\exists$ конечный $\lim_{n \rightarrow \infty} \sigma_n = S$, то говорят, что ряд $\sum a_k$ сходится к $S$ в смысле средних арифметических (СА, по Чезаро).

Теорема Коши о перманентности метода СА гласит, что если ряд сходится (в обычном смысле) к $S$, то в смысле СА он сходится к тому же $S$.

\subsubsection{Суммы Фейера}

\[\sigma_n(f, x) = \frac{S_0(f, x) + S_1(f, x) + \ldots + S_n(f, x)}{n + 1}\]

--- суммы Фейера

\[\sigma_n(f, x) = \int_{-\pi}^\pi f(x + t) \Phi_n(t) dt = [t := -t] = \int_{-\pi}^{\pi} f(x - t)\Phi_n(t)dt\]

Можно сделать замену $t := -t$ и по чётности ядра Фейера получить такую же формулу, только пригодную для использования в качестве свёртки (границы интегрирования просто перевернулись).

\subsubsection{Преобразование Фурье}

\[f \in L^1(\mathbb{R}^m), \quad \hat{f}(y) = \int_{\mathbb{R}^m} f(x)e^{-2\pi i\sk{y}{x}}dx, \quad y \in \mathbb{R}\]

Также для него верна теорема Римана-Лебега (более общий случай):

\[X \subset \mathbb{R}^m, \dbl f \in L^1(X, \lambda_m), \dbl I_f(y) = \int_{X} f(x)e^{-2\pi i \sk{y}{x}} dx\]

Тогда $I_f(y) \goesto{|y| \rightarrow \infty} 0$. То есть, $\hat{f}(y) \goesto{|y| \rightarrow \infty}$

\textit{Примеры: }

\begin{enumerate}
    \item $m = 1$
    
    $\chi = \chi[-1, 1]$

    \[\hat{\chi}(y) = \int_{-\infty}^{\infty} \chi(x)e^{-2\pi i yx} dx = \int_{-1}^{1} e^{-2 \pi i y x} dy = \left(\frac{e^{-2\pi i y x}}{-2\pi i x}\right)_{-1}^{1} = \frac{\sin 2\pi x}{\pi x}\]

    \item $f = e^{-\pi a^2 x^2}$, $x \in \mathbb{R}$
    
    Тогда:

    \[\widehat{f_a}(y) = \frac{1}{a}f_{\frac{1}{a}}(a) = \frac{1}{a}e^{\frac{-\pi y^2}{a^2}}\] 

    Доказательство было на лекции, но можно пж не надо...
\end{enumerate}
\subsubsection{Свертка в $L^1(\mathbb R^m)$}

\[f, K \in L^1(\mathbb{R}^m) \quad (f * K)(x) = \int_{\mathbb{R}^m} f(x - t)K(t) dt\]

--- лемма о корректности доказывается точно так же, как и для случая $L^1[-\pi, \pi]$.

\subsubsection{Интеграл Фурье, частичный интеграл Фурье}

$m = 1$

\[I_A(f, x) = \int_{-A}^{A} \hat{f}(y) e^{2 \pi i y x} dy\]

--- частичный интеграл Фурье

\[I(f, x) = \lim_{A \rightarrow \infty} I_A(x)\]

--- интеграл Фурье

\newpage

\subsection{Важные теоремы}

\subsubsection{Признак Дини. Следствия}
\textit{Формулировка:}

\begin{itemize}
    \item $f \in L^1[-\pi, \pi]$
    \item $x_0 \in \mathbb{R}$
    \item $S \in \mathbb{R}$ (или $\mathbb{C}$)
    \item Выполнено:
    

    \[\int_{0}^{\pi} \frac{|f(x_0 + t) - 2S + f(x_0 - t)|}{t}dt < +\infty\]
\end{itemize}

Тогда:

\[S_n(f, x_0) \ntoinf S\]

\textit{Доказательство:}

Пусть $\varphi(t) := f(x_0 + t) - 2S + f(x_0 - t)$. Тогда нам дан: $\int_0^\infty \frac{|\varphi(t)|}{t}dt < +\infty$. Сравним $S_n$ и $S$ (вспоминаем, что $\int_{-\pi}^{\pi} \DD_n(t) dt = 1$ по лемме из определения ядра Дирихле):

\[(S_n(f, x_0) - S) = \int_{-\pi}^{\pi} f(x_0 + t)\DD_n(t) dt - S\int_{-\pi}^{\pi} \DD_n(t) dt = \int_{-\pi}^{\pi} (f(x_0 + t) - S)\DD_n(t) dt = \int_{-\pi}^{0} + \int_{0}^{\pi} = \]

В первом интеграле делаем замену $t := -t$, тогда внутри $f$ поменяется знак, а ядро Дирихле останется таким же, ведь оно чётное! (почему минус из дифференциала не вылез? Сократился с границами интегрирования?)

\[ = \int_{0}^{\pi}(f(x_0 - t) - S)\DD_n(t) dt + \int_{0}^{\pi} (f(x_0 + t) - S)\DD_n(t) dt = \]

\[ = \int_{0}^{\pi}(f(x_0 + t) - 2S + f(x_0 - t))\DD_n(t) dt = \int_{0}^{\pi} \varphi(t)\DD_n(t)dt = \]

И применяем (*) из принципа локализации Римана:
\[ = \frac{1}{2\pi} \int_{0}^{\pi} \varphi(t) \sin kt \ctg \frac{t}{2} + \varphi(t) \cos kt dt = \]

Чтобы свести эти штуки к коэффициентам Фурье, надо, чтобы границы интегрирования были $[-\pi, \pi]$. Давайте заведём вспомогательные функции:

\[h_1(t) = \begin{dcases}
    \varphi(t)\sin kt \ctg \frac{t}{2}, &t \in [0, \pi] \\
    0, &t \in[-\pi, 0)
\end{dcases}, h_2(t) = \begin{dcases}
    \varphi(t)\cos kt, &t \in [0, \pi] \\
    0, &t \in[-\pi, 0)
\end{dcases}\]

Сразу нужно проверить их принадлежность к $L^1[-\pi, \pi]$, чтобы считать от них коэффициенты. И если $h_2$ лежит в этом классе очевидным образом, то в $h_1$ у нас опять котангенс, который неограничен в нуле. Но тут нам на помощь приходит древнее знание, что $\tg x > x, x \in (0, \frac{\pi}{2})$ (доказывается визуализацией на единичной окружности, и там треугольник будет больше $x$). Ну вот, а значит: $\ctg x < \frac{1}{x}$. Оцениваем $h_1$ на $t \in [0, \pi]$ (на остальном он и так 0):

\[|h_1| = |\varphi(t)\ctg \frac{t}{2}| \le \frac{|\varphi(t)|}{|2t|}\]

Спрашивается, а суммируемая ли последняя функция ($L^1$ есть суммируемость)? Да конечно, в условии же написано, интгерал конечен! Ну и всё, тогда по теореме Римана-Лебега:

\[ = b_n(h_1) + a_n(h_2) \ntoinf 0\]

ч. т. д. 

\textit{Замечание:}

Вместо интеграла по всему промежутку можно было взять $\int_0^\delta \frac{|\varphi(t)|}{t} dt < + \infty, \forall \delta > 0$

\textit{Следствия:}

\begin{enumerate}
    \item Условия те же самые, только вместо интеграла существуют 4 конечных предела:
    \[f(x_0 - 0), f(x_0 + 0), \alpha_+ = \lim_{h \rightarrow 0 + 0} \frac{f(x_0 + h) - f(x_0 + 0)}{t}, \alpha_- = \lim_{h \rightarrow 0 - 0} \frac{f(x_0 + h) - f(x_0 - 0)}{t}\]

    Тогда всё тоже сработает, и ряд сойдётся к:

    \[S = \frac{1}{2}(f(x_0 + 0) - f(x_0 - 0))\]

    \textit{Доказательство: }


    Нам дали $S$, проверяем $\varphi$:

    \[\varphi(t) = f(x_0 + t) - 2S + f(x_0 - t)\]

    \[\frac{\varphi(t)}{t} = \frac{f(x_0 + t) - f(x_0 - 0)}{t} - \frac{f(x_0 + t) - f(x_0 - 0)}{-t} \goesto{t \rightarrow 0} \alpha_+ - \alpha_-\]

    Таким образом, у нас подынтегральное выражение из интеграла в теореме конечно, значит сам интеграл тоже конечен, ну и всё, запускаем теорему и всё получается!

    ч. т. д.

    Этим следствием мы обслужили точки разрыва 1го рода, когда у нас происходит скачок. В таком случае, ряд Фурье сходится к среднему значению.

    \images{0.3}{priznak_dini.jpg}

    \item Условия всё те же, только вместо 4х пределов существуют конечные односторонние прозводные $f'_-(x_0)$ и $f'_+(x_0)$, тогда всё получается и сходится.
    
    \textit{Доказательство:}

    Очевидно вытекает из предыдущего.

    ч. т. д.
\end{enumerate}

\newpage
\subsection{Теоремы}

\subsubsection{Теорема о свойствах частичных сумм ряда Фурье. Неравенство Бесселя}
\textit{Формулировка:}

\begin{itemize}
    \item $x \in \mathcal{H}, \{e_k\}$ --- ОС
    \item $n \in \mathbb{N}$
    \item $S_n = \sum_{i = 1}^{n} c_i(x)e_i$
    \item $\mathcal{L}_n = \text{Lin}(e_1, e_2, \ldots, e_n)$ --- линейная оболочка
\end{itemize}

Тогда:

\begin{enumerate}
    \item $S_n$ --- проекция $x$ на $\mathcal{L}$:
    
    \[x = S_n + z, \quad z \perp \mathcal{L}\]

    \images{0.3}{ner_bessel.jpg}

    Получается, у нас есть пространство поверх ОС, и $S_n$ это проекция (компонента внутри $\mathcal{L}_n$), а $z$ не лежит в этом пространстве и как-бы отправляет их проекции в сам $x$, своего рода перпендикуляр.

    \item $||x - S_n|| = \min_{y \in \mathcal{L}} ||x - y||$ --- лучшее приближение $x$ в $\mathcal{L}_n$
    \item $||S_n|| \le ||x||$
\end{enumerate}

\textit{Доказательство:}

\textbf{1. }

Проверим то, что $\forall j: z \perp e_j$: 

\[\sk{z}{e_j} = \sk{x - S_n}{e_j} = \sk{x}{e_j} - \sk{S_n}{e_j} = \sk{x}{e_j} - \sk{\sum_{i = 1}^{n} c_i}{e_i} =\]

\[= \sk{x}{e_j} - c_j ||e_j||^2 = 0\]

Там в сумме уничтожились почти все слагаемые, кроме $e_j$ (по ортогональности), а в конце просто получилось определение $c_j$.

\textbf{2. }

\[(x - y) = \underbrace{(S_n - y)}_{(1)}+ \underbrace{z}_{(2)}\]

(1) $\in \mathcal{L}_n$, а $(2) \perp \mathcal{L}_n$, поэтому валидна теорема Пифагора:

\[||x - y||^2 = ||S_n - y||^2 + ||z||^2 \ge ||z||^2 = ||S_n - x||^2\]

Финт ушами! Минимум как раз и достигается в $S_n$.

\textbf{3. }

Продолжаем 2й пункт:

\[||x||^2 = ||S_n||^2 + ||z||^2 \ge ||S_n||^2\]

Нормы положительные, поэтому с квадратами всё работает.

ч. т. д. 

\textit{Следствие (неравенство Бесселя): }

\[\sum_{i = 1}^{\infty} |c_i(x)|^2 \cdot ||e_i||^2 \le ||x||^2\]

Фактически это пункт 3, только счётная сумма, но всё работает, так как для каждого $n$ работает теорема.

\subsubsection{Теорема Рисса -- Фишера о сумме ряда Фурье. Равенство Парсеваля}
\textit{Формулировка:}

\begin{itemize}
    \item $x \in \mathcal{H}, \{e_k\}$ --- ОС
\end{itemize}

Тогда:

\begin{enumerate}
    \item Ряд Фурье вектора $x$ сходится
    \item \[x = \sum_{k = 1}^{\infty} c_k(x)e_k + z, \quad \forall k: z \perp e_k\]
    \item \[x = \sum_{k = 1}^{\infty} c_k(x)e_k \Leftrightarrow ||x||^2 = \sum_{k = 1}^{\infty} |c_k|^2 \cdot ||e_k||^2\]
    
    Это равенство назывется \textbf{равенством Парсеваля} или \textbf{уравнением замкнутости}.
\end{enumerate}

\textit{Доказательство:}

\textbf{1.}

Вспоминаем одну из первых теорем, там был пункт, что сходимость ортогонального ряда эквивалентна сходимости ряд квадрата его норм. Ряд Фурье как раз ортогональный, поэтому навешиваем на каждое слагаемое норму, возводим в квадрат:

\[\sum_{k = 1}^{\infty} |c_k|^2 \cdot ||e_k||^2 \le\]

И... оно меньше:

\[\le ||x||^2\] 

по неравенству Бесселя! Значит, такой ряд сходится, ну тогда и просто ряд Фурье сходится.

\textbf{2.}

Опять проверяем! Домножим $z$ скалярно на какой-нибудь $e_j$:

\[\sk{z}{e_j} = \sk{x - \sum_{k = 1}^{\infty} c_k e_k}{e_j} = \sk{x}{e_j} - \sk{\sum_{k = 1}^{\infty} c_k e_k}{e_j} =\]

Счётная сумма раскрывается по пункту одной из первых теорем:

\[= \sk{x}{e_j} - \sum_{k = 1}^{\infty} c_k \sk{e_k}{e_j} = \sk{x}{e_j} - c_j ||e_j||^2 = 0\]

Опять всё посокращалось, кроме одного $e_j$ и вылезло определение $c_j$.

\textbf{3. }

$\Rightarrow$

Всё то же следствие про эквивалентность сходимости ряда Фурье и ряда из его норм в квадрате.

$\Leftarrow$

Нам дано то, что справа. Ну дак давайте применим теорему Пифагора к левому:

\[||x||^2 = \left|\left| \sum_{k = 1}^{\infty} c_k e_k\right|\right|^2 + ||z||^2 = \sum_{k = 1}^{\infty} |c_k|^2||e_k||^2 + ||z||^2\]

Внутри суммы всё тоже друг другу ортоганально, поэтому применение теоремы Пифагора ещё раз валидно! Ну и всё, справа нам дано, слева это. Следовательно, $||z||^2 = 0$.

ч. т. д. 

\subsubsection{Теорема о характеристике базиса}
\textit{Формулировка:}

\begin{itemize}
    \item $\{e_k\}$ --- ОС
\end{itemize}

Тогда следующие утверждения эквивалентны:

\begin{enumerate}
    \item $\{e_k\}$ --- базис
    \item \[\forall x, y \in \mathcal{H} \quad \sk{x}{y} = \sum_{k = 1}^{\infty} c_k(x) \overline{c_k(y)} ||e_k||^2\]
    \item $\{e_k\}$ --- замкнутая
    \item $\{e_k\}$ --- полная
    \item $\mathcal{L} = \text{Lin}(e_1, e_2, \ldots)$ плотно в $\mathcal{H}$, то есть Cl$(\mathcal{L}) = \mathcal{H}$ (ну и по определению плотности)
\end{enumerate}

\textit{Доказательство:}

\textbf{1 $\Rightarrow$ 2}

Раз это базис, раскладываем вектора:

\[x = \sum_{k = 1}^{\infty} c_k(x)e_k, \quad y = \sum_{k = 1}^{\infty} c_k(y)e_k\]

Чтобы поддержать комплекснозначность разовьём вот эту тему с сопряжением при скалярном произведении (в сумме сокращается как обычно):

\[\sk{e_k}{y} = \overline{\sk{y}{e_k}} = \overline{\sk{\sum_{i = 1}^{\infty} c_i(y)e_i}{e_k}} = \sum_{i = 1}^{\infty}\overline{c_i(y)}\sk{\overline{e_i}}{\overline{e_k}} = \overline{c_k(y)}||e_k||^2\]

Ну и то, что нужно:

\[\sk{x}{y} = \sk{\sum_{k = 1}^{\infty} c_k(x)e_k}{y} = \sum_{k = 1}^{\infty} c_k(x)\sk{e_k}{y} = \sum_{k = 1}^{\infty} c_k(x) \overline{c_k(y)}||e_k||^2\]

\textbf{2 $\Rightarrow$ 3}

Тривиально, подставляем в формулу $x = y$ и получаем равенство Парсеваля

\textbf{3 $\Rightarrow$ 4}

Экстравагантный способ, хотя вроде почти то же самое. Ну, пусть у нас таки завёлся такой вектор $z$, $\forall k : z \perp e_k$. По определению, $\forall k: c_k(z) = 0$ (там внутри он скадярно множится на базисный вектор). Распишем равенство Парсеваля (оно нам гарантируется) для $z$!

\[||z||^2 = \sum_{k = 1}^{\infty} |c_k(z)|^2 ||e_k||^2 = 0\]

А $z$ оказывается есть просто 0! Всё, базис полный по определению (если и нашёлся, то он нулевой).

\textbf{4 $\Rightarrow$ 1}

По теореме Рисса-Фишера, любой $x$ раскладывается как:

\[x = \sum_{k = 1}^{\infty} c_k e_k + z, \quad \forall k: z \perp e_k\]

Ну а по полноте $z = 0$, по этому по определению $\{e_k\}$ --- базис.


Такс, всё, закольцевали первые 4. Теперь докажем эквивалентность 4 и 5.

\textbf{4 $\Rightarrow$ 5}

Пусть это не так. Тогда возьмём $x \in \mathcal{H} \setminus $ Cl($\mathcal{L}$). Раз он не лежит в замыкании линейной оболочки, он построен с привлечением ещё какой-то размерности (ещё какого-то вектора $z, \forall k : z \perp e_k$). Но по полноте такой $z$ есть 0, поэтому $x \notin \mathcal{H} \setminus $ Cl($\mathcal{L}$). Противоречие.

\textbf{5 $\Rightarrow$ 4}

Давайте попробуем сочинить такой $x_0$, что он $\forall k: x_0 \perp e_k$. Ну, раз он таков, то он $x_0 \perp \mathcal{L}$, и, следовательно, $x_0 \perp$ Cl$(\mathcal{L})$. Ну и что же это за вектор такой, который всему ортагонален, в том числе и всему пространству Cl$(\mathcal{L}) = \mathcal{H}$ (по условию 5)? $x_0 \perp x_0 \Rightarrow \sk{x_0}{x_0} = 0 \Rightarrow x_0 = 0$ (по аксиоме скалярного произведения).

ч. т. д. 

\newpage


\subsubsection{Теорема о свойствах сходимости в гильбертовом пространстве}
\textit{Формулировка:}

\begin{itemize}
    \item $x_n \rightarrow x_0, y_n \rightarrow y_0 \quad \langle x_n, y_n \rangle \rightarrow \langle x_0, y_0 \rangle$
    \item $\sum_n x_n$ --- сходится
    
    Тогда $\forall y \in \mathcal{H} \quad \langle \sum x_n , y\rangle = \sum \langle x_n, y \rangle$

    \item $\sum x_n$ --- ортогональный ряд
    
    Тогда $\sum x_n$ --- сходится $\Leftrightarrow \sum ||x_n||^2$ --- сходится
\end{itemize}

\textit{Доказательство:}

\textbf{1.}

Как в 1м семестре по неравенству треугольника:

\[|\sk{x_n}{y_n} -\sk{x_0}{y_0}| \le |\sk{x_n}{y_n} - \sk{x_n}{y_0}| + |\sk{x_n}{y_0} - \sk{x_0}{y_0}| \le |\sk{x_n}{y_n - y_0}| + |\sk{x_n - x_0}{y_0}| \le \]

КБШ:

\[ \le \underbrace{||x_n||}_{\text{огр, т. к. сходится (?)}} \cdot \underbrace{||y_n - y_0||}_{\rightarrow 0} + \underbrace{||x_n - x_0||}_{\rightarrow 0} \cdot \underbrace{||y_0||}_{\const} \ntoinf 0\]

\textbf{2.}

$S_n = \sum_n x_k \ntoinf S$ (раз сходится, по условию). Посмотрим на то, что пытаемся доказать (последнее по пункту 1):

\[\sum_n \sk{x_k}{y} = \sk{\sum_n x_k}{y} = \sk{S_n}{y} \ntoinf \sk{S}{y}\]

\textbf{3.}

Посмотрим на норму частичных сумм (у нас именно частичные суммы, поэтому при перемножении никаких проблем не будет):

\[||S_n||^2 = \sk{\sum_{i = 1}^n x_i}{\sum_{k = 1}^n x_k} = \sum_{i, k = 1}^n \sk{x_i}{x_k} =\]

Потом вспоминаем, что ряд то у нас был ортоганальный, значит в сумме останутся только элементы с одинаковыми индексами:

\[=\sum_{k = 1}^{n} ||x_k||^2 =: C_n\]

Обозначим такую сумму как $C_n$. Аналогично получается, что:

\[||S_n - S_m||^2 = |C_n - C_m|\]

А из этого получается, что $(S_n)$ и $(C_n)$ --- фундаментальны одновременно. Так как мы работаем в гильбертовом пространстве, оно полное, и каждая фундаментальная последовательность должна иметь предел. Ну и в нашей ситуации, если одна из сторон равенства меньше китайского $\varepsilon$ (может корня, суть не меняется), то и вторая автоматически тоже. И если одна сходится, то автоматически сходится и другая.

ч. т. д. 

\subsubsection{Теорема о коэффициентах разложения по ортогональной системе}
\textit{Формулировка:}

\begin{itemize}
    \item ${e_k}$ --- ортогональная система в $\mathcal{H}$
    \item $x \in \mathcal{H}$
    \item $x = \sum_{k = 1}^{\infty} c_k e_k$
\end{itemize}

Тогда:

\begin{enumerate}
    \item ${e_k}$ --- ЛНЗ
    \item $c_k = \frac{\langle x, e_k \rangle}{||e_k||^2}$
    \item $c_ke_k$ --- это проекция на прямую $l_k = {te_k, t \in \mathbb{R}}, x = c_ke_k + z, $ где $z \perp e_k$
\end{enumerate}

\textit{Доказательство:}

\textbf{1.}

Ну, а пусть он ЛЗ. Так как это всё дело в терминологии ЛинАла, следует рассматривать что-то конечное. Ну давайте, не умаляя общности, предположим, что случился ЛЗ набор среди $N$ первых векторов: $\sum_N \alpha_k e_k = 0$. Домножим тогда разложение на какой-нибудь $j$-й вектор из ОС:

\[0 = \sk{e_j}{\sum_{k = 1}^{N} \alpha_k e_k} = \alpha_k ||e_k||^2\]

Ну тогда либо коэффициент был равен нулю, либо норма вектора. Оба этих случая забанены при ЛНЗ.

\textbf{2.}

Ну просто посчитаем:

\[\sk{x}{e_k} = \sk{\sum c_k e_k}{e_k} = c_k ||e_k||^2\]
\[c_k = \frac{\sk{x}{e_k}}{||e_k||^2}\]

Очев)

\textbf{3.}

\images{0.3}{ort_sys.jpg}

Тут на картинке показано, что это за проекция и как оно вообще работает. А нам достаточно проверить лишь ортогональность $z$ и $l_k$:

\[\sk{z}{e_k} = \sk{x}{e_k} - \sk{c_ke_k}{e_k} = c_k ||x_k||^2 - c_k ||x_k||^2 = 0\]

ч. т. д. 

\textit{Замечание:}

При перегонке ОС в ОНС (деление на норму), ряд Фурье не меняется. Так как раз поменялись вектора из ОС, то и коэффициенты Фурье поменялись (типа, базисный уменьшили в 2 раза, коэффициент увеличился в 2 раза).

\subsubsection{Лемма о вычислении коэффициентов тригонометрического ряда}
\textit{Формулировка:}

\begin{itemize}
    \item тригонометрический ряд с частичными суммами $S_n$ (вещественный или комплексный)
    \item $f \in L^1[-\pi, \pi]$
    \item $S_n \rightarrow f$ в смысле $L^1: ||S_n - f||_1 \ntoinf 0$
\end{itemize}

Тогда:

\begin{enumerate}
    \item \[a_k(f) = \frac{1}{\pi} \int_{-\pi}^{\pi} f(t) \cos kt dt, \quad k = 0, 1, 2, \ldots\]
    \item \[b_k(f) = \frac{1}{\pi} \int_{-\pi}^{\pi} f(t) \sin kt dt, \quad k = 1, 2, \ldots\]
    \item \[c_k(f) = \frac{1}{2\pi} \int_{-\pi}^{\pi} f(t) e^{-ikt} dt\]
\end{enumerate}

\textit{Доказательство:}

Докажем только (1), остальное аналогично. Посчитаем следующую штуку ($n \ge k$):

\[\int_{-\pi}^{\pi} S_n(t) \cos kt dt = \]

Вспомним, что $\{1, \cos kt, \sin kt\}_{k = 1}^{\infty}$ задаёт ортогональную систему! Это значит, что все произведения этих векторов, кроме одинаковых, дают 0! А $S_n = \frac{a_0}{2} + \sum_{k = 1}^{n} a_k \cos kt + b_k \sin kt$, значит она даст везде нуля кроме косинуса с индексом $k = j$!

\[ = \int_{-\pi}^{\pi} a_k \cos^2 kt dt = a_k \pi\]

Офигеть, так это почти то, что мы искали!

\[a_k = \frac{1}{\pi} \int_{-\pi}^{\pi} S_n(t) \cos kt dt\]

Посмотрим на разницу:

\[\left|\frac{1}{\pi} \left( \int_{-\pi}^{\pi} (S_n - f) \cos kt dt \right)\right| \le \int_{-\pi}^{\pi} |S_n - f| dt = ||S_n - f||_1 \ntoinf 0\]

А она стремится к 0 по условию.

ч. т. д. 

\subsubsection{Теорема Римана--Лебега}
\textit{Формулировка:}

\begin{itemize}
    \item $E \subset \mathbb{R}$
    \item $f \in L^1(E)$
\end{itemize}

Тогда:

\begin{enumerate}
    \item \[\int_{E} f(t) \cos \lambda t dt \goesto{\lambda \rightarrow \infty} 0\]
    \item \[\int_{E} f(t) \sin \lambda t dt \goesto{\lambda \rightarrow \infty} 0\]
    \item \[\int_{E} f(t) e^{i \lambda t} dt \goesto{\lambda \rightarrow \infty} 0\]
\end{enumerate}

Причём бесконечность может быть в любую сторону.

\textit{Доказательство:}

Будем доказывать случай (3), остальные выводятся из него. Также давайте сразу достроим нашу функцию до $\mathbb{R}$, тогда на $\mathbb{R} \setminus E$ функция будет принимать значение 0, что позволяет нам бесплатно интегрировать по всему $\mathbb{R}$.

\[\int_{E} = \int_{\mathbb{R}} f(t) e^{i \lambda t} dt = \begin{bmatrix}
    t = \tau + \frac{\pi}{\lambda}\\
    dt = d\tau
\end{bmatrix} = \int_{\mathbb{R}} f\left(\tau + \frac{\pi}{\lambda}\right) e^{i\lambda \tau + \pi i} d\tau =\]

Заметим, что $e^{\pi i } = -1$ (тождество Эйлера, ну или формула Эйлера):

\[= -\int_{\mathbb{R}} f\left(\tau + \frac{\pi}{\lambda}\right) e^{i\lambda \tau } d\tau\]

Ну а вот теперь финт ушами, оценим исходный интеграл как сумму двух половинок (оригинального и преобразованного интеграла, причём во втором ``переназываем'' $t := \tau$):

\[\int_{\mathbb{R}} f(t) e^{i \lambda t} dt = \frac{1}{2} \left(\int_{\mathbb{R}} f(t) e^{i \lambda t} dt - \int_{\mathbb{R}} f\left(t + \frac{\pi}{\lambda}\right)e^{i \lambda t} dt\right) = \frac{1}{2} \left(\int_{\mathbb{R}} e^{i \lambda t} \left(f(t) -  f\left(t + \frac{\pi}{\lambda}\right)\right) dt\right) \le\] 

\[ \le \int_{\mathbb{R}} |e^{i \lambda t}| \cdot \left|f(t) -  f\left(t + \frac{\pi}{\lambda}\right)\right| dt \le 1 \cdot \left|\left|f(t) - f\left(t + \frac{\pi}{\lambda}\right)\right|\right|_1 \goesto{\lambda \rightarrow \infty} 0 \]

Экспоненту оценили сверху единицей, оставшийся интеграл --- это просто норма разности в $L^1(\mathbb{R})$. А последний переход откуда? Так это же теорема о непрерывности сдвига! И правда, сдвиг стремится к 0, значит и норма тоже.

ч. т. д. 

\subsubsection{Следствия об оценке коэффициентов Фурье}
\textit{Формулировка (следствие 1):}

\begin{itemize}
    \item $f \in \tilde{C}[-\pi, \pi]$
\end{itemize}

Тогда $\forall k \neq 0: |a_k(f)|, |b_k(f)|, |2c_k(f)| \le \omega(f, \frac{\pi}{k})$

\textit{Доказательство:}

Оценим аналогично $2c_{-k}(f)$ (с минусом, чтобы была положительная степень):

\[|2c_{-k}(f)| = \frac{1}{\pi} \int_{-\pi}^{\pi} f(t) e^{ikt} dt = \frac{1}{2\pi} \int_{-\pi}^{\pi}e^{ikt}\left(f(t) - f(t + \frac{\pi}{k})\right) dt \le \frac{1}{2\pi} \cdot 2\pi \cdot \omega(f, \frac{\pi}{2})\]

(оценили интеграл сверху супремумом на меру множества)

ч. т. д. 

\textit{Формулировка (следствие 2):}

\begin{itemize}
    \item $f \in$ Lip$_M^\alpha[-\pi, \pi]$
\end{itemize}

Тогда $\forall k \neq 0: |a_k(f)|, |b_k(f)|, |2c_k(f)| \le \frac{M\pi^\alpha}{|k|^\alpha}$

\textit{Доказательство:}

Да то же самое всё:

\[|2c_{-k}(f)| = \frac{1}{\pi} \int_{-\pi}^{\pi} f(t) e^{ikt} dt = \frac{1}{2\pi} \int_{-\pi}^{\pi}e^{ikt}\left(f(t) - f(t + \frac{\pi}{k})\right) dt \le \frac{1}{2\pi} \cdot 2\pi \cdot \frac{M \pi^\alpha}{|k|^\alpha}\]

(оценили интеграл сверху супремумом на меру множества)

ч. т. д. 

\textit{Формулировка (следствие 3):}

\begin{itemize}
    \item $f \in C^1$ (видимо $\tilde{C}^1[-\pi, \pi]$)
\end{itemize}

Тогда $\forall k \neq 0$:

\begin{enumerate}
    \item \[a_k(f') = kb_k(f)\]
    \item \[b_k(f') = -ka_k(f)\]
    \item \[c_k(f') = ikc_k(f)\]
\end{enumerate}

\textit{Доказательство: }

Докажем (2) интегрированием по частям:

\[b_k(f') = \frac{1}{\pi}\int_{-\pi}^{\pi} f'(t) \sin kt dt = \underbrace{\frac{1}{\pi}f(t)\sin kt|_{-\pi}^{\pi}}_{0} - \frac{k}{\pi} \int_{-\pi}^{\pi} f(t) \cos kt dt = -ka_k(f) \]

Остальное аналогично.

ч. т. д. 

\textit{Следствие из следствия:}

\begin{enumerate}
    \item $f \in \tilde{C}^r[-\pi, \pi]$, тогда коэффициенты оцениваются сверху $\frac{\const}{|k|^r}$ (ну типа запускаем подсчёт коэффициентов по производным, пока можем).
    \item $f \in \tilde{C}^r[-\pi, \pi], f^{(r)} \in$ Lip$_m^\alpha, 0 < \alpha \le 1$, тогда коэффициенты оцениваются сверху $\frac{M \pi^\alpha}{|k|^{(r + \alpha)}}$
\end{enumerate}

\subsubsection{Принцип локализации Римана}
\textit{Формулировка:}

\begin{itemize}
    \item $f, g \in L^1[-\pi, \pi]$
    \item $\exists \delta > 0$
    \item $x_0 \in [-\pi, \pi]$
    \item $f \equiv g$ на почти всех $x \in (x_0 - \delta, x_0 + \delta)$
\end{itemize}

Тогда $S_n(f, x_0)$ и $S_n(g, x_0)$ ведут себя одинаково (расходятся или сходятся к одному и тому же).

\textit{Доказательство:}

Переформулируем и докажем:

\[h := f - g, \quad h \equiv 0 \text{ при п. в. } x \in (x_0 - \delta, x_0 + \delta), \quad S_n(h, x_0) \ntoinf 0\]

Воспользуемся подсчётом с ядром Дирихле (сверху раскрываем синус суммы):

\[\DD_n = \frac{\sin \left( n + \frac{1}{2}\right)t}{2\pi\sin \frac{t}{2}} = \frac{\sin nt \cos \frac{t}{2} + \sin\frac{t}{2}\cos nt}{2\pi\sin \frac{t}{2}} = \frac{1}{2\pi}(\sin nt \ctg \frac{t}{2} + \cos nt) \quad (*)\]

Ну и запускаем подсчёт суммы:

\[S_n(h, x_0) = \int_{-\pi}^{\pi} h(x_0 + t) \DD_n(t) dt = \frac{1}{2\pi}\int_{-\pi}^{\pi}\underbrace{h(x_0 + t) \ctg \left(\frac{t}{2}\right)}_{h_1(t)}\sin nt + \underbrace{h(x_0 + t)}_{h_2(t)}\cos nt = \]

\[= b_n(h_1(t)) + a_n(h_2(t)) \ntoinf 0\]

Стремление к нулю получается по теореме Римана-Лебега. Однако, это не всё. Чтобы мы могли засунуть в коэффициенты $a_k$ и $b_k$ функции, необходимо чтобы они были из $L^1[-\pi, \pi]$. $h_2$ очевидно оттуда, так как она есть разность двух суммируемых функций. А вот с $h_1$ могут быть проблемы из-за котангенса. Или не могут быть? На самом деле не могут, так как на $(x_0 - \delta, x_0 + \delta)$ $h$ есть тождественный ноль, поэтому бесконечная огромность котангенса нам не мешает. В свою очередь, за пределами это окрестности тоже всё гуд, т. к. там существует оценка (котангенс там убывает):

\[\forall t \notin (x_0 - \delta, x_0 + \delta): \left|h(x_0 + t)\ctg \frac{t}{2}\right| \le |h(x_0 + t)|\ctg \frac{\delta}{2}\]

ч. т. д. 


\subsubsection{Корректность определения свертки}
\textit{Формулировка:}

\begin{itemize}
    \item Выражение $(f * K)(x)$ определено для почти всех $x$ и лежит в $L^1[-\pi, \pi]$.
\end{itemize}

\textit{Доказательство:}

Пусть:

\[g(x, t) = f(x - t)K(t)\]

Доказывать будем в 2 шага, сначала измеримость $g$, а потом её принадлежность классу $L^1$.

\textbf{1. Измеримость}

Нам хочется проверить, что $g: \mathbb{R}^2 \rightarrow \rinf$ измерима. Давайте подробим на 2 функции, и установим измеримость каждой. Например, $h(x, t) = K(t)$ точно измерима, потому что мы можем построить множество Лебега для неё:

\[\mathbb{R}^2(h < (x, t)) = \mathbb{R} \times \mathbb{R}(K < t)\]

Фактически, у нас получился измеримый прямоугольник (декартово произведение всего $\mathbb{R}$ (т. к. $x$ игнорируется, измеримо) и измеримое множество для $K$). Теперь разбираемся с $\varphi(x, t) = f(x - t)$. Давайте построим множества $E_a:= \mathbb{R}(f(x) < a)$ (при фиксированном $t$). Такое множество измеримо.

\images{0.7}{korr_sv.jpg}

На картинке видно, что для каждого $t$ это будет какой-то отрезок на $x$. И все такие отрезки будут лежать между двумя прямыми, так как там есть сдвиг на каждый следующий по $x$. Ну и всё, можно даже сравнять эту картинку, введя линейное отображение $V(x, t) = (x - t, t)$ (там сдвиг, на $\mathbb{R}$ пофигу). Таким образом, это множество тоже измеримо (по картнике).

\textbf{2. Принадлежность $L^1$}

Проверим, что $g \in L^1 [-\pi, \pi] \times [-\pi, \pi]$. Как? Да по определению, его норма должна быть конечна.

\[\int_{-\pi}^{\pi} \int_{-\pi}^{\pi} |f(x - t)| \cdot |K(t)| dx dt = \int_{-\pi}^{\pi}|K(t)| dt \int_{-\pi}^{\pi} |f(x - t)| dx =\]

Вау, да внутри у нас норма $f$ (на сдвиг пофигу, сделаем замену и у нас всё периодическое)!

\[\int_{-\pi}^{\pi} K(t) ||f||_1 dt = ||f||_1 ||K||_1\]

Получается, норма $g$ --- конечна (так как $f, K$ у нас по условию из $L^1[-\pi, \pi]$). А теперь нетипичное применение теоремы Фубини --- у нас есть функция из произведения пространств ($L^1[-\pi, \pi] \times [-\pi, \pi]$), суммиремая, и мы хоиим поинтегрировать её по одной из переменных. А теорема нам как раз и говорит (первые 2 пункта), что при почти всех $x$ это можно сделать, и что результат будет суммируем ($\in L^1[-\pi, \pi]$). Ура, получилось!

ч. т. д. 

\subsubsection{Свойства свертки}
\textit{Формулировка:}

\begin{itemize}
    \item Повествовательная теорема :)
\end{itemize}

\textit{Доказательство:}

\textbf{1. Дистрибутивность}

\[(f * K)(x) = \int_{-\pi}^{\pi} f(x - t)K(t) dt = \begin{bmatrix}
    \tau = x - t\\
    t = x - \tau
\end{bmatrix} = \int_{x + \pi}^{x - \pi} f(\tau)K(x - \tau) d\tau = \int_{-\pi}^{\pi} f(\tau)K(x - \tau) d\tau = (K * f)(x)\]

(на промежутки интегрирования всем пофигу, потому что функции периодические).

\textbf{2. Вычисление коэффициентов}

\[c_k(f * K) = 2\pi c_k(f)c_k(K)\]

\[2\pi c_k(f * K) = \int_{-\pi}^{\pi} (f * K)(t) e^{-ikt}dt = \int_{-\pi}^{\pi} dt \int_{-\pi}^{\pi} \underbrace{f(t - x)K(x) e^{-ik(t - x) -ikx}}_{\in L^1[-\pi, \pi] \times [-\pi, \pi]}dx = \]

Ну вот тут формально надо сказать, что функция (выделенная скобочкой) суммируемая, поэтому по Фубини можно переставить интегралы (всё как в определении корректности).

\[= \int_{-\pi}^{\pi} K(x) e^{-ikx} dx \left(\int_{-\pi}^{\pi} f(t - x) e^{-ik(t - x)} dt\right) =\]

Ну всё, внутри как обычно делаем замену, сдвигаем границы и в итоге получаем:

\[ = \int_{-\pi}^{\pi} K(x) e^{-ikx} (2\pi c_k(f)) dx = 2\pi2\pi c_k(f) c_k(K)\]

\textbf{3. Оценка свёртки}

Пусть $f \in L^p[-\pi, \pi], \dbl g \in L^q[-\pi, \pi], \dbl \frac{1}{p} + \frac{1}{q} = 1, \dbl 1 \le p \le + \infty$

Тогда $f * g$ --- непрерывна, и $||f * g||_{\infty} \le ||f||_p \cdot ||g||_q$

\textit{Доказательство:}

Неравенство --- суть есть неравенство Гёльдера!

\[\left|\int_{-\pi}^{\pi} f(x - t) K (t) dt \right| \le \int_{-\pi}^{\pi} |f(x - t)| \cdot |K(t)| dt \le \left(\int_{-\pi}^{\pi} |f(x - t)|^p\right)^{\frac{1}{p}} \cdot \left(\int_{-\pi}^{\pi}|g(t)|^q\right)^{\frac{1}{q}} = ||f||_p \cdot ||g||_q\] 

То есть, мы оценили модуль интеграла неравенством Гёльдера (видимо, из этого следует и оценка на существенный супремум?). Теперь посмотрим на непрерывность:

\[(f * g)(x + h) - (f * g)(x) = \int_{-\pi}^{\pi} f(x + h - t)g(t) dt - \int_{-\pi}^{\pi} f(x - t)g(t) dt = \int_{-\pi}^{\pi} (f(x + h - t) - f(x - t))g(t)dt\]

Оценим по Гёльдеру:

\[\int_{-\pi}^{\pi} |f(x + h - t) - f(x - t)| \cdot |g(t)| dt \le ||f(x + h) - f(x)||_p \cdot ||g||_q \goesto{h \rightarrow 0} 0 \]

По теореме о непрерывности сдвига. (Для случая $p = \infty$ надо поменять $f$ и $g$ местами, тогда у нормы разности размерность пространства будет 1 и в ней тоже всё работает).

\textbf{4. Принадлежность классу более высокого ранга}

$f \in L^p[-\pi, \pi], \dbl 1 \le p < +\infty, \dbl K \in L^1[-\pi, \pi]$. Тогда $f * K \in L^p[-\pi, \pi]$.

При $p = 1$ --- по определению. А если $p = \infty$, то это пункт (3). Оценим:

\[\left|\int_{-\pi}^{\pi}f(x - t)K(t)dt\right|^p \le \left( \underbrace{\int_{-\pi}^{\pi} |f(x - t)| \cdot |K(t)|^{\frac{1}{p}}} \cdot |K(t)|^{\frac{1}{q}}\right)^p \le \]

Вы видите финт ушами? А он есть. Посмотрите, мы раздробили $|K(t)|$ на две части, по правилу для коэффициентов неравенства Гёльдера $\frac{1}{p} + \frac{1}{q} = 1$! Применяем Гёльдера ($(\int fg)^p = \int f^p (\int g^q)^{\frac{1}{q}}$):

\[\le \left( \int_{-\pi}^{\pi} |f(x - t)|^p|K(t)| dt\right)||K(t)||_1^{\frac{p}{q}}\]

Посмотрите и переосознайте ещё раз, как получился правый множитель, ощибки в вычислениях нет, просто кучу раз друг на друга наложились степени. Теперь оценим норму самой свёртки:

\[||(f * K)(x)||_p^p = \int_{-\pi}^{\pi} \left|\int_{-\pi}^{\pi}f(x - t)K(t)dt \right|^pdx \le\]

Применяем нашу свежеполученную оценку:

\[\le ||K(t)||_1^{\frac{p}{q}}\int_{-\pi}^{\pi} \left( \int_{-\pi}^{\pi} |f(x - t)|^p|K(t)| dt\right) dx = ||K(t)||_1^{\frac{p}{q}}\int_{-\pi}^{\pi}|K(t)| dt \left( \int_{-\pi}^{\pi} |f(x - t)|^p dx\right) =\]

\[= ||K||_1^{\frac{p}{q} + 1}||f||_p^p =\]

Заметим, что $\frac{p}{q} + 1 = p\left(1 - \frac{1}{p}\right) + 1= p$:

\[=||K||_1^p||f||_p^p\]

Таким образом:

\[||f * K||_p \le ||K||_1 \cdot ||f||_p \]

Значит, $p$-норма свёртки ограничена, значит она находится в классе $L^p[-\pi, \pi]$ (её интеграл меньше функции, уже лежащей в ней, значит он конечен и всё получилось).

ч. т. д. 

\subsubsection{Теорема о свойствах аппроксимативной единицы}
\textit{Формулировка:}

$(K_h)$ --- АЕ

Тогда:

\begin{enumerate}
    \item $f \in \tilde{C}[-\pi, \pi]: (f * K_h) \rsh{h \rightarrow h_0} f$
    \item $f \in L^1[-\pi, \pi]: ||(f * K_h) - f||_1 \goesto{h \rightarrow h_0} 0$
    \item Если $f \in L^1[-\pi, \pi]$ непрерывна в точке $x$ и $(K_h)$ --- УАЕ (именно такой порядок букв, не перепутайте ;) ), тогда $(f * K)$ тоже непрерывна в точке $x_0$ и:
    \[(f * g)(x) - f(x) \goesto{h \rightarrow h_0} 0\]
\end{enumerate}

\textit{Доказательство:}

Для начала, заметим, что такую конструкцию можно представлять следующим образом (по АЕ1 $\int_{-\pi}^{\pi} K_h = 1$):

\[(f * g)(x) - f(x) = \int_{-\pi}^{\pi} f(x - t)K_h(t)dt - \left(\int_{-\pi}^{\pi} K_h(t) dt \right) f(x) = \int_{-\pi}^{\pi} (f(x - t) - f(x))K(t)dt \dbl (*)\]

\textbf{1.}

$f$ --- равномерно-непрерывна. На каком основании? Да на таком же, как и в теореме о непрерывности сдвига, у нас непрерывная функция на компакте --- теорема Кантора. Распишем:

\[\forall \varepsilon > 0 \dbl \exists \delta > 0 \dbl \forall |t| < \delta \dbl \forall x \in [-\pi, \pi] : |f(x + t) - f(x)| < \frac{\varepsilon}{2M}\]

$M$ --- из АЕ2. Ну чтож, супер, давайте теперь оценивать (*), для равномерной сходимости, если заюзаем услоавие равномерное непрерывности, будет самое то:

\[ \le \underbrace{\int_{-\delta}^{\delta} |f(x - t) - f(x)| \cdot |K_h(t)| dt}_{I_1} + \underbrace{\int_{E_\delta} |f(x - t) - f(x)| \cdot |K_h(t)| dt}_{I_2}\]

С первым разбираемся по равномерной непрерывности, благо, $\delta$ взяли прямо из этого определения:

\[ I_1 \le \frac{\varepsilon}{2M} \int_{-\delta}^{\delta} |K_h(t)|dt \le \frac{\varepsilon}{2M} \int_{-\pi}^{\pi} |K_h(t)| dt \underset{\text{АЕ2}}{\le} \frac{\varepsilon}{2}\]

Ну а со вторым по АЕ3:

\[I_2 \le \underbrace{\esssup_{E_\delta} |f(x - t) - f(x)|}_{2||f||_{\infty}} \cdot \int_{E_\delta} |K_h(t)| dt \goesto{h \rightarrow h_0} 0\]

Выбор $\esssup$ для оценки максимума функций кажется достаточно удобным (и вроде очевидным), причём функции ограничены по условию! (они непрерывные периодические, значит ограниченные! (гоняются типа по окружности, компакт + теорема Вейерштрасса ?)). Ну а раз всё это дело радостно стремится к 0 при приближении $h$ к $h_0$. ну тогда уж точно найдётся такая окрестность, что:

\[\exists U(h_0) \dbl \forall h \in U(h_0): |\ldots| \le \frac{\varepsilon}{2}\]

Готово.

\textbf{3. }

Тут почти то же самое, только записываем определение для непрерывности в точке $x$:

\[\forall \varepsilon > 0 \dbl \exists \delta > 0 \dbl \forall |t| < \delta \dbl : |f(x + t) - f(x)| < \frac{\varepsilon}{2M}\]

Тогда $I_1$ оценивается так же, буквально дословно. Ну а для $I_2$ применяем АЕЗ', раз уж есть такая возможность:

\[I_2 \le \esssup_{t \in E_\delta}|K_h(t)| \underbrace{\int_{E_\delta} \underbrace{|f(x - t) - f(x)|}_{\le |f(x - t)| + |f(t)|}}_{2\pi|f(x)| + ||f||_1} dt  \goesto{h \rightarrow h_0} 0\]

Внутри интеграла сначала оценили по неравенству треугольника. Замечаем, что $f(x)$ --- просто константа, и оцениваем ещё сверху длиной промежутка. А $|f(x - t)|$ завернули в норму, приенив трюк со сдвигом. Всё конечное, стремится в совокупности к 0. Победа!

\textbf{2. }

Вот тут немного глинисто :( . Ну штош, давайте оценивать норму, применяя (*):

\[||(f * K_h) - f||_1 = \int_{-\pi}^{\pi} \left| \int_{-\pi}^{\pi} (f(x - t) - f(x))K_h(t) dt\right| dx \le \] 

\[ \le \int_{-\pi}^{\pi} \left(\int_{-\pi}^{\pi} |f(x - t) - f(x)| \cdot |K_h(t)| dt \right)dx = \]

В этом месте НЕ НАДО пытаться свернуть это в произведение $L^1$ норм, это нам ничего не даст (теорема о непрерывности свдига для $t$ не сработает, оно никуда не стремится :( )) Вместо этого предлагается вспомнить свойство АЕ, что если её модуль доделить на $L^1$ норму, то это тоже будет АЕ. Ну давайте доделим, и при этом обернём внутренний интеграл в функцию:

\[g(t) := \int_{-\pi}^{\pi} |f(x + t) - f(x)| dx\]

\[ = ||K_h(t)||_1\int_{-\pi}^{\pi} g(-t) \frac{|K_h(t)|}{||K_h(t)||}dt \quad (**)\]

Докажем непрерывность $g(t)$ в нуле (да и во всех точках, по определению) (потом будем туда устремлять), рассмотрим $g(t) := ||f_t - f||_1$ (это она на самом деле и есть, ну посмотрите на неё!):

\[\text{фиксируем } t_0: |g(t_0 + h) - g(t_0)| = | \,||f_{t_0 + h} - f|| - ||f_{t_0} - f|| \,| \le \]

Неравенство треугольника и теорема о непрерывности сдвига:

\[\le ||f_{t_0 + h} - f_t|| \goesto{h \rightarrow h_0} 0\]

Супер, она непрерывна. Тогда возвращаемся к (**). Вы мб не заметили, но это свёртка в нуле!

\[ (**) = ||K_h(t)||_1\int_{-\pi}^{\pi} g(0 - t) \frac{|K_h(t)|}{||K_h(t)||}dt = ||K_h(t)||_1 \cdot \left(g * \frac{|K_h|}{||K_h||_1}\right)(0)\]

По пункту 1 мы знаем, что у непрерывной пероиодической функции свёртка сходится равномерно к самой функции! Так устремим же наконец, ёлки палки!

\[\goesto{h \rightarrow h_0} \underbrace{||K_h(t)||_1}_{\text{огр. по АЕ2}} \cdot g(0) = 0\]

ч. т. д.

\textit{Замечания:} 

\begin{enumerate}
    \item В пункте 2 сработает и $f \in L^p[-\pi, \pi]$ (с соответствующей нормой)!!!! Но доказывать мы это не будем, там примерно то же самое только громоздкое.
    \item Пункт 3: $(K_h)$ --- УАЕ, чётная, $f \in L^1[-\pi, \pi]$, в $x_0$ существуют конечные $f(x_0 + 0), f(x_0  - 0)$, тогда верно (что-то подонобное признаку Дини):
    
    \[|(f * K_h) - f|_1 \goesto{h \rightarrow h_0} \frac{1}{2}(f(x_0 + 0) - f(x_0 - 0))\]
\end{enumerate}

\subsubsection{Теорема Фейера }
\textit{Формулировка:}

Аналог теоремы о свойствах АЕ для сумм Фейера

\begin{itemize}
    \item $f \in \tilde{C}[-\pi, \pi]: \sigma_n(f, x) \rsh{n \rightarrow \infty} f$
    \item $f \in L^p[-\pi, \pi]: ||\sigma_n(f, x) - f||_1 \ntoinf 0$
    \item Если $f \in L^1[-\pi, \pi]$ непрерывна в точке $x$, тогда:
    \[\sigma_n(f, x) \ntoinf f(x)\]
\end{itemize}

\textit{Доказательство:}

Достаточно лишь доказать, что $\Phi_n(t)$ является УАЕ. Давайте пойдём поаксиомно:

\textbf{AE1 + AE2}

$\Phi_n(t) \in L^1[-\pi, \pi], \Phi_n(t) \ge 0, \int_{-\pi}^{\pi} \Phi_n(t) = 1$ --- по свойствам ядра Фейера (см. в опредлении).

\textbf{АЕ3'}

Ну оценим, почему бы и нет: 

\[\forall t \in E_\delta \quad \Phi_n(t) = \frac{1}{2\pi (n + 1)} \cdot \frac{\sin^2\left(\frac{n + 1}{2}\right)}{\sin^2 \frac{t}{2}} \le \frac{1}{2\pi (n + 1)} \cdot \frac{1}{\sin^2 \frac{\delta}{2}} \ntoinf 0\]

Заметим, что знаменатель мы оценили исходя из строения множества $E_\delta$, ну и тогда при каждом $t$ у нас всё равно всё сходится, значит супремум стремится к нулю (это мы так считали супремум, не написав его явно))! 

Тогда по предыдущей теоерме все условия выполняются.

ч. т. д.

\textit{Замечание:}

$\DD_n(t)$ --- не АЕ, у него $L^1$ нормы неограничены!


\subsubsection{Полнота тригонометрической системы и другие следствия теоремы Фейера}
\textit{Формулировка (следствие 1):}

Если $f L^1[-\pi, \pi]$ непрерывна в точке $x_0$ и ряд Фурье в ней сходится, то

\[S_n(f, x_0) \ntoinf f(x_0)\]

(а в противном случае утверждение не имеет смысла)

\textit{Доказательство:}

По теореме Фейера:

\[\sigma_n(f, x_0) \ntoinf f(x_0)\]

Ну тогда, раз уж по условию $S_n(f, x_0)$ сходится к $f(x_0)$, то по теореме Коши о перманентности метода СА, $\sigma_n$ должен сходиться туда же, куда и $S_n$.

ч. т. д. 

\textit{Формулировка (полнота тригонометрической системы):}

\begin{enumerate}
    \item \[\left\{\frac{1}{2}, \cos t, \sin t, \cos 2t, \sin 2t \ldots \right\}\]

    --- полная ОС в $L^2[-\pi, \pi]$
    \item $f \in L^1[-\pi, \pi]$
    
    Если $\forall k: a_k(f) = 0$ и $b_k(f) = 0$, то $f \equiv 0$.
\end{enumerate}

\textit{Доказательство:}

\textbf{1.} следует из \textbf{2.} (ну, типа, мы взяли функцию $z$ из более широкого класса, и попробовали посчитать $\forall k: \sk{z}{e_k} = 0$. То есть, мы изобрели новый вектор, ортогональный всем остальным. Ну, и он всегода оказывается нулевым, значит, система полная).

\textbf{2.}

Ну, раз $a_k(f) = b_k(f) = 0$, значит и $S_n \equiv 0, \dbl \sigma_n \equiv 0$. Ну а по теореме Фейера (пункту 2) оно стремится к $0$ в смысле $L^1$. 

ч. т. д. 

\textit{Следствие: }

$f \in L^2[-\pi, \pi]$, тогда $S_n(f, x) \goesto{L^2} f$

\textit{Формулировка (теорема Вейерштрасса):}

Тригонометрические полиномы плотны в $C[-\pi, \pi]$ и $L^p[-\pi, \pi]$. (наверное $\tilde{C}$?)

\textit{Доказательство:}

$\sigma_n(f, x)$ --- суть есть тригонометрический полином (ну там внутри чёто поскладывалось, поделилось и всё получилось). Ну а тогда, по пунктам 1 и 2 теоремы Фейера выполнены запросы (соостветственно). Равномерная сходимость --- прямо то, что нужно (в терминах плотности), ну и $p$-норма тоже.

ч. т. д. 

\textit{Формулировка (равенства Парсеваля):}

$f, g \in L^2[-\pi, \pi]$

\begin{enumerate}
    \item \[\int_{-\pi}^{\pi} f \overline{g} = 2\pi \sum c_k(f) \overline{c_k(y)}\]
    \item \[\int_{-\pi}^{\pi} |f|^2 = 2\pi \sum |c_k(f)|^2\]
    \item \[\int_{-\pi}^{\pi} fg = \pi \left(\frac{a_0(f)a_0(g)}{2} + \sum a_k(f)a_k(g) + b_k(f)b_k(g)\right)\]
    \item \[\int_{-\pi}^{\pi} f^2 = \pi \left( \frac{a_0(f)^2}{2} + \sum a_k(f)^2 + b_k(f)^2 \right)\]
\end{enumerate}

\textit{Доказательство:}

Ну там что-то тривиальное, на самом деле, это вариации $\sk{f}{g} = \sum c_k(f) \overline{c_k(g)}$ для скалярного произведения в $L^2$. И нормировочные коэффициенты вылезли.

\subsubsection{Лемма об оценке интеграла ядра Дирихле}
\textit{Формулировка:}

\begin{itemize}
    \item $\DD_n(t) = \frac{\sin nt}{\pi t} + \frac{1}{2\pi}(\cos nt + h(t)\sin t)$, где $h(t)$ не зависит от $n$ и её интеграл ограничен: $|h(t)| \le 1$ на $t \in [-\pi, \pi]$
    \item $\forall x \in [0, 2\pi] : \int_{0}^{x} \DD_n(t) dt \le 2$
\end{itemize}

\textit{Доказательство:}

\textbf{1.}

Посмотрим на формулу ядра Дирихле и раскроем верхний синус суммы:

\[\DD_n(t) = \frac{\sin \left(n + \frac{1}{2}\right)t}{2\pi\sin \frac{t}{2}} = \frac{\sin nt \cos\frac{t}{2} + \sin \frac{t}{2} \cos nt}{2\pi\sin \frac{t}{2}} = \frac{1}{2\pi}\frac{\sin nt}{\tg \frac{t}{2}} + \frac{\cos nt}{2\pi} = (*)\]

Теперь просто надо просто написать то, что мы хотим получить и подогнать ответ. Заметим только, что:

\[ \frac{\sin nt}{2 \pi \tg \frac{t}{2}} = \frac{\sin nt}{\pi t} + \frac{\sin nt}{2 \pi} \left(\underbrace{\frac{1}{\tg \frac{t}{2}} - \frac{1}{\frac{t}{2}}}_{h(t)}\right)\]

Если не влом, пересчитайте, проверьте, но оно так. Заодно вылезла и $h(t)$ --- действительно, не зависящая от $n$. Посомтрим на итоговую запись разложения нашего ядра:

\[(*) = \frac{\sin nt}{\pi t} + \frac{1}{2\pi} \left( \cos nt + \sin nt \left(\underbrace{\frac{1}{\tg \frac{t}{2}} - \frac{1}{\frac{t}{2}}}_{h(t)}\right) \right)\]

Оценка на $h(t)$ вылезает сама собой, особенно, учитывая что это нечётная функция, $h(0) = 0$, убывающая:

\[|h(t)| \le \frac{2}{\pi} < 1\]

\textbf{2.}

Во-первых, давайте скажем, что интеграл ядра Дирихле не сильно отличается от интеграла первого синуса на $x \in (0, \pi]$:

\[\left| \int_{0}^{x} \DD_n(t) dt - \underbrace{\int_{0}^{x} \frac{\sin nt}{\pi t} dt}_{(***)}\right| \le \frac{1}{2\pi}\left|\int_{0}^{x} \cos nt + h(t)\sin nt dt\right| \le \frac{1}{2\pi} 2 \le 1\]

Тут надо помахать руками, сказать, что сумма оценивается двойкой, так как каждое слагаемое меньше 1, и всё получится. (Если сесть с бумажкой, то оно тоже нормально оценится, но КПК этого не делал).

Ну ок, оценили, дальше то что? Давайте посмотрим, что это за интеграл с синусом такой вообще?

\[J_n(x) = \int_{0}^{x} \frac{\sin nt}{\pi t} dt = \begin{dcases}
    y = nt \\
    dt = \frac{dy}{n}
\end{dcases} = \int_{0}^{nx} \frac{n \sin y}{\pi y} \frac{dy}{n} = \int_0^{nx} \frac{\sin y}{\pi y} dy\]

Это вот такая какая-то периодическая функция, причём колебания у неё затухают с каждым $\pi$:

\images{0.8}{lemm_sin.jpg}

Причём, раз каждое следующее колебание по площади меньше, чем предыдущее, значит, самое большое --- первое. Им и оценим весь интеграл:

\[0 \le J_n(x) \le \int_{0}^{\pi} \frac{\sin y}{\pi y} dy \le \int_{0}^{\pi} \frac{1}{\pi} dy = 1\]

Ну супер, значит всегда $(***)$ в промежутке $[0, 1]$, и из этих оценок следует, что $\int_0^x \DD_n(t)dt$ лежит в промежутке $[-2, 1]$. Почти готово, надо лишь рассмотреть теперь $x$ на промежутке $[0, 2\pi]$, как в условии:

\[\forall x \in [\pi, 2\pi] : \int_0^{x} \DD_n(t) = \int_0^{2\pi} - \int_x^{2\pi} = 1 -\int_x^{2\pi} = 1 - \int_{x  -2\pi}^{0} = 1 - \underbrace{\int_0^{2\pi - x}}_{\in [-2, 1]}\]

Объясняю, что произошло: чтобы добить до $[0, 2\pi]$ надо поисследовать интгеграл на $[\pi, 2\pi]$, т. к. на остальном участке мы его уже поисследовали. Далее, мы разбили на разность двух интегралов, первый --- из определения ядра Дирихле (ну, в другой нотации видимо, сместили границы интегрирования, но всё равно работает! Не верите --- проверьте)))) ). Потом, сдвигаем по периодичности границы у второго интеграла и по чётности ядра Дирехле подменяем $x := -x$, получая нормальный интеграл с границами $[0, t]$, где $t \in [0, \pi]$. А для такого то мы всё уже знаем, его значение лежит в промежутке $[-1, 2]$. Ну и всё, две половинки ($\int_0^{\pi} + \int_\pi^{2\pi}$) складываем, и получаем, что в максимуме они могут дать 2.

ч. т. д.

\subsubsection{Теорема об интегрировании ряда Фурье}
\textit{Формулировка:}

\begin{itemize}
    \item $f \in L^1[-\pi, \pi]$
    \item $\forall a, b \in \mathbb{R}$
\end{itemize}

Тогда:

\[\int_a^b f(x)dx = \sum_{k \in \mathbb{Z}} c_k(f)\int_{a}^{b} e^{-ikx} dx \]

(в смысле главного значения: $\sum_{k = -n}^{n}$)

\textit{Доказательство:}

Во-первых, давайте, не умаляя общности будем считать, что:

\[-\pi \le a < b \le \pi\]

Иначе по периодичности сдвинем да и всё. Тогда введём $\chi = \chi[a, b]$ (характеристическая функция отрезка), и преобразуем то, что лежит внутри суммы:

\[c_k(f)\int_{a}^{b}e^{-ikx} dx = c_k(f)\int_{-\pi}^{\pi} \chi(x) e^{-ikx} dx = c_k(f)2\pi c_k(\chi) = \]

Как вам финт ушами? Действительно, просто ввели характеристическую функцию отрезка (тем самым занулили всё что лежит вне $[a, b]$, чем и обеспечивается корректность подсчёта интеграла), и тем самым упаковали в коэффициент! А первый коэффициент давайте развернём:

\[= \frac{2\pi}{2\pi}c_k(\chi)\int_{-\pi}^{\pi} f(x)e^{-ikx}dx\]

Возвращаемся обратно к сумме:

\[\sum_{k = - n}^{n} \int_{-\pi}^{\pi} c_k(\chi)f(x)e^{-ikx}dx \underset{(1)}{=}\int_{-\pi}^{\pi} f(x) \underbrace{\sum_{k = -n}^{n} c_k(\chi)e^{-ikx}}_{S_n(\chi, x)} dx = \int_{-\pi}^{\pi} f(x)S_n(\chi, x) \underset{(2)}{\ntoinf} \int_{-\pi}^{\pi} f(x)\chi(x) dx = \int_{a}^{b} f(x) dx\]

(1) работает, так как суммы конечные. А так, воу, что-то даже получилось. Заменили на частичную сумму Фурье, потом случилось (2) и всё получилось. Однако, как случилось (2)? Во-первых, сработало 2е следствие признака Дини (там, где существуют односторонние производные (а они существуют! У нас суперфункция, которая на отрезке равна 1, а вне него 0. Ну, там неоч с дифференцированностью в точках $a$ и $b$, н и пофигу, нас спасает приписка ``при почти всех $x \in [-\pi, \pi]$'' (спасает, потому что сейчас будем запускать теорему Лебега для сходимоссти почти везде!))).

Ну ок, суммы сходятся, а интегралы то почему сходятся? Ну, надо суммируемую мажоранту, и тогда сработает Лебег. Придумываем (используя предыдущую лемму):

\[|S_n(\chi, x)| = \left| \int_{-\pi}^{\pi} \chi(x - t)\DD_n(t) dt\right| = \left|\int_{x - b}^{x - a} \DD_n(t) dt \right| \le \left| \int_{0}^{x - a} \right| + \left| \int_{x - b}^{0}\right| \le 2 + 2 = 4\]

Что произвошло: оцениваем частичную сумму Фурье по определению, применяем характеристическую функцию, тем самым меняем границы интегрирования ($a \le x - t \le b \Rightarrow x - a \ge t \ge x - b$). $x \in [-\pi, \pi]$. Тут опять надо помахать руками, сказать, что всё будет хорошо, даже если $0 \notin [x - b, x - a]$, упомянуть чётность ядра Дирихле и разбить наконец-таки на 2 интгерала, которые мы оценивали в прошлой лемме.

\[|f(x)S_n| \le 4||f||_1\]

Суммируемая мажоранта найдена ($f$ --- из $L^1$ по условию), значит верна теорема Лебега и всё получается!

ч. т. д. 

\subsubsection{Следствие о синус-коэффициентах интегрируемой функции}

\textit{Экзотика от КПК!}

\textit{Формулировка:}

\begin{itemize}
    \item $f \in L^1[-\pi, \pi]$ 
\end{itemize}

Тогда $\sum \frac{b_n(f)}{n}$ --- сходится

\textit{Доказательство:}

\textbf{Обратите внимание!} Здесь экспоненциальная часть будет использоваться без минуса $\int e^{ikx}$ (так можно делать, ведь индексы мы гоняем от $-k$ до $k$), но тут так удобнее.

Давайте поинтегрируем в стиле прошлой теоремы такое:

\[g(u) := \int_{0}^{u} f(x)dx = \sum_{k \in \mathbb{Z}} c_k(f) \int_0^u e^{ikx} dx\]

\[\int_{-\pi}^{\pi} g(u) du = \int_{-\pi}^{\pi} \sum \underset{(1)}{=} \sum \int_{-\pi}^{\pi} = \sum c_k(g) \int_{-\pi}^{\pi} \int_{0}^{u} e^{ikx}dxdu = (*)\]

Посчитаем интегралы внутри по отдельности:

\[\int_{0}^{u} e^{-ikx}dx = \begin{cases}
    \left(\frac{e^{ikx}}{ik} \right)_0^{u} = \frac{e^{iku} - 1}{ik}, &k \neq 0\\
    u, &k = 0
\end{cases}\]

\[\begin{cases}
    \int_{-\pi}^{\pi} \frac{e^{iku} - 1}{ik} du = -\frac{2 \pi}{ik}, &k \neq 0\\
    \int_{-\pi}^{\pi} u du = 0, &k = 0
\end{cases}\]

Таким образом:

\[(*) = \sum \frac{-c_k(f)2\pi}{ki}\]

И такой ряд сходится, ведь $f$ --- суммируемая, значит $g$ конечна для любого $u \in [\pi, \pi]$ (она меньше, чем $\int_{-\pi}^{\pi} |f| = ||f||_1$), ну, получается, интеграл ограниченной величины --- всё хорошо (вроде бы?).

Ну зашибись, сходится и сходится, а как это помогает доказать сходимость ряда синус-коэффициентов. Предлагается вспомнить, что синус выражается через экспоненты:

\[\sin kx = \frac{e^{ikx} - e^{-ikx}}{2i}\]

И если подставить, то получается, что:

\[b_k(f) = i(c_k(f) - c_{-k}(f))\]

Утверждается, что:

\[\sum \frac{-2 \pi c_k(f)}{ki} = 2\pi \sum \frac{b_n(f)}{n}\]

Доказывается методом пристального взгляда, и указанием на то, что мы можем комбинировать суммы (суммировать $c_k$ и $c_{-k}$ вместе в левой сумме). Значит, исходный ряд сходится.

И всё было бы замечательно, только мы до сих пор не научились обосновывать переход (1). Действительно, а почему это можно так просто взять и переставить счётную сумм с интегралом? Ну давайте посмотрим, что это такое:

\[\int \sum = \lim \int \sum_{-N}^{N} = \lim \sum_{-N}^{N} \int \goesto{n \rightarrow \infty} \sum \int\]

Расписали как предел, внутри поменяли местами, так как конечная сумма. Ну а то, что дальше, мы фактически уже проделывали в предыдущей теореме, там признак Дини, оценка ядра Дирихле и теорема Лебега для случая сходимости почти везде.

ч. т. д. 

\subsubsection{Лемма о слабой сходимости сумм Фурье}

КПК до этой леммы сделал небольшое отступление об ``обобщённых'' функциях. Стоит его посмотреть, чтобы понимать, зачем оно вообще надо. Но на понимание леммы это влиять не должно.
\textit{Формулировка:}

\begin{itemize}
    \item $f \in L^1[-\pi, \pi]$
\end{itemize}

Тогда:

\[\forall h \in \tilde{C}^{\infty}[-\pi, \pi]: \int_{-\pi}^{\pi} S_n(f, x)h(x) \ntoinf \int_{-\pi}^{-\pi} f(x)h(x)dx\]

\textit{Доказательство:}

Тут тоже везде используется $e^{ikt}$. Может так вообще и надо в частичных суммах Фурье? Но вообще вроде пофигу, суммируется то в мысле главного значения, от $-n$ до $n$.


Заметим сразу (хоть нас никто и не просил), что свёртка этих функций есть функция непрерывная и даже гладкая!

\[(f * g)(x) = \int_{-\pi}^{\pi} f(x - t)h(t) dt\]

Первое обуславливается $\tilde{C}^{\infty} \subset L^{\infty}$ и 3им свойством свёртки. Ну и дифференцируемость по Лейбницу:

\[\frac{d}{dx}(f * g)(x) = \left( \int_{-\pi}^{\pi} f(x - t)h(t) dt =  \int_{-\pi}^{\pi} f(t)h(x - t) dt\right) =  \int_{-\pi}^{\pi} f(x)h'_x(t) dt\]

Внутри там переставили $x$ (так можно делать по коммуникативности свёртки), чтобы заиспользовать бесконечную дифференцируемость $h$. Но ещё нужна суммируемая мажоранта, не зависящая от $x:$. Ну оценим сверху $||f||_1$ и $h_x$ какой-то константой, раз уж она непрерывно-периодическая.

Ну всё, поехали. Давайте посмотрим, что нам надо доказать (суммы конечные, переставляем без стеснения):

\[\int_{-\pi}^{\pi} S_n(f, t) h(t) dt = \int_{-\pi}^{\pi} \sum_{k = -n}^{n} c_k(f)  e^{ikt} h(t) dt = \sum_{k = -n}^{n}c_k(f)\int_{-\pi}^{\pi}h(t) e^{ikt}dt = (*)\]

Обратите внимание, что в коэффициентах Фурье у нас зафиксировано, что экспонента с минусом.  Поэтому надо подогнать под это:

\[\overline{h}(t) = h(-t), \quad c_k(\overline{h}) = \int_{-\pi}^{\pi} \overline{h}(t) e^{-ikt} dt = \int_{-\pi}^{\pi} h(-t)e^{-ikt} dt = \int_{-\pi}^{\pi} h(t) e^{ikt} dt\]

Ну вот, всё свернулось и у нас получились коэффициенты свёртки!

\[= \sum_{k = -n}^{n} c_k(f)c_k(\overline{h})2 \pi = \sum_{k  =-n}^{n} c_k(f * \overline{h}) e^{ikx} = \]

Воу, это ещё что такое? Откуда взялась экспонента? Так она всегда там и была, мы её смотрим в $x = 0$ (там она единица :) )!

\[ = S_n(f * \overline{h}, 0) \ntoinf (f * \overline{h})(0) = \int_{-\pi}^{\pi} f (0 - t)\overline{h}(t) dt = [t := -t] = \int_{-\pi}^{\pi} f(t)h(t) dt\]

Предельный переход нам обеспечивает следствие признака Дини, свёртка у нас как раз гладкая, поэтому сходится поточечно.  

ч. т. д. 

\subsubsection{Свойства преобразования Фурье: непрерывность, ограниченность, сдвиг}

Опять повествовательная теорема(

\textbf{1. Непрерывность}

По следствию теоремы ``Предельный переход по параметру в несобственном интеграле'' надо выдать суммируемую мажоранту, не зависящую от параметра (воспринимаем преобразоавние как интеграл с параметром):

\[|\hat{f}(y)| = \left| \int_{\mathbb{R}^m} f(x) e^{-2\pi i\sk{y}{x}}dx\right| \le \int_{\mathbb{R}^m} |f(x)| = ||f||_1\]

\textbf{2. Ограниченность}

Да то же самое:

\[|\hat{f}(y)| \le \int_{\mathbb{R}^m} |f(x)| = ||f||_1\]

\textbf{3. Сдвиг и сжатие}

Сдвиг $f_h = f(x + h)$, тогда $\hat{f_h}(y) = e^{-2 \pi i \sk{y}{h}} \hat{f}(y)$.

\textit{Доказательство:}

\[\hat{f_h}(y) = \int_{\mathbb{R}^m}f(x)e^{-2 \pi i \sk{y}{x + h}}dx = e^{-2\pi i \sk{y}{h}}\int_{\mathbb{R}^m}f(x)e^{-2 \pi i \sk{y}{x}}dx = e^{-2\pi i \sk{y}{h}} \hat{f}(y)\]

ч. т. д.

Сжатие $g_a = f(ax), a \in \mathbb{R} \setminus \{0\}$, тогда $\hat{g_a} = \frac{1}{a^m} \hat{f}\left(\frac{y}{a}\right)$

\textit{Доказательство:}

\[\hat{g_a}(y) = \int_{\mathbb{R}^m}f(ax)e^{-2 \pi i \sk{y}{x}}dx = \begin{dcases}
x = \frac{\tilde{x}}{a}\\
dx = \frac{d\tilde{x}}{a}
\end{dcases} = \int_{\mathbb{R}^m}f(\tilde{x})e^{-2 \pi i \sk{y}{\frac{\tilde{x}}{a}}}\frac{d\tilde{x}}{a} = \]
\[ = \frac{1}{a^m} \int_{\mathbb{R}^m}f(\tilde{x})e^{-2 \pi i \sk{\frac{y}{a}}{\tilde{x}}}\frac{d\tilde{x}}{a} = \frac{1}{a^m}\hat{f}\left(\frac{y}{a}\right)\]

Сделали замену и в скалярном произведении перетащили константу. Причём, когда выносили из-под интеграла, не забыли, что у нас $\mathbb{R}^m$.

ч. т. д.

\subsubsection{Преобразование Фурье свертки}
\textit{Формулировка:}

\begin{itemize}
    \item $f, g \in L^1(\mathbb{R}^m)$
\end{itemize}

Тогда:

\begin{enumerate}
    \item \[\widehat{f * g} = \hat{f}\cdot \hat{g}\]
    \item \[\int_{\mathbb{R}^m} f(y) \hat{g}(y) dy = \int_{\mathbb{R}^m} \hat{f}(y) g(y) dy\]
\end{enumerate}

(Во втором случае может возникать гемор из-за области определения (?), но вроде всё хорошо. Под интеграл можно ставить, т. к. $\hat{f}$ ведёт как константа, а $g$ --- суммируема по определению).

\textit{Доказательство:}

\textbf{1.}

\[\int_{\mathbb{R}^m} \left(\int_{\mathbb{R}^m} f(x - t)g(t) dt\right) e^{-2 \pi i \sk{y}{x}}dx =\]

Запускаем Фубини, чекрыжим экспоненту, ничего интересного:

\[ = \int_{\mathbb{R}^m} \left(\int_{\mathbb{R}^m} f(x - t)g(t) dt\right) e^{-2 \pi i \sk{y}{x - t}} e^{-2\pi i \sk{y}{t}}dx =\]

\[\int_{\mathbb{R}^m}g(t)e^{-2 \pi i \sk{y}{t}} \left( \int_{\mathbb{R}^m} f(x - t) e^{-2\pi i \sk{y}{x - t}} dx \right) dt = \hat{g}(y) \cdot \hat{f}(y)\]

\textbf{2. } доказывается аналогично, попередвигать шапочки и всё получится.

ч. т. д. 


\subsubsection{Преобразование Фурье и дифференцирование}
\textit{Формулировка:}

$f \in L^1(\mathbb{R}^m)$

\begin{enumerate}
    \item Пусть при некотором $k = 1, 2, \ldots, m \dbl \exists g = \frac{\partial f}{\partial x_k}$ --- суммируемая, непрерывная. Тогда $\hat{g}(y) = 2 \pi i y_k \cdot \hat{f}(y)$
    \item Если $|x| \cdot f$ --- суммируемая, то $\hat{f} \in C^1(\mathbb{R}^m)$ и для $k = 1, 2, \ldots, m: \frac{\partial \hat{f}}{\partial y_k}(y) = -2\pi i \hat{F_k}(y)$, где $F_k(x) = x_kf(x)$
\end{enumerate}

\textit{Доказательство:}

\textit{Минилемма:}

Сначала докажем предположение о том, что если $\varphi(u, t) \in L^1(\mathbb{R}^m)$ (суммируемая), то для почти всех $u : \varphi(u, t) \goesto{t \rightarrow \infty} 0$, если $\exists \psi(u, t) = \varphi'_t(u, t)$ --- непрерывная. Тогда напишем формулу Ньютона-Лейбница (имеем право, всё непрерывно):

\[\varphi(u, t) - \varphi(u, 0) = \int_{0}^{t} \psi(u, \tau) d\tau \]

Заметим, что функция $\psi(u, t)$ суммируема по Фубини (её интеграл равен разности суммируемых функций, они из $L^1(\mathbb{R}^m)$). Ну значит и интеграл по ней тоже суммируем (пункт 2 теоремы Фубини)! Давайте сделаем предельный переход по $t$. Справа всё хорошо, у нас суммируемая функция интегрируется, ну подвинули предели интегрирования, что такого, она всё так же даёт конечное значение. Из этого следует, что слева тоже существует конечный предел $\lim_{t \rightarrow \pm \infty}$! Такс, получается, что при почти всех $u \dbl \varphi$ суммируема (см. определение $L^1$) и имеет конечный предел. Хм, ну, если бы на бесконечности функция стремилась, например, к 2023, у неё не получилось бы быть суммируемой (ну, фактически $f(x) = 2023$ подойдёт. Какой же будет интеграл, конечным???? Не думаю.) Получается, раз этот предел существует, то он может быть только 0.

ч. т. д. 

\textbf{1.}

Теперь можно переходить к основному доказательству. Не умаляя общности, давайте считать, что частная производная случилась на $m$-й координате. Тогда удобно будет записывать нашу функцию как:

\[f(\underbrace{x_1, x_2, \ldots, x_{m - 1}}_{u}, \underbrace{x_m}_{t}) = f(u, t)\]

И введём:

\[g(u, t) = f'_t(u, t)\]

Давайте посчитаем преобразование Фурье $g(x)$:

\[\int_{\mathbb{R}^{m-1}} \ldots \int_{-\infty}^{\infty} dx_m g(x) \underbrace{e^{- 2 \pi i (y_1 x_1 + y_2 x_2 + \ldots + y_m x_m)}}_{e^{- 2 \pi \sk{y}{x}}} = \]

Давайте оставим под последним интегралом только то, что действительно зависит от $x_m$ (константы уносимв влево, по Фубини всё раскрывается в повторный, зашибись):

\[ = \int_{\mathbb{R}^{m-1}} \ldots e^{- 2 \pi i (y_1 x_1 + y_2 x_2 + \ldots + y_{m - 1} x_{m - 1})}\int_{-\infty}^{\infty} g(x) e^{-2 \pi i y_m x_m} dx_m = (*)\]

Посмотрим на последний интеграл отдельно, по частям:

\[\int_{-\infty}^{\infty} g(u , t) e^{-2 \pi i y_m t} dt = \int_{-\infty}^{\infty} f'_t(u , t) e^{-2 \pi i y_m t} dt = \] 
\[ = \underbrace{f(u, t)e^{-2 \pi i y_m t}|_{-\infty}^{\infty}}_{=0} + 2 \pi i y_m\int_{-\infty}^{\infty}f(u, t)e^{-2 \pi i y_m t}dt\]

Наша минилемма была нужна как раз, чтобы занулить вот этот член посередине. И он зануляется при подстановке, благодаря ней! Ну и все, если $m = 1$ то мы всё доказали. А если больше, то подставляем обратно в интеграл (*):

\[(*) = \int_{\mathbb{R}^{m-1}} \ldots e^{- 2 \pi i (y_1 x_1 + y_2 x_2 + \ldots + y_{m - 1} x_{m - 1})} (2\pi i y_m)\int_{-\infty}^{\infty} f(x) e^{-2 \pi i y_m x_m} dx_m \]

Видно, что константа вылезла (надо пропихнуть её влево до начала), и, фактически, получается обычная свёртка $f$! Доказали.

\textbf{2.} 

Давайте проверим:

\[\hat{f}(y) = \int_{\mathbb{R}^m} f(x)e^{-2\pi i (y_1 x_1 + y_2 x_2 + \ldots + y_m x_m)} dx\]
\[\frac{\partial \hat{f}}{\partial y_k}(y) = -2\pi i\int_{\mathbb{R}^m} x_kf(x)e^{-2\pi i (y_1 x_1 + y_2 x_2 + \ldots + y_m x_m)} dx\]

Получилось! А почему законно? Да по правилу Лейбница, функция то суммируемая!

\[|x_kf(x)e^{-2\pi i \sk{y}{x}}| \le |x| \cdot |f|\]

ч. т. д. 

\subsubsection{Следствие о преобразовании Фурье финитных и бесконечно гладких функций}
\textit{Формулировка:}

\begin{enumerate}
    \item $f \in L^1(\mathbb{R}^m)$ --- финитная. Тогда $\hat{f}(y) \in C^{\infty}(\mathbb{R}^m)$
    \item $f \in C^{\infty}$. Тогда $\forall p > 0: |y|^p \hat{f}(y)$ --- суммируемая в $\mathbb{R}^m$
\end{enumerate}

\textit{Доказательство:}

\textbf{1.}

Заметим, что если несколько раз дифференцировать по пункту 2 из теоремы, у нас всё время будут получаться преобразования Фурье одной и той же функции (шлавное, чтобы дифференцирвоать было можно) ):

\[\frac{\partial \hat{f}}{\partial y_k}(y) = \widehat{(-2\pi i)x_kf(x)}\]
\[\frac{\partial^2 \hat{f}}{\partial y_k \partial y_j}(y) = \widehat{(-2\pi i)^2x_kx_jf(x)}\]
\[\vdots\]

Ну супер, значит, $\hat{f} \in C^{\infty}(\mathbb{R}^m)$ --- дифференцируй сколько хочешь, это всего лишь домножение на координату и взятие очердного преобразования Фурье исходной функции. Только пункт 2 предыдущей теоремы можно применять, если $|x| f$ оказалась суммируемой. Заметим, что в пункте 1 у нас прекрасный класс функций --- финитные! Они внутри шарика что-то значат, а за пределами --- 0. Получается, что раз $f$ была суммируемой (по условию), домножив её на какую-то ограниченную величину ($|x|$ именно такова, ведь она не может уйти в бесконечность --- ограниченных шаров такого радиуса не бывает :) ), мы ей суммируемость точно не испортим. Ну, сдвинется она вверх-вниз немного, ничего страшного.

Итого: следствие из пункта 2 теоремы.

\textbf{2.}

Вспмоним страшное --- мультииндекс $\alpha$. $|\alpha| = x_1 + x_2 + \ldots + x_m, y^\alpha = y_1^{x_1}y_2^{x_2}\ldots y_m^{x_m}$ и запишем с его помощью, что же там у нас получается при многократном дифференцировании $f$ и взятии преобразования Фурье:

\[\widehat{\left(\frac{\partial^\alpha f}{\partial x^\alpha}\right)} = (2\pi i)^{|\alpha|} |y|^{\alpha} \hat{f}(y)\]

Причём, как мы помним из свойств преобразования Фурье, каждая такая функция ограничена. Помня о том, что нам нужно научиться доказывать для $\forall p : |y|^p$, ограничим каждую функцию таким образом:

\[\forall n > 0 \dbl \exists C \dbl \forall y \in \mathbb{R}^m: i \in [1, m]: |y_i|^n|\hat{f}(y)| < C \text{ и } |\hat{f}(y)| < C\]

Вообще, они могут быть ограничены разнми константами, ну давайте просто выберем из них наибольшую. Тогда суммируем неравенства и получаем: 

\[|\hat{f}(y)| \le \frac{C}{1 + |y_1|^n + |y_2|^n + \ldots + |y_m|^n}\]

Домножим на $|y|^p$:

\[|y|^p \cdot |\hat{f}(y)| \le \frac{C \cdot |y|^p}{1 + |y_1|^n + |y_2|^n + \ldots + |y_m|^n} = (*)\]

Отвлечёмся на момент, и посмотрим на это неравенство:

\[\max_{i} |y_i|^2 \ge \frac{|y|^2}{m}\]

Оно очевидно, так как мы взяли квадрат наибольшей координаты и сравнили его со средним квадратом по больнице. Возведём в $\frac{n}{2}$:

\[\max_{i} |y_i|^n \ge \frac{|y|^n}{m^{\frac{n}{2}}}\]

Подгоним наше неравенство:

\[(*) = \frac{C |y|^n m^{\frac{n}{2}}}{m^{\frac{n}{2}}|y|^{n - p} (\ldots)} \le \frac{Cm^{\frac{n}{2}}\max_{i}|y_i|^n}{|y|^{n - p}(\ldots)} = \frac{C_1}{|y|^{n - p}}\]

Мы оценили сверху, и упихали все константы в $C_1$ ($n$ у нас фиксировано на момент исследования, проверяется $p$). Теперь вспоминаем оценочное неравенство для интеграла по дополнению шара:

\[\int_{\mathbb{R}^m \setminus B(0, 1)} \frac{1}{|x|^q} d\lambda_m\]

Доказывается оно разложением в сферические координаты в $\mathbb{R}^m$:

\[\int_{\mathbb{R}^m \setminus B(0, 1)} = \int_{1}^{\infty} dr \int_{0}^{\pi} d\varphi_1\int_{0}^{\pi} d\varphi_2\ldots\int_{0}^{2\pi} d\varphi_{m - 1} \frac{1}{|r|^{p - m + 1}}(\sin \varphi_{1})^{m - 2} \ldots \]

Оно суммируемо при $q > m$ (можно посмотреть на аналогию при $m = 1$, чтобы убедиться в верности). Тогда, выбираем $n - p > m$ и получаем суммируемость нашей функции.

ч. т. д. 


\subsubsection{Лемма о функции с дифференцируемым преобразованием Фурье}

\textit{Эта лемма нам идеологически не нужна, просто пример использования преобразований Фурье}

\textit{Формулировка:}

\begin{itemize}
    \item $f \in L^1(\mathbb{R}^m) \ge 0$ (вещественная, не комплексная)
    \item $f(x) \sim \frac{1}{|x|^p} \quad x \rightarrow \infty$
    \item $\hat{f}(y)$ --- дифференцируема в точке $y  = 0$
\end{itemize}

Тогда: 

\[p > m + 1\]

\textit{Доказательство:}

Пусть $B = B(0, 1), \dbl V = \lambda_m(B)$. Посмотрим на такой интеграл:

\[I(r) = \frac{1}{V} \int_{B} \hat{f}(0) - \hat{f}(ry)dy\]

По определению дифференцируемости:

\[\hat{f}(y) = \hat{f}(0) + \text{ лин. часть от }y + \underbrace{o(y)}_{\alpha(y)|y|}, \quad y \rightarrow 0\]

Тогда, если выразить $\hat{f}(0) - \hat{f}(ry)$ и поинтегрировать, окажется, что линейная часть сократится (типа, интегрируем по симметричному шару, значит у нас эта линейная часть войдёт и с +, и с -. Да, машем руками как можем). И окажется, что:

\[\hat{f}(0) - \hat{f}(ry) = o(r), \quad r \rightarrow 0\]

\[I(r) = o(r)\]

Ну, окэй. Давайте оценим этот интеграл снизу:

\[\hat{f}(0) - \hat{f}(ry) = \int_{\mathbb{R}^m}\underbrace{f(x)\underbrace{(1 - e^{ - 2 \pi i \sk{x}{y}})}_{\le 2}}_{\le 2 ||f||_1} dx\]

Оценим разность этих преобразований, и окажется, что в скобочке всё меньше 2х (оцениваем экспоненту снизу), ну а функция оценивается нормой. Получается, она суммируемая. Тогда в $I(r)$ по теореме Фубини переставляем интегралы:

\[I(r) = \int_{B}\frac{1}{V} \int_{\mathbb{R}^m}f(x)(1 - e^{ - 2 \pi i r\sk{x}{y}}) dx dy = \int_{\mathbb{R}^m} f(x)\frac{1}{V}\int_{B}(1 - e^{ - 2 \pi i r\sk{x}{y}}) dy dx = \]

\[ = \int_{\mathbb{R}^m} f(x)\left(\frac{V}{V} - \frac{1}{V}\int_{B}\underbrace{e^{ - 2 r\pi i \sk{x}{y}}}_{\int_{\mathbb{R}^m} \chi_B(y)e^{ - 2 \pi ir \sk{x}{y}} = \widehat{\chi_B}(rx) }\right) dy dx = \int_{\mathbb{R}^m}f(x)\left(1 - \frac{1}{V}\widehat{\chi_B}(x)\right)dx = (*)\]

Заметим, что $\widehat{\chi_B}(x) \rightarrow 0$ при $x \rightarrow \infty$ (по свойствам преобразования Фурье). А также $|\widehat{\chi_B}| \le V$ (так как интегрируется экспонента, по величине не большая 1, по шару). Тогда давайте возьмём такое $R$, чтобы при $x > R$, $f(x) \ge \frac{1}{2|x|^p}$ (оценка из условия сработала) и $\widehat{\chi_B}(rx) \le \frac{V}{2}$. Тогда продолжаем оценку:

\[(*) \le \int_{|x| \ge \frac{R}{r}} f(x)\left(1 - \frac{1}{V}\widehat{\chi_B}(x)\right)dx = \int_{|x| \ge \frac{R}{r}} \frac{f(x)}{2}dx \ge \]

Дальше, мы закаывали чтобы оценки выполнялись при $|x| > R$, значит, надо, чтобы выполнялось $r \le 1$, тогда верно:

\[\ge \frac{1}{4} \int_{|x| \ge \frac{R}{r}} \frac{1}{|x|^p} dx \ge \]

Тогда применяем ``воспоминание'' из предыдущего следствия (про сферические координаты), и оцениваем. Заметьте, что $R$ фиксировано на уровне функции $f$, мы его вычисляем один раз:

\[\le C \int_{\frac{R}{r}}^{\infty} \frac{dt}{t^{p - m + 1}} = C_1 r^{ p - m + 1}\]

Ну вот, у нас с одной стороны:

\[I(r) \ge C_1 r^{p - m}\]

А с другой:

\[I(r) = o(r)\]

Значит, $p - m$ обязан быть больше 1, иначе ``о''-маленькое не случится.

ч. т. д. 
\subsubsection{Лемма ``о ядре Дирихле''}
\textit{Формулировка:}

\begin{itemize}
    \item $f \in L^1(\mathbb{R})$
    \item $x \in \mathbb{R}$
\end{itemize}

Тогда $\forall A > 0$:

\[I_A(f, x) = \int_{-\infty}^{\infty}f(x)\frac{\sin 2\pi A t}{\pi t}\]

\textit{Доказательство:}

Переброс шапочки (так можно по теореме о своёствах свёртки)!

\[\chi_A = \chi_{[-A, A]}\]

\[\int_{-A}^{A} \hat{f}(y)e^{2\pi i y x}dy = \int_{-\infty}^{\infty} \hat{f}(y)\chi_Ae^{2\pi i y x}dy = \int_{-\infty}^{\infty} f(y)\widehat{\chi_Ae^{2\pi i y x}}dy = (*)\]

Мы когда-то считали преобразование Фурье характеристичческой функции (см. в определении). Ну вот тут ровно оно, только у нас вылезла добавка (экспонента). Заметим, что это просто сдвиг:

\[\widehat{\chi_Ae^{2\pi i y x}} = \int_{-\infty}^{\infty} \chi_A(z)e^{2\pi i z x}e^{- 2 \pi i z y}dz = \int_{-\infty}^{\infty} \chi_A(z)e^{-2\pi i z (y - x)} = \hat{\chi_A}(y - x)\]

\[(*) = \int_{-\infty}^{\infty} f(y) \hat{\chi_A}(y - x) dy = \int_{-\infty}^{\infty} f(y) \frac{\sin 2\pi A (y - x)}{\pi(y - x)} dy = \]

По чётности правой функции меняем:

\[ = \int_{-\infty}^{\infty} f(y) \frac{\sin 2\pi A (x - y)}{\pi(x - y)} dy = \begin{dcases}
    t &= x - y\\
    dy &= -dt
\end{dcases} = \int_{-\infty}^{\infty} f(x - t) \frac{\sin 2\pi At}{\pi t} dt\]

(куда минус пропал после замены????)

ч. т. д. 

\textit{Следствие: }

\[\forall \delta > 0: I_A(f, x) = \int_{-\delta}^{\delta} f(x - t)\frac{\sin 2At}{\pi t} + o(1)_{A \rightarrow \infty}\]

\textit{Доказательство:}

Давайте посмотрим на функцию $g(t)$:

\[g(t) := \begin{dcases}
    \frac{f(x - t)}{\pi t} &t \in (-\infty, -\delta) \cup (\delta, \infty)\\
    0 &t \in (-\delta, \delta)
\end{dcases}\]

Заметим, что она суммируемая (ну, возле нуля там оно пытается уйти в бесконечноть, но мы просто отрезаем на дельта-окрестности, и зануляем; $f$ суммируемая по условию). Ну и всё, тогда запускаем теорему Римана-Лебега, которая говорит, что:

\[\int_{\mathbb{R}} g(t)\sin \underbrace{2\pi A}_{\lambda}t \goesto{\lambda \rightarrow \infty} 0\]

Это обосновывает ``о''-маленькое.

ч. т. д.

\textit{Замечание: }

Подобным образом возникает оценка на частиные суммы Фурье (в т. ч. из леммы об оценке ядра Дирихле):

\[\forall \delta > 0: S_n(f, x) = \int_{-\delta}^{\delta} f(x - t)\frac{\sin nt}{\pi t} + o(1)_{n \rightarrow \infty}\]

\subsubsection{Теорема о равносходимости ряда Фурье и интеграла Фурье}
\textit{Формулировка:}

\begin{itemize}
    \item $f \in L^1[-\pi, \pi]$, $f_0 \in L^1(\mathbb{R})$
    \item $x \in \mathbb{R}$
    \item Пусть $f(x) = f_0(x)$ в $U(x) = (x - \delta, x + \delta)$
\end{itemize}

Тогда в точке $x$ сходимость ряда Фурье равносильна сходимости интеграла Фурье, и если она есть, то:

\[\int_{-\infty}^{\infty}\hat{f}(y)e^{2 \pi i x y }dy = \sum_{k \in \mathbb{Z}} c_k(f_0) e^{ikx}\]

\textit{Доказательство:}
Проверим на $\delta$-окрестности $x$ (там $f = f_0$, поэтому не пишем индекс):

\[I_A(f, x) = \int_{-\delta}^{\delta}f(x - t)\frac{\sin 2 \pi A t}{\pi t} dt + o(1)\]
\[S_n(f, x) = \int_{-\delta}^{\delta}f(x - t)\frac{\sin n t}{\pi t} dt + o(1)\]


\[I_A(f, x) - S_{[2\pi A]}(f_0, x) \goesto{A \rightarrow \infty} 0\]

\textbf{1. $n = 2 \pi A \in \mathbb{N}$}

Всё работает, интегралы сократились, ``о''-шки занулились, так как гоним на бесконесность. Жалко, такое случается нечасто)

\textbf{2. $n = [2 \pi A], n \le 2\pi A < n + 1$}

\[\frac{n}{2 \pi} \le A < \frac{n}{2\pi} + 1, \quad \alpha = A - \frac{n}{2 \pi} \le \frac{1}{2\pi}\]
Разобьём интеграл Фурье на части:

\[I_A(f, x) = \int_{-A}^{A} \hat{f}(y)e^{2 \pi i y x} dy = \int_{-\frac{n}{2 \pi}}^{\frac{n}{2\pi}} + \int_{\frac{n}{2\pi}}^{\frac{n}{2\pi} + \alpha} + \int_{-\frac{n}{2 \pi} - \alpha}^{-\frac{n}{2 \pi}} = I_{\frac{n}{2\pi}}(f, x) + \int_{\frac{n}{2\pi}}^{\frac{n}{2\pi} + \alpha} + \int_{-\frac{n}{2 \pi} - \alpha}^{-\frac{n}{2 \pi}} \]

Ну всё, теперь опять оцениваем разность. Причём для интеграла Фурье, первый в разбиении считаем через ``Дирихле'', а остальные --- по определению:

\[I_A(f, x) - S_A(f, x) = \int_{\frac{n}{2\pi}}^{\frac{n}{2\pi} + \alpha} + \int_{-\frac{n}{2 \pi} - \alpha}^{-\frac{n}{2 \pi}} + o(1)\]

Заметьте, что сумму Фурье мы считаем именно от $A$. Мне кажется, что в лекции КПК ошибка, т. к. если считать в точке $[2 \pi A]$,  то первый интеграл из разложения $I_A$ не сократится. Осталось доказать, что эти вот миниинтегральчики стремятся к нулю, когда $A \rightarrow \infty$. Оценим один из них, без разницы какой:

\[\left|  \int_{\frac{n}{2\pi}}^{\frac{n}{2\pi} + \alpha} \hat{f}(y)e^{2 \pi i y x} dy\right| \le \int_{\frac{n}{2\pi}}^{\frac{n}{2\pi} + \alpha} |\hat{f}(y)| \cdot |e^{2 \pi i y x}|dy \le \int_{\frac{n}{2\pi}}^{\frac{n}{2\pi} + \alpha} |\hat{f}(y)| dy \le\]

Оценка на $\alpha$ сверху:

\[\le \sup_{y > \frac{n}{2\pi}} |\hat{f}(y)| \cdot \frac{1}{2\pi} \goesto{A \rightarrow \infty} 0\]

В супремуме границы ``с запасом'', можно было и просто по границам интегрирования гонять. Ну и всё, при $A \rightarrow \infty$ к бесконечности движется и $n$ (они связаны)! А $\hat{f}(y) \goesto{y \rightarrow \infty} 0$ по свойствам преобразования Фурье (следствие из теоремы Римана-Лебега).

ч. т. д. 

\subsubsection{Уравнение теплопроводности}
\textit{Формулировка:}

\begin{itemize}
    \item $t \ge 0$
    \item $x \in \mathbb{R}$
    \item $u|_{t = 0} = u_0$
    \item $\forall$ фиксированного $t \ge 0 \dbl u, u'_x, u''_{xx} \in L^1(\mathbb{R})$
    \item $u'_t$ имеет на каждом $[0, T]$ суммируемую мажоранту:
    
    \[\forall T \dbl \exists f_T : \forall t \in [0, T] \dbl \forall x \dbl |u'_t(x, t)| \le f_T(x) \in L^1(\mathbb{R})\]
\end{itemize}

\[\frac{\partial u}{\partial t} = \frac{\partial^2 u}{\partial x^2}\]

\textit{Доказательство:}

Ну, нам надо найти $u$, вспоминаем диффуры :).

Сначала, раз уж у нас есть такое равенство, давайте применим к обоим частям преобразования Фурье (по $x$):

\[(u''_{xx})^{\hat{\,}}(y, t) = (4 \pi^2 i^2 y^2)\hat{u}(y, t) = - 4 \pi^2 y^2 \hat{u}(y, t)\]

Тут мы применили теорему о преобразовании Фурье и дифференцировании (пункт 1).

\[\hat{u'_t}(y, t) = \int_{-\infty}^{\infty} u'_t(x, t)e^{-2\pi i yx} dx = \frac{\partial}{\partial t} \left(\int_{-\infty}^{\infty} u(x, t)e^{-2\pi i yx} dx \right) = \hat{u}'_t(y, t)\]

Ну а здесь дифференцирование было по $t$, поэтому мы получили фактически правало Лейбница (оно выполняется по пункту про суммируемые мажоранты по условию).

Супер, давайте приравняем:

\[- 4 \pi^2 y^2 \hat{u}(y, t) = \hat{u}'_t(y, t)\]

Получилось обычное дифференциальное уравнение. Заметим, что в точке $t = 0$ мы знаем решение (по условию, с преобразованием Фурье и подстановкой проблем не будет, преобразовали-подставили):

\[\hat{u}(y, 0) = \widehat{u_0}(y)\]

Ну и что же может быть решением нашего диффура? Мы продифференцировали, вылезла константа и функкция осталась такой же. Да это же экспонента!

\[\hat{u}(y, t) = e^{-4\pi^2 y^2 t}\hat{u_0}(y) = (*)\]

Не забываем домножить на изначальное решение, чтобы при $t = 0$, когда экспонента уйдёт в 1, решение оставалось корректным. Ну всё, супер, мы нашли преобразование Фурье. Теперь надо найти саму функцию $u(x, t)$. Вспомним пример 2 из определения преоборазования Фурье:

\[f_a(x) = e^{-\pi a^2x^2}, \quad \widehat{f_a}(y) = \frac{1}{a}f_{\frac{1}{a}}(y)\]

Подгоняем под наш случай (хотим запихать экспоненту под преобразование):

\[a = \sqrt{4 \pi t}, \quad \left(\underbrace{\frac{1}{\sqrt{4 \pi t}} e^{-\frac{\pi x^2}{4 \pi t}}}_{g(x, t)}\right)^{\hat{\,}} = e^{-4\pi^2 x^2 t}\]

\[(*) = \hat{g}(y, t) \cdot \widehat{u_0}(y) = \widehat{g * u_0}(y, t)\]

Последнее по теореме для преобразования Фурье свёртки. Ну что ж, применяем обратное преобразование и получаем:

\[u(x, t) = \int_{\mathbb{R}} \frac{1}{\sqrt{4 \pi t}} e^{-\frac{\pi (x - s)^2}{4 \pi t}} u_0(s)ds\]

ч. т. д. 

\subsubsection{Признак Дирихле--Жордана}
\textit{Формулировка:}

\begin{itemize}
    \item $f \in L^1[-\pi, \pi]$ (или $f \in L^1(\mathbb{R})$)
    \item $f$ имеет ограниченную вариацию в окрестности точки $x \in \mathbb{R}$
\end{itemize}

Тогда:

\[S_n(f, x) \ntoinf \frac{f(x + 0) + f(x - 0)}{2}\]

(вместо $S_n(f, x)$ работает и $I_A(f, x)$)

\textit{Доказательство (рукомахательское):}

Напоминание о давно забытом прошлом:

\[\Var{a}{b} f  = \sup_{\text{дроблениям } x} \sum_{N} |f(x_{i + 1}) - f(x_i)|\]

Утверждается, что функцию с конечной вариацией можно представить в виде линейной комбинации двух монотонно убывающих. Как? Ну, сначала представим в виде разности двух возрастающих $g - h$. Так точно можно, набросок доказательство:

\[g(x) := \Var{a}{x} f + f(x), \quad h(x) := \Var{a}{x} f + f(x)\]

\[\forall y > x : g(y) - g(x) = \Var{a}{y} f + f(y) - \Var{a}{x} - f(x) = \Var{x}{y} f + (f(y) - f(x)) \ge 0\]

Например, рассмотрим функцию $g$ и докажем её монотонность. Взяли две точки, одну больше другой, и смотрим на разность значений функций в них. По аддитивности вариации сводим, и видим, что у нас получается разница вариации (супремума по всем дроблениям) и какого-то одного конкретного дробления $\{x, y\}$. Очевидно, что их разность не может быть меньше нуля, в крайнем случпе они равны. С $h$ аналогично. Получается, они возрастающие.

Так, и как же нам представить нашу функцию в виде двух убывающих? Выясняется, что при желании можно это сделать. Раз у нас функция ограниченной вариации, она ограничена (вроде очев). Ну дак давайте зададим так:

\[\psi(x) = (C_g - g(x)) + (C_h - h(x))\]

Где констаты подобраны так, чтобы всегда быть больше соответствующих функций. Ну, вроде справились, что-то такое.

Так, теперь надо доказать признак. Применим трюк как в доказательстве признака Дини. Пусть $\varphi(x, t) = f(x + t) + f(x - t)$, тогда:

\[S_n(f, x) = \int_{-\delta}^{\delta} f(x - t) \frac{\sin nt}{\pi t} dt + o(1) = \int_{0}^{\delta} \varphi(x, t) \frac{\sin nt}{\pi t} dt + o(1)\]

Заметили? Мы подменили границы интегрирования, раздробив на два интеграла $\int_{0}^{\delta} + \int_{-\delta}^{0}$. Теперь воспользуемся ограниченной вариативностью, и скажем, что $\varphi$ представляется в виде линейной комбинации убывающих функций $\{\tilde{\varphi}\}$ ($x$ фиксирован). Посмотрим на интеграл с одной из таких:

\[\int_{0}^{\delta} \tilde{\varphi}(t)\frac{\sin nt}{\pi t}dt + o(1) = [\Phi(t) = \tilde{\varphi}(t)\chi_{[0, \delta]}(t)] = \int_{0}^{\infty} \Phi(t) \frac{\sin nt}{\pi t} dt + o(1) = \]

Если что, мы бесплатно дополнили функцию $\Phi(t)$ на всём промежутке, где она не определена нулём, это никак не влияет на её монотонность (когда разбивали в линейную комбинацию убывающих фукнций можно было пошаманить с константой, подвинуть вверх, все дела. Вообще, можно считать, что она неотрицательная, за счёт наших сдвигов). Замена:

\[= \begin{dcases}
    \tau = nt \\
    dt = \frac{d\tau}{n}
\end{dcases} = \int_{0}^{\infty} \Phi\left(\frac{\tau}{n}\right) \frac{\sin \tau}{\pi \tau} d\tau + o(1)\]

Ну, собственно, тут становится понятно, зачем мы поменяли границы интегрирования на именно от 0 до бесконечности: справа от $\Phi\left(\frac{\tau}{n}\right)$ стоит интеграл Дирихле! Можно сказать, что он сходится независимо от $n$. А $\Phi$ --- монотонная (убывающая) и ограниченая (как минимум нулём). Получается, по теореме Абеля такой интеграл сходится равномерно. Тогда законен предельный переход по $n$ (по теореме о непрерывном переходе в несоственно интеграле). Переходим:

\[\Phi(+ 0) \frac{\pi}{\pi2} = \frac{\Phi(0)}{2}\]

Ну вот. Мы собираем всю нашу линейную комбинацию теперь, после предельного перехода, и получаем то, что хотели (на 2 поделилось, $\varphi(0)$ --- то, что надо).

ч. т. д. 


\end{document}

\begin{comment}
    \subsubsection{Теорема X-N}
    \textit{Формулировка:}

    \begin{itemize}
        \item 
    \end{itemize}

    \textit{Доказательство:}
\end{comment}

