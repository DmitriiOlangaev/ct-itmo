\documentclass{article}
\usepackage[utf8]{inputenc}
\usepackage[T2A]{fontenc}
\usepackage[russian]{babel}
\usepackage{amsfonts}
\usepackage{amsmath}
\usepackage{amssymb}
\usepackage{arcs}
\usepackage{fancyhdr}
\usepackage{float}
\usepackage[left=3cm,right=3cm,top=3cm,bottom=3cm]{geometry}
\usepackage{graphicx}
\usepackage{hyperref}
\usepackage{multicol}
\usepackage{stackrel}
\usepackage{xcolor}
\usepackage{epigraph}
\usepackage{tikz}
\usepackage{amsthm}
\usepackage{graphics}
\usepackage{draftwatermark}
\usepackage{ marvosym }
\usepackage{physics}
\usepackage{pdfpages}
\usepackage{ulem}


\def\letus{%
\mathord{\setbox0=\hbox{$\exists$}%
         \hbox{\kern 0.125\wd0%
               \vbox to \ht0{%
                  \hrule width 0.75\wd0%
                  \vfill%
                  \hrule width 0.75\wd0}%
               \vrule height \ht0%
               \kern 0.125\wd0}%
       }%
        }
\def\dbl{\,\,}
\def\image#1{\includegraphics[width=\linewidth]{static/#1}}
\def\imageh#1{\includegraphics[width=\linewidth, height=0.95\textheight]{static/#1}}
\def\images#1#2{\begin{center}\includegraphics[width=#1\linewidth]{static/#2}\end{center}}

\def\rsh#1{\underset{#1}{\rightrightarrows}}
\def\rshe{\rsh{E}}
\def\eqby#1{\underset{#1}{=}}
\def\rinf{\overline{\mathbb{R}}}
\def\goesto#1{\underset{#1}{\longrightarrow}}
\def\toinf#1{\goesto{#1 \rightarrow \infty}}
\def\ntoinf{\toinf{n}}

\DeclareMathOperator{\sign}{sign}
\DeclareMathOperator{\const}{const}
\DeclareMathOperator{\segm}{Segm}\DeclareMathOperator*{\esssup}{ess\,sup}
\DeclareMathOperator{\supp}{supp}



\newcommand*\lateraleye{%
       \scalebox{0.15}{
    \tikzset{every picture/.style={line width=0.75pt}} 
    \begin{tikzpicture}[x=0.75pt,y=0.75pt,yscale=-1,xscale=1]
    \draw  [line width=1.5]  (300,100.33) .. controls (326,122) and (352,135) .. (378,139.33) .. controls (352,143.67) and (326,156.67) .. (300,178.33) ;
    \draw  [fill={rgb, 255:red, 0; green, 0; blue, 0 }  ,fill opacity=1 ] (308.94,116.33) .. controls (313.87,116.33) and (317.86,125.51) .. (317.85,136.83) .. controls (317.84,148.15) and (313.84,157.33) .. (308.91,157.33) .. controls (303.99,157.32) and (300,148.14) .. (300.01,136.82) .. controls (300.02,125.5) and (304.02,116.32) .. (308.94,116.33) -- cycle ;
    \draw  [draw opacity=0][line width=1.5]  (314.84,166.6) .. controls (311.87,164.64) and (309.14,162.18) .. (306.76,159.24) .. controls (295.12,144.82) and (296.6,124.33) .. (310.07,113.45) .. controls (311.48,112.32) and (312.96,111.33) .. (314.5,110.49) -- (331.14,139.55) -- cycle ; \draw  [line width=1.5]  (314.84,166.6) .. controls (311.87,164.64) and (309.14,162.18) .. (306.76,159.24) .. controls (295.12,144.82) and (296.6,124.33) .. (310.07,113.45) .. controls (311.48,112.32) and (312.96,111.33) .. (314.5,110.49) ;
    \draw  [fill={rgb, 255:red, 255; green, 255; blue, 255 }  ,fill opacity=1 ] (304.43,124.2) .. controls (306.09,124.25) and (307.32,128.01) .. (307.18,132.6) .. controls (307.05,137.19) and (305.59,140.88) .. (303.93,140.83) .. controls (302.27,140.78) and (301.03,137.02) .. (301.17,132.43) .. controls (301.31,127.83) and (302.76,124.15) .. (304.43,124.2) -- cycle ;
    \end{tikzpicture}
    }\,}
    
\def\D{\,\mathrm{d}}

\let\vanillaparagraph\paragraph
\let\vanillasubparagraph\subparagraph
\renewcommand{\paragraph}[1]{\vanillaparagraph{#1}\mbox{}\\}
\renewcommand{\subparagraph}[1]{\vanillasubparagraph{#1}\mbox{}\\}

\graphicspath{{./images/}}

\setlength{\parindent}{0pt}

\setcounter{tocdepth}{4}
\setcounter{secnumdepth}{4}

\SetWatermarkText{$\underset{\text{@imodre @snitron}}{\text{ПРОДАМ ГАРАЖ}}$}
\SetWatermarkScale{2}
\SetWatermarkLightness{0.9}

\begin{document}
\DraftwatermarkOptions{stamp=false}
\begin{titlepage}
    \centering
    \vspace*{\baselineskip}
    \rule{\textwidth}{1.6pt}\vspace*{-\baselineskip}\vspace*{2pt}
    \rule{\textwidth}{0.4pt}\\[\baselineskip]
    {\LARGE СВЯТОЙ КПК\\ [0.3\baselineskip] \#BlessRNG}\\[0.2\baselineskip]
    \rule{\textwidth}{0.4pt}\vspace*{-\baselineskip}\vspace{3.2pt}
    \rule{\textwidth}{1.6pt}\\[\baselineskip]
    \scshape
    Или как не сдохнуть на 4 семе из-за матана \\
    \vspace*{2\baselineskip}
    Разработал \\[\baselineskip]
    {\Large Никита Варламов\quad @snitron}
        \vspace*{2\baselineskip}\par
    Почётный автор \\[\baselineskip]
    {\Large Тимофей Белоусов\quad @imodre}
    \vfill
    v0.0\\
    {\scshape Февраль-??? 2023} \par
\end{titlepage}

Вы в любой момент можете добавить любую недостающую теорему, затехав её и отправив код (фотографии письменного текста запрещены) в телегу любому из указанных авторов, или создав Pull Request в \href{https://github.com/snitron/ct-itmo}{Git-репозиторий конспекта (click)}. Ваше авторство также будет указано, с вашего разрешения.

\newpage

\begin{flushright}
\emph{Ah shit\\
Here we go again!\\
And again...\\
Oh, fuck.}
\end{flushright}


\tableofcontents


\setlength{\parskip}{6pt}%
\newpage
\DraftwatermarkOptions{stamp=false}


\section{Период Палеозойский}
\subsection{Важные определения}

\subsubsection{Пространство $L^p(E,\mu)$}
$1 \le p < +\infty$, $(X, \mathfrak{A}, \mu)$, $E \in \mathfrak{A}$

Тогда $\mathfrak{L}_p(E, \mu) = \{f:$ почти всех $E \rightarrow \mathbb{R} (\mathbb{C}), f$ --- измерима. $\int_{E} |f|^{p} < + \infty\}$

\begin{enumerate}
    \item $\mathfrak{L}_p(E, \mu)$ --- линейное пространство
    \item $f \equiv g$, если $f = g$ почти всюду
\end{enumerate}

$L_p := \mathfrak{L}_p /_{\equiv}$ --- точки этого пространства

$[f] = \{g: f \equiv g\}$
$[f_1] + [f_2] = [f_1 + f_2]$

И введём норму $||[f]|| = \left(\int_{E} |f|^{p}\right)^{\frac{1}{p}}$

\subsubsection{Пространство $L^\infty(E,\mu)$}

$\mathfrak{L}^{\infty}(E, \mu) = \{f: $ почти всех $E \rightarrow \mathbb{R} (\mathbb{C})$, измерима, $\esssup |f| < + \infty\}$

$||f||_{\infty} = \esssup f$

\textbf{Дописать всё}

\subsubsection{Существенный супремум}

$\esssup f = \inf \{a: f \le a $ почти всюду $\}$

$a$ --- существенная верхняя граница функции $f$, если при почти всех $x \dbl f(x) \le a$

\textit{Свойства: }
\begin{enumerate}
    \item $\esssup f(x) \le \sup f(x)$
    \item при почти всех $x: f(x) \le \esssup f(x)$
    \item $f$ --- суммируемая, $g$ --- измерима: $\esssup |g| < + \infty$
    \[\left|\int_{E} fg\right| \le \esssup |g| \cdot \int_{E} |f|\]
\end{enumerate}

\subsubsection{Гильбертово пространство}

$\mathfrak{H}$ --- линейное пространство, в котором задано скалярное произведение и соответствующая норма. Если $\mathfrak{H}$ --- полное, то оно называется гильбертовым.

\subsubsection{Ортонормированная система, примеры}

${e_k}$ --- О. С. , тогда ${\frac{e_k}{|| e_k ||}}$ --- ортонормированная система.

Примеры: 

\begin{enumerate}
    \item $l^2 \quad e_k = (0, \ldots, 0, 1, 0, \ldots)$
    \item $L^2[0, 2\pi] \quad \{1, \cos t, \sin t, \cos 2t, \sin 2t, \cos 3t, \sin 3t, ldots\}$
    \item $\left(\frac{e^{ikt}}{\sqrt{2\pi}}\right)_{k \in \mathbb{Z}}$
\end{enumerate}
\newpage

\subsection{Определения}
\subsubsection{Произведение мер}
$(X, \mathfrak{A}, \mu)$, $(Y, \mathfrak{B}, \nu)$ --- пространства с мерой.

\textit{Лемма: }
$\mathcal{A}, \mathcal{B}$ --- полукольца. Тогда $\mathcal{A} \times \mathcal{B} = \{A \times B, A \in \mathcal{A}, B \in \mathcal{B}\}$ --- полукольцо.

Также, множества из $\mathcal{A} \times \mathcal{B}$ являются измеримыми прямоугольниками.

$\mu, \nu$ --- $\sigma$-конечные меры. Тогда стандартное продолжение $m_{0}$ (в смысле теоремы о продолжении меры (?)) с полукольца $\mathfrak{A} \times \mathfrak{B}$, определённой на некоторой $\sigma$-алгебре $\mathfrak{A} \otimes \mathfrak{B}$, и являющееся $\sigma$-конечной полной мерой --- обзначается просто $m$.

И тогда $m$ --- и есть произведение мер $\mu$ и $\nu$ $(\mu \times \nu)$.

\textit{Замечание: }

\[(\mu \times \nu) \times \rho = \mu \times (\nu \times \rho)\]

\subsubsection{Сечения множества}

$X, Y$ --- множества. $C \subset X \times Y$

Тогда: 

\[C_{x} := \{y \in Y: (x, y) \in C\}\]

\[C^{y} := \{x \in X: (x, y) \in C\}\]

--- сечения множества $C$ (1 и 2 рода)

\textit{Замечания: }

\[\left(\bigcup_{\alpha \in A} C_{\alpha}\right)_{x} = \bigcup_{\alpha \in A} \left(C_{\alpha}\right)_{x}\]

\[\left(\bigcap_{\alpha \in A} C_{\alpha}\right)_{x} = \bigcap_{\alpha \in A} \left(C_{\alpha}\right)_{x}\]

\[\left(C \setminus C'\right)_{x} = C_{x} \setminus C'_{x}\]

\subsubsection{Полная мера, сигма-конечная мера}

См. \href{http://gg.gg/holykpksem3}{\color{blue}{конспект прошлого семестра}}

\subsubsection{Образ меры при отображении}

Пусть у нас есть $(X, \mathfrak{A}, \mu)$, $(Y, \mathfrak{B}, \_ )$ --- пространства с мерой, $\Phi: X \rightarrow Y$.

\begin{enumerate}
    \item $\forall \Phi \quad \Phi^{-1}(\mathfrak{B})$ --- $\sigma$-алгебра (это предлагается доказать как уражнение)
    \item Пусть $\Phi$ --- ``измеримо'' $\left(\Phi^{-1}(\mathfrak{B}) \subset \mathfrak{A}\right)$
\end{enumerate}

Для $E \in \mathfrak{B}$ зададим $\nu R := \mu\left(\Phi^{-1}(E)\right) = \int_{\Phi^{-1}(E)} 1 d \mu$

$\nu$ --- образ меры $\mu$ при отображении $\Phi$

\textbf{NB: ДОПИСАТЬ НА СЕССИИ, ТУТ ЕЩË ЕСТЬ ДОКАЗАТЕЛЬСТВО, ЧТО ЭТО МЕРА}

\subsubsection{Взвешенный образ меры}

$\omega: X \rightarrow \mathbb{R} \ge 0$, измерима на $X$

$B \in \mathfrak{B}, \tilde{\nu}(B) := \int_{\Phi^{-1}(B)} \omega d\mu$ --- тоже мера, это и есть взвешенный образ меры $\mu$ при отображении $\Phi$

\subsubsection{Плотность одной меры по отношению к другой}

$X = Y, \mathfrak{A} = \mathfrak{B}, \Phi = $ id

$\nu b = \int_{B} \omega d \mu$ --- ещё одна мера в $X$

Здесь $\omega$ называется плотностью меры $\nu$ относительно меры $\mu$. И в этом случае:

\[\int_{X}f(x)d\nu(x) = \int f(x) \cdot \omega(x) d\mu(x)\]

\subsubsection{Сферические координаты в $\mathbb{R}^3$}

На основе земных координат, $\varphi \in (0, 2\pi), \dbl \theta \in \left(-\frac{\pi}{2}, \frac{\pi}{2}\right)$

\images{0.3}{sph_3d.jpg}

\[x = \rho \cos \varphi \cos \theta\]
\[y = \rho \cos \varphi \sin \theta\]
\[z = \rho \sin \varphi\]
\[J_{\Phi} = \rho ^ 2 \cos \varphi\]

\subsubsection{Сферические координаты в $\mathbb{R}^m$}


\subsubsection{Условие $L_{loc}$}

$f: X \times \tilde{Y} \rightarrow \rinf, Y \subset \tilde{Y}, a$ --- предельная точка $Y$ в $\tilde{Y}$. 

$f$ удовлетворяет условию $L_{loc}(a): \exists g: X \rightarrow \rinf$ --- суммируемая, $\exists U(a): \forall$ почти всех $x \forall y \in U(a)$:

\[|f(x, y)| \le g(x)\]

\subsubsection{Интеграл комплекснозначной функции}

База базовая: $(X, \mathfrak{A}, \mu), f: X \rightarrow \mathbb{C}$

\[\int_{E}f(z)d\mu = \int_{E}\Re(f(z))d\mu + i\int_{E}\Im(f(z))d\mu\]

Также измеримость и суммируемость следует из соттветствующих свойств реальной и мнимой частей функций.

\subsubsection{Фундаментальная последовательность, полное пространство}

$A \subset X$ --- нормированное пространство

$A$ --- (всюду) плотное в $X$

\[\forall x \in X \dbl \exists \varepsilon > 0 \quad B(x, \varepsilon) \cap A\text{--- непусто}\]

\subsubsection{Мера Лебега-Стилтьеса, мера Бореля-Стилтьеса}

\begin{enumerate}
    \item $\mathcal{P}^{1}, g: \mathbb{R} \rightarrow \mathbb{R}, $ возрастает, непрерывно
    \[\mu_g[a, b) := g(b) - g(a)\]

    --- счётно аддитивная мера
    \item $g$ --- возрастает, не обязательно непрпрывно
    
    \[\mu_g[a, b) = g(b - 0) - g(a - 0)\]

    --- мера
\end{enumerate}

Запускаем теорему о продолжении, тогда

$\exists \mathfrak{A} \supset \mathcal{P}^{1} \exists$ продолжение $\mu_g \subset \mathcal{P}$ на $\mathfrak{A}$

$\mu_g$ --- полная мера на $\mathfrak{A}$ --- мера Лебега-Стильтьеса

Если нассмотреть $\mu_g$ на борелевском $\mathfrak{B} \rightarrow \rinf$ --- мера Бореля


\subsubsection{Функция распределения}
$(X, \mathfrak{A}, \mu)$, $h: X \rightarrow \rinf$, измерима, вочти всюду конечна

$\forall t \in \mathbb{R} \quad \mu X(h < t) < +\infty$

Пусть $H(t) = \mu X(h < t)$ --- возрастающая

$H(t)$ --- называется функцией распределеиния по мере $\mu$

\subsubsection{Ортогональный ряд}

Ряд $\sum a_k$ --- ортогональный, если $\forall k, l a_k \perp a_l$

\subsubsection{Сходящийся ряд в гильбертовом пространстве}

$\sum a_n, a_n \in \mathfrak{H}$

$S_N := \sum_{1 \le n \le N} a_n$, если $\exists S \in \mathfrak{H}: S_N \goesto{\mathfrak{H}} S$

Такой ряд называется сходящимся.

\subsubsection{Ортогональная система (семейство) векторов}

${e_k} \subset \mathcal{H}$ --- ортогональная система, если:

\begin{enumerate}
    \item $k \neq j \dbl e_k \perp e_j$
    \item $\forall k \dbl e_k \neq 0$
\end{enumerate}

\subsubsection{Коэффициенты Фурье}

\[c_k(x) = \frac{\langle x, e_k \rangle}{||e_k||^2}\]

--- коэффициент Фурье вектора $x$ по О. С. $e_k$

\subsubsection{Ряд Фурье в Гильбертовом пространстве}

\[c_k \cdot e_k\]
--- ряд Фурье ветора $x$ по О. С. $e_k$

\newpage

\subsection{Важные теоремы}

\subsubsection{Теорема Лебега о мажорированной сходимости для случая сходимости почти везде}
\textit{Формулировка:}

\begin{itemize}
    \item $(X, \mathfrak{A}, \mu)$ --- пространство с мерой
    \item $f_n, f: X \rightarrow \rinf$ --- измеримые
    \item $f_n \rightarrow f$ почти всюду
    \item $\exists g: X \rightarrow \rinf$ --- суммируемая, и $\forall n$ и при почти всех $x \dbl |f_n(x)| \le g(x)$
\end{itemize}

Тогда:

\[\int_{X}|f_n - f| d \mu \dbl \ntoinf \dbl 0\]

И, как очевидное (``уж тем более''):

\[\int_{X} f_n d\mu \dbl \ntoinf \dbl \int_{X} f d\mu\]


\textsc{Дисклеймер:}

Развеем все сомнения насчёт корректности условия (вдруг они у вас были):

\[\left| \int f_n - \int f \right| = \left| \int f_n - f \right| \le \int |f_n - f| \text{ (уж тем более)}\]

А также, наши функции из условия на самом деле даже суммируемые, не просто измеримые. Давайте для каждого $n$ соберём точки, на который $f_n$ не сходится к $f$, сложим (это всё будет множемтво меры 0) и вычтем, а на остатке сделаем предельный переход:

\[|f_n(x)| \le g(x)\]
\[|f(x)| \le g(x) < +\infty\]

\textit{Доказательство:}

Заведём последовательность $h_n := \sup (|f_n - f|, |f_{n + 1} - f|, |f_{n + 2} - f|, \ldots)$. Она убывает, так как по условию у нас есть сходимость почти везде. Также, можно ограничить её: $0 \ge h_n \ge 2 g$ (модули больше нуля и по условию все $|f_n| \ge g$). А ещё это просто определение последовательности из верхнего предела:

\[\lim_{n \rightarrow \infty} h_n = \overline{\lim_{n \rightarrow \infty}} |f_n - f| = 0 \text{ (почти везде)}\]

Теперь берём положительную возврастающую последовательность $2g - h_n$ и запускаем теорему Леви (см. 3 семестр, там как раз нужна возрастающая последовательность):

\[\int_{X} (2g - h_n) d\mu \ntoinf \int_{X} 2g d\mu\]

Откуда по линейности первого интеграла следует, что $\int_{X} h_n \ntoinf 0$, ну и добиваем:

\[0 \underset{n \rightarrow \infty}{\longleftarrow} \int_{X} h_n \ge \int_{X} |f_n - f| d \mu\]

ч. т. д. 

\subsubsection{Теорема Лебега о мажорированной сходимости для случая сходимости по мере}
\textit{Формулировка (то же самое, что и выше, только сходится по мере теперь):}

\begin{itemize}
    \item $(X, \mathfrak{A}, \mu)$ --- пространство с мерой
    \item $f_n, f: X \rightarrow \rinf$ --- измеримые
    \item $f_n \underset{\mu}{\Longrightarrow} f$
    \item $\exists g: X \rightarrow \rinf$ --- суммируемая, и $\forall n$ и при почти всех $x \dbl |f_n(x)| \le g(x)$
\end{itemize}

Тогда:

\[\int_{X}|f_n - f| d \mu \dbl \ntoinf \dbl 0\]

\textit{Доказательство:}

Рассмотрим 2 случая.

\textbf{1. $\mu X < + \infty$}

Зафиксируем $\varepsilon > 0$ и сооружаем множества $X_n = X(|f_n - f| \ge \varepsilon)$. Сделовательно, $\mu X_n \ntoinf 0$, т.к. есть сходимость по мере. Расписываем:

\[\int_{X} |f_n - f| d\mu = \int_{X_n} |f_n - f| d\mu + \int_{X^c_n} |f_n - f| d\mu \le \underbrace{\int_{X_n} 2g d\mu}_{(1)} + \underbrace{\int_{X^c_n} \varepsilon d\mu}_{(2)}\]

$(1)$ --- оценка разности по условию, и ещё при больших $n$ меньше эпсилона по абсолютной непрерывности интеграла. (2) --- из условия о сходимости по мере выше оцениваем эпсилоном.

\[\le \varepsilon + \varepsilon \cdot \mu X_n^c \le \varepsilon \cdot (1 + \mu X)\]

(оцениваем меру дополнения просто всем пространством)

\textbf{2. $\mu X = \infty$}

Сначала докажем небольшое свойство интеграла по мере:

\[\forall \varepsilon > 0 \dbl \exists A \subset X \text{ измеримое } \mu A < + \infty \quad \int_{X \setminus A} g < \varepsilon\]

Если по-русски, то существует некоторое множество в исходном, на котором в основном концентрируется интеграл, следовательно, на остальном кусочке интеграл крайне мал. И мы можем предъявить такое для сколь угодно малого $\varepsilon$.

Рассмотрим интеграл как супремум ступенчатых функций:

\[\int_{X} g = \sup_{0 \ge g_n \ge |g|} \int_{X} g_n d\mu\]

Этот супремум значит, что существует какая-то $g_{n_0}$, хорошо ($\varepsilon$) оценивающая нашу функцию:

\[\exists g_{n_0}: \dbl \int_{X} g - g_{n_0} < \varepsilon\]

Давайте возьмём за $A$ носитель функции $g_{n_0}$:

\[A := \supp g_{n_0} = \{x: g_{n_0}(x) \neq 0\}\]

Так как ступенчатая функция есть сумма константы на характеристическую функцию, её интеграл конечен (?). Ну, а на ``хвостиках'' где она равна нулю нам не особо интересно. Таким образом, $\mu A < + \infty$:

\[\int_{X \setminus A} g d\mu = \int_{X \setminus A} g \underbrace{- g_{n_0}}_{\text{так как вне } A \,\, g_{n_0} = 0} \le \int_{X} g - g_{n_0} < \varepsilon\]

Ну и всё, раз доказали, давайте разобъём на два интеграла:

\[\int_{X} |f_n - f| d\mu = \underbrace{\int_{A} |f_n - f| d\mu}_{< \varepsilon \text{ по пункту 1}} - \underbrace{\int_{X \setminus A} |f_n - f| d\mu}_{<2 \varepsilon \text{ по доказанному выше}} \le 3 \varepsilon\]

ч. т. д. 

\subsubsection{Принцип Кавальери}
\textit{Формулировка:}

\begin{itemize}
    \item $(X, \mathfrak{A}, \mu), (Y, \mathfrak{B}, \nu)$ --- пространства с мерой
    \item $\mu, \nu$ --- $\sigma$-конечные, полные меры
    \item $C \in \mathfrak{C}$
    \item $m = \mu \times \nu, \mathfrak{C} = \mathfrak{A} \otimes \mathfrak{B}$
\end{itemize}

Тогда: 

\begin{enumerate}
    \item при почти всех $x \quad C_{x} \in \mathfrak{B}$ 
    \item $x \mapsto \nu C_{X}$ --- измеримо на $X$ (сама функция задана почти везде)
    \item $m C = \int_{X} \nu (C_{x}) d\mu(x)$
\end{enumerate}

Аналогично для сечений $C^{y}$

\textit{Замечания:}
\begin{enumerate}
    \item $C$ --- измеримо $\nRightarrow \forall x C_x$ --- измеримое 
    \item $\forall x \forall y: C_x, C^y$ --- измеримы $\nRightarrow C$ --- измеримо
\end{enumerate}

\textit{Доказательство:}

Введём $D$ --- это множество тех множеств, которые удовлетворяют принципу Кавальери :) . Давайте докажем, что разные типы множеств содержатся в $D$. А потом (внезапно) окажется, что это все множества.

\textbf{1. $G = A \times B$ (измеримые прямоугольники)}

Проверяем здесь и далее попунктно:

\begin{enumerate}
    \item Так как это прямоугольники, $C_x = \begin{cases}
        B, \dbl x \in A\\
        \varnothing, \dbl x \notin A
        \end{cases}$ (очев). Ну, значит, при всех $x: \dbl C_x \in \mathfrak{B}$
    \item Берём в качестве такой функции $\nu(B) \cdot \chi_A(x)$. Она измерима на $X$.
    \item Ну давайте поинтегрируем) $\int_{X} \nu(C_x) d\mu = \int_{X} \nu(B) \cdot \chi_A(x) d\mu = \nu(B) \cdot \mu(A) = m(A \times B)$
\end{enumerate}

\textbf{2. $E_i \in D, E_i$ дизъюнктны, $E = \bigsqcup_{\text{НБЧС}} E_i$. Тогда $E \in D$}

\begin{enumerate}
    \item $E_x = \bigsqcup (E_i)_x$. Обратите внимание, что все множества справа уже лежат в $D$, поэтому они ``измеримы'' (лежат в $\mathfrak{B}$) при почти всех $x$. Ну, значит и объединение их тоже.
    \item Если вы ещё не поняли, мы в этом пункте фактически хотим предоставить функцию вычисления меры сечения по заданному $x$. $\nu E_x = \sum \nu E_{i_x}$. Это сумма измеримых неотрицательных функций, определённых на почти всех $x$ (потому что кусочки уже лежат в $D$).
    \item $\int_{X} \nu(C_x) d\mu = \int_{X} \sum \nu E_{i_x} d\mu$. Тут напрашивается переставить местами сумму и интегрирование, и это можно сделать по теореме об интегрировании положительных рядов!. $\sum \int_{X} \nu E_{i_x} d\mu = \sum mE_i = (*)$ (кусочки уже в $D$, и по счётной аддитивности) $(*) = m E$
\end{enumerate}

\textbf{3. $E_i \in D, \dbl E_i \supset E_{i + 1} \supset \ldots, \dbl \bigcap E_i = E, \dbl mE_1 < +\infty$. Тогда $E \in D$}
\begin{enumerate}
    \item $E_x = \bigcap (E_i)_x$. Аналогично предыдущему.
    \item По теореме о непрерывности меры сверху (условия подходят): $\lim \nu E_{i_x} = \nu E_x$. Ну и тогда, добавляя оговорку о том, что всё это работает на тех $x$, на которых определены функции для кусочков, то и наша функция сопоставления измерима.
    \item $\int_{X} \nu(C_x) d\mu = \int_{X} \lim_{i \rightarrow \infty} \nu E_{i_x} d\mu = $. Замечаем, что все наши функции в пределе положительны (меры) и суммируемы (т. к. $0 \le \nu E_{i_x} \le \nu E_1 < +\infty$ по условию, значит суммируемы). Тогда запускаем теорему Лебега о мажорированной сходимости для случая почти везде (в обратку) и выигрываем! $= \lim_{i \rightarrow \infty} \int_{X} \nu E_{i_x} d\mu = \lim_{i \rightarrow \infty} m E_i = mE$ (последнее тоже по непрерывности меры сверху).
\end{enumerate}

Сделаем небольшое лирическое отступление в прошлое. Как мы помним, у нас есть теорема о продолжении меры, по которой, в частности, строилась и мера Лебега. По одному из её пунктов, меру предлагалось высчитывать, выбирая всё лучше оценивающее покрытие ячейками, и беря по всем таким покрытиям инфимум: $(\mathcal{P} $(п-к.), $\mu_0) \rightarrow (\mathfrak{A} (\sigma-$алг.$), \mu); \quad \mu A = \inf \{\sum \mu P_k, \dbl A \subset \bigcup P_k\}$. Также, если мы рассмотрим конкретно меру Лебега, то измеримое про ней множество можно представить (по теореме о регуляризации?) в виде $A \in \mathfrak{A}, \dbl A = B \setminus C$, где $B$ --- ``борелевское'', а $C$ --- ``меры 0'' (кавычки тут не просто так, ведь мы не задавали никаких топологий и прочего, чтобы их снять. Тут это для общего понимания происходящего). Ну и получается, что если берём за основу ``измеримости'' вот это определение с инфимумом, то $B$ представляется в виде $\bigcap_i \bigcup_j P_{ij}$ (типа взяли всевозможные покрытия и пересекли, получив тем самым наилучшее, чтоли). И некоторый остаток меры 0. Однако, не стоит его недооценивать, у нас мера по условию принципа --- полная, а это значит, что ``иерархия'' на этих множествах должна соблюдаться (см. определение полной меры из 3 сем.). Рассматриваем всё это далее!

\textbf{4. $mE = 0$. Тогда $E \in D$}

То же самое: $\dbl mD = 0, \dbl H = \bigcap_i \bigcup_j P_{ij}, \dbl P_{ij} \in \mathfrak{A} \times \mathfrak{B}, \dbl E \subset H$. Заметим, что $H \in D$ по пункту 3.

\begin{enumerate}
    \item $0 = mH = \int_{X} \nu (H_x) d\mu$. Если так случилось, то логично, что $\nu(H_x) = 0$ п. в. $x$. Ну тогда $\nu(E_x) = 0$ при этих $x$, так как $E_x \subset H_x$ по полноте меры. 
    \item Доказано предыдущим пунктом, всё 0.
    \item Как следствие, $\int_{X} \nu (E_x) = 0 = mE$
\end{enumerate}

\textbf{5. $A \in \mathfrak{A} \otimes \mathfrak{B}, mA < + \infty$. Тогда $A \in D$}

Пользуясь лирическим отступлением (и ``обобщённой регулярностью''): $A = B \setminus C, \dbl B = \bigcap_{i} \bigcup_{j} P_{ij} \in D, \dbl mC = 0 \Rightarrow C \in D$

\begin{enumerate}
    \item $mA = mB - mC = mB$, сечения: $A_x = B_x \setminus C_x$ (измеримы при п. в. $x$, т. к. составляющие уже в $D$)
    \item Из общих соображений, $\nu B_x - \nu C_x \ge \nu A_x$. С другой стороны, по монотонности ($A \subset B$): $\nu A_x \le \nu B_x$. А т. к. $\nu C_x = 0$ при п. в. $x$, то при тех же $x: \nu A_x = \nu B_x$.
    \item $\int_{X} \nu A_x d\mu = \int_{X} \nu B_x d\mu = $ (оно уже в $D$) $ = mB = mA$ (из начала).
\end{enumerate}

Ну и всё, осталось обощить всё вышеперечисленное и показать, что всё-таки любое множество лежит в нашем классе $D$ (фактически, остались только множества бесконечной меры).

\textbf{6. $A \in \mathfrak{A} \times \mathfrak{B}$ --- любое $ \in D$}

$\mu A = +\infty$. Запускаем $\sigma-$конечность: $X = \bigsqcup X_k, Y = \bigsqcup Y_i$. С другой стороны, $X \times Y = \bigsqcup X_k \times Y_i$. Тогда $A \cap (X_k \times Y_i) \in D$ по пункту 5 (конечная мера), а их дизъюнктное объединение $\bigsqcup A \cap (X_k \times Y_i) \in D$ по пункту 2.

ч. т. д. 

\textit{Следствия:}

\begin{enumerate}
    \item $C \in \mathfrak{A} \otimes \mathfrak{B}, \dbl P_1(C) = \{x \in X: C_x \neq 0\}$ (проекция на $X$) и она измерима на нём, то меру можно считать по ней $mC = \int_{P_1(C)} \nu (C_x) d\mu$. Это очевидно (ну просто проекция удаляет те точки, где сечение и так было равно нулю).
    \item $f: [a, b] \rightarrow \mathbb{R}$. Тогда $\int_a^b f(x) = \int_{[a, b]} f d\lambda_1$
    
    \textit{Доказательство:}

    Достаточно рассмотреть неотрицательную функцию, т. к. оба интеграла аддитивны и можно просто разбить. Тогда, ПГ$(f, [a, b]) = C$ --- измеримое множество (очев). А $C_x = [0, f(x)]$ (см. картинку). Причём, если вспомнить 2й сем, то окажется, что той загадочной площадью $\sigma$, которую мы использовали в рассуждениях, может быть и $\lambda$! Давайте посмотрим поближе: $\lambda(C_x) = \lambda([0, f(x)]) = f(x)$.

    \images{0.3}{kavalieri.jpg}

    $\int_a^b f(x) dx = \lambda_2(\text{ПГ}(f, [a, b])) = $ (по следствию 1 можем считать просто на проекции) $= \int_{[a, b]} \lambda(C_x) d \lambda_1 = \int_{[a, b]} f(x) d\lambda_1$
\end{enumerate}

\subsubsection{Теорема Фубини}
\textit{Формулировка:}

\begin{itemize}
    \item $(X, \mathfrak{A}, \mu), (Y, \mathfrak{B}, \nu)$ --- пространства с мерой
    \item $\mu, \nu$ --- $\sigma$-конечные меры
    \item $m = \mu \times \nu$
    \item $f: X \times Y \rightarrow \rinf$, суммируема на $X \times Y$ по мере $m$
\end{itemize}

Тогда:

\begin{enumerate}
    \item при почти всех $x$ функция $f_{x}$ суммируема на $Y$
    \item $x \mapsto \varphi(x) = \int_{Y} f_{x} d\nu$ --- это суммируемая функция на $X$
    \item \[\int_{X \times Y} f dm= \int_{X} \varphi(x) d \mu(x) = \int_{X} \left( \int_{Y} f(x, y) d  \nu (y)\right) d \mu(x)\]
\end{enumerate}

\textit{Доказательство:}

Теорема-клон Тонелли, и выводится ровно из неё.

\textbf{Подготовка}

$0 \le f_-, f_+ \le |f|$ --- по определению суммируемых функций. Также сразу заметим, что:

\[\underbrace{\int_{X \times Y} f_{-, +} dm}_{(1)} = \underbrace{\int_{X} \left( \int_{Y} f_{-, +} d\nu \right) d\mu}_{(2)} < +\infty \] 

(это всё потому что они измеримы, поэтому применяем Тонелли. Ну и интегралы конечны почти везде в силу суммируемости самой $f$).

$(f_-)_x, (f_+)_x$ --- измеримы по Тонелли. Можно также рассмотреть и 2й пункт:

\[\varphi_- = \int_{Y} (f_-)_x d\nu, \quad \varphi_+ = \int_{Y} (f_+)_x d\nu \]

Эти функции точно так же измеримы по Тонелли \textit{(note для душнил: да, измеримо\textbf{*}, на области определения и почти везде, но кажется, что это уже и так все поняли)}.

По гига-неравенству с интегралами сразу делаем вывод, что $f_-, f_+$ --- суммируемы (интеграл (1) конечен). Но также и $\varphi_-, \varphi_+$ --- суммируемы по интегралу $(2)$.

\textbf{Содержательная часть}

\begin{enumerate}
    \item $f_x = (f_+)_x - (f_-)_x$ --- суммируемая как сумма суммируемых.
    \item $\varphi = \varphi_+ - \varphi_+$ --- аналогично.
    \item $\int_{X \times Y} f dm =  \int_{X \times Y} f_+ dm - \int_{X \times Y} f_- dm$ (по определению). Ну и дальше можно расписать $\int_{X} \int{Y} \ldots$, погруппировав, но всем уже и так всё понятно.
\end{enumerate}

ч. т. д. 

\textit{Следствие (аналогичное принципу Кавальери) [валидно также и для Тонелли]:}

$C \in \mathfrak{A} \otimes \mathfrak{B}, f$ --- суммируемая [измеримая, $\ge 0$]. Если $P_1(C)$ --- измеримо на $X$, тогда:

\[\int_{C} f dm = \int_{P_1(C)} \int_{C_x} fd\nu d\mu\]

\textit{Доказательство:}

Полагаем, что $f$ вне проекции равно 0 и не вносит ничего в результат.


\subsubsection{Теорема о преобразовании меры при диффеоморфизме}
\textit{Формулировка:}

\begin{itemize}
    \item $\Phi: O \subset \mathbb{R}^{m} \rightarrow \mathbb{R}^{m}$, диффеоморфизм
\end{itemize}

Тогда $\forall A \in \mathbb{M}^{m}, A \subset O$

\[\lambda \Phi(a) = \int_{A} |\det \Phi'(x)| dx\]

\textit{Доказательство:}

Введём обозначения: $J_\Phi = \det \Phi'(x), \nu A = \lambda \Phi(A)$. Тогда необходимо проверить, что $J_{\Phi}$ является плотностью $\nu$ относительно $\lambda$:

\[\lambda A \inf_{A} |J_{\Phi}| \le \nu A \le \lambda A\sup_{A} |J_{\Phi}|\]

Сразу скажем, что нам достаточно доказать лишь правую часть. У нас отображение --- диффеоморфизм, поэтому супремум и инфимум как бы ``обратны друг другу'' (в смысле обратности функций):

\[\inf_{A} |\det \Phi'(x)| \le \nu A \dbl | \dbl pow\left(-1\right)\]

\[\nu A \le \frac{1}{\inf_{A} |\det \Phi'(x)|}\]

Но, тк у нас диффеоморфизм, мы можем рассмотреть также $|\det \Phi'^{-1}(x)| = \frac{1}{|\det \Phi'(x)|}$. И логично, что в точке инфимума будет достигаться супремум обратного оператора.

\[\nu A \le\sup_{A} |\det \Phi'(x)|\]

\textbf{1. $A \subset \overline{A} \subset Q$  --- кубическая ячейка}

Доказываем от противного. Пусть предпосылка не выполняется:

\[\nu Q > \lambda Q \sup_{Q} |J_{\Phi}|\]

Давайте выберем такое $c$, чтобы $\nu Q > c \lambda Q > \lambda Q \sup_{Q} |J_{\Phi}|$. Запускаем половинное деление на кубе (покоординатное, $2^m$ частей). Утверждается, что найдётся хоть один кусочек куба, на котором это неравенство выполняется (а если не найдётся, то мы можем посуммировать неравенства, и предпосылка перестанет выполняться). Выбираем этт кусочек, и запускаем деление на нём и так далее до посинения. По теореме Кантора в пересечении их замыканий будет точка $a: \bigcap \overline{Q_i} = a$. И вдруг оказывается, что тогда не выполняется лемма об оценке малых кубов в точке $a$! ($c > \sup_{Q} |J_{\Phi}| = \sup_{\overline{Q}} |J_{\Phi}| > |J_{\Phi}| = |\det \Phi'(a)|$)А вообще-то она должа выполняться. Таким образом --- противоречие, значит неравенство выполняется.

\textbf{2. $A$ --- открытое}

Опять вспоминаем 3й сем. Любое открытое множество можно разбить на дизъюнктное объединение кубических ячеек. $A = \bigsqcup Q_i$. Оценим каждую:

\[\nu Q_i < \lambda Q_i \sup_{Q_i} |J_{\Phi}| \le \lambda Q_i \sup_{A} |J_{\Phi}|\]

Оценили супремумом по всему множеству. Ну и теперь сумммируем всё по $i$:

\[\nu A \le \lambda A \sup_{A} |J_{\Phi}|\]

\textbf{3. $A$ --- измеримое}

Возьмём открытое $G$, такое что $A \subset G$ и запустим по предыдущему пункту:

\[\nu G \le \lambda G \sup_{G} |J_{\Phi}|\]

Берём инифмум по обоим частям, левая остигаяется по регулярности меры Лебега, а правая --- по Лемме 2 (из леммы об оценке малых кубов).

\[\nu A \le \lambda A \sup_{A} |J_{\Phi}|\]

ч. т. д. 

\subsubsection{Теорема о гладкой замене переменной в интеграле Лебега}
\textit{Формулировка:}

\begin{itemize}
    \item $\Phi: O \subset \mathbb{R}^{m} \rightarrow \mathbb{R}^{m}$, диффеоморфизм
    \item $\Phi(O) = O'$
    \item $f: O' \rightarrow \mathbb{R} \ge 0$ --- измерима
\end{itemize}

Тогда:

\[\int_{O'} fdx = \int_{O} f\left(\Phi(x)\right) \cdot |\det \Phi'(x) d\lambda(x)|\]

\textit{Доказательство:}

Запустим теорему о вычислении интеграла по взвешенному образу меры ($\omega(x) = |\det \Phi'(x)|, \nu(A) = \lambda \Phi(A)$). Тогда искомое --- результат её работы.

ч. т. д.

\textit{Следствие:}

Если $A' \subset O', \Phi(A) = A'$, то:

\[\int_{A'} f = \int_{A} f \circ \Phi |J_{\Phi}| dx\]

\subsubsection{Теорема о непрерывности сдвига}
    \textit{Формулировка:}

    \begin{itemize}
        \item $f: \mathbb{R}^{m} \rightarrow \rinf$
        \item $h \in \mathbb{R}^{m}$
        \item $f_{h}(x) := f(x + h)$
    \end{itemize}

    \begin{enumerate}
        \item $f$ --- равномерно непрерывна в $\mathbb{R}^{m}$
        
        Тогда $||f_h - f||_{\infty} \goesto{h \rightarrow 0} 0$

        \item $1 \le p < +\infty, \dbl f \in L^{p}(\mathbb{R}^{m})$
        
        Тогда $||f_h - f||_{p} \goesto{h \rightarrow 0} 0$

        \item $f \in \tilde{C}[0, \tau]$
        
        Тогда $||f_h - f||_{\infty} \goesto{} 0$

        \item $1 \le p < +\infty, \dbl f \in L^{p}[0, \tau]$
        
        Тогда $||f_h - f||_{p} \goesto{} 0$
    \end{enumerate}

    \textit{Доказательство:}
\newpage

\subsection{Теоремы}

\subsubsection{Теорема об интегрировании положительных рядов}
\textit{Формулировка:}

\begin{itemize}
    \item $(X, \mathfrak{A}, \mu)$ --- пространство с мерой
    \item $u_n: X \rightarrow \rinf, u_n \ge 0$ (при почти всех $x$ ?)
    \item $u_n$ --- измеримы на $E \in \mathfrak{A}$
\end{itemize}

Тогда: 

\[\int_{E} \left(\sum_{n = 1}^{\infty} u_n(x)\right)d\mu(x) = \sum_{n = 1}^{\infty} \left(\int_{E} u_n(x) d\mu(x)\right)\]

\textit{Доказательство:}

Подгоним под теорему Леви 3 (3 семестр). Пусть $S_{N}(x) = \sum_{n = 1}^{N} u_n(x)$ --- последовательность частичных сумм. Очевидно, что эта последовательность --- монотонно неубывающая (так как функции у нас неотрицательные): 

\[0 \le S_{N} \le S_{N + 1} \le S_{N + 2} \le \ldots\]

Тогда, делаем предельный переход (вот тут есть вопрос, почему должен существовать предел, но если подумать: если его не существует, вообще вся эта теорема не имеет смысла (ну бесконечности, чел, смысл их интегрировать)). А так же, измеримость сохраняется, так как у нас исходные функции все были измеримы (ну и по теореме о пределе измеирмых функций): 

\[S_{N}(x) \toinf{N} S(x)\]

Ну и всё, значи, по теореме Леви можем перейти к предельному преходу интегралов: 

\[\int_{E} S_{N}(x) d\mu(x) \toinf{N} \int_{E} S(x) d\mu(x)\]

Левую часть можно расписать по линейности интеграла (там у нас конечное число членов): 

\[\int_{E} S_{N}(x) d\mu(x) = \sum_{n = 1}^{N} \int_{E} u_n(x) d\mu(x)\]

Ну, а раз интграл суммы стремится к интегралу предельной функции, то и сумма интегралов обязана туда стремиться.

\[\sum_{n = 1}^{N} \int_{E} u_n(x) d\mu(x) \toinf{N} \sum_{n = 1}^{\infty} \int_{E} u_n(x) d\mu(x)\]

ч. т. д. 


\textit{Следствие: }

\begin{itemize}
    \item $u_n: X \rightarrow \mathbb{R}$, измеримы на $E \in \mathfrak{A}$
    \item $\sum \int_{E} |u_n(x)| d\mu < +\infty$ (конечна)
\end{itemize}

Тогда $\sum u_n(x)$ --- абсолютно сходящийся при почти всех $x$

\textit{Доказательство: }

Пусть: 

\[S(x) = \int_{n = 1}^{\infty} \left|u_n(x)\right|\]

Тогда, по предыдущей теореме: 

\[\int_{E} S(x) d\mu = \sum_{n = 1}^{\infty} \left(\int_{E} |u_n(x)| d\mu\right) < +\infty\]

Раз интеграл конечен, значит $S(x)$ --- суммируема, а это значит, что $S(x)$ --- почти везде конечна. Ну значит и сходится.

ч. т. д.

\textit{Пример: }

\begin{itemize}
    \item $(x_n)$ --- вещественная последовательность
    \item $\sum a_n$ --- абсолютно сходящийся числовой ряд
\end{itemize}

Тогда функциональный ряд $\sum \frac{a_n}{\sqrt{|x - x_n|}}$  --- абсолютно сходится при почти всех $x$ (в $\mathbb{R}$ по мере Лебега)

\textit{Доказательство: }

Во-первых, можно доказать, что если для $\forall A$ на $[-A, A]$ абсолютно сходится почти везде, то и везде (на $\mathbb{R}$) почти везде сходится (лол). Счётное количество п. в. $\Rightarrow$ п. в. (чтобы количество отрезков было счётным, надо чтобы $A$ были хотя бы рациональными. Кажется, что это не сильная проблема, так как отрезки включают в себя и все вещественные числа на отрезке тоже).

Попробуем подогнать под предыдущую теорему: 

\[\int_{[-A, A]}\frac{|a_n|}{\sqrt{|x - x_n|}} d\lambda = |a_n| \int_{-A}^{A} \frac{dx}{\sqrt{|x - x_n|}} \le\]

Так, стоп. А как мы перешли к определённому интегралу? Оказывается, что так можно делать, на доказано это будет позже (в курсе).

\[\underset{x := x - x_n}{\le} |a_n| \int_{-A - x_n}^{A - x_n} \frac{dx}{\sqrt{|x|}} \le |a_n| \int_{-A}^{A} \frac{dx}{\sqrt{|x|}} \le\]

Почему верен последний переход? Посмотрим на картинке: 

\images{0.5}{sh_pol_r.png}

Ну, по ней очевидно, что мы откусили кусочек поменьше, а добавили побольше. Тогда оценим модуль: 

\[ \le 2 \cdot |a_n| \int_{0}^{A}\frac{dx}{\sqrt{|x|}} = 4 \cdot \sqrt{A} \cdot |a_n|\]

Всё, абсолютный интеграл ограничен, значит сходится (при почти всех $x$).

ч. т. д. 

\subsubsection{Абсолютная непрерывность интеграла}
\textit{Формулировка:}

\begin{itemize}
    \item $(X, \mathfrak{A}, \mu)$ --- пространство с мерой
    \item $f: X \rightarrow \rinf$ --- суммируемая
\end{itemize}

Тогда:

\[\forall \varepsilon > 0 \dbl \exists \delta > 0, \quad \forall E\text{--- измеримое} \dbl \mu E < \delta \qquad \left|\int_{E} f d \mu\right| < \varepsilon\]

\textit{Доказательство:}

Для доказательства сего факта нам бы хотелось поисследовать, как на таких множествах ведёт себя функция в зависимости он величиные её значений на соответствующих множествах. Давайте заведём множества $X_n$:

\[X_n = X(|f| \ge n)\]

Заметим, что $\ldots \supset X_n \supset X_{n + 1} \supset \ldots$. Причём:

\[\bigcap X_n = X_{\infty} = X(|f| = \infty)\]

А также, ведь по условию наша функция $f$ суммируема, значит она почти везде конечна (а там, где не конечна --- множество меры 0):

\[\mu \left( \bigcap X_n \right) = 0\]

Теперь заведём вспомогательную меру:

\[ \nu(A) = \int_{A} f d\mu\]

И внезапно заметим, что для неё выполняется теорема об непрерывности меры сверху! ($X_0 = X$, так как там у нас условие модуль больший нуля, и интеграл по нему конечен, так как функция суммируема):

\[\nu(X_0) = \int_{X_0 = X} |f| \mu < +\infty\]

Ну а в пересечении, как мы уже выяснили, у нас множество меры ноль (а на нём интеграл тоже нулевой):

\[\nu\left( \bigcap X_n \right) = 0\]

Таким образом, $\nu(X_n) \ntoinf 0$. И это даёт нам право с полной уверенностью сказать, что:

\[\forall \varepsilon > 0 \dbl \exists n_\varepsilon \quad \int_{X_{n_\varepsilon}} |f| d\mu < \frac{\varepsilon}{2}\]

Все приготовления сделаны, давайте оценивать:

\[\forall \varepsilon > 0 \dbl \delta := \frac{\varepsilon}{2 n_{\varepsilon}} \dbl \mu E < \delta \qquad \left|\int_{X_{n_\varepsilon}} f d\mu\right| \le \int_{X_{n_\varepsilon}} |f| d\mu = \int_{E \cap X_{n_\varepsilon}} |f| d\mu + \int_{E \cap X^c_{n_{\varepsilon}}} |f| d\mu\]

Первое слагаемое оценим $X_{n_{\varepsilon}}$, для которого у нас уже есть готовое утвверждение выше. А второе оценим мерой, умноженной на $n_\varepsilon$. Так можно сделать, ведь дополнение $X_{n_\varepsilon}$ есть множество точек, на котором функция $< n_{\varepsilon}$

\[\le \frac{\varepsilon}{2} + n_{\varepsilon} \cdot \overbrace{\underbrace{\mu \left( E \cap X^c_{n_{\varepsilon}}\right) \le \mu\left( E \right) < \delta}} \le \frac{\varepsilon}{2} + \frac{\varepsilon}{2} = \varepsilon\]

ч. т. д. 

\textit{Следствие:}
\begin{itemize}
    \item $(e_n) \in \mathfrak{A}$ --- последовательность (?) множеств\
    \item $\mu e_n \ntoinf 0$
    \item $f$ --- суммируемая на $X$
\end{itemize}

Тогда:

\[\int_{e_n} f d \mu \ntoinf 0\]

\textit{Доказательство:}


Очевидно следует из теоремы, ну камон)

\subsubsection{Теорема о произведении мер}
\textit{Формулировка:}

\begin{itemize}
    \item $(X, \mathfrak{A}, \mu)$, $(Y, \mathfrak{B}, \nu)$ --- пространства с мерой (полукольца (?))
    \item Зададим $m_0(A \times B) = \mu A \cdot \nu B$
\end{itemize}

Тогда:

\begin{enumerate}
    \item $m_0$ --- мера на $\mathfrak{A} \times \mathfrak{B}$
    \item $\mu, \nu$ --- $\sigma$-конечные меры $\Longrightarrow$ $m_0$ --- $\sigma$-конечная
\end{enumerate}

\textit{Доказательство:}

\textbf{1.}

Давайте рассмотрим какой-то $P = \bigsqcup P_k$ --- измеримые прямоугольники. Чтобы доказать, что это действительно мера на $\mathfrak{A} \times \mathfrak{B}$, необходимо доказать счётную аддитивность: $m_0(P) \underset{?}{=} \sum m_0(P_k)$

Верно, что $P = A \times B, P_k = A_k \times B_k$ (наше множество есть результат перемножение множеств из каждого пространства). Также из этого следует, что:

\[\chi_P = \sum \chi_{P_k}\]
\[\chi_A(x)\chi_B(y) = \sum \chi_{A_k}(x)\chi_{B_k}(y)\]

Поинтегрируем это по $Y$!

\[\chi_A(x) \nu(B) = \sum \chi_{A_k}(x)\nu(B_k)\]

А теперь по $X$!

\[\mu(A)\nu(B) = \sum \mu(A_k) \nu(B_k)\]

Всё проверили, это действительно мера.

\textbf{2.}

По сигма-конечности исходных мер, мы можем расбить исходные простанства на счётное объединение множеств, имеющих конечную меру.

\[X = \bigcup X_k, \dbl \mu X_k < +\infty\]
\[Y = \bigcup Y_k, \dbl \nu Y_n < +\infty\]

Ну и тогда мера перемножения двух этих множеств будет просто резуьльтатом перемножения нескольких конечных чисел и их сумма, что, очевидно, конечно:

\[X \times Y = \bigcup_{(i, j)} X_i \times Y_j\]
\[m_0(X \times Y) = \sum_{(i, j)} \mu(X_i) \cdot \nu(Y_j)\]

ч. т. д. 

\subsubsection{Теорема Тонелли}
\textit{Формулировка:}

$f_x(y) = f^y(x) = f(x, y)$ --- новая нотация для функций с фиксированным аргументом.

\begin{itemize}
    \item $(X, \mathfrak{A}, \mu), (Y, \mathfrak{B}, \nu)$ --- пространства с мерой
    \item $\mu, \nu$ --- $\sigma$-конечные меры
    \item $m = \mu \times \nu$
    \item $f: X \times Y \rightarrow \rinf \ge 0$, измерима относительно $\mathfrak{A} \otimes \mathfrak{B}$
\end{itemize}

Тогда:

\begin{enumerate}
    \item при почти всех $x$ функция$f_{x}$ измерима на $Y$
    \item $x \mapsto \varphi(x) = \int_{Y} f_{x} d\nu$ --- это измеримая функция на $X$
    \item \[\int_{X \times Y} f dm= \int_{X} \varphi(x) d \mu(x) = \int_{X} \left( \int_{Y} f(x, y) d  \nu (y)\right) d \mu(x)\]
\end{enumerate}

Всё то же самое валидно и для $y$.

\textit{Доказательство:}

будет принципом ``конструктора''. Соберём измеримую функцию из кусочков.

\textbf{1. $C \in \mathfrak{A} \otimes \mathfrak{B}, f = \chi_C$}

То есть, сначала рассмотрим функцию-``ступеньку''. 

\begin{enumerate}
    \item $f_x = \chi_{C_x}$ --- она измерима тогда, когда измеримо $C_x$. А оно измеримо по принципу Кавальери!
    \item $\int_{Y} f_x d\nu = \int_{Y} \chi_{C_x} d\nu = \nu(C_x)$ --- измеримо по принципу Кавальери!
    \item $\int_{X} \varphi(x) d\mu = \int_{X} \nu(C_x) d\mu = mC$ (по принципу Кавальери). Тогда в обратную сторону $= \int_{X \times Y} \chi_{C} dm = \int_{X \times Y} f dm$
\end{enumerate}

\textbf{2. $f = \sum_{i = 1}^{n} c_i \chi_i \ge 0$ --- ступенчатая}

\begin{enumerate}
    \item $f_x = \sum c_i (\chi_{C_I})_x$ --- аналогично предыдущему пункту, сумма измеримых почти везде.
    \item $\int_{Y} f_x d\nu = \int_{Y} \sum c_i (\chi_{C_i})_x d\nu = \sum c_i \int_{Y} (\chi_{C_i})_x d\nu$. Ну и собственно говоря, у нас конечная сумма почти везде измеримых функций. Всё хорошо.
    \item $\int_{X \times Y} f dm = \sum c_i \int_{X \times Y} \chi_{C_i} dm=$ вот тут просто раскрываем по пункту для ``ступеньки'' и заносим сумму внутрь $=\sum c_i \int_{X} \left(\int_{Y} \chi_{C_i} d\nu\right) d\mu = \int_{X} \sum c_i \left( \int_{Y} \chi_{C_i} d\nu \right) d\mu = \int_{X} \int_{Y} \left( \sum c_i \chi_{C_i}\right) d\nu d\mu$
\end{enumerate}

\textbf{3. $f \ge 0$ --- ступенчатая}

Идея: аппроксимация + теорема Леви $\times \, 2$. \textit{(в этом разделе постоянно используется такой приём --- прим. авт.)}

Запускаем теорему о характеризации измеримых функций ступенчатыми (?), $f = \lim_{n \rightarrow \infty} g_n, g_n$ --- ступенчатые, возрастающие.

\begin{enumerate}
    \item $f_x = \lim_{n \rightarrow \infty} (g_n)_x$ --- измерима как предел измеримых функций (3 сем).
    \item $\int_{Y} f_x d\nu \underset{\infty \leftarrow n}{\longleftarrow} \int_{Y} (g_n)_x d\nu$ (по теореме Леви). Предел измеримых.
    \item Обозначим интеграл каждой ступенчатой функции как $\varphi_n(x) = \int_{Y} g_n d\nu$. Так вот, оказывается $\varphi_n(x) \le \varphi_{n + 1}(x) \le \varphi_{n + 2}(x)$ (так как там подынтегральные функции возрастающие, все дела), и при этом $\varphi_n(x) \ntoinf \varphi(x)$ (предыдущий пункт). Тогда давайте просто $\int_{X} \varphi(x) d\mu = \lim_{n \rightarrow \infty} \int_{X} \varphi_n(x) d\mu$ по теореме Леви (типа в обратную сторону). А ещё, зная что в основе $\varphi_n$ лежит ступенчатая функция, мы понимаем, что для неё уже выполняется наша теорема, таким образом применив пункты 2 и 3 мы можем перейти к равенству $= \lim \int_{X \times Y} g_n dm = $ и опять по Леви $ = \int_{X \times Y} fdm$
\end{enumerate}

ч. т. д. 

\subsubsection{Формула для бета-функции}
\textit{Формулировка:}

Бета-функция задаётся следующим образом: 

\[B(s, t) = \int_{0}^{1}x^{s - 1}(1 - x)^{t - 1}dx, \quad s, t > 0\]

Тогда:

\[B(s, t) = \frac{\Gamma(s)\Gamma(t)}{\Gamma(s + t)}\]

\textit{Доказательство:}

Рассмотрим:

\[\Gamma(s)\Gamma(t) = \int_{0}^{\infty}x^{s - 1}e^{-x}dx \cdot \int_{0}^{\infty} y^{t - 1}e^{-y} dy = \]

Заметим, что второй интеграл есть ничто иное, как константа! Внесём его внутрь:

\[= \int_{0}^{\infty} x^{s - 1}e^{-x} \left(\int_{0}^{\infty} y^{t - 1}e^{-y}dy\right)dx = \int_{0}^{\infty}  \left(\int_{0}^{\infty} x^{s - 1} y^{t - 1}e^{-(x + y)}dy\right)dx = \]

Заменим $y = u - x$:

\[= \int_{0}^{\infty} \left(\int_{x}^{\infty}x^{s - 1}(u - x)^{t - 1}e^{-u}du\right)dx = \]

А теперь финт ушами! По теореме Тонелли, этот повторный интеграл является двойным интегралом по некоторой области $C$:

\images{0.3}{betta.jpg}

Так давайте просто поменяем пределы интегрирования: 

\[= \int_{0}^{\infty} \left(\int_{0}^{u}x^{s - 1}(u - x)^{t - 1}e^{-u}dx\right)du = \]

И ещё раз заменим: $x = uv, \dbl dx = udv$ ($u$ типа как константа, пределы интегрирования тоже поменялись!)

\[= \int_{0}^{\infty} \left(\int_{0}^{1}(uv)^{s - 1}(u - uv)^{t - 1}e^{-u}dv\right)udu = \int_{0}^{\infty} \left(\int_{0}^{1}u^{s - 1}v^{s - 1}u^{t - 1}(1 - v)^{t - 1}e^{-u}dv\right)udu = \]

\[\int_{0}^{\infty}u^{s + t - 1}e^{-u} du \cdot \int_{0}^{1}v^{s - 1}(1 - v)^{t - 1}dv = \Gamma(s + t)B(s, t)\]

ч. т. д.

\subsubsection{Объем шара в $\mathbb{R}^m$}
\textit{Формулировка:}

\begin{itemize}
    \item $B(0, R) = \{x \in \mathbb{R}^{m}: x_1^{2} + x_2^{2} + \ldots + x_m^{2} \le R^{2}\}$
    \item $\alpha_{m} \ \lambda_{m}(B(0, 1))$
\end{itemize}

Тогда: 

\[\mu\left(B(0, R)\right) = \alpha_m R^{m}\]

\textit{Доказательство:}

Почему вылез радиус в степени $m$ --- это при линейном растяжении шарика $B(0, 1)$ просто вылез множитель (по прошлому сему (?)). Поэтому достаточно рассмотреть только этот базированный шар единичного радиуса. Будем же наконец искать его объём, интегрируя!

\[\alpha_m = \int_{-1}^{1} \lambda_{m - 1} \left(B(0, 1)_{x_1}\right) dx_1 = \]

А почему так? Да очень просто. Дело в том, что сечение шара размерности $m$ есть подпространство размерности $m - 1$, а именно --- шар радиуса $\sqrt{1 - x_1^2}$.

\images{0.5}{objom.jpg}

\[= \int_{-1}^{1} \alpha_{m - 1} (1 - x_1^2)^{\frac{m - 1}{2}} dx_1 = \]

Делаем замену $x_1^2 = x, \dbl dx_1 = \frac{dx}{2\sqrt{x}}$:

\[= \frac{\alpha_{m - 1}}{2} \int_{-1}^{1} x^{\frac{1}{2}}(1 - x)^{\frac{m - 1}{2}}dx = \alpha_{m - 1} \cdot B\left(\frac{1}{2}, \frac{m + 1}{2}\right) = \alpha_{m - 1}\frac{\Gamma\left(\frac{1}{2}\right)\Gamma\left(\frac{m + 1}{2}\right)}{\Gamma\left(\frac{m}{2} + 1\right)}\]

Двойка из знаменателя пропала из-за того, что подинтергальная функция чётна, значит, изначальный интеграл можно разбить на два на промежутках $(-1, 0)$ и $(0, 1)$ и они будут равны, и равны бета-функции. Ну и всё, двойка сократилась. Гораздо интереснее, что же там будет, если мы будем раскрывать ``альфы'' до талого. Сразу заметим, что $\alpha_1 = 2$ (ну просто длина промежутка $(-1, 1)$). Посмотрим (пары, эквивалентные ``подчёркнутым'' сократятся, и так далее со сдвигом на один через один, лол):

\[\alpha_{m} = \frac{\Gamma\left(\frac{1}{2}\right)\uwave{\Gamma\left(\frac{m + 1}{2}\right)}}{\Gamma\left(\frac{m}{2} + 1\right)} \cdot \frac{\Gamma\left(\frac{1}{2}\right)\Gamma\left(\frac{m}{2}\right)}{\uwave{\Gamma\left(\frac{m - 1}{2} + 1\right)}} \cdot \frac{\Gamma\left(\frac{1}{2}\right)\Gamma\left(\frac{m - 1}{2}\right)}{\Gamma\left(\frac{m - 2}{2} + 1\right)} \cdot \ldots \cdot 2 = \]

Вспоминаем ``факториальность'' гамма-функции $\Gamma(n + 1) = n\Gamma(n)$ и формулу из темы про бесконечные произведения $\Gamma(x)\Gamma(1 - x) = \frac{\pi}{\sin \pi x}$:

\[= 2\frac{\Gamma\left(\frac{1}{2}\right)^{m - 1}\Gamma\left(\frac{3}{2}\right)}{\Gamma\left(\frac{m}{2} + 1\right)} = 2\frac{\Gamma\left(\frac{1}{2}\right)^{m - 1}\cdot \frac{1}{2} \cdot\Gamma\left(\frac{1}{2}\right)}{\Gamma\left(\frac{m}{2} + 1\right)} = \frac{\Gamma\left(\frac{1}{2}\right)^{m}}{\Gamma\left(\frac{m}{2} + 1\right)} = \frac{\left(\frac{\pi}{\sin \frac{\pi}{2}}\right)^{\frac{m}{2}}}{\Gamma\left(\frac{m}{2} + 1\right)} = \frac{\pi^{\frac{m}{2}}}{\Gamma\left(\frac{m}{2} + 1\right)}\]

(можно прогнать ещё для первых размерностей 2, 3)

ч. т. д.

\subsubsection{Теорема Фату. Следствия}
\textit{Формулировка:}

\begin{itemize}
    \item $(X, \mathfrak{A}, \mu)$ --- пространство с мерой
    \item $f_n \ge 0$ --- измерима
    \item $f_n \rightarrow f$ почти везде
    \item Если $\exists C > 0\ \dbl \forall n \int_{X} f_n d\mu \le C$  
\end{itemize}

Тогда:

\[\int_{X} f d\mu \le C\]

(тут, вообще говоря, не предполагается, что интегрально функции сходятся)

\textit{Доказательство:}

Заведём $g_n = \inf \{f_n, f_{n + 1}, f_{n + 2}, \ldots\}$ (должно уже на что-то намекать). Очевидно, что эта последовательность возрастающая, так как у нас есть сходимость почти везде изначально. Также $\forall n : 0 \le g_n \le f_n$

Очевидно, что $g_n \le f_n$, интегрируем!

\[\int_{X} g_n \le \int_{X} f_n \le C \quad (*)\]

С другой стороны, так как $\lim_{n \rightarrow \infty} g_n$ суть есть $\underline{\lim} f_n = f$ (ну раз у нас есть сходимость, то и нижний предел сходится к $f$). Тогда по теореме Леви:

\[\lim_{n \rightarrow \infty} \int_{X} g_n = \int_{X} f \le C\]

ч. т. д.

\textit{Следствие: }

То же самое, только меняем сходимость почти везде на: 

\begin{itemize}
    \item $f_n, f \ge 0$, измеримы, почти везде конечны
    \item $f_n \underset{\mu}{\Longrightarrow} f$
\end{itemize}

\textit{Доказательство: }

Запускаем теорему Рисса, выбираем сходящуюся подпоследовательность и доказательство сработает.

\textit{Следствие: }

\begin{itemize}
    \item $f_n \ge 0$, измеримы
\end{itemize}

Тогда:

\[\int_{X} \underline{\lim}f_n \le \underline{\lim}\int_{X} f_n\]

\textit{Доказательство: }
Вспоминаем 2й сем, там было несколько теорем о частичном пределе. Одна из них говорит, что если существует нижний предел, то существует и подпоследовательность к нему ведущая. А раз он существует (почему?), то всё гуд:

\[\exists n_k: \int_{X} f_{n_k} \ntoinf \underline{\lim} \int_{X} f_n\]

Тогда запускаем (*):

\[\int_{X} g_n \le \int_{X} f_n\]

Левый интеграл по выкладкам из основной теоремы стремится к тому, чему надо. А для правого написано выше:

\[\int_{X} \underline{\lim}f_n \le \underline{\lim}\int_{X} f_n\]

ч. т. д. 

\subsubsection{Теорема о вычислении интеграла по взвешенному образу меры}
\textit{Формулировка:}

\begin{itemize}
    \item $(X, \mathfrak{A}, \mu), (Y, \mathfrak{B}, \_)$ --- пространства с мерой
    \item $\omega: X \rightarrow \rinf \ge 0$ --- измеримо
    \item $\Phi: X \rightarrow Y$ --- ``измеримое''
    \item $\nu$ --- взвешенный образ $\mu$ (с весом $\omega$)
\end{itemize}

Тогда для $\forall f: Y \rightarrow \rinf \ge 0$ --- измеримых:
\begin{enumerate}
    \item $f \circ \Phi$ --- измеримо (относительно $\mathfrak{A}$)
    \item $\int_{Y} f d\nu = \int_{X} f(\Phi(x))\cdot\omega(x) d\mu(x)$
\end{enumerate}
\textit{Доказательство:}

\textbf{1.}

Ну тут всё достаточно просто, нам дано, что $f$ --- измерима относительно $\mathfrak{B}$. Выводим измеримость через данное:

\[X(f \circ \Phi < a) = \Phi^{-1}(Y(f < a)) \in \mathfrak{A}, \quad (Y(f < a) \in \mathfrak{B})\]

Ну типа, мы перегоняем каждую точку из пространства, в котором нам известна измеримость, в новое. Причём, важно что прообраз этих множеств точно лежит в $\mathfrak{A}$ (по ``измеримости'' отображения $\Phi$)

\textbf{2. ``Зоологическая теорема'' }

Запускаем классическое ``ступенчатое'' доказательство.

\textbf{2.1 $B \in \mathfrak{B}, f = \chi_{B}$ -- ступенька}

По условию: $f \circ \Phi(x) = \chi_{B}(\Phi(x))$. Вообразим это в голове, и поймём, что это характеристическая функция образа $B_k: = \chi_{\Phi^{-1}(B)}(x)$. С другой стороны, мы можем поинтегрировать функцию по $Y:$

\[\int_{Y} f d\nu = \nu(B) = \]

$\nu$ --- взвешенная мера (по определению). Распишем:

\[= \int_{\Phi^{-1}(B)} \omega d\mu = \int_{X} \chi_{\Phi^{-1}(B)} \omega d\mu = \]

Мы расширили интеграл, но добавили характеристическую функцию, чтобы занулить его вне искомой области. Ну и теперь подинтегральная функция просто и есть $f$:

\[ = \int_{X} f \circ \Phi(x) \omega(x) d\mu\]

\textbf{2. $f = \sum \alpha_k \chi_{B_k}(x)$ --- ступенчатая}

По линейности интеграла всё работает.

\[\int_{Y} \sum \alpha_k \chi_{B_k} d\nu = \sum \alpha_k \int_{Y} \chi_{B_k} d\nu = \sum \alpha_k \nu(B_k) =\]

\[= \sum \alpha_k \int_{X} (f_{B_k} \circ \Phi)(x) \omega(x) d\mu = \int_{X} \sum \alpha_k = \ldots\]

Ну короче, всё хорошо.

\textbf{3. $f \ge 0$ --- измеримая}

$g_n \ntoinf f, g_n \ge 0$ --- ступенчатые. Запускаем теорему Леви и всё получается.

ч. т. д. 

\textit{Следствие:}

Вместо измеримости $\ge 0$ можно взять и суммируемость.

\textit{Доказательство:}

$|f|$ подходит по условию теоремы.

$|f|$ --- суммируема относительно $\nu \Leftrightarrow f \circ \Phi$ суммируема относительно $\mu$ (почему?). ``Тогда с задачей срезок не будет никаких проблем''

\subsubsection{Критерий плотности}

Тут мы резко свернули с абстрактных рельс на $\mathfrak{A} = \mathfrak{B}, \Phi = \text{id}$

\textit{Формулировка:}

\begin{itemize}
    \item $(X, \mathfrak{A}, \mu)$ --- пространство с мерой
    \item $\nu$ --- ещё одна мера на $\mathfrak{A}$
    \item $\omega: X \rightarrow \rinf \ge 0$, измеримо 
\end{itemize}

Тогда эквивалентно:

\begin{enumerate}
    \item $\omega$ --- плотность $\mu$ отностительно $\mu$
    \item $\forall A \in \mathfrak{A} \quad \inf_{A} \omega \cdot \mu A \le \nu A \le \sup_{A} \omega \cdot \mu A$
\end{enumerate}

\textit{Доказательство:}

\textbf{$1 \Rightarrow 2$}

Очевидно (расписать по определению).

\textbf{$2 \Rightarrow 1$}

Проще рассматривать множество на ненулевых $\omega$. Для начала, на нулевых всё выполняется: $B = X(\omega = 0)$.

\[\nu B = 0 = \int_{B} 0 d\mu\]

Ещё надо бы показать, что если множесто оказалось на перечении ``нулевого'' и ``пложительного'' веса, то оно не испортит нам оценку. 

\images{0.5}{kr_pl_1.jpg}

$A$ --- синее множество. Тогда $A \cap B$ --- часть, где вес равен нулю, а $A \setminus B$ --- где положительный. Посмотрим методом пристального взгляда на неравенства:

\[\nu A \le \sup \omega \mu A\]
\[\nu A  \setminus B + \underbrace{\nu A \cap B}_{=0} \le \sup \omega \cdot ( \mu (A \setminus B) + \mu (A \cap B))\]

Как видно, мы только усилили неравенство (во втором случае мы проверяем только часть множества).

Зафиксируем $q \in (0, 1)$. Рассмотрим $A_j = A(q^j < \omega < q^{j - 1}), \dbl j \in \mathbb{Z}$. Так как степень пробегает целые числа, то такое замощение покрывает всю положительную ось $\mathbb{R}$ ($A = \bigsqcup A_j$).

\images{0.4}{kr_pl_2.jpg}

\[q^j \mu A_j \underbrace{\le}_{1} \nu A_j \underbrace{\le}_{2} q^{j - 1} \mu A_j\]

\[q^j \mu A_j \underbrace{\le}_{3} \int_{A_j} \omega d\mu \underbrace{\le}_{4} q^{j - 1} \mu A_j\]

Откуда взялись эти неравенства? Ну, первое просто напросто вытекает из предпосылки, и т. к. мы ограничили множество, очевидно, какие у него инфимум и супремум. А второе --- просто расписали взвешенную меру. Записываем \textit{очень} длинное оценочное неравенство:

\[q\int_{A} \omega d\mu = q\sum \int_{A_j} \omega d\mu \underbrace{\le}_{4} q \sum q^{j - 1} \mu A_j = \sum q^j \mu A_j \le\]
\[\underbrace{\le}_{1} \sum \nu A_j \underbrace{\le}_{2} \sum q^{j - 1} \mu A_j = q^{-1} \sum q^j \mu A_j \le \]

\[\underbrace{\le}_{3} q^{-1} \sum \int_{A_j} \omega d\mu = q^{-1} \int_{A} \omega d \mu\]

Таким образом, мы окольцевали:

\[q\int_{A} \omega d\mu \le \sum \nu A_j = \nu A \le q^{-1} \int_{A} \omega d \mu\]

Устремляем $q \rightarrow 1$ и получаем искомое.

ч. т. д. 

\subsubsection{Лемма о единственности плотности}
\textit{Формулировка:}

\begin{itemize}
    \item $f, g$ --- суммируемы на $X$
    \item $\forall A$ --- измеримое, $\int_{A} f = \int_{A} g$
\end{itemize}

Тогда $f = g$ почти везде

\textit{Доказательство:}

Для удобства будем расматривать $h := f - g$. Тогда по условию теоремы $\forall A: \int_{A} |h| = 0$

Заведём $X_+ := X(h \ge 0), X_- := X(h < 0)$. Очевидно, что $X = X_+ \sqcup X_-$.

$\int_{X_+} |h| = \int_{X_+} h = 0$ (как и по любому измеримому множеству), $\int_{X_-} |h| = -\int_{X_-} h = 0$

Ну и значит и по всему пространству: $\int_{X} |h| = \int_{X_+} |h| + \int_{X_-} |h| = 0 - 0 = 0$. Получается, что $h = 0$ почти везде.

ч. т. д.

\textit{Следствие: }

Плотность меры определяется однозначно с точностью до изменения на множестве меры 0.

\subsubsection{Лемма об оценке мер образов малых кубов}
\textit{Формулировка:}

\begin{itemize}
    \item $\Phi: O \subset \mathbb{R}^{m} \rightarrow \mathbb{R}^{m}$
    \item $\Phi \in C^{1}$
    \item $a \in O$
    \item Пусть $c > |\det \Phi'(a)| \neq 0$
\end{itemize}

Тогда $\exists \delta > 0 \dbl \forall$ Куб $Q \subset B(a, \delta)$, $a \in Q$ (кубик задевает за точку)

\[\lambda \cdot \Phi(Q) < c \cdot \lambda Q\]

\textit{Доказательство:}

Пусть $L = \Phi'(a)$ (и оно ещё и обратимое, выводится из условия). Так как по условию $\Phi \in C^{-1}$, вблизи точки $a$ она представляется как:

\[\Phi(x) = \Phi(a) + L (x - a) + o(x - a)\]

Преобразуем (там мы применили обратный оператор $L^{-1}$ к ``о''-шке, но ничего страшного, она осталась ``о''-шкой):

\[\underbrace{a + L^{-1}(\Phi(x) - \Phi(a))}_{\Psi(x)} = x + o(x - a)\]

$\Psi(x)$ представляет из себя сдвинутый (на $a$ и $\Phi(a)$) $\Phi$ под действием обратного отображения $L$. И вот так получается, что он мапит иксы почти в себя самих. Давайте посмотрим на определение ``о''-маленького (эпсилон немного отнормирован):

\[\forall \varepsilon > 0 \dbl \exists \text{ шар } B_{\varepsilon}(a) \dbl \forall x \in B_{\varepsilon}(a): |\Psi(x) - x| = \frac{\varepsilon}{\sqrt{m}}|x - a|\]

Пусть у нас есть куб $Q$ внутри этого шара со стороной $h$: $Q \subset B_{\varepsilon}(a)$. Тогда $\forall x \in Q: |x - a| < \sqrt{m}h$. (точка $a$ лежит внутри куба, оценили диагональю). Применяя выкладку из ``о''-маленького, получаем:


\[|x - a| = \frac{\sqrt{m}}{\varepsilon}|\Psi(x) - x| < \sqrt{m}h\]

\[|\Psi(x) - x| < \varepsilon h\]

Теперь неочевидное: оценим для произвольных $x, y \in Q$ насколько близко они лежат друг к другу. Оценка на покоординатные функции по неравенству тругольника:

\[|\Psi_i(x) - \Psi_i(y)| = \underbrace{|\Psi_i(x) - x_i|}_{(1)} + \underbrace{|\Psi_i(y) - y_i|}_{(2)} + \underbrace{|x_i - y_i|}_{(3)} \le\]

Сразу скажем, что покоординатные функции можно оценить сверху нормами на полную функцию (по последним достижениям), а две точки внутри куба уж точно лежат не более чем на $h$ друг от друга. Таким образом:

\[\le \varepsilon h + \varepsilon h + h = (1 + 2\varepsilon) h \]

Ну и всё, тогда $\Psi(Q)$ лежит внутри куба со стороной $(1 + 2\varepsilon)h$ (ну, мы же только что узнали, насколько сильно развозит точки при отображении). Оценим меру (мера $Q$ тривиальна):

\[\lambda\Psi(Q) \le (1 + 2\varepsilon)^mh^m = (1 + 2\varepsilon)^m\lambda Q\]

А как мы обсудили ранее, $\Psi$ отличается от $\Phi$ только линейным отображением. Вспоминая прошлый сем, нам известно, что тогда мера Лебега множества после линейного отобрадения обязана домножиться на определитель оператора:

\[\lambda\Phi(Q) = |\det L|\lambda \Psi(Q) \le |\det L|(1 + 2\varepsilon)^m \lambda Q\]

Последний штрих, так как нам дали $c$, подберём такой $\varepsilon$, чтобы $|\det L|(1 + 2\varepsilon)^m < c$ (у нас есть такая возможность, так как по условию коэффициент $c$ больше определителя). А в качестве $\delta$ выберем радиус $B_{\varepsilon}(a)$.

ч. т. д.

\textit{Лемма 2 (без доказательства):}

Пусть $f: O \subset \mathbb{R}^m \rightarrow \mathbb{R}$ --- непрерывна, $A \subset O$ --- измеримое множество. Тогда:

\[\inf_{G \subset O\text{ --- открытые}, A \subset G} \left(\lambda G \cdot \sup_{G}f\right) = \lambda A \cdot \sup_{A} f  \]
\subsubsection{Предельный переход по параметру в несобственном интеграле}
\textit{Формулировка:}

\begin{itemize}
    \item $f: \langle a, b \rangle \times Y \rightarrow \rinf$
    \item $Y \subset \tilde{Y}$ --- метризуемое
    \item $y_0 \in \tilde{Y}$ --- предельная точка $Y$
\end{itemize}

\begin{enumerate}
    \item при почти всех $x \exists f_0(x) = \lim_{y \rightarrow y_0} f(x, y)$
    \item $\forall t \in (a, b) \dbl \forall f(x, y_0), f(x, y)$ --- суммируемые по $x$ на $(a, t)$ и $\int_a^{t} f(x, y) dx \goesto{y \rightarrow y_0} \int_a^{t} f_0(x) dx$
    \item $J(y) = \int_a^{\rightarrow b} f(x, y)$ --- равномерно сходящаяся при $y \in Y$
\end{enumerate}

Тогда $\int_a^{\rightarrow b} f_0(x) dx$ --- существует (как несобственный)

\textit{Доказательство:}


\subsubsection{Предельный переход под знаком интеграла при наличии равномерной сходимости или $L_{loc}$}
\textit{Формулировка:}

\begin{itemize}
    \item $f: X \times \tilde{Y} \rightarrow \rinf$
    \item $X$ --- пространство с мерой, $\mu X < + \rinf$
    \item $\tilde{Y}$ --- метрезуемое топологическое пространство
    \item $Y \subset \tilde{Y}$
    \item $a \in \tilde{Y}$ --- предельная точка  $Y$
    \item $\forall y \in Y \quad x \mapsto f(x, y)$ --- суммируема на $X$
    \item Пусть $f(x, y) \rsh{y \rightarrow a} \varphi(x)$
\end{itemize}

Тогда $\varphi$ --- суммируема на $X$ и 

\[\lim_{y \rightarrow a} \int_{X} f(x, y) d \mu(x) = \int_{X} \varphi(x) d \mu(x)\]
 
\textit{Доказательство:}

\subsubsection{Правило Лейбница дифференцирования интеграла по параметру}
\textit{Формулировка:}

\begin{itemize}
    \item $Y$ --- промежуток $\subset \mathbb{R}$
    \item $f: X \times Y \rightarrow \mathbb{R}$
    \item $\forall \quad f(x, y)$ --- суммируемая функция от $x$
    \item При почти всех $x \dbl \forall y \exists f'_y(x, y)$
    \item $f'_y$ --- удовлетворяет условию $L_{loc}(y_0)$
\end{itemize}

Тогда:

\begin{itemize}
    \item $J(y) = \int_{X} f(x, y) d\mu(x)$ --- дифференцируема в $y_0$
    \item $J'(y_0) = \int_{X} f'_y(x, y) d \mu(x)$
\end{itemize}

\textit{Доказательство:}

\subsubsection{Теорема о вложении пространств $L^p$}
\textit{Формулировка:}

\begin{itemize}
    \item $\mu E < + \infty, 1 \le s < r \le +\infty$
\end{itemize}

Тогда:

\begin{enumerate}
    \item $L_r(E, \mu) \subset L_s(E, \mu)$
    \item $||f||_s \le \left(\mu E\right)^{\frac{1}{s} - \frac{1}{r}} \cdot ||f||_{r}$
\end{enumerate}

\textbf{fix}

\textit{Доказательство:}

\subsubsection{Теорема о сходимости в $L^p$ и по мере}
\textit{Формулировка:}

$1 \le p < +\infty \quad f_n \in L_p(E, \mu)$: 

\begin{enumerate}
    \item $f \in L_p \quad f_n \goesto{L_p} f$, тогда $f_n \underset{\mu}{\Longrightarrow} f$
    \item $f_n \underset{\mu}{\Longrightarrow} f$ [либо $f_n \rightarrow f$ почти всюду], $|f_n| \le g$ почти всюду, при всех $n$, где $g \in L^p$. Тогда $f_n \goesto{L_p} f$
\end{enumerate}
\textit{Доказательство:}



\subsubsection{Полнота $L^p$}
\textit{Формулировка:}

$L^p(E, \mu)$ ---- полное ($1 \le p < + \infty$)

\textit{Доказательство:}

\subsubsection{Плотность в $L^p$ множества ступенчатых функций}
\textit{Формулировка:}

\begin{itemize}
    \item $(X, \mathfrak{A}, \mu), 1 \le p \le +\infty$
\end{itemize}

Тогда множество ступенчатых функций плотно в $L_p(X, \mu)$

\textit{Доказательство:}

\subsubsection{Лемма Урысона}
\textit{Формулировка:}

\begin{itemize}
    \item $X$ --- нормированное топологическое пространство (например, $\mathbb{R}^m$)
    \item $F_0, F_1 \subset X$ --- замкнутое
    \item $F_0 \cap F_1 = \varnothing$
\end{itemize}

Тогда: $f: X \rightarrow \mathbb{R}, \quad 0 \le f \le 1$ --- непрерывное

$f|_{F_0} \equiv  0$, $f|_{F_1} \equiv 1$

\textit{Доказательство:}

\subsubsection{Плотность в $L^p$ непрерывных финитных функций}
\textit{Формулировка:}

\begin{itemize}
    \item $(\mathbb{R}^{m}, \mathfrak{M}^{m}, \lambda_m)$
\end{itemize}

Тогда $C_0(\mathbb{R}^{m})$ плотно в $L^{p}(\mathbb{R}^{m}, \lambda_m)$

\textit{Доказательство:}

\subsubsection{Интегрирование по мере Бореля--Стильтьеса, порожденной функцией распределения (с леммой)}
\textit{Формулировка:}

\begin{itemize}
    \item $f: \mathbb{R} \rightarrow \rinf \ge 0$, измерима по Борелю
    \item $h: X \rightarrow \rinf$, измерима, почти везде конечна
    \item $H$ --- функция распеределения
    \item $\mu_{H}$ --- мера Бореля-Стильетса
\end{itemize}

Тогда:

\[\int_{X} f\left(h(x)\right)d\mu(x) = \int_{\mathbb{R}} f(t)d\mu_{H}(t)\]

\textit{Доказательство:}

\subsubsection{Теорема об интегрировании по частям}

\textit{Формулировка: }

\begin{itemize}
    \item $g: [a, b] \rightarrow \mathbb{R}$, возрастающая
    \item $f$ --- абсолютно непрерывная функция ($C^{1}$) на $[a, b]$
    \item $\mu_{H}$ --- мера Бореля(Лебега ?)-Стильетса
\end{itemize}

Тогда:

\[\int_{[a, b)} f(x) dg(x) = fg|_{a}^{b} - \int_{[a, c]} f'(x)g(x)dx\]

\textit{Доказательство:}
\newpage


\subsubsection{Лемма о  ``почти признаке Дирихле''}
\textit{Формулировка:}

\begin{itemize}
    \item $-\inf < a < b \le +\inf$
    \item $f$ --- ``доп.'' на $[a, b)$ ($\forall A \in (a, b) f$ --- суммируема на $(a, A)$)
    \item $g(x)$ --- монотонно стремится к 0 при $x \rightarrow b - 0$
    \item Пусть функция $F(t) = \int_{a}^{t} fdx$ --- ограничена
\end{itemize}

Тогда:

\[\int_{a}^{\rightarrow b} fg dx\] --- сходится

\[\left|\int_{a}^{\rightarrow b} fg dx\right| \le |g(a)| \cdot \sup_{t \in (a, b)} \left|\int_{t}^{\rightarrow b} f dx\right|\]

\textit{Доказательство:}


\subsubsection{Следствие о ``почти признаке Абеля''}
\textit{Формулировка:}

\begin{itemize}
    \item $\int_{a}^{\rightarrow b} f dx$ --- сходится, $g$ --- монотонна и ограничена на $[a, b)$
\end{itemize}

Тогда: $\int_{a}^{\rightarrow b} f dx$ --- сходится, и к тому же: 

\[\left|\int_{a}^{\rightarrow b} fg\right| \le 5 (???) \cdot \sup_{(a, b)} \left|g(t)\right| \cdot \sup_{(a, b)} \left|\int_{t}^{\rightarrow b} f(x) dx \right|\]

\textit{Доказательство:}

\subsubsection{Признак Абеля равномерной сходимости интеграла}
\textit{Формулировка:}

\begin{itemize}
    \item $f, g: [a, b) \times Y \rightarrow \mathbb{R}$
    \item $\int_{a}^{\rightarrow b} f(x, y) dx$ --- равномерно сходящийся на $Y$
    \item $g(x, y)$ --- ограничена на $[a, b) \times Y$
\end{itemize}

Тогда:

\[\int_{a}^{\rightarrow b} f(x, y) g(x, y) dx\]

--- равномерно сходящийся на $Y$.

\textit{Доказательство:}

\newpage

\section{Период Мезозойский}
\subsection{Важные определения}

\newpage

\subsection{Определения}

\subsubsection{Кусочно-гладкий путь}

$\gamma$ --- кусочно-гладкий, $V$ --- непрерывно, $V = (V_1, V_2, \ldots, V_m)$

\[I(V, \gamma) = \int_a^b\langle V\left(\gamma(t)\right), \gamma'(t)\rangle dt = \]

\[\int_a^b V_1\left(\gamma(t)\right)\gamma'_1(t) + V_2\left(\gamma(t)\right)\gamma'_2(t) + \ldots + V_m\left(\gamma(t)\right)\gamma'_m(t) dt =\]

Делаем замену: $x = \gamma(t)$; $x_1 = \gamma_1(t), x_2 = \gamma_2(t)$; $dx_m = \gamma'_m(t) dt$

\[\int_{\gamma} V_1 dx_1 + V_2 dx_2 + \ldots + D_m dm\]

\newpage

\subsection{Важные теоремы}
\newpage

\subsection{Теоремы}
\subsubsection{Теорема о свойствах сходимости в гильбертовом пространстве}
\textit{Формулировка:}

\begin{itemize}
    \item $x_n \rightarrow x_0, y_n \rightarrow y_0 \quad \langle x_n, y_n \rangle \rightarrow \langle x_0, y_0 \rangle$
    \item $\sum_n x_n$ --- сходится
    
    Тогда $\forall y \in \mathcal{H} \quad \langle \sum x_n , y\rangle = \sum \langle x_n, y \rangle$

    \item $\sum x_n$ --- ортогональный ряд
    
    Тогда $\sum x_n$ --- сходится $\Leftrightarrow \sum ||x_n||^2$ --- сходится
\end{itemize}

\textit{Доказательство:}

\subsubsection{Теорема о коэффициентах разложения по ортогональной системе}
\textit{Формулировка:}

\begin{itemize}
    \item ${e_k}$ --- ортогональная система в $\mathcal{H}$
    \item $x \in \mathcal{H}$
    \item $x = \sum_{x = 1}^{\infty} c_k e_k$
\end{itemize}

Тогда:

\begin{enumerate}
    \item ${e_k}$ --- ЛНЗ
    \item $c_k = \frac{\langle x, e_k \rangle}{||e_k||^2}$
    \item $c_ke_k$ --- это проекция на прямую $l_k = {te_k, t \in \mathbb{R}}, x = c_ke_k + z, $ где $z \perp l_k$
\end{enumerate}

\textit{Доказательство:}


\subsubsection{Теорема о свойствах частичных сумм ряда Фурье. Неравенство Бесселя}
\textit{Формулировка:}

\begin{itemize}
    \item 
\end{itemize}

\textit{Доказательство:}
\end{document}

\begin{comment}
    \subsubsection{Теорема X-N}
    \textit{Формулировка:}

    \begin{itemize}
        \item 
    \end{itemize}

    \textit{Доказательство:}
\end{comment}