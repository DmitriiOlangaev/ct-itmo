\documentclass{article}
\usepackage[utf8]{inputenc}
\usepackage[T2A]{fontenc}
\usepackage[russian]{babel}
\usepackage{amsfonts}
\usepackage{amsmath}
\usepackage{amssymb}
\usepackage{arcs}
\usepackage{fancyhdr}
\usepackage{float}
\usepackage[left=3cm,right=3cm,top=3cm,bottom=3cm]{geometry}
\usepackage{graphicx}
\usepackage{hyperref}
\usepackage{multicol}
\usepackage{stackrel}
\usepackage{xcolor}
\usepackage{epigraph}
\usepackage{tikz}
\usepackage{amsthm}
\usepackage{graphics}
\usepackage{draftwatermark}
\usepackage{ marvosym }
\usepackage{physics}
\usepackage{pdfpages}



\def\letus{%
\mathord{\setbox0=\hbox{$\exists$}%
         \hbox{\kern 0.125\wd0%
               \vbox to \ht0{%
                  \hrule width 0.75\wd0%
                  \vfill%
                  \hrule width 0.75\wd0}%
               \vrule height \ht0%
               \kern 0.125\wd0}%
       }%
        }
\def\dbl{\,\,}
\def\image#1{\includegraphics[width=\linewidth]{static/#1}}
\def\imageh#1{\includegraphics[width=\linewidth, height=0.95\textheight]{static/#1}}
\def\images#1#2{\begin{center}\includegraphics[width=#1\linewidth]{static/#2}\end{center}}

\def\rsh#1{\underset{#1}{\rightrightarrows}}
\def\rshe{\rsh{E}}
\def\eqby#1{\underset{#1}{=}}
\def\rinf{\overline{\mathbb{R}}}
\def\goesto#1{\underset{#1}{\longrightarrow}}
\def\toinf#1{\goesto{#1 \rightarrow \infty}}
\def\ntoinf{\toinf{n}}

\DeclareMathOperator{\sign}{sign}
\DeclareMathOperator{\const}{const}
\DeclareMathOperator{\segm}{Segm}\DeclareMathOperator*{\esssup}{ess\,sup}



\newcommand*\lateraleye{%
       \scalebox{0.15}{
    \tikzset{every picture/.style={line width=0.75pt}} 
    \begin{tikzpicture}[x=0.75pt,y=0.75pt,yscale=-1,xscale=1]
    \draw  [line width=1.5]  (300,100.33) .. controls (326,122) and (352,135) .. (378,139.33) .. controls (352,143.67) and (326,156.67) .. (300,178.33) ;
    \draw  [fill={rgb, 255:red, 0; green, 0; blue, 0 }  ,fill opacity=1 ] (308.94,116.33) .. controls (313.87,116.33) and (317.86,125.51) .. (317.85,136.83) .. controls (317.84,148.15) and (313.84,157.33) .. (308.91,157.33) .. controls (303.99,157.32) and (300,148.14) .. (300.01,136.82) .. controls (300.02,125.5) and (304.02,116.32) .. (308.94,116.33) -- cycle ;
    \draw  [draw opacity=0][line width=1.5]  (314.84,166.6) .. controls (311.87,164.64) and (309.14,162.18) .. (306.76,159.24) .. controls (295.12,144.82) and (296.6,124.33) .. (310.07,113.45) .. controls (311.48,112.32) and (312.96,111.33) .. (314.5,110.49) -- (331.14,139.55) -- cycle ; \draw  [line width=1.5]  (314.84,166.6) .. controls (311.87,164.64) and (309.14,162.18) .. (306.76,159.24) .. controls (295.12,144.82) and (296.6,124.33) .. (310.07,113.45) .. controls (311.48,112.32) and (312.96,111.33) .. (314.5,110.49) ;
    \draw  [fill={rgb, 255:red, 255; green, 255; blue, 255 }  ,fill opacity=1 ] (304.43,124.2) .. controls (306.09,124.25) and (307.32,128.01) .. (307.18,132.6) .. controls (307.05,137.19) and (305.59,140.88) .. (303.93,140.83) .. controls (302.27,140.78) and (301.03,137.02) .. (301.17,132.43) .. controls (301.31,127.83) and (302.76,124.15) .. (304.43,124.2) -- cycle ;
    \end{tikzpicture}
    }\,}
    
\def\D{\,\mathrm{d}}

\let\vanillaparagraph\paragraph
\let\vanillasubparagraph\subparagraph
\renewcommand{\paragraph}[1]{\vanillaparagraph{#1}\mbox{}\\}
\renewcommand{\subparagraph}[1]{\vanillasubparagraph{#1}\mbox{}\\}

\graphicspath{{./images/}}

\setlength{\parindent}{0pt}

\setcounter{tocdepth}{4}
\setcounter{secnumdepth}{4}

\SetWatermarkText{$\underset{\text{@imodre @snitron}}{\text{ПРОДАМ ГАРАЖ}}$}
\SetWatermarkScale{2}
\SetWatermarkLightness{0.9}

\begin{document}
\DraftwatermarkOptions{stamp=false}
\begin{titlepage}
    \centering
    \vspace*{\baselineskip}
    \rule{\textwidth}{1.6pt}\vspace*{-\baselineskip}\vspace*{2pt}
    \rule{\textwidth}{0.4pt}\\[\baselineskip]
    {\LARGE СВЯТОЙ КПК\\ [0.3\baselineskip] \#BlessRNG}\\[0.2\baselineskip]
    \rule{\textwidth}{0.4pt}\vspace*{-\baselineskip}\vspace{3.2pt}
    \rule{\textwidth}{1.6pt}\\[\baselineskip]
    \scshape
    Или как не сдохнуть на 4 семе из-за матана \\
    \vspace*{2\baselineskip}
    Разработал \\[\baselineskip]
    {\Large Никита Варламов\quad @snitron}
        \vspace*{2\baselineskip}\par
    Почётный автор \\[\baselineskip]
    {\Large Тимофей Белоусов\quad @imodre}
    \vfill
    v0.0\\
    {\scshape Февраль-??? 2023} \par
\end{titlepage}

Вы в любой момент можете добавить любую недостающую теорему, затехав её и отправив код (фотографии письменного текста запрещены) в телегу любому из указанных авторов, или создав Pull Request в \href{https://github.com/snitron/ct-itmo}{Git-репозиторий конспекта (click)}. Ваше авторство также будет указано, с вашего разрешения.

\newpage

\begin{flushright}
\emph{Ah shit\\
Here we go again!\\
And again...\\
Oh, fuck.}
\end{flushright}


\tableofcontents


\setlength{\parskip}{6pt}%
\newpage
\DraftwatermarkOptions{stamp=false}


\section{Период Палеозойский}
\subsection{Важные определения}

\subsubsection{Пространство $L^p(E,\mu)$}
$1 \le p < +\infty$, $(X, \mathfrak{A}, \mu)$, $E \in \mathfrak{A}$

Тогда $\mathfrak{L}_p(E, \mu) = \{f:$ почти всех $E \rightarrow \mathbb{R} (\mathbb{C}), f$ --- измерима. $\int_{E} |f|^{p} < + \infty\}$

\begin{enumerate}
    \item $\mathfrak{L}_p(E, \mu)$ --- линейное пространство
    \item $f \equiv g$, если $f = g$ почти всюду
\end{enumerate}

$L_p := \mathfrak{L}_p /_{\equiv}$ --- точки этого пространства

$[f] = \{g: f \equiv g\}$
$[f_1] + [f_2] = [f_1 + f_2]$

И введём норму $||[f]|| = \left(\int_{E} |f|^{p}\right)^{\frac{1}{p}}$

\subsubsection{Пространство $L^\infty(E,\mu)$}

$\mathfrak{L}^{\infty}(E, \mu) = \{f: $ почти всех $E \rightarrow \mathbb{R} (\mathbb{C})$, измерима, $\esssup |f| < + \infty\}$

$||f||_{\infty} = \esssup f$

\textbf{Дописать всё}

\subsubsection{Существенный супремум}

$\esssup f = \inf \{a: f \le a $ почти всюду $\}$

$a$ --- существенная верхняя граница функции $f$, если при почти всех $x \dbl f(x) \le a$

\textit{Свойства: }
\begin{enumerate}
    \item $\esssup f(x) \le \sup f(x)$
    \item при почти всех $x: f(x) \le \esssup f(x)$
    \item $f$ --- суммируемая, $g$ --- измерима: $\esssup |g| < + \infty$
    \[\left|\int_{E} fg\right| \le \esssup |g| \cdot \int_{E} |f|\]
\end{enumerate}

\subsubsection{Гильбертово пространство}

$\mathfrak{H}$ --- линейное пространство, в котором задано скалярное произведение и соответствующая норма. Если $\mathfrak{H}$ --- полное, то оно называется гильбертовым.

\newpage

\subsection{Определения}
\subsubsection{Произведение мер}
$(X, \mathfrak{A}, \mu)$, $(Y, \mathfrak{B}, \nu)$ --- пространства с мерой.

\textit{Лемма: }
$\mathcal{A}, \mathcal{B}$ --- полукольца. Тогда $\mathcal{A} \times \mathcal{B} = \{A \times B, A \in \mathcal{A}, B \in \mathcal{B}\}$ --- полукольцо.

Также, множества из $\mathcal{A} \times \mathcal{B}$ являются измеримыми прямоугольниками.

$\mu, \nu$ --- $\sigma$-конечные меры. Тогда стандартное продолжение $m_{0}$ (в смысле теоремы о продолжении меры (?)) с полукольца $\mathfrak{A} \times \mathfrak{B}$, определённой на некоторой $\sigma$-алгебре $\mathfrak{A} \otimes \mathfrak{B}$, и являющееся $\sigma$-конечной полной мерой --- обзначается просто $m$.

И тогда $m$ --- и есть произведение мер $\mu$ и $\nu$ $(\mu \times \nu)$.

\textit{Замечание: }

\[(\mu \times \nu) \times \rho = \mu \times (\nu \times \rho)\]

\subsubsection{Сечения множества}

$X, Y$ --- множества. $C \subset X \times Y$

Тогда: 

\[C_{x} := \{y \in Y: (x, y) \in C\}\]

\[C^{y} := \{x \in X: (x, y) \in C\}\]

--- сечения множества $C$ (1 и 2 рода)

\textit{Замечания: }

\[\left(\bigcup_{\alpha \in A} C_{\alpha}\right)_{x} = \bigcup_{\alpha \in A} \left(C_{\alpha}\right)_{x}\]

\[\left(\bigcap_{\alpha \in A} C_{\alpha}\right)_{x} = \bigcap_{\alpha \in A} \left(C_{\alpha}\right)_{x}\]

\[\left(C \setminus C'\right)_{x} = C_{x} \setminus C'_{x}\]

\subsubsection{Полная мера, сигма-конечная мера}

См. \href{http://gg.gg/holykpksem3}{\color{blue}{конспект прошлого семестра}}

\subsubsection{Образ меры при отображении}

Пусть у нас есть $(X, \mathfrak{A}, \mu)$, $(Y, \mathfrak{B}, \_ )$ --- пространства с мерой, $\Phi: X \rightarrow Y$.

\begin{enumerate}
    \item $\forall \Phi \quad \Phi^{-1}(\mathfrak{B})$ --- $\sigma$-алгебра (это предлагается доказать как уражнение)
    \item Пусть $\Phi$ --- ``измеримо'' $\left(\Phi^{-1}(\mathfrak{B}) \subset \mathfrak{A}\right)$
\end{enumerate}

Для $E \in \mathfrak{B}$ зададим $\nu R := \mu\left(\Phi^{-1}(E)\right) = \int_{\Phi^{-1}(E)} 1 d \mu$

$\nu$ --- образ меры $\mu$ при отображении $\Phi$

\textbf{NB: ДОПИСАТЬ НА СЕССИИ, ТУТ ЕЩË ЕСТЬ ДОКАЗАТЕЛЬСТВО, ЧТО ЭТО МЕРА}

\subsubsection{Взвешенный образ меры}

$\omega: X \rightarrow \mathbb{R} \ge 0$, измерима на $X$

$B \in \mathfrak{B}, \tilde{\nu}(B) := \int_{\Phi^{-1}(B)} \omega d\mu$ --- тоже мера, это и есть взвешенный образ меры $\mu$ при отображении $\Phi$

\subsubsection{Плотность одной меры по отношению к другой}

$X = Y, \mathfrak{A} = \mathfrak{B}, \Phi = $ id

$\nu b = \int_{B} \omega d \mu$ --- ещё одна мера в $X$

Здесь $\omega$ называется плотностью меры $\nu$ относительно меры $\mu$. И в этом случае:

\[\int_{X}f(x)d\nu(x) = \int f(x) \cdot \omega(x) d\mu(x)\]

\subsubsection{Условие $L_{loc}$}

$f: X \times \tilde{Y} \rightarrow \rinf, Y \subset \tilde{Y}, a$ --- предельная точка $Y$ в $\tilde{Y}$. 

$f$ удовлетворяет условию $L_{loc}(a): \exists g: X \rightarrow \rinf$ --- суммируемая, $\exists U(a): \forall$ почти всех $x \forall y \in U(a)$:

\[|f(x, y)| \le g(x)\]

\subsubsection{Интеграл комплекснозначной функции}

База базовая: $(X, \mathfrak{A}, \mu), f: X \rightarrow \mathbb{C}$

\[\int_{E}f(z)d\mu = \int_{E}\Re(f(z))d\mu + i\int_{E}\Im(f(z))d\mu\]

Также измеримость и суммируемость следует из соттветствующих свойств реальной и мнимой частей функций.

\subsubsection{Фундаментальная последовательность, полное пространство}

$A \subset X$ --- нормированное пространство

$A$ --- (всюду) плотное в $X$

\[\forall x \in X \dbl \exists \varepsilon > 0 \quad B(x, \varepsilon) \cap A\text{--- непусто}\]

\subsubsection{Мера Лебега-Стилтьеса, мера Бореля-Стилтьеса}

\begin{enumerate}
    \item $\mathcal{P}^{1}, g: \mathbb{R} \rightarrow \mathbb{R}, $ возрастает, непрерывно
    \[\mu_g[a, b) := g(b) - g(a)\]

    --- счётно аддитивная мера
    \item $g$ --- возрастает, не обязательно непрпрывно
    
    \[\mu_g[a, b) = g(b - 0) - g(a - 0)\]

    --- мера
\end{enumerate}

Запускаем теорему о продолжении, тогда

$\exists \mathfrak{A} \supset \mathcal{P}^{1} \exists$ продолжение $\mu_g \subset \mathcal{P}$ на $\mathfrak{A}$

$\mu_g$ --- полная мера на $\mathfrak{A}$ --- мера Лебега-Стильтьеса

Если нассмотреть $\mu_g$ на борелевском $\mathfrak{B} \rightarrow \rinf$ --- мера Бореля


\subsubsection{Функция распределения}
$(X, \mathfrak{A}, \mu)$, $h: X \rightarrow \rinf$, измерима, вочти всюду конечна

$\forall t \in \mathbb{R} \quad \mu X(h < t) < +\infty$

Пусть $H(t) = \mu X(h < t)$ --- возрастающая

$H(t)$ --- называется функцией распределеиния по мере $\mu$

\subsubsection{Ортогональный ряд}

Ряд $\sum a_k$ --- ортогональный, если $\forall k, l a_k \perp a_l$

\subsubsection{Сходящийся ряд в гильбертовом пространстве}

$\sum a_n, a_n \in \mathfrak{H}$

$S_N := \sum_{1 \le n \le N} a_n$, если $\exists S \in \mathfrak{H}: S_N \goesto{\mathfrak{H}} S$

Такой ряд называется сходящимся.

\newpage

\subsection{Важные теоремы}

\subsubsection{Теорема Лебега о мажорированной сходимости для случая сходимости почти везде}
\textit{Формулировка:}

\begin{itemize}
    \item $(X, \mathfrak{A}, \mu)$ --- пространство с мерой
    \item $f_n, f: X \rightarrow \rinf$ --- измеримые
    \item $f_n \rightarrow f$ почти всюду
    \item $\exists g: X \rightarrow \rinf$ --- суммируемая, и $\forall n$ и при почти всех $x \dbl |f_n(x)| \le g(x)$
\end{itemize}

Тогда:

\[\int_{X}|f_n - f| d \mu \dbl \ntoinf \dbl 0\]

И, как очевидное (``уж тем более''):

\[\int_{X} f_n d\mu \dbl \ntoinf \dbl \int_{X} f d\mu\]

\textit{Доказательство:}

\subsubsection{Теорема Лебега о мажорированной сходимости для случая сходимости по мере}
\textit{Формулировка (то же самое, что и выше, только сходится по мере теперь):}

\begin{itemize}
    \item $(X, \mathfrak{A}, \mu)$ --- пространство с мерой
    \item $f_n, f: X \rightarrow \rinf$ --- измеримые
    \item $f_n \underset{\mu}{\Longrightarrow} f$
    \item $\exists g: X \rightarrow \rinf$ --- суммируемая, и $\forall n$ и при почти всех $x \dbl |f_n(x)| \le g(x)$
\end{itemize}

Тогда:

\[\int_{X}|f_n - f| d \mu \dbl \ntoinf \dbl 0\]

\textit{Доказательство:}

\subsubsection{Принцип Кавальери}
\textit{Формулировка:}

\begin{itemize}
    \item $(X, \mathfrak{A}, \mu), (Y, \mathfrak{B}, \nu)$ --- пространства с мерой
    \item $\mu, \nu$ --- $\sigma$-конечные меры
    \item $m = \mu \times \nu, \mathfrak{C} = \mathfrak{A} \otimes \mathfrak{B}$
\end{itemize}

Тогда: 

\begin{enumerate}
    \item при почти всех $x \quad C_{x} \in \mathfrak{B}$ 
    \item $x \mapsto \nu C_{X}$ --- измеримо на $C_{x}$
    \item $m C = \int_{X} \nu (C_{x}) d\mu(x)$
\end{enumerate}

Аналогично для сечений $C^{y}$

\textit{Доказательство:}


\subsubsection{Теорема Фубини}
\textit{Формулировка:}

\begin{itemize}
    \item $(X, \mathfrak{A}, \mu), (Y, \mathfrak{B}, \nu)$ --- пространства с мерой
    \item $\mu, \nu$ --- $\sigma$-конечные меры
    \item $m = \mu \times \nu$
    \item $f: X \times Y \rightarrow \rinf$, суммируема на $X \times Y$ по мере $m$
\end{itemize}

Тогда:

\begin{enumerate}
    \item при почти всех $x$ функция $f_{x}$ суммируема на $Y$
    \item $x \mapsto \varphi(x) = \int_{Y} f_{x} d\nu$ --- это суммируемая функция на $X$
    \item \[\int_{X \times Y} f dm= \int_{X} \varphi(x) d \mu(x) = \int_{X} \left( \int_{Y} f(x, y) d  \nu (y)\right) d \mu(x)\]
\end{enumerate}

\subsubsection{Теорема о преобразовании меры при диффеоморфизме}
\textit{Формулировка:}

\begin{itemize}
    \item $\Phi: O \subset \mathbb{R}^{m} \rightarrow \mathbb{R}^{m}$, диффеоморфизм
\end{itemize}

Тогда $\forall A \in \mathbb{M}^{m}, A \subset O$

\[\lambda \Phi(a) = \int_{A} |\det \Phi'(x)| dx\]

\textit{Доказательство:}



\subsubsection{Теорема о гладкой замене переменной в интеграле Лебега}
\textit{Формулировка:}

\begin{itemize}
    \item $\Phi: O \subset \mathbb{R}^{m} \rightarrow \mathbb{R}^{m}$, диффеоморфизм
    \item $\Phi(O) = O'$
    \item $f: O' \rightarrow \mathbb{R} \ge 0$ --- измерима
\end{itemize}

Тогда:

\[\int_{O'} fdx = \int_{O} f\left(\Phi(x)\right) \cdot |\det \Phi'(x) d\lambda(x)|\]

\textit{Доказательство:}

\newpage

\subsection{Теоремы}

\subsubsection{Теорема об интегрировании положительных рядов}
\textit{Формулировка:}

\begin{itemize}
    \item $(X, \mathfrak{A}, \mu)$ --- пространство с мерой
    \item $u_n: X \rightarrow \rinf, u_n \ge 0$ (при почти всех $x$ ?)
    \item $u_n$ --- измеримы на $E \in \mathfrak{A}$
\end{itemize}

Тогда: 

\[\int_{E} \left(\sum_{n = 1}^{\infty} u_n(x)\right)d\mu(x) = \sum_{n = 1}^{\infty} \left(\int_{E} u_n(x) d\mu(x)\right)\]

\textit{Доказательство:}

Подгоним под теорему Леви 3 (3 семестр). Пусть $S_{N}(x) = \sum_{n = 1}^{N} u_n(x)$ --- последовательность частичных сумм. Очевидно, что эта последовательность --- монотонно неубывающая (так как функции у нас неотрицательные): 

\[0 \le S_{N} \le S_{N + 1} \le S_{N + 2} \le \ldots\]

Тогда, делаем предельный переход (вот тут есть вопрос, почему должен существовать предел, но если подумать: если его не существует, вообще вся эта теорема не имеет смысла (ну бесконечности, чел, смысл их интегрировать)). А так же, измеримость сохраняется, так как у нас исходные функции все были измеримы (ну и по теореме о пределе измеирмых функций): 

\[S_{N}(x) \toinf{N} S(x)\]

Ну и всё, значи, по теореме Леви можем перейти к предельному преходу интегралов: 

\[\int_{E} S_{N}(x) d\mu(x) \toinf{N} \int_{E} S(x) d\mu(x)\]

Левую часть можно расписать по линейности интеграла (там у нас конечное число членов): 

\[\int_{E} S_{N}(x) d\mu(x) = \sum_{n = 1}^{N} \int_{E} u_n(x) d\mu(x)\]

Ну, а раз интграл суммы стремится к интегралу предельной функции, то и сумма интегралов обязана туда стремиться.

\[\sum_{n = 1}^{N} \int_{E} u_n(x) d\mu(x) \toinf{N} \sum_{n = 1}^{\infty} \int_{E} u_n(x) d\mu(x)\]

ч. т. д. 


\textit{Следствие: }

\begin{itemize}
    \item $u_n: X \rightarrow \mathbb{R}$, измеримы на $E \in \mathfrak{A}$
    \item $\sum \int_{E} |u_n(x)| d\mu < +\infty$ (конечна)
\end{itemize}

Тогда $\sum u_n(x)$ --- абсолютно сходящийся при почти всех $x$

\textit{Доказательство: }

Пусть: 

\[S(x) = \int_{n = 1}^{\infty} \left|u_n(x)\right|\]

Тогда, по предыдущей теореме: 

\[\int_{E} S(x) d\mu = \sum_{n = 1}^{\infty} \left(\int_{E} |u_n(x)| d\mu\right) < +\infty\]

Раз интеграл конечен, значит $S(x)$ --- суммируема, а это значит, что $S(x)$ --- почти везде конечна. Ну значит и сходится.

ч. т. д.

\textit{Пример: }

\begin{itemize}
    \item $(x_n)$ --- вещественная последовательность
    \item $\sum a_n$ --- абсолютно сходящийся числовой ряд
\end{itemize}

Тогда функциональный ряд $\sum \frac{a_n}{\sqrt{|x - x_n|}}$  --- абсолютно сходится при почти всех $x$ (в $\mathbb{R}$ по мере Лебега)

\textit{Доказательство: }

Во-первых, можно доказать, что если для $\forall A$ на $[-A, A]$ абсолютно сходится почти везде, то и везде (на $\mathbb{R}$) почти везде сходится (лол). Счётное количество п. в. $\Rightarrow$ п. в. (чтобы количество отрезков было счётным, надо чтобы $A$ были хотя бы рациональными. Кажется, что это не сильная проблема, так как отрезки включают в себя и все вещественные числа на отрезке тоже).

Попробуем подогнать под предыдущую теорему: 

\[\int_{[-A, A]}\frac{|a_n|}{\sqrt{|x - x_n|}} d\lambda = |a_n| \int_{-A}^{A} \frac{dx}{\sqrt{|x - x_n|}} \le\]

Так, стоп. А как мы перешли к определённому интегралу? Оказывается, что так можно делать, на доказано это будет позже (в курсе).

\[\underset{x := x - x_n}{\le} |a_n| \int_{-A - x_n}^{A - x_n} \frac{dx}{\sqrt{|x|}} \le |a_n| \int_{-A}^{A} \frac{dx}{\sqrt{|x|}} \le\]

Почему верен последний переход? Посмотрим на картинке: 

\images{0.5}{sh_pol_r.png}

Ну, по ней очевидно, что мы откусили кусочек поменьше, а добавили побольше. Тогда оценим модуль: 

\[ \le 2 \cdot |a_n| \int_{0}^{A}\frac{dx}{\sqrt{|x|}} = 4 \cdot \sqrt{A} \cdot |a_n|\]

Всё, абсолютный интеграл ограничен, значит сходится (при почти всех $x$).

ч. т. д. 

\subsubsection{Абсолютная непрерывность интеграла}
\textit{Формулировка:}

\begin{itemize}
    \item $(X, \mathfrak{A}, \mu)$ --- пространство с мерой
    \item $f: X \rightarrow \rinf$ --- суммируемая
\end{itemize}

Тогда:

\[\forall \varepsilon > 0 \dbl \exists \delta > 0, \quad \forall E\text{--- измеримое} \dbl \mu E < \delta \qquad \left|\int_{E} f d \mu\right| < \varepsilon\]

\textit{Доказательство:}

\textit{Следствие:}
\begin{itemize}
    \item $(e_n) \in \mathfrak{A}$ --- последовательность (?) множеств\
    \item $\mu e_n \ntoinf 0$
    \item $f$ --- суммируемая на $X$
\end{itemize}

Тогда:

\[\int_{e_n} f d \mu \ntoinf 0\]

\textit{Доказательство:}

\subsubsection{Теорема о произведении мер}
\textit{Формулировка:}

\begin{itemize}
    \item $(X, \mathfrak{A}, \mu)$, $(Y, \mathfrak{B}, \nu)$ --- пространства с мерой
    \item Зададим $m_0(A \times B) = \mu A \cdot \nu B$
\end{itemize}

Тогда:

\begin{enumerate}
    \item $m_0$ --- мера на $\mathfrak{A} \times \mathfrak{B}$
    \item $\mu, \nu$ --- $\sigma$-конечные меры $\Longrightarrow$ $m_0$ --- $\sigma$-конечная
\end{enumerate}

\textit{Доказательство:}

\subsubsection{Теорема Тонелли}
\textit{Формулировка:}

\begin{itemize}
    \item $(X, \mathfrak{A}, \mu), (Y, \mathfrak{B}, \nu)$ --- пространства с мерой
    \item $\mu, \nu$ --- $\sigma$-конечные меры
    \item $m = \mu \times \nu$
    \item $f: X \times Y \rightarrow \rinf \ge 0$, измерима относительно $\mathfrak{A} \otimes \mathfrak{B}$
\end{itemize}

Тогда:

\begin{enumerate}
    \item при почти всех $x$ функция$f_{x}$ измерима на $Y$
    \item $x \mapsto \varphi(x) = \int_{Y} f_{x} d\nu$ --- это измеримая функция на $X$
    \item \[\int_{X \times Y} f dm= \int_{X} \varphi(x) d \mu(x) = \int_{X} \left( \int_{Y} f(x, y) d  \nu (y)\right) d \mu(x)\]
\end{enumerate}

\textit{Доказательство:}



\subsubsection{Формула для бета-функции}
\textit{Формулировка:}
Бета-функция задаётся следующим образом: 

\[B(s, t) = \int_{0}^{1}x^{s - 1}(1 - x)^{t - 1}dx, \quad s, t > 0\]

Тогда:

\[B(s, t) = \frac{\Gamma(s)\Gamma(t)}{\Gamma(s + t)}\]

\textit{Доказательство:}

\subsubsection{Объем шара в $\mathbb{R}^m$}
\textit{Формулировка:}

\begin{itemize}
    \item $B(0, R) = \{x \in \mathbb{R}^{m}: x_1^{2} + x_2^{2} + \ldots + x_m^{2} \le R^{2}\}$
    \item $\alpha_{m} \ \lambda_{m}(B(0, 1))$
\end{itemize}

Тогда: 

\[\mu\left(B(0, R)\right) = \alpha_m R^{m}\]

\textit{Доказательство:}

\subsubsection{Теорема Фату. Следствия}
\textit{Формулировка:}

\begin{itemize}
    \item $(X, \mathfrak{A}, \mu)$ --- пространство с мерой
    \item $f_n \ge 0$ --- измерима
    \item $f_n \rightarrow f$ почти везде
    \item Если $\exists C > 0\ \dbl \forall n \int_{X} f_n d\mu \le C$  
\end{itemize}

Тогда:

\[\int_{X} f d\mu \le C\]

\textit{Доказательство:}

\textit{Следствие: }

То же самое, только меняем сходимость почти везде на: 

\begin{itemize}
    \item $f_n, f \ge 0$, измеримы, почти везде конечны
    \item $f_n \underset{\mu}{\Longrightarrow} f$
\end{itemize}

\textit{Следствие: }

\begin{itemize}
    \item $f_n \ge 0$, измеримы
\end{itemize}

Тогда:

\[\int_{X} \underline{\lim}f_n \le \underline{\lim}\int_{X} f_n\]


\subsubsection{Теорема о вычислении интеграла по взвешенному образу меры}
\textit{Формулировка:}

\begin{itemize}
    \item $(X, \mathfrak{A}, \mu), (Y, \mathfrak{B}, \_)$ --- пространства с мерой
    \item $\omega: X \rightarrow \rinf \ge 0$ --- измеримо
    \item $\Phi: X \rightarrow Y$ --- ``измеримое''
    \item $\nu$ --- взвешенный образ $\mu$ (с весом $\omega$)
\end{itemize}

Тогда для $\forall f: Y \rightarrow \rinf \ge 0$ --- измеримых:
\begin{enumerate}
    \item $f \circ \Phi$ --- измеримо (относительно $\mathfrak{A}$)
    \item $\int_{Y} f d\nu = \int_{X} f(\Phi(x))\cdot\omega(x) d\mu(x)$
\end{enumerate}
\textit{Доказательство:}

\subsubsection{Критерий плотности}
\textit{Формулировка:}

\begin{itemize}
    \item $(X, \mathfrak{A}, \mu)$ --- пространство с мерой
    \item $\nu$ --- ещё одна мера на $\mathfrak{A}$
    \item $\omega: X \rightarrow \rinf \ge 0$, измеримо 
\end{itemize}

Тогда эквивалентно:

\begin{enumerate}
    \item $\omega$ --- плотность $\mu$ отностительно $\mu$
    \item $\forall A \in \mathfrak{A} \quad \inf_{A} \omega \cdot \mu A \le \nu A \le \sup_{A} \omega \cdot \mu A$
\end{enumerate}

\textit{Доказательство:}

\subsubsection{Лемма о единственности плотности}
\textit{Формулировка:}

\begin{itemize}
    \item $f, g$ --- суммируемы на $X$
    \item $\forall A$ --- измеримое, $\int_{A} f = \int_{A} g$
\end{itemize}

Тогда $f = g$ почти везде

\textit{Доказательство:}

\textit{Следствие: }

Плотность меры определяется однозначно с точностью до изменения на множестве меры 0.

\subsubsection{Лемма об оценке мер образов малых кубов}
\textit{Формулировка:}

\begin{itemize}
    \item $\Phi: O \subset \mathbb{R}^{m} \rightarrow \mathbb{R}^{m}$
    \item $\Phi \in C^{1}$
    \item $a \in O$
    \item Пусть $c > |\det \Phi'(a)| \neq 0$
\end{itemize}

Тогда $\exists \delta > 0 \dbl \forall$ Куб $Q \subset B(a, \delta)$

\[\lambda \cdot \Phi(Q) < c \cdot \lambda Q\]

\textit{Доказательство:}

\subsubsection{Предельный переход по параметру в несобственном интеграле}
\textit{Формулировка:}

\begin{itemize}
    \item $f: \langle a, b \rangle \times Y \rightarrow \rinf$
    \item $Y \subset \tilde{Y}$ --- метризуемое
    \item $y_0 \in \tilde{Y}$ --- предельная точка $Y$
\end{itemize}

\begin{enumerate}
    \item при почти всех $x \exists f_0(x) = \lim_{y \rightarrow y_0} f(x, y)$
    \item $\forall t \in (a, b) \dbl \forall f(x, y_0), f(x, y)$ --- суммируемые по $x$ на $(a, t)$ и $\int_a^{t} f(x, y) dx \goesto{y \rightarrow y_0} \int_a^{t} f_0(x) dx$
    \item $J(y) = \int_a^{\rightarrow b} f(x, y)$ --- равномерно сходящаяся при $y \in Y$
\end{enumerate}

Тогда $\int_a^{\rightarrow b} f_0(x) dx$ --- существует (как несобственный)

\textit{Доказательство:}


\subsubsection{Предельный переход под знаком интеграла при наличии равномерной сходимости или $L_{loc}$}
\textit{Формулировка:}

\begin{itemize}
    \item $f: X \times \tilde{Y} \rightarrow \rinf$
    \item $X$ --- пространство с мерой, $\mu X < + \rinf$
    \item $\tilde{Y}$ --- метрезуемое топологическое пространство
    \item $Y \subset \tilde{Y}$
    \item $a \in \tilde{Y}$ --- предельная точка  $Y$
    \item $\forall y \in Y \quad x \mapsto f(x, y)$ --- суммируема на $X$
    \item Пусть $f(x, y) \rsh{y \rightarrow a} \varphi(x)$
\end{itemize}

Тогда $\varphi$ --- суммируема на $X$ и 

\[\lim_{y \rightarrow a} \int_{X} f(x, y) d \mu(x) = \int_{X} \varphi(x) d \mu(x)\]
 
\textit{Доказательство:}

\subsubsection{Правило Лейбница дифференцирования интеграла по параметру}
\textit{Формулировка:}

\begin{itemize}
    \item $Y$ --- промежуток $\subset \mathbb{R}$
    \item $f: X \times Y \rightarrow \mathbb{R}$
    \item $\forall \quad f(x, y)$ --- суммируемая функция от $x$
    \item При почти всех $x \dbl \forall y \exists f'_y(x, y)$
    \item $f'_y$ --- удовлетворяет условию $L_{loc}(y_0)$
\end{itemize}

Тогда:

\begin{itemize}
    \item $J(y) = \int_{X} f(x, y) d\mu(x)$ --- дифференцируема в $y_0$
    \item $J'(y_0) = \int_{X} f'_y(x, y) d \mu(x)$
\end{itemize}

\textit{Доказательство:}

\subsubsection{Теорема о вложении пространств $L^p$}
\textit{Формулировка:}

\begin{itemize}
    \item $\mu E < + \infty, 1 \le s < r \le +\infty$
\end{itemize}

Тогда:

\begin{enumerate}
    \item $L_r(E, \mu) \subset L_s(E, \mu)$
    \item $||f||_s \le \left(\mu E\right)^{\frac{1}{s} - \frac{1}{r}} \cdot ||f||_{r}$
\end{enumerate}

\textbf{fix}

\textit{Доказательство:}

\subsubsection{Теорема о сходимости в $L^p$ и по мере}
\textit{Формулировка:}

$1 \le p < +\infty \quad f_n \in L_p(E, \mu)$: 

\begin{enumerate}
    \item $f \in L_p \quad f_n \goesto{L_p} f$, тогда $f_n \underset{\mu}{\Longrightarrow} f$
    \item $f_n \underset{\mu}{\Longrightarrow} f$ [либо $f_n \rightarrow f$ почти всюду], $|f_n| \le g$ почти всюду, при всех $n$, где $g \in L^p$. Тогда $f_n \goesto{L_p} f$
\end{enumerate}
\textit{Доказательство:}



\subsubsection{Полнота $L^p$}
\textit{Формулировка:}

$L^p(E, \mu)$ ---- полное ($1 \le p < + \infty$)

\textit{Доказательство:}

\subsubsection{Плотность в $L^p$ множества ступенчатых функций}
\textit{Формулировка:}

\begin{itemize}
    \item $(X, \mathfrak{A}, \mu), 1 \le p \le +\infty$
\end{itemize}

Тогда множество ступенчатых функций плотно в $L_p(X, \mu)$

\textit{Доказательство:}

\subsubsection{Лемма Урысона}
\textit{Формулировка:}

\begin{itemize}
    \item $X$ --- нормированное топологическое пространство (например, $\mathbb{R}^m$)
    \item $F_0, F_1 \subset X$ --- замкнутое
    \item $F_0 \cap F_1 = \varnothing$
\end{itemize}

Тогда: $f: X \rightarrow \mathbb{R}, \quad 0 \le f \le 1$ --- непрерывное

$f|_{F_0} \equiv  0$, $f|_{F_1} \equiv 1$

\textit{Доказательство:}

\subsubsection{Плотность в $L^p$ непрерывных финитных функций}
\textit{Формулировка:}

\begin{itemize}
    \item $(\mathbb{R}^{m}, \mathfrak{M}^{m}, \lambda_m)$
\end{itemize}

Тогда $C_0(\mathbb{R}^{m})$ плотно в $L^{p}(\mathbb{R}^{m}, \lambda_m)$

\textit{Доказательство:}

\subsubsection{Интегрирование по мере Бореля--Стильтьеса, порожденной функцией распределения (с леммой)}
\textit{Формулировка:}

\begin{itemize}
    \item 
\end{itemize}

\textit{Доказательство:}
\newpage


\end{document}

\begin{comment}
    \subsubsection{Теорема X-N}
    \textit{Формулировка:}

    \begin{itemize}
        \item 
    \end{itemize}

    \textit{Доказательство:}
\end{comment}