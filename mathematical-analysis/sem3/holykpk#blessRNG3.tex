\documentclass{article}
\usepackage[utf8]{inputenc}
\usepackage[T2A]{fontenc}
\usepackage[russian]{babel}
\usepackage{amsfonts}
\usepackage{amsmath}
\usepackage{amssymb}
\usepackage{arcs}
\usepackage{fancyhdr}
\usepackage{float}
\usepackage[left=3cm,right=3cm,top=3cm,bottom=3cm]{geometry}
\usepackage{graphicx}
\usepackage{hyperref}
\usepackage{multicol}
\usepackage{stackrel}
\usepackage{xcolor}
\usepackage{epigraph}
\usepackage{tikz}
\usepackage{amsthm}
\usepackage{graphics}
\usepackage{draftwatermark}
\usepackage{ marvosym }
\usepackage{physics}
\usepackage{pdfpages}
\usepackage{yfonts}
\usepackage{ulem}

\def\letus{%
\mathord{\setbox0=\hbox{$\exists$}%
         \hbox{\kern 0.125\wd0%
               \vbox to \ht0{%
                  \hrule width 0.75\wd0%
                  \vfill%
                  \hrule width 0.75\wd0}%
               \vrule height \ht0%
               \kern 0.125\wd0}%
       }%
        }
\def\dbl{\,\,}
\def\image#1{\includegraphics[width=\linewidth]{static/#1}}
\def\imageh#1{\includegraphics[width=\linewidth, height=0.95\textheight]{static/#1}}
\def\images#1#2{\begin{center}\includegraphics[width=#1\linewidth]{static/#2}\end{center}}

\def\rsh#1{\underset{#1}{\rightrightarrows}}
\def\rshe{\rsh{E}}
\def\eqby#1{\underset{#1}{=}}

\DeclareMathOperator{\sign}{sign}
\DeclareMathOperator{\const}{const}
\DeclareMathOperator{\segm}{Segm}


\newcommand*\lateraleye{%
       \scalebox{0.15}{
    \tikzset{every picture/.style={line width=0.75pt}} 
    \begin{tikzpicture}[x=0.75pt,y=0.75pt,yscale=-1,xscale=1]
    \draw  [line width=1.5]  (300,100.33) .. controls (326,122) and (352,135) .. (378,139.33) .. controls (352,143.67) and (326,156.67) .. (300,178.33) ;
    \draw  [fill={rgb, 255:red, 0; green, 0; blue, 0 }  ,fill opacity=1 ] (308.94,116.33) .. controls (313.87,116.33) and (317.86,125.51) .. (317.85,136.83) .. controls (317.84,148.15) and (313.84,157.33) .. (308.91,157.33) .. controls (303.99,157.32) and (300,148.14) .. (300.01,136.82) .. controls (300.02,125.5) and (304.02,116.32) .. (308.94,116.33) -- cycle ;
    \draw  [draw opacity=0][line width=1.5]  (314.84,166.6) .. controls (311.87,164.64) and (309.14,162.18) .. (306.76,159.24) .. controls (295.12,144.82) and (296.6,124.33) .. (310.07,113.45) .. controls (311.48,112.32) and (312.96,111.33) .. (314.5,110.49) -- (331.14,139.55) -- cycle ; \draw  [line width=1.5]  (314.84,166.6) .. controls (311.87,164.64) and (309.14,162.18) .. (306.76,159.24) .. controls (295.12,144.82) and (296.6,124.33) .. (310.07,113.45) .. controls (311.48,112.32) and (312.96,111.33) .. (314.5,110.49) ;
    \draw  [fill={rgb, 255:red, 255; green, 255; blue, 255 }  ,fill opacity=1 ] (304.43,124.2) .. controls (306.09,124.25) and (307.32,128.01) .. (307.18,132.6) .. controls (307.05,137.19) and (305.59,140.88) .. (303.93,140.83) .. controls (302.27,140.78) and (301.03,137.02) .. (301.17,132.43) .. controls (301.31,127.83) and (302.76,124.15) .. (304.43,124.2) -- cycle ;
    \end{tikzpicture}
    }\,}
    
\def\D{\,\mathrm{d}}

\let\vanillaparagraph\paragraph
\let\vanillasubparagraph\subparagraph
\renewcommand{\paragraph}[1]{\vanillaparagraph{#1}\mbox{}\\}
\renewcommand{\subparagraph}[1]{\vanillasubparagraph{#1}\mbox{}\\}

\graphicspath{{./images/}}

\setlength{\parindent}{0pt}

\setcounter{tocdepth}{4}
\setcounter{secnumdepth}{4}

\SetWatermarkText{$\underset{\text{@imodre @snitron}}{\text{ПРОДАМ ГАРАЖ}}$}
\SetWatermarkScale{2}
\SetWatermarkLightness{0.9}

\begin{document}
\DraftwatermarkOptions{stamp=false}
\begin{titlepage}
    \centering
    \vspace*{\baselineskip}
    \rule{\textwidth}{1.6pt}\vspace*{-\baselineskip}\vspace*{2pt}
    \rule{\textwidth}{0.4pt}\\[\baselineskip]
    {\LARGE СВЯТОЙ КПК\\ [0.3\baselineskip] \#BlessRNG}\\[0.2\baselineskip]
    \rule{\textwidth}{0.4pt}\vspace*{-\baselineskip}\vspace{3.2pt}
    \rule{\textwidth}{1.6pt}\\[\baselineskip]
    \scshape
    Или как не сдохнуть на 3 семе из-за матана \\
    \vspace*{2\baselineskip}
    Разработал \\[\baselineskip]
    {\Large Никита Варламов\quad @snitron}
        \vspace*{2\baselineskip}\par
    Почётный автор \\[\baselineskip]
    {\Large Тимофей Белоусов\quad @imodre}
    \vfill
    v1.0\\
    {\scshape Октябрь-Январь 2022-2023} \par
\end{titlepage}

\textbf{Заметки авторов}

В данном конспекте названия всех задач имеют ссылку на своего автора в виде верхнего индекса:
\begin{enumerate}
    \item @imodre
    \item @snitron
\end{enumerate}
По любым вопросам и предложениям/улучшениям обращаться в телеграмм к соответвующему автору.

\textbf{Known Issues}



Вы в любой момент можете добавить любую недостающую теорему, затехав её и отправив код (фотографии письменного текста запрещены) в телегу любому из указанных авторов. Ваше авторство также будет указано, с вашего разрешения.
\newpage

\begin{flushright}
\emph{Ah shit\\
Here we go again!\\
And again...}
\end{flushright}

\tableofcontents


\setlength{\parskip}{6pt}%
\newpage
\DraftwatermarkOptions{stamp=false}


\section{Период Палеозойский}
\subsection{Важные определения}
\subsubsection{Норма линейного оператора}
Пусть $X, Y$ --- нормированные линейные пространства, $A \in \mathbb{L}(X, Y)$ (это множество линейных отображений над $X \rightarrow Y$). Тогда нормой линейного оператора называется $||A||_{X, Y} = \sup_{{x \in X}_{|x| = 1}}{|Ax|}_Y$

Замечания (для $X = \mathbb{R}^m, Y = \mathbb{R}^n$):
\begin{enumerate}
\item По лемме об ограниченности нормы линейного оператора ($L = (l_{i, j}), |Lx| \le C_L|x| = C_L \cdot 1 = \sqrt{\sum{l_{i, j}^2}}$) --- всегда ограничена!
\item $x \rightarrow |Lx|$ --- непрерывная функция, заданная на компакте ($|x| = 1 \Leftrightarrow x \in S(0, 1)$ --- сфера), причём по Вейерштрассу, максимум достигается. (напоминаю, мы в $\mathbb{R}^m!$)
\item Верно неравенство $\forall x \in \mathbb{R}^m: |Lx| \le ||L|| \cdot |x|$ (тут у нас важно различать евклидову и неевклидову норму). КПК считает, что это очевидно: 
    \begin{enumerate}
        \item $x = 0$ --- равенство
        \item $x \neq 0$ --- делим на норму $x: |L \frac{x}{|x|}| \le ||L||$, это очевидно, т.к. наша новая норма задаётся как супремум значений $|x| = 1$, ну и мы вот сравниваем супрермум с меньшими значениями.
    \end{enumerate}
\item $\forall x \in \mathbb{R}^m$, если нашлось $C > 0: |Lx| \le C \cdot |x| \Rightarrow ||L|| \le C$ --- тупо по пункту 3, очевидно.
\end{enumerate}

\subsubsection{Простое k-мерное гладкое многообразие в $\mathbb{R}^m$}

\textit{Обобщение вот всей этой темы с диффеоморфизмами в одно толковое определение}

$M \subset \mathbb{R}^m$ --- простое k-мерное $C^r$-гладкое многообразие в $\mathbb{R}^m$, если:

\begin{itemize}
    \item $\exists O \subset \mathbb{R}^k$ --- открытое (область?)
    \item $\exists \Phi: O \rightarrow \mathbb{R}^m, \Phi(O) = M$ --- гомеоморфизм (непрерывная биекция)
    \item $\Phi \in C^r(O)$
    \item $\forall x \in O: \rank \Phi'(x) = k$
\end{itemize}

$\Phi$ --- гладкая параметризация.

\subsubsection{Формулировка достаточного условия относительного экстремума}
\begin{itemize}
    \item $f: E \subset \mathbb{R}^{m + n} \rightarrow \mathbb{R}, \Phi: E \rightarrow \mathbb{R}^n, \quad f, \Phi \in C^1$
    \item $M_\Phi \subset E: \{x \dbl | \dbl \Phi(x) = 0\}$
    \item $a \in E$ --- точка относительного локального экстремума ($\forall x \in U(a) \cap M_\Phi, f(x_0) \le f(x)$ --- это нестрогий максимум, остальное аналогично)
    \item $\Phi(a) = 0$ --- уравнение связи
    \item $\rank \Phi'(a) = n$
\end{itemize}

\textit{это условия из необходимого условия}

\begin{itemize}
    \item $G(x) = f(x) - \lambda_1\Phi_1(x) - \lambda_2\Phi_2'(x) \ldots \lambda_n\Phi_n(x) = f - \langle \lambda, \Phi \rangle$
    \item $\lambda$ из необходимого условия
    \item $h \in \mathbb{R}^{m + n}, h = (h_x, h_y)$
    \item $\Phi'(a) \cdot h \neq 0 \Rightarrow$ можно выразить $h_y = \Psi(h_x)$
    \item $Q(h_x) = d^2G(a, (h_x, \Psi(h_x)))$ --- это квадратичная форма
\end{itemize}

Тогда: 

\begin{enumerate}
    \item $Q(h_x)$ --- положительно-определённая, тогда $a$ --- точка относительного локального минимума
    \item $Q(h_x)$ --- отрицательно-определённая, тогда $a$ --- точка относительного локального максимума
    \item $Q(h_x)$ --- незнако-определённая, тогда $a$ --- не точка относительного локального экстремума
    \item $Q(h_x)$ --- полу-определённая, тогда информации недостаточно (может быть и так, и так)
\end{enumerate}

\newpage

\subsection{Определения}

\subsubsection{Положительно-, отрицательно-, незнако- определенная квадратичная форма}

Квадратичная форма: $Q: \mathbb{R}^m \rightarrow \mathbb{R}$

\[Q(h) = \sum_{1 \le i, j \le m}{a_{ij}h_i h_j}\]


\begin{itemize}
    \item Положительно-: $\forall h \in \mathbb{R}^m \neq 0: Q(h) > 0$
    \item Отрицательно-: $\forall h \in \mathbb{R}^m \neq 0: Q(h) < 0$
    \item Незнако-: $\exists h \in \mathbb{R}^m \neq 0: Q(h) < 0, \exists \widetilde{h} \neq 0: Q(\widetilde{h}) > 0$
    \item Полуопределённая (положительно определённая вырожденая): $Q(h) \ge 0, \exists h \in \mathbb{R}^m \neq 0: Q(h) = 0$

\end{itemize}

\subsubsection{Локальный максимум, минимум, экстремум}
Рассмотрим только максимум, остальное аналогично (+ строгий)

$f: D \subset \mathbb{R}^m \rightarrow \mathbb{R}, a \in D$

Если $\exists U(a): \forall x \in U(a) \quad f(x) \le f(a)$, то $a$ --- точка локального максимума.

\subsubsection{Диффеоморфизм}
$F: O \subset \mathbb{R}^m \rightarrow \mathbb{R}^m, O$ --- открыто и связно (область)
\begin{itemize}
    \item $F$ --- обратимо
    \item $F$ --- дифференцируемо
    \item $F^{-1}$ --- дифференцируемо
\end{itemize}

Тогда $F$ --- диффеоморфизм

\subsubsection{Теорема о локальной обратимости}

\begin{itemize}
    \item $F: O \subset \mathbb{R}^m \rightarrow \mathbb{R}^m$
    \item $F \in C^1(O)$
    \item $x_0 \in O: \det F'(x_0) \neq 0$
\end{itemize}

Тогда $\exists U(x_0): F|_{U(x_0)}$ --- диффеоморфизм

\subsubsection{Формулировка теоремы о гладкости обратного отображения в терминах систем уравнений}

\begin{itemize}
    \item $F = (f_1, f_2, \ldots, f_m)$
    \item $\begin{cases}
        f_1(x_1, x_2, \ldots, x_m) = y_1\\
        f_2(x_1, x_2, \ldots, x_m) = y_2\\
        \vdots\\
        f_m(x_1, x_2, \ldots, x_m) = y_m
    \end{cases}$
    \item $(x_0, y_0): F(x_0) = y_0, \det \frac{\partial f_i}{\partial x_j} \neq 0$
    \item $\exists U(x_0), W(y_0): \exists F: U \rightarrow W$ --- диффеоморфизм $: \exists $ гладкое решение $ \begin{cases}
        x_1(y_1, y_2, \ldots, y_m)\\
        x_2(y_1, y_2, \ldots, y_m)\\
        \vdots\\
        x_m(y_1, y_2, \ldots, y_m)
    \end{cases}$
\end{itemize}

\subsubsection{Формулировка теоремы о неявном отображении в терминах систем уравнений}
\begin{itemize}
    \item $F = (x_1, x_2, \ldots, x_m, y_1, y_2, \ldots, y_n)$
    \item $\begin{cases}
        f_1(x_1, x_2, \ldots, x_m, y_1, y_2, \ldots, y_n) ,= 0\\
        f_2(x_1, x_2, \ldots, x_m, y_1, y_2, \ldots, y_n) = 0\\
        \vdots\\
        f_n(x_1, x_2, \ldots, x_m, y_1, y_2, \ldots, y_n) = 0
    \end{cases}$
    \item $(x^0, y^0): F(x^0, y^0) = 0, \det \frac{\partial f_i}{\partial y_j} \neq 0$
    \item $\exists U(x^0) \in \mathbb{R}^m, \dbl \varphi(x): F(x, \varphi(x)) = 0, x \in U(x_0)$ --- гладкие решения
\end{itemize}

\subsubsection{Касательное пространство к $k$-мерному многообразию в $\mathbb{R}^m$}

\begin{itemize}
    \item $M \subset \mathbb{R}^m$ --- простое $k$-мерное $C^r$-гладкое многообразие в $\mathbb{R}^m$
    \item $p \in M$
    \item $\Phi: O \subset \mathbb{R}^k \rightarrow \mathbb{R}^m$ --- параметризация $M \cap U(p)$
    \item $t^0 \in O: \Phi(t^0) = p$
\end{itemize}

Тогда $\Phi'(t^0): \mathbb{R}^k \rightarrow \mathbb{R}^m$ --- линейный оператор, образ $\Phi'(t^0)$ --- линейное подпространство в $\mathbb{R}^m$, не зависящее от $\Phi$. Ну вот оно и называется \textit{касательным пространством} ($T_p M$).

Причём важно, что это пространство не обязано проходить через точку $p$. Это просто пространство касательных векторов, откладываемых от начала координат (???). 

\subsubsection{Набор функций, независимый в окрестности точки}

% https://youtu.be/AWZbCBfOlt4?t=805

Набор функций $f_1 \ldots f_n: O \subset \mathbb{R}^m \rightarrow \mathbb{R}$ называется независимым в окрестности $x_0$, если:

$F := (f_1 \ldots f_n): O \rightarrow \mathbb{R}^n$

$F(x_0) := y_0$

$\forall$ достаточно малой окрестности $V(y_0)$ $\forall$ непрерывного G: $V(y_0) \rightarrow \mathbb{R}$ равенство $G(f_1(x) \ldots f_n(x)) \equiv 0 $ в $ U(x_0)$ выполняется только если $G \equiv 0$.

\newpage
\subsection{Важные теоремы}
\subsubsection{Достаточное условие экстремума}

\textit{Формулировка:}

\begin{itemize}
    \item $f: D \subset \mathbb{R}^m \rightarrow \mathbb{R}$
    \item $a \in Int(D)$
    \item $\nabla f (a) = 0$
    \item $f \in C^2(D)$
    \item $Q(h) := d^2f(a, h)$
\end{itemize}

Тогда: 

\begin{enumerate}
    \item $Q(h)$ --- положительно-определённая, тогда $a$ --- точка локального минимума
    \item $Q(h)$ --- отрицательно-определённая, тогда $a$ --- точка локального максимума
    \item $Q(h)$ --- незнако-определённая, тогда $a$ --- не точка локального экстремума
    \item $Q(h)$ --- полу-определённая, тогда информации недостаточно (может быть и так, и так)
\end{enumerate}


\textit{Доказательство:}

\textbf{(1)}

Давайте поближе присмотримся к $\forall h \in \mathbb{R}^m \dbl \forall t \in [0, 1]: \quad f(a + h) = f(a) + df(a, h) + \frac{1}{2!}d^2f(a + th, h)$ --- это типа формула Тейлора с остатком в форме Лагранжа.

Теперь рассмотрим разность $f(a + h) - f(a)$, и заметим, что $df(a, h) = 0$ по условию.

\begin{align*}
    f(a + h) &= f(a) + \frac{1}{2!}(f_{x_1, x_1}''(a+th)h_1^2 + f_{x_1, x_2}''(a + th)h_1h_2 + \ldots) \\
    &= f(a) + \frac{1}{2!}d^2f(a + th, h) \\
    &= f(a) + \frac{1}{2!}Q(h) + \frac{1}{2!}(d^2f(a + th, h) - Q(h))\\
    &= f(a) + \frac{1}{2!}Q(h) + \frac{1}{2!}(d^2f(a + th, h) - d^2f(a, h))\\
    &= f(a) + \frac{1}{2!}Q(h) + \frac{1}{2!}(f_{x_1, x_1}''(a+th)h_1^2 - f_{x_1, x_1}''(a)h_1^2 + f_{x_1, x_2}''(a + th)h_1h_2 - \ldots)
\end{align*}

Теперь заметим, что если повыносить коэффициенты при двойных производных, получится что-то в стиле $(f''_1 - f''_2)(\sum_{i, j}{h_i h_j})$, где левая скобка --- б.м. при $h \rightarrow 0$, а правая оценивается $|h|^2$. Таким образом, все эти штуки есть ничто иное, как $\alpha(h)|h|^2$, где $\alpha(h)$ --- б.м. при $h \rightarrow 0$.

В итоге получаем:

\[f(a + h) - f(a) \ge \frac{1}{2}Q(h) + \alpha(h)|h|^2 \underset{\text{по лемме об оценке кв. формы}}{\ge} \frac{\gamma_Q}{2}|h|^2 + \alpha(h)|h|^2\]
\[\underset{\text{при } h \rightarrow 0 }{\ge} \frac{\gamma_Q}{4}|h|^2 \underset{h \neq 0}{>} 0\]

Получается, что в окрестности нашей точки $a$ все значения больше, чем в ней самой. Получается, это по определению это точка локального минимимума.

\textbf{(2)}

Всё то же самое, только пусть мы рассматриваем функцию $g := -f$. С учётом отрицательно определённой квадратичной формы всё получится, и тут у нас точка локального максимума.

\textbf{(3)}

Шизофазия начинается тут. Т.к. у нас незнакоопределённая форма, значит $\exists h > 0: Q(h) > 0, \quad \exists \widetilde{h} > 0: Q(\widetilde{h}) < 0$

Раньше мы с вами считали, что $h$ может быть любым. Теперь же давайте рассмотрим относительно вот этих существующих $h, \widetilde{h}$. Но чтобы устремлять всё это дело к 0, нам необходим некоторый параметр. Пусть он будет $s$.
Тогда рассматриваем по тому же принципу: $f(a + sh) - f(a)$, рассуждения такие же, только там везде дополнительно вылезает $s^2$, и, таким образом, функции станут зависеть от него:
\[f(a + sh) - f(a) \ge \frac{1}{2}Q(sh) - |\alpha(s)|s^2 = \frac{s^2}{2}Q(h) - |\alpha(s)|s^2 \ge \frac{1}{4}Q(h)s^2\]

Вот, тут у нас получилось, что это минимум. А если отработаем с $\widetilde{h}$, то получится наоборот.

\textbf{(4)}

Ну а тут, слава Богу, достаточно привести пример.

Пусть $f(x) := x_1^2-x_2^4, \quad a = (0, 0)$

$df(a, h) = 0, \quad  d^2f(a,h) = 2h_1^2$

Видно, что в этом случае мы можем бегать и по $x_1$, и по $x_2$, и в итоге получим разные значения, потому что форма вообще зависит только от одной компоненты.

А для почти идентичной $g(x) := x_1^2 + x_2^4$ уже всё наоборот, и существует строгий локальный минимум.

ч. т. д. 

\subsubsection{Теорема о неявном отображении}
\textit{Формулировка:}

\begin{itemize}
    \item $F: O \subset \mathbb{R}^{m + n} \rightarrow \mathbb{R}^n$
    \item $(a, b) \in O, \quad a \in \mathbb{R}^m, b \in \mathbb{R}^n$
    \item $F(a, b) = 0 \in \mathbb{R}^{m + n}$
    \item $F \in C^r, r \in \mathbb{N} \cup \{\infty\}$
    \item $\det F_y'(a, b) \neq 0$
\end{itemize}

Тогда $\exists P(a) \subset \mathbb{R}^m, Q(b) \subset \mathbb{R}^n$ --- окрестности, и $\exists !\varphi : P \rightarrow Q \in C^r$ гладкое:

\[\forall x \in P: F(x, \varphi(x)) = 0\]

\textit{Бонус:}

\[\varphi'(x) = -(F_y'(x, \varphi(x)))^{-1} \cdot F_x'(x, \varphi(x)) \Leftrightarrow F_x'(x, \varphi(x)) + F_y' \cdot \varphi'(x) = 0 \text{ (продифференцировали условие)}\]

\textit{Доказательство:}

\image{neyav_otobr}

Нет, это не шутка. Всё доказательство строится вокруг одной картинки и яростного махания руками со знанием дела.

Заведём $\Phi(x, y): O \subset \mathbb{R}^{m + n} \rightarrow \mathbb{R}^{m + n}, \quad \Phi(x, y) = (x, F(x, y))$. Логично, что по условию $\Phi(a, b) = (a, 0)$. Если посмотреть на производный оператор (а она дифференцируема, так как $F$ --- дифференцируема (?)), то прекрасно видно, что матрица квадратная, да ещё и блочная $\Rightarrow \det \Phi'(a, b) = \det E_m \cdot \det F_y'(a, b)$. По условию ничего из этого не 0, следовательно определитель невырожден. А поэтому, по теореме о локальной обратимости: $\Phi$ --- локальный диффеоморфизм класса $C^r$.

Заведём окрестность (как декартово произведение, почему бы и нет) $\widetilde{U} = P_1 \times Q$. $P_1$ немного большевата для $P$, поэтому потом мы её немного подрежем. $ \widetilde{V} = \Phi(\widetilde{U})$. Заметим, что все эти окрестности открыты по предыдущим теоремам.

Т.к. у нас $\Phi|_{\widetilde{U}}$ --- диффеоморфизм, на прообразе и образе имеет место быть обратное отображение $\Psi: \widetilde{V} \rightarrow \widetilde{U} = \Phi^{-1}$.

Заметим, что отображение $\Phi$ не меняет ``$x$''-овые координаты (по построению функции ,см. рисунок), ``$y$''-овые же как-то колбасит, как показано зелёной областью. Значит и $\Psi$ их тоже не меняет, т.к. диффеоморфизм. Именно поэтому справа у нас координаты $(x, v)$. Можно представить $\Psi(x, v) = (x, H(x, v)), \quad H: \widetilde{V} \rightarrow \mathbb{R}^n \in C^r$. Поэтому давайте выберем окрестность $P \subset \mathbb{R}^m := \widetilde{V} \cap (\mathbb{R}^n \times \{0\})$. Она открыта по теореме (1 сем) о свойствах открытых множеств (конечное пересечение открытых открыто). $U = P \times Q$

Вооот. А теперь давайте предложим в качестве $\varphi(x): P \rightarrow Q := H(x, 0)$. Она прнадлежит классу $C^r$, т.к. все функции до этого в нём лежали. А почему выполняется условие $F(x, \varphi(x)) = 0, x \in P$? Ну давайте проследим путь. Что такое вообще $H(x, 0)$ --- мы берём все точки вида $(x, 0)$ (см. картинку), и взаимно-однозначно отправляем их обратно в левую часть, тем самым вычисляя им значение $b_0 \in Q(b)$ (этим и занимается $H(x, v)$ по своей сути). Ну вот. А потом мы отправляем точку $(x, b_0)$ в правую часть, и куда же она должна приехать, если уезжала из 0? Правильно, в 0. Ура, условие выполняется.

Осталось доказать едиственность, опять давайте помашем руками:

\[x \in P, y \in Q: F(x, y) = 0, \quad \Phi(x, y) = (x, 0)\]

\[(x, y) = \Psi\Phi(x, y) = \Psi(x, 0) = (x, H(x, 0)) = (x, \varphi(x))\]

ч. т. д.

\subsubsection{Необходимое условие относительного локального экстремума}
\textit{Формулировка:}

\begin{itemize}
    \item $f: E \subset \mathbb{R}^{m + n} \rightarrow \mathbb{R}, \Phi: E \rightarrow \mathbb{R}^n, \quad f, \Phi \in C^1$
    \item $M_\Phi \subset E: \{x \dbl | \dbl \Phi(x) = 0\}$
    \item $a \in E$ --- точка относительного локального экстремума ($\forall x \in U(a) \cap M_\Phi, f(x_0) \le f(x)$ --- это нестрогий максимум, остальное аналогично)
    \item $\Phi(a) = 0$ --- уравнение связи
    \item $\rank \Phi'(a) = n$
\end{itemize}

Тогда $\lambda \in \mathbb{R}^n (\lambda_1, \lambda_2, \ldots, \lambda_n): \begin{cases}
    f'(a) + \lambda\Phi'(a) = 0 \\
    \Phi(a) = 0
\end{cases}$

Второе условие бесплатное, оно из условия.

\textit{Доказательство:}

Так как у нас ранг $n$ на матрице производного оператора $\Phi'$, давайте считать, что он достигается на $m + 1 \ldots m + n$ ($n$ штук) столбцах матрицы (это матрица $n$ строк $\times \dbl (m + n)$ столбцов). Тогда в стиле всех предыдущих теорем а-ля ``неявное отображение'' разделим переменные: $(x_1, x_2, \ldots, x_m), (x_{m + 1}, x_{m + 1}, \ldots, x_{m + n}) \mapsto (x, y)$. Точку $a$ тоже: $(a_x, a_y)$. 

Запускаем теорему о неявном отображении: $\Phi(a) = 0, \frac{\partial \Phi}{\partial y}$ --- невырожденный оператор. Тогда существует $! \varphi: U(a_x) \rightarrow P(a_y), \Phi(x, \varphi(x)) = 0$. Замечаем, что $x \mapsto (x, \varphi(x))$ --- параметризация простого гладкого $m$-мерного многобразия в $M_\Phi \cap \{U(a_x) \times P(a_y)\}$.

Тогда для $g(x) = f(x, \varphi(x))$ точка $a_x$ --- просто точка локального экстремума. Почему? Управляя теперь точкой $x$, мы с помощью $g$ попадаем в $M_\Phi$, внутри которого $\Phi(x) = 0$ всегда! Поэтому внутри хорошего (в рамках этой задачи) множества мы и ищем экстремум. Это можно легко понять, если представить поиск экстремума на какой-то области графика (ради этого всё и делается же).

Хорошо, давайте его искать. По необходимому условию экстремума, ЧП $g$ должны быть равны нулю ($\varphi'$ бывает только по $x$):

\[f_x'(a) + f_y'(a)\cdot\varphi'(a_x) = 0 \qquad (\in \mathbb{R}^m)\]

Начииная с этого места опускаем подстановку точек, но они там есть! Вспоминаем, что у нас есть $\Phi(x, \varphi(x)) = 0$. Также дифференцируем: 

\[\Phi_x' + \Phi_y'\varphi' = 0 \qquad (\in Mat(n, m))\]

\[\forall \lambda \in \mathbb{R}^n: \qquad \lambda\Phi'_x + \lambda\Phi_y'\varphi' = 0 \qquad (\in \mathbb{R}^m)\]

Тогда можно вычесть из уравнения с $f$ уравнение с $\Phi$ --- размерности сошлись:

\[f_x' - \lambda\Phi_x' + (f_y' - \lambda\Phi_y')\varphi' = 0\]

Пусть $f_y' - \lambda\Phi_y' = 0$. Тогда:

\[\lambda = f_y' \cdot \left(\Phi_y'\right)^{-1}\]

Если мы берём это $\lambda$ (а нас и просят её предъявить), то наше предположение верно. Раз разность 0, то и иксовая разность равна нулю:

\[\begin{cases}
    f_y' - \lambda\Phi_y' = 0\\
    f_x' - \lambda\Phi_x' = 0
\end{cases}\]

Это векторная запись точек, которые мы когда-то разъединили. Давайте соединим обратно: 

\[f' - \lambda\Phi' = 0 \qquad \lambda = f_y' \cdot \left(\Phi_y'\right)^{-1}\]

ч. т. д. 

\newpage

\subsection{Теоремы}

\subsubsection{Лемма об условиях, эквивалентных непрерывности линейного оператора}

\textit{Формулировка:}

Пусть $X, Y$ --- нормированные линейные пространства, $A \in \mathbb{L}(X, Y)$.

Тогда следующие утверждения эквиваленты:

\begin{enumerate}
\item $A$ --- ограниченный оператор, в том смысле, что $||A||$ --- конечно
\item $A$ --- непрерывно в нуле
\item $A$ --- непрерывно на всём $X$
\item $A$ --- равномерно непрерывно
\end{enumerate}


\textit{Доказательство:}
Для $||A|| \equiv 0$ --- тривиально (супремум = 0, следовательно 0), поэтому далее считаем норму оператора ненулевой.
Ну, во-первых, $4 \Rightarrow 3 \Rightarrow 2$ --- очевидно, просто одно следует из другого.

Во-вторых,  $2 \Rightarrow 1$:

По определению непрерывности в нуле: $\forall \varepsilon > 0, \dbl \exists \delta  \forall x \in B(0, \delta): |Ax| < \varepsilon$ (это нам дано, значит можем пользоваться, как хотим)

Давайте рассмотрим $\varepsilon = 1: |Ax| < 1$, потом делим на $\delta$:

\[|A\frac{x}{\delta}| < \frac{1}{\delta}\]
Переназначим $x$ и заметим, что $x \in \overline{B(0, 1)}: |Ax| \le \frac{1}{\delta}$ (обратите внимание, мы взяли замыкание шара и получили нестрогое неравенство)

Тогда для $x \in S(0, 1): |Ax| \le \frac{1}{\delta} = \frac{1}{\delta} \cdot |x|$ --- по замечанию 4 из определения, $||L|| \le \frac{1}{\delta}$.

В-третих, $1 \Rightarrow 4$:

Давайте опять запишем определение равномерной непрерывности:

\[\forall \varepsilon > 0 \exists \delta: \forall x_1, x_2: |x_1 - x_2| < \delta |f(x_1) - f(x_2)| < \varepsilon\]

Назначим $\delta := \frac{\varepsilon}{||A||}$
\[|Ax_1 - Ax_2| < \varepsilon\]

По линейности:

\[|A(x_1 - x_2)| < ||A|| \cdot |x_1 - x_2| < ||A|| \delta = ||A|| \frac{\varepsilon}{||A||} = \varepsilon\]

ч.т.д.

\subsubsection{Теорема Лагранжа для отображений}
\textit{Формулировка:}
$F: D \subset \mathbb{R}^m \rightarrow \mathbb{R}^l, D$ --- открытое

$F$ --- дифференцируемо на $D$, $[a, b] \subset D$

Тогда $\exists c \in [a, b]: |F(a) - F(b)| \le ||F'(c)|| \cdot |b - a|$
\textit{Доказательство:}
Заведём функцию $f(t) = F(a + t(b - a)), t \in [0, 1] \subset \mathbb{R}$. То есть как-бы двигаем точку по $[a, b]$.

\[f'(t) = F'(a + t(b - a))(b - a)\]

Заметим, что это оператор $\mathbb{R} \rightarrow \mathbb{R}^l$, т.к. $F'(a + t(b - a))$ --- $l$, а $b - a$ --- $m$ (???)

Вспомним также теорему Лагранжа для векторнозначных функций:

$F: [a, b] \rightarrow \mathbb{R}^n, F$ --- дифференцируема на $[a, b], \exists c \in [a, b]$
\[|F(a) - F(b)| \le |F'(c)| \cdot |b - a|\]

Рассмотрим нашу функцию $f(t)$ по этой теореме в точках 0 и 1:

\[|f(1) - f(0)| = |f'(c)| \cdot |1 - 0|\]
Подставим:
\[|F(b) - F(a)| \le |F'(a + c(b - a))\cdot(b - a)| \underset{\text{по замечанию 3}}{\le} ||F'(a + c(b - a))|| \cdot |b - a|\]
Ну а дальше, пусть $c := a + c(b - a)$ и всё супер.

ч.т.д.



\subsubsection{Теорема об обратимости линейного отображения, близкого к обратимому}
\textit{Формулировка (безымянная лемма):}

\textit{Возможно, она нахер не нужна, но пусть всё же будет}

Пусть $B \in \mathbb{L}(\mathbb{R}^m, \mathbb{R}^m)$.

Если $c > 0: \forall x \in \mathbb{R}^m: |Bx| \ge c|x|$, тогда $B \in \Omega_m$ и $||B^{-1}|| \le \frac{1}{c}$

\textit{Доказательство:}

$B$ --- очевидно инъективен, т.к. любой ненулевой вектор у нас отправляется в разные точки $\Rightarrow$ биекция $\Rightarrow$ обратимый $\Rightarrow \exists B^{-1}$

Теперь пусть $x = B^{-1}y \Rightarrow |Bx| = |y| \ge c|x| = c |B^{-1}y| \Rightarrow |B^{-1}y| \le \frac{1}{c} \cdot |y| \underset{\text{по замечанию 3}}{\Rightarrow} ||B^{-1}|| \le \frac{1}{c}$

ч.т.д.

\textit{Замечание:}

Если $A \in \Omega_m$, то можно провенуть такую штуку: $|x| = |A^{-1}Ax| \le ||A^{-1}||\cdot|Ax|$ (по 3 замечанию). Тогда:
\[|Ax|\ge \frac{1}{||A^{-1}||} |x|\]

\textit{Формулировка:}

Пусть $L \in \Omega_m$ --- обратимый оператор, $M \in \mathbb{L}(\mathbb{R}^m, \mathbb{R}^m)$, $||L - M|| < \frac{1}{||L^{-1}||}$

Тогда: \begin{enumerate}
    \item $M \in \Omega_{m}$ --- обратимый
    \item $||M^{-1}|| \le \frac{1}{\frac{1}{|L^{-1}|} - ||L - M||}$
    \item $||L^{-1} - M^{-1}|| \le \frac{||L^{-1}||}{\frac{1}{|L^{-1}|} - ||L - M||} \cdot ||L - M||$
\end{enumerate}

\textit{Доказательство:}

\textbf{(1) и (2)}

Рассмотрим $|Mx|$ с рандомным возможным $x$. По неравенству треугольника (это всё же норма) и оценкам по замечаниям сверху:

\[|Mx| \ge |Lx| - |(M - L)x| \ge \frac{1}{||L^{-1}||}|x| - ||M - L||\cdot|x| = \left(\frac{1}{||L^{-1}||} - ||M - L||\right)|x|\]

По безымянной лемме всё доказано (заметим, что выражение в скобочках --- положительная константа).

\textbf{(3)}

Неповторимый оригинал:
\[\frac{1}{l} - \frac{1}{m} = \frac{m - l}{ml}\]

Жалкая копия (доказывается тривиально, раскрытием скобок):
\[L^{-1}  - M^{-1} = M^{-1}(M - L)L^{-1}\]

Отнормируем:
\[||L^{-1} - M^{-1}|| \le ||M^{-1}|| \cdot ||L - M|| \cdot ||L^{-1}||\]

Ну и просто подставим (2).

ч.т.д.

\textit{Следствие:}

Отображение $\Omega_m \rightarrow \Omega_m: L \rightarrow L^{-1}$ непрерывно.

\textit{Доказательство:}

Давайте по Гейне: если $B_k \rightarrow L$, то сходится ли $B^{-1}_k \rightarrow L^{-1}$????

Во-первых, начиная с некоторого места:

\[|B_k - L| \le \frac{1}{||L^{-1}||}\]

\[|B^{-1}_k - L^{-1}| \le \frac{||L^{-1}||}{\underbrace{\frac{1}{|L^{-1}|} - \underbrace{||L - B_k||}_{\rightarrow 0}}_{\text{огр.}}} \cdot ||L - B_k|| \rightarrow 0\]

ч.т.д.


\subsubsection{Теорема о непрерывно дифференцируемых отображениях}
\textit{Формулировка:}

$F: D \subset \mathbb{R}^m \rightarrow \mathbb{R}^l, F$ дифференцируема на $D, F': D \rightarrow \mathbb{L}(\mathbb{R}^m, \mathbb{R}^l)$

Тогда следующие утверждения эквивалентны:

\begin{enumerate}
    \item $F \in C^1(D) \Leftrightarrow \forall i, j: \frac{\partial f_i}{\partial x_j}$ --- непрерывны
    \item $F'$ --- непрерывно на $D: \forall x: \mathbb{R}^m \dbl \forall \varepsilon > 0 \dbl \exists \delta > 0: \dbl \forall \widetilde{x} \dbl |x - \widetilde{x}| < \delta \dbl ||F'(x) - F'(\widetilde{x})|| < \varepsilon$
\end{enumerate}

\textit{Доказательство:}

\textbf{(1) $\Rightarrow$ (2)}

Давайте зафиксируем какие-то $i, j$ и относительно них рассмотрим наше условие непрерывности частных производных по отдельности. Также, применим китайскую грамоту и возьмём немного другой эпсилон:
\[\forall x: \mathbb{R}^m \dbl \forall \varepsilon > 0 \dbl \exists \delta > 0: \dbl \forall \widetilde{x} \dbl |x - \widetilde{x}| < \delta \dbl \left|\frac{\partial f_i}{\partial x_j}(x) - \frac{\partial f_i}{\partial x_j}(\widetilde{x})\right| < \frac{\varepsilon}{\sqrt{ml}}\]

Тогда, так как нам это уже известно, проверим условие $(2)$:
\[||F'(x) - F'(\widetilde{x})|| \underset{\text{по лемме об ограниченности нормы}}{\le} \sqrt{\sum_{i, j}{\left(\frac{\partial f_i}{\partial x_j}(x) - \frac{\partial f_i}{\partial x_j}(\widetilde{x})\right)^2}}\]

Ну а теперь просто оцениваем всё это дело эпсилоном!
\[\le \sqrt{\sum_{i, j}{\frac{\varepsilon^2}{ml}}}=\sqrt{ml\cdot\frac{\varepsilon^2}{ml}} = \varepsilon\]

\textbf{(2) $\Rightarrow$ (1)}

Ну а вот тут душный пиздец. Идея в том, что мы хотим проверить для каждой частной производной с индексами $(v, u)$ наше предположение.

Давайте выберем $h \in \mathbb{R}^m = (0,0,0, \ldots, 0, \underbrace{1}_{u\text{-ое число}}, 0, \ldots, 0, 0)^T$.
Теперь нам известно, что:
\[|(F'(x) - F'(\widetilde{x}))h| \le ||F'(x) - F'(\widetilde{x})|| \cdot |h| \underset{|h| = 1}{\le} \varepsilon\]

Ну а с другой стороны, $(F'(x) - F'(\widetilde{x}))h$ есть ничто иное, как вектор $\left(\frac{\partial f_i}{\partial x_u}(x) - \frac{\partial f_i}{\partial x_u}(\widetilde{x})\right)_{i = 1 \ldots l}$. Поэтому давайте рассмотрим его норму по вышеиспользованной лемме:
\[\sqrt{\sum_{i = 1}^l{\left(\frac{\partial f_i}{\partial x_u}(x) - \frac{\partial f_i}{\partial x_u}(\widetilde{x})\right)^2}} \le \varepsilon\]

Ну, раз уж у нас корень суммы квадратов меньше, то и каждое слагаемое по отдельности тоже меньше. Давайте зафиксируем $i = v$ и получим долгожданное:
\[\left|\frac{\partial f_v}{\partial x_u}(x) - \frac{\partial f_v}{\partial x_u}(\widetilde{x})\right| \le \varepsilon\]

Так как данные эпсилон-дельта преамбуды везде были одинаковыми, то и тут всё супер. Доказано, не умаляя общности!!!!

ч. т. д.

\subsubsection{Теорема Ферма. Необходимое условие экстремума. Теорема Ролля}
\textit{Формулировка (Ферма):}

$f: D \subset \mathbb{R}^m \rightarrow \mathbb{R}, a \in Int(D), f$ --- дифференцируема в точке $a$ (точка локального экстремума)

Тогда $\forall l \in \mathbb{R}^m: |l| = 1$ (направление) $\frac{\partial f}{\partial l}(a) = 0$


\textit{Доказательство:}

Тривиалити, для $f|_{\text{прямая через }a\text{ по направлению }l} a$ --- тоже точка локального экстремума, поэтому по одномерной теореме Ферма всё работает!

ч. т. д.

\textit{Следствие (Необходимое условие экстремума)}

$a$ --- точка локального экстремума $\Rightarrow \forall k \in [1, m]: \frac{\partial f}{\partial x_k} = 0$

\textit{Следствие (Ролль)}
\begin{itemize}
    \item $f: D \subset \mathbb{R}^m \rightarrow \mathbb{R}$
    \item $K \subset D$ --- компакт
    \item $f$ --- дифференцируема в $Int(K)$, непрерывна на $K$
    \item $f|_{\text{граница }K} = \const$
\end{itemize}

Тогда $\exists a \in Int(K): \nabla f \equiv 0$

\textit{Доказательство}

По теореме Вейерштрасса (привет, 1 сем), на компакте функция достигает своего минимимума и максимума.
Тогда либо у нас на $K f \equiv const$, тогда такая точка --- любая, либо же по теореме Ферма она существует где-то внутри компакта.

ч. т. д.



\subsubsection{Лемма об оценке квадратичной формы и об эквивалентных нормах}
\textit{Формулировка (Лемма об оценке квадратичной формы):}

$Q$ --- положительно определённая квадратичная форма.

Тогда $\exists \gamma_Q: \forall h \quad Q(h) \ge \gamma_Q \cdot |h|^2$

\textit{Доказательство:}

А давайте так: $$\gamma_Q := \min_{|x| = 1}{Q(x)}$$. Он достигается, так как мы гоняем по компакту (сфере), следовательно по Вейерштрассу всё хорошо.

Для $x = 0$ всё тривиально, поэтому при $x \neq 0: Q(x) = |x|^2Q(\frac{x}{|x|}) \underset{\frac{x}{|x|}\text{по модулю равен 1 (сфера)}}{\ge} \gamma_Q|x|^2$

\textit{Формулировка (Лемма об эквивалентных нормах):}

$p: \mathbb{R}^m \rightarrow \mathbb{R}$ --- норма

Тогда $\exists \dbl C_1, C_2 > 0: \forall x \quad C_1|x| \le p(x) \le C_2|x|$

\textit{Доказательство:}

То же самое: $$C_1 := \min_{|x| = 1}{p(x)}, \quad C_2 := \max_{|x| = 1}{p(x)}$$

почти локальной инъективностиДля минимума: $\forall x: p(x) = |x| \cdot p(\frac{x}{|x|}) \ge C_1|x|$, для максимума аналогично.

Осталось лишь доказать, что норма непрерывна, чтобы максимум и минимум достигался. Оценим (разложим по базису $\{e_i\}_{i = 1}^{n}$):

\[p(x, y) \le p\left(\sum (x_i - y_i)e_i\right) \underset{\text{неравенство треугольника}}{\le} \sum|x_i - y_i|p(e_i) \underset{\text{КБШ}, M = \sqrt{\sum p^2(e_k)}}{\le} M \cdot |x - y|\]

Ну всё, фигня, изменяется всего на какую-то константу, значит непрерывно.

ч. т. д.

\subsubsection{Лемма о ``почти локальной инъективности''}
\textit{Формулировка:}

\begin{itemize}
    \item $F: O \subset \mathbb{R}^m \rightarrow \mathbb{R}^m$
    \item $x_0 \in O$
    \item $F$ --- дифференцируема в $x_0$
    \item $\det F'(x_0) \neq 0 $
\end{itemize}

Тогда $\exists C > 0, \delta > 0 \quad \forall h \in B(0, \delta) \dbl |F(x_0 + h) - F(x_0)| \ge C|h|$

\textit{Доказательство:}

\begin{enumerate}
    \item Если $F$ --- линейное отображение, то рассмотрим: $|h| = |F^{-1}Fh| \le ||F^{-1}|| \cdot |Fh|$. По линейности: 
    \[|F(x_0 + h) - F(x_0)| = |Fh| \ge \underbrace{\frac{1}{||F^{-1}||}}_{C}|h| \quad \forall \delta\]
    \item В противном случае, запишем определение дифферецируемости: $|F(x_0 + h) - F(x_0)| = |\underbrace{F'(x_0)h}_{> 0} + |h| \cdot \underbrace{\alpha(h)}_{\text{б. м.}}| \underset{\text{нер-во треугольника}}{\ge} \underbrace{C}_{\text{из пункта 1}}|h| - \alpha(h) \cdot |h|$. Давайте выберем $\delta$ так, чтобы $\alpha(h) < \frac{C}{2}$
    
    \[\ldots\ge \frac{C}{2}|h|\]
\end{enumerate}

ч.т.д.

\textit{Замечание}

При $\forall x \dbl \det F'(x) \neq 0$ не следует инъективность!

\subsubsection{Теорема о сохранении области}

\textit{Формулировка:}
\begin{itemize}
    \item $F: O \subset \mathbb{R}^m \rightarrow \mathbb{R}^m, O$ --- открытое
    \item $F$ --- дифференцируемо
    \item $\forall x \in O: \dbl \det F'(x) \neq 0$
\end{itemize}

Тогда $F(O)$ --- открытое множество.

\textit{Замечания}
\begin{enumerate}
    \item Если $O$ --- связное и $F$ --- непрерывное, то $F(O)$ --- связное\\\\
    \textit{Доказательство:}

    Ну, типа очев. Если у нас есть $W_1, W_2 \subset F(O)$, причём они не связны, то логично что получиться они могли только вследствие $F^{-1}(W_1) \cap F^{-1}(W_2) = \emptyset$
    \item $F$ --- непрерывное $\Leftrightarrow \forall W \subset F(O)$ --- открытого, $F^{-1}(W)$ --- открыто
    \textit{Доказательство:}

    По топологическому определению непрерывности (привет, 1 сем!).\\\\
    \image{sohr_obl_1.png}
\end{enumerate}


\textit{Доказательство:}

В общем, основная идея доказательства состоит в том, чтобы доказать, что любая точка из образа является внутренней, тогда по определению открытого множества мы докажем и вывод.
$\forall x_0 \in O: \dbl y_0 = F(x_0)$. 

По лемме выше, $\exists C > 0, \exists \delta > 0: \forall h \in \overline{B(0, \delta)}: |F(x_0 + h) - F(x_0)| \ge C|h|$. Не стоит смущаться при виде замкнутого шара, это мы просто провели двойную бухгалтерию. Причём, как видно на картинке, граница нашей области отображается куда-то далеко (аж на константу) больше, чем просто на $\delta$.

\image{sohr_obl_2.png}

Заведём расстояние $dist(x, A) = \inf_{y \in A} \rho(x, y)$ между точкой и множеством. Пусть $r = \frac{1}{2} \cdot dist(y_0, \underbrace{F(\underbrace{S(x_0, \delta)}_{\text{компакт}})}_{\text{непр. } \Rightarrow \text{ компакт}})$. Так как у нас там всё компакты то минимум достигается, и, что важнее всего, всё это больше нуля.

Теперь самое интересное: докажем, что $B(y_0, r) \subset F(O): \forall y \in B(y_0, r) \dbl \exists x \in B(x_0, \delta): F(x) = y$. Это докажет нам всё остальное.

$\forall y \in B(y_0, r): \rho(y, F(S(x, \delta))) > r$. Это очевидно, на рисунке всё видно. Рассмотрим $g(x) := |F(x) - y|^2, x \in B(x_0, \delta)$. Как было сказано выше, мы доказываем, что у нас $\exists x \Leftrightarrow g(x) = 0$ возможно. Ну, очевидно, что, видимо, в если там и есть ноль, то это экстремум функции (модуль же, лол).

\[g(x_0) = |F(x_0) - y|^2 = |y_0 - y|^2 \underset{\text{очевидно по рисунку}}{\le} r^2\]

Также, по рисунку очевидно, что для всех $x$ с границы, наша функция отправляет их сильно дальше.
\[\forall x \in S(x, \delta) \quad g(x) \ge r^2\]

Получается, наш минимум лежит где-то внутри сферы. Поищем его. По определению евклидовой нормы:

\[g(x) = (F_1(x) - y_1)^2 + (F_2(x) - y_2)^2 + \ldots + (F_m(x) - y_m)^2\]

По необходимому условию экстремума, $\nabla F(x) = 0 \Rightarrow \forall i \in [1, m]: \frac{\partial f}{\partial x_i} = 0$

\[g'(x) = 2(F_1(x) - y_1)\frac{\partial f}{\partial x_1} + 2(F_2(x) - y_2)\frac{\partial f}{\partial x_2} + \ldots + 2(F_m(x) - y_m)\frac{\partial f}{\partial x_m} = 0\]

Или в векторной форме:
\[2 \cdot (F(x) - y) \cdot F'(x) = 0\]

Однако, по условию у нас производный оператор невырожденный! Следовательно, остаётся только $F(x) = y$. А это то, что мы и искали!!!!

ч. т. д.


\subsubsection{Следствие о сохранении области для отображений в пространство меньшей размерности}
\textit{Формулировка:}

\begin{itemize}
    \item $f: O \subset \mathbb{R}^m \rightarrow \mathbb{R}^l$
    \item $O$ --- открыто
    \item $l < m$    
    \item $F \in C^1(O)$
    \item $\forall x \in O: \rank(F') = l$
\end{itemize}

Тогда $F(O)$ --- открыто

\textit{Доказательство:}

Зафиксируем $x_0 \in O$. Так как у нас матрица производного оператора теперь имеет вид не квадратный, а прямоугольный ($l \times m$), просто так применить предыдущую теорему не получится. Поэтому, не умаляя общности, давайте считать, что вот этот ЛНЗ набор векторов в матрице реализуется на позициях $1 \ldots l$.
Тогда мы можем посчитать определитель этой матрицы: $$\det \left(\frac{\partial F_i}{\partial x_j}\right)_{1 \le i, j \le l}(x_0) \neq 0$$
При этом, так как мы потребовали непрерывность, немножко пошевелив $x_0$ всё также будет работать:

\[\exists U(x_0): \forall x \in U(x_0) \quad \det \left(\frac{\partial F_i}{\partial x_j}\right)_{1 \le i, j \le l}(x) \neq 0\]

Мы уже доказали, что $F(x_0)$ --- внутренняя в $F(U(x_0))$ (по предыдущей теореме). Осталось немного пошаманить, чтобы доказать, что действительно из пространства большей в меньшую всё корректно отобразится.

\image{sohr_obl_men_raz.png}

Давайте заведём такую окрестность $U_l = {(t_1, t_2, \ldots, t_l): (t_1, t_2, \ldots, t_l, (x_0)_{l + 1}, \ldots, (x_0)_m)}$. Как видно на рисунке, это такая проекция в пространстве большей размерности на пространство меньшей. Теперь заведём $\widetilde{F}: U_l \rightarrow \mathbb{R}^l$ и посмотрим на её матрицу производных: $$\frac{\partial \widetilde{F}_i}{\partial t_j} = \left(\frac{\partial F_i}{\partial x_j}(t_1, t_2, \ldots, t_l, (x_0)_{l + 1}, \ldots, (x_0)_m)\right)$$

И вот теперь, по непрерывности $\widetilde{F}$ и прошлой теореме, всё по идее работает.

ч. т. д.

\subsubsection{Теорема о гладкости обратного отображения}
\textit{Формулировка:}

\begin{itemize}
    \item $F: O \subset \mathbb{R}^m \rightarrow \mathbb{R}^m$
    \item $F$ --- обратимо
    \item $F \in C^r(O), r \in 1, 2, \ldots$
    \item $\forall x \in C: \det F'(x) \neq 0$
\end{itemize}

Тогда $F^{-1} \in C^r, \quad ((F^{-1}(y))' = (F'(x))^{-1})$ при $F(x) = y$

\textit{Доказательство:}

Докажем по индукции по $r$. Замое запарное --- база.

\textbf{База:}

Пусть $x_0 \in O, \quad F(x_0) = y_0$. $S := F^{-1}$. Заметим, что $S$ --- непрерывно по теореме о сохранении области и теореме о топологическом определении непрерывности (типа для любого открытого из прообраза образ тоже открыт)

\image{glad_obr.png}

По лемме о ``почти'' локальной инъективности:
\[\exists C, \delta > 0: \forall x \in B(x_0, \delta) \quad |F(x) - F(x_0)| \ge C|x - x_0| \Rightarrow |x - x_0| \le \frac{1}{C}|F(x) - F(x_0)|\]

Запишем определение дифференцируемости для $F$ и сразу распишем всё в терминах $y$:

\[A = F'(x_0), \quad \underbrace{F(x) - F(x_0)}_{y - y_0} = A(\underbrace{x - x_0}_{S(y) - S(y_0)}) + \alpha(\underbrace{x}_{S(y)})|x - x_0|\]

Выражаем $(S(y) - S(y_0)))$:

\[S(y) - S(y_0) = A^{-1}(y - y_0) - \underbrace{A^{-1}\alpha(S(y))|S(y) - S(y_0)|}_{\beta(y) \underset{???}{=} o(|y - y_0|)}\]

Получилось вполне себе нормальное определение для дифференцируемости $S$. Надо лишь доказать ``о''-шку при $y \rightarrow y_0$. Оценим её с помощью вывода из леммы выше и стандартной оценки операторной нормы (не забываем, что мы как-бы управляем $y$ ???):

\begin{align*}
    |x - x_0| = |S(y) - S(y_0)| < \delta \underset{\text{при }y\text{ близких к }y_0}{\Rightarrow} |\beta(y)| &= |A^{-1}\alpha(S(y))|\cdot|S(y) - S(y_0)| \\
    &\le \underbrace{\frac{||A^{-1}||}{C}}_{\const}\cdot\underbrace{|y - y_0|}_{|F(x) - F(x_0)|}\cdot|\alpha(S(y))| \\
    &= o(|y - y_0|)
\end{align*}

Фактически ``о''-шка доказана по определению. Тем самым доказана дифференцируемость. А что с непрерывностью производной то? Этого мы ещё не доказывали. Построим цепочку непрерывных отображений:

\[y \mapsto S(y) = x \mapsto A(x) \mapsto A^{-1}(x) = S'(y)\]

Непрерывность дифференцирования обратного производного оператора доказывается маханием руками на тему отдельнызх производных в матрице. Тем самым база доказана.

\textbf{Переход}

Достаточно тривиальный. Посмотрим при $m = 1: (f^{-1}(y))' = \frac{1}{f(x(y))}$. То есть, пусть $f \in C^{r + 1}$, тогда надо доказать, что $f' \in C^r$. Ну там вот это и написано, обратная функция вообще $C^\infty, f'(x) \in C^r$ --- очев. Для многомерного случая всё тоже самое, только формула выглядит пафоснее $\ldots = (F'(x(y)))^{-1}$

ч. т. д.

\subsubsection{Теорема о задании гладкого многообразия системой уравнений}
\textit{Формулировка:}

$M \subset \mathbb{R}^m, 1 \le k \le m, 1 \le r \le \infty$

Тогда следующие утверждения эквивалентны:

\begin{enumerate}
    \item $\exists U(p) \in \mathbb{R}^m: M \cap U(p)$ --- гладкое $k$-мерное $C^r$-гладкое многообразие
    \item $\exists \widetilde{U}(p) \in \mathbb{R}^m: \exists (F_1, F_2, \ldots, F_{m - k}): \widetilde{U} \rightarrow \mathbb{R}, F_i \in C^r$ \begin{enumerate}
        \item $\forall x \in \widetilde{U} \cap M \Leftrightarrow F_1(x) = F_2(x) = \ldots = F_{m - k} = 0$
        \item $\grad F_1, \grad F_2, \ldots, \grad F_{m - k}$ --- ЛНЗ
    \end{enumerate}
\end{enumerate}
\textit{Доказательство (оставь надежду всяк сюда идущий):}

\textbf{(1) $\Rightarrow$ (2)}

\image{glad_mnogoobr.png}

Нам дано многобразие. А что это значит? $\Phi: O \subset \mathbb{R}^k \rightarrow \mathbb{R}^m \in C^r$ --- гомеоморфизм. Давайте посмотрим на неё в смысле координатных функций: $\exists \Phi = (\varphi_1, \varphi_2, \ldots, \varphi_l), p = \Phi(t^0), \rank \Phi'(t^0) = k$. Всё по определнию. 

У нас тут ЛНЗ набор (ранг $k$), поэтому давайте опять считать, что он реализуется на первых $k$ векторов, поэтому: 
\[\left(\det \frac{\partial \Phi_i}{\partial t_j}\right)_{i = 1 \ldots k} = 0\]

Теперь давайте, во-первых, примем за $\mathbb{R}^m = \mathbb{R}^{m - k} \times \mathbb{R}^k$ (на рисунке справа, всё логично). И заведём $L: \mathbb{R}^m \rightarrow \mathbb{R}^k: (x_1, x_2, \ldots, x_m) \mapsto (x_1, x_2, \ldots, x_k)$ --- просто проекция первых $k$ координат. Тогда заметим, что $(L \circ \Phi)'(t^0)$ --- невырожденный оператор: всё просто, он мапит первые $k$ координат, а оператор по ним невырожден по определению многообразия, вон, наверху написано. Значит, это локальный диффеоморфизм (по соответствующей теореме). А если $W(t^0)$ --- окрестность, то $L \circ \Phi: W \rightarrow V \subset \mathbb{R}^k \in C^r$ --- диффеорморфизм (класс гладкости сохраняется).

Тогда давайте введём ещё парочку отображений: $\Psi: V \rightarrow W := (L \circ \Phi)^{-1}$ --- обратное отображение, также диффеоморфизм, т. к. оно там всё диффеоморфизм, следовательно биекция сохраняется. Также, получается, раз у нас биекция, над $V$ множество в $R^{m - k}$ это график какого-то отображения. Оно точно существует, ведь $L$ --- биективно. Назовём его $H: V \rightarrow \mathbb{R}^{m - k}$

При $x' \in V: (\underbrace{x'}_{1 \ldots k}, \underbrace{H(x')}_{k + 1 \ldots m - k}) = \Phi\Psi(x')$ --- это правда, просто проехались по путям и вернулись. В $L$ у нас только первые $k$ координат, а $H$ нам дорисовывает остальные $m - k$ штук. Ну и вот, в правой стороне равенства у нас диффеоморфизмы, слева проекция (там вообще всё гуд) и $H \Rightarrow$ это тоже диффеоморфизм класса $C^r$. 

Почти всё. Осталось чётко определить, на какой окрестности будут определены наши функции. Смотрите, вообще наш график $H$ может в принципе быть и шире, чем $W(t^0)$, и тогда $L($\textit{график} $H)$ может быть больше, чем $V$, и мы не хотим со всем этим разбираться --- зачем? Поэтому давайте аккуратненько всё подрежем. $V \times \mathbb{R}^{m - k}$ --- открытое, такой типа цилиндр вверх. $\Phi$ --- гомеоморфизм, поэтому $\Phi(W)$ --- открытое. Но в $M$ --- это важно! Оно может и не быть открыто во всём $\mathbb{R}^m$, а конкретно на $M$ с индуцированной метрикой точно открыто. Тогда вспоминаем теорему из 1-го семестра об открытом множестве в пространстве и подпространстве: $M \subset \mathbb{R}^m, \Phi(W) \subset M$ --- открытое, тогда $\exists G \subset \mathbb{R}^m: G \cap M = \Phi(W), G$ --- открытое. И тогда пусть область определения $\widetilde{U}(p) = G \cap \{V \times \mathbb{R}^{m - k}\}$ --- открытое в $\mathbb{R}^m$, отрезали всё лишнее. 

Ну всё, совсем немного осталось. Надо задать такие функции, что они будут нулевыми при $x \in \widetilde{U} \cap M$. Пусть $F_j(x) = H_j(L(x)) - x_{j + k}$. Что тут написано: мы берём $x$, отпрявляем его в $L$, оставляя только первые $k$ координат. Потом $H$ отправляем его обратно наверх, причём конкретно $H_j$ вернёт нам $x_{k + j}$-ю координату, ведь, как мы писали выше, точки из графика $H$ выглядят как $(\underbrace{x'}_{1 \ldots k}, \underbrace{H(x')}_{k + 1 \ldots m - k})$. Ну всё, \textsc{(a)} выполнено автоматически. А что там с градиентами? Давайте просто их построим и увидим, что в конце будет просто $-E$, что и даст нам $m - k$ независимых векторов (ну, ранг такой).

\textbf{(2) $\Rightarrow$ (1)}

Тут нам сильно помогут наработки предков. Давайте подгоним наше условие под условие теоремы о неявном отображении (в смысле системы уравнений). У нас там была система из уравнений $F(x, y) = 0$, где $x$ --- ``переменные'', а $y$ --- ``функции'' и решение $(x^0, y^0)$, такое что при $\forall x \in P(x^0), y \in Q(y^0): F(x, y) = 0 \Leftrightarrow \exists \varphi: P \rightarrow Q: \phi(x) = y$. Давайте назначим первые $k$ координат переменными, а следующие $m - k$ --- функциями. Опять же, у нас ЛНЗ лабор этих градиентов этих функций, а именно: 

\[\left(\det \frac{\partial F_i}{\partial x_{j + k}}\right)_{1 \le i, k \le m - k}(x^0, y^0) \neq 0\]

Значит, условие теоремы выполнено, и параметризация есть ничто иное, как $\Phi: U(p_1, p_2, \ldots, p_k) \rightarrow \mathbb{R}^m \quad x' \mapsto (x', \varphi(x'))$ на $x \in M \cap \widetilde{U} \cap \{P \times Q\}$ (по сути график $\varphi$). В том числе это и гомеоморфизм, так как в одну сторону всё непрерывно, так как функции непрервыны $(x', \varphi(x'))$, а обратно --- это по сути проекция, так что всё тоже непрерывно. Классы гладкости тоже переезжают из прошлой теоремы.

ч. т. д.

\subsubsection{Следствие о двух параметризациях}
\textit{Формулировка:}

$M \subset \mathbb{R}^m$ --- $k$-мерное $C^r$-гладкое многообразие в $\mathbb{R}^m$

\begin{enumerate}
    \item $\exists \Phi_1: O_1 \subset \mathbb{R}^k \rightarrow \mathbb{R}^m$
    \item $\exists \Phi_2: O_2 \subset \mathbb{R}^k \rightarrow \mathbb{R}^m$
\end{enumerate}

--- гладкие параметризации.

Тогда $\exists \Theta: O_1 \rightarrow O_2: \Phi_1 = \Phi_2 \circ \Theta$ --- диффеоморфизм класса $C^r$

\textit{Доказательство:}

\image{sl_o_2_param.png}

Продолжаем повествование из прошлой теоремы. Гомеоморфизм $O_1 \rightarrow O_2$, вообще говоря, существует тривиально: $\Phi_2^{-1} \circ \Phi_1$. Однако, так говорить не совсем правильно, потому что для корректного взятия обратной функции, необходимо сузить образ $\Phi_2$ на его реальную область значений. Поэтому давайте поступим умнее: нарисуем возможные пути точки (крестика) на рисунке (кстати, важно заметить, что разные параметризации могут отправлять точки в разные пространства $\mathbb{R}^k$, ведь ранг может реализоваываться на произвольных строчках матрицы произвожного опреатора; поэтому у нас народилось 2 пространства и соответствующие отображения между ними (см. картинку)).

\[\Phi_1 = \Phi_2 \circ (\Psi_2 \circ L_2 \circ \Phi_1) = \Psi_2 \circ \Theta\]

Супер, гомеоморфизм есть. А обратим ли он? Да пожалуйста:

\[\Theta^{-1} = \Psi_1 \circ L_1 \circ \Phi_2\]

А всякие гладкости и классы приходят просто из предыдущих отображений, всё там супер.

ч. т. д.

\subsubsection{Лемма о корректности определения касательного пространства}
\textit{Формулировка:}

\begin{itemize}
    \item $M \subset \mathbb{R}^m$ --- простое $k$-мерное $C^r$-гладкое многообразие в $\mathbb{R}^m$
    \item $p \in M$
    \item $\Phi: O \subset \mathbb{R}^k \rightarrow \mathbb{R}^m$ --- параметризация $M \cap U(p)$
    \item $t^0 \in O: \Phi(t^0) = p$
    \item $\Phi'(t^0): \mathbb{R}^k \rightarrow \mathbb{R}^m$ --- линейный оператор
\end{itemize}

Тогда образ $\Phi'(t^0)$ --- линейное подпространство в $\mathbb{R}^m$, не зависящее от $\Phi$.

\textit{Доказательство:}

Так как $\Phi$ --- параметризация, $\rank \Phi = k$. Ну и тогда всё очевидно по знаниям из линейной алгебры, размерность пространства определяется количеством ЛНЗ столбцов.

По поводу независимости, по следствию о двух параметризациях:

\[\Phi_2 = \Phi \circ \Theta \Rightarrow \Phi_2' = \Phi' \Theta'\]

$\Theta$ --- диффеоморфизм, следовательно $\Theta'(t^0)$ --- невырожденный. Поэтому образ $\Phi_2' = \Phi'$ (см. картинку)

\image{lemm_corr_par.png}

ч. т. д.

\subsubsection{Касательное пространство в терминах векторов скорости гладких путей}
\textit{Формулировка (Лемма):}

$v \in T_p M$

Тогда $\exists $ гладкий $ \gamma: [-\varepsilon, \varepsilon] \rightarrow M: \gamma(0) = p, \gamma'(0) = v$

\textit{Доказательство:}

Раз у нас есть $v$ в образе, значит оно откуда-то пришло. Давайте найдём: $u = (\Phi'(t_0))^{-1} v$.

Тогда предъявим путь в $O: \widetilde{\gamma}(s) = t^0 + su, s \in [-\varepsilon, \varepsilon]$. Типа мы выбрали направление, и гоняем по нему в $O$.

А настоящий путь будет таким: $\gamma(s) = \Phi \circ \widetilde{\gamma}(s)$. Тогда $\gamma'(s) = \Phi' \circ \widetilde{\gamma}(s)$.

Проверим: $\gamma(0) = \Phi (t^0 + 0) = p, \quad \gamma'(0) = \Phi' u = v$

\image{lemm_gl_p.png}

ч. т. д.

\textit{Формулировка:}

$\exists $ гладкий путь $ \gamma: [-1, 1] \rightarrow M, \gamma(0) = p$

Тогда $\gamma'(0) \in T_p M$

\text{Доказательство:}

\image{lemm_sk.png}

Давайте опять прогуляемся по картинке из теоремы о задачи параметризации: 

\[\gamma(s) = \Phi \circ \Psi \circ L \circ \gamma(s)\]

Это очевидно, просто прошли по кругу.

\[\gamma'(0) = \Phi'(t^0)\Psi' L' \gamma'\]

Всё лежит в образе $\Phi'(t^0)$, так что по определению касательного пространства всё супер.

ч. т. д.

\subsubsection{Касательное пространство к графику функции и к поверхности уровня}
\textit{Формулировка (к графику функции):}

\begin{itemize}
    \item $f: \mathbb{R}^n \rightarrow \mathbb{R}$
    \item $f \in C^1$
    \item $y = f(x)$ --- задаёт простое гладкое $n$-мерное многообразие в $\mathbb{R}^{n + 1}$ (???)
    \item есть точка $f(x^0) = y^0$
\end{itemize}

Тогда $y - y^0 = \sum_{i = 1}^n{f_{x_i}'(x^0)(x_i - x^0_i)}$ задаёт аффинное касательное пространство

\textit{Доказательство:}

Пусть $\Phi: x \mapsto (x, f(x))$. Посмотрим на производный оператор этого отображения:

\[\Phi' = \begin{pmatrix}
    1 & 0 & 0 & \ldots & 0 \\
    0 & 1 & 0 & \ldots & 0 \\
    \vdots & \vdots & \vdots & \ddots & \vdots \\
    f_{x_1}' & f_{x_2}' & f_{x_3}' & \ldots & f_{x_n}'
\end{pmatrix}\]

Заметим, что в нашей формуле неизвестные --- $y$ и $x_i$. Давайте рассмотрим вектор, образующийся перед $x_i$:

\[\begin{pmatrix}
    f_{x_1}' \\
    f_{x_2}' \\
    f_{x_3}' \\
    \vdots \\
    f_{x_n}' \\
    -1
\end{pmatrix}\]

Заметим, что этот вектор ортогонален матрице производного оператора (при перемножении даёт нуль-вектор, следовательно косинус 0, по скалярному произведению). Ну это то, что нам нужно. Вектор (фактически нормаль) к касательному пространству. Ещё и через точку начальную проходит $(x^0, y^0)$.

ч. т. д.

\textit{Формулировка (к уровню):}

\begin{itemize}
    \item $f: \mathbb{R}^m \rightarrow \mathbb{R}$ --- гладкая
    \item $f(x_1, x_2, \ldots, x_m) = 0$ --- функция
    \item $x^0$ --- точка, в которой ищем касательное пространство
\end{itemize}

Тогда касательное пространство задаётся уравнением $f_{x_1}'(x^0)(x_1 - x^0_1) + \ldots + f_{x_m}'(x_m - x^0_m)) = 0$

\textit{Доказательство:}

Во-первых, давайте опять прогоним трюк с теоремой о неявном отображении: будем считать первые $m - 1$ координату ``неизвестными'', а $x_m$ --- ``функцией''.

Тогда пусть $f_{x_m}'(x^0) \neq 0$. Значит, существует $x_m = \varphi(x_1, x_2, \ldots, x_{m - 1})$. Ещё один трюк: $(x_1, x_2, \ldots, x_{m - 1}) \mapsto (x_1, x_2, \ldots, x_{m - 1}, \varphi(x_1, x_2, \ldots, x_{m - 1}))$ --- параметризация многообразия $f(x) = 0$ в окрестности точки $x^0$.

Тогда по предыдущему, что мы доказали, касательная плоскость задаётся $\sum_{i = 1}^{m - 1} \varphi_i'(x^0)(x_i - x^0_i) = x_m - x^0_m$, или:

\[\sum_{i = 1}^{m - 1} \varphi_{x_i}'(x^0)(x_i - x^0_i) - (x_m - x^0_m) = 0\]

Это всё замечательно, но условие требует вывод в терминах $f$. А как они связаны? По условию:

\[f(x_1, x_2, \ldots, x_{m - 1}, \varphi(x_1, x_2, \ldots, x_{m - 1})) = 0\]

Давайте вычислим рецепт замены $\varphi_{x_i}$:

\[\frac{\partial f}{\partial x_i}: \quad f_{x_i}' + f_{x_m}' \cdot \varphi_{x_i}' = 0 \Rightarrow \varphi_{x_i}' = - \frac{f_{x_i}'}{f_{x_m}'} \qquad \left(f_{x_m}'(x^0) \neq 0 \text{ по усл.}\right)\]

Итого:
\[-\sum_{i = 1}^{m - 1} \frac{f_{x_i}'}{f_{x_m}'}(x^0)(x_i - x^0_i) - (x_m - x^0_m) = 0 \qquad | \cdot -f_{x_m}'\]

ч. т. д.

\subsubsection{Вычисление нормы линейного оператора с помощью собственных чисел}
\textit{Формулировка:}

$A: \mathbb{R}^m \rightarrow \mathbb{R}^n$

Тогда $||A|| = \max {\sqrt{\lambda}}, \lambda \in \sigma(A^TA)$ --- множество собственных чисел. 

\textit{Доказательство:}

Рассмотрим $x \in S^{m - 1}: \{y \in \mathbb{R}^m: |y| = 1\}$.

\[||A||^2 = \sup_{x \in S^{m - 1}}|Ax|^2 = \sup_{x \in S^{m - 1}}\langle Ax, Ax \rangle = \sup_{x \in S^{m - 1}}\langle \underbrace{A^TA}_{\text{симметричная}}x, x \rangle = \sup_{x \in S^{m - 1}, \lambda \in \sigma(A^TA)} \lambda |x|^2 = \max_{\lambda \in \sigma(A^TA)} \lambda\]

Немного контекста: собственное число, это такое число, что $A$ отображает $x$ в $\lambda x$. Матрица $A^TA$ --- симметричная $(m \times n) \times (n \times m) = m \times m$. Так как у нас эта матрица вещественная, то и собственные числа у неё вещественные. Ну и значит, что максимальный вектор, который может получится, это вектор, домноженный на максимальное собственное число.

ч. т. д

\subsubsection{Теорема о функциональной зависимости}
\textit{Формулировка:}

\begin{itemize}
    \item $f_1, f_2, \ldots, f_n: O \mathbb{R}^m \rightarrow \mathbb{R} \in C^1$
    \item $F = (f_1, f_2, \ldots, f_n): O \rightarrow \mathbb{R}^n$
    \item $\forall x \in O: \rank F'(x) \le k$
    \item $x^0 \in O: \rank F'(x^0) = k$
    \item $y^0 = F(x^0)$
\end{itemize}

Тогда: $\exists U(x^0), \exists g_{k + 1}, g_{k + 2}, \ldots, g_n: V(y^0_1, y^0_2, \ldots, y^0_k) \subset \mathbb{R}^l \rightarrow \mathbb{R}$

Что: $f_i = g(f_1(x), f_2(x), \ldots, f_k(x)), i = k + 1 \ldots n, x \in U(x^0)$

\textit{Доказательство:}

Пусть в точке $x^0$ ранг реализуется на первом $k$-миноре (строчки $1 \ldots k$, столбцы $1 \ldots k$).

Введём дополнительную функцию $\Phi: O \subset \mathbb{R}^m \rightarrow \mathbb{R}^m, \Phi(x) = (f_1(x), f_2(x), \ldots, f_k(x), x_{k + 1}, x_{k + 1}, \ldots, x_k)$.

Тогда, если посмотреть на матрицу производного оператора $\Phi'(x^0)$ (см. рисунок), то окажется, что она невырождена: $\det \Phi'(x^0) \neq 0$. Поэтому $\Phi$ действует как локальный диффеоморфизм класса $C^1$ (по определению там внутри функции минимум $C^1$). Опять начинаем рисовать:

\image{func_zav.png}

$\Phi(U(x^0)) = W, \Phi(x^0) = w_0$. Рассмотрим функцию $\widetilde{F}: W \rightarrow \mathbb{R}^n: F \circ \Phi^{-1}$. Посмотрим поподробнее на точку $w_0 = (u, v)$. Координата $u$ вычислялась как $f_1(x_0), f_2(x_0), \ldots, f_k(x_0)$, теперь мы снова отправляем её обратно, получая $x^0_1, x^0_2, \ldots, x^0_k$, и шлём в $F$, снова применяя к точке функции $f_i$. Получается, что координата $u$ отображается в саму себя. $v$ же под действием какого-то отображения отображается во что-то другое: $\widetilde{F}(u, v) = (u, \Theta(u, v))$.

Рассмотрим производный оператор $\widetilde{F}' = F' \underbrace{\left(\Phi^{-1}\right)'}_{\text{невырожден}}$. Невырожденный оператор (кстати, не только в точке $w$, но и во всей окрестности, на которой работает локальный диффеоморфизм ($W$)) не меняет ранг матрицы, поэтому $\rank \widetilde{F} = k$. С другой стороны, если посмотреть на матрицу производного опреатора (см. рисунок), $\Theta_v'$ обязана быть тождественно равна 0, в противном случае мы могли бы сочинить минор большего ранга, чем $k$. Таким образом, $\Theta_v' = 0 \Rightarrow \Theta = \Theta(u)$ (зависит только от $u$).

Тогда давайте перенесём (домножим на обратную), выразим $F$ и аккуратно распишем:

\begin{align*}
        F(x) &= \widetilde{F} \circ \Phi (x) \\
        &= \widetilde{F}(f_1(x), f_2(x), \ldots, f_k(x), x_{k + 1}, x_{k + 2}, \ldots, x_m) \\
        &= (f_1(x), \ldots, f_k(x), \Theta\left(f_1(x), f_2(x), \ldots, f_k(x)\right)_{k + 1}, \ldots, \Theta\left(f_1(x), f_2(x), \ldots, f_k(x)\right)_n)
\end{align*}

ч. т. д.

\newpage

\section{Период Мезозойский}
\subsection{Важные определения}

\subsubsection{Равномерная сходимость последовательности функций на множестве}

\begin{itemize}
    \item $(f_n): \mathbb{N} \rightarrow \mathbb{F}$
    \item $f_n: E \subset \underbrace{X}_{\text{мн-во}} \rightarrow \mathbb{R}$
\end{itemize}

Если $\exists f(x)$:

\[\forall \varepsilon > 0 \dbl \exists N: \forall \dbl n > N \dbl \forall x \in E \quad |f(x) - f_n(x)| < \varepsilon\]

--- тогда $f_n$ равномерно сходится к $f$ на $E$ ($f_n \rshe f$).

Это же условие равносильно $M_n := \sup_{x \in E} |f_n(x) - f(x)| \underset{n \rightarrow \infty}{\rightarrow} 0 $

\subsubsection{Степенной ряд, радиус сходимости степенного ряда, формула Адамара}

\[a, z, z_0 \in \mathbb{C} \qquad \sum_{k = 0}^{\infty} a_k (z - z_0)^k\]

Вот такой ряд в шаре $B(z_0, R)$ называется \textit{степенным рядом}. 

\textit{Примечания: }

Мы не особо раньше дружили с комплексными числами, но бояться их не нужно, потому что это числа вида $z = a + ib, a, b \in \mathbb{R}, i^2 = -1$. А $|z|$ --- и вовсе вещественное число, поэтому можно оценивать всё по модулю и работать как бы с вещественными числами.

Давайте попробуем абсолютно оценить сходимость такого ряда (по признаку Коши): 

\[\lim_{n \rightarrow \infty}{\sqrt[n]{|a_n| \cdot |z - z_0|^n}} = |z - z_0| \cdot \lim_{n \rightarrow \infty}\sqrt[n]{|a_n|} \Rightarrow \begin{cases}
    |z - z_0| < \frac{1}{\lim_{n \rightarrow \infty}\sqrt[n]{|a_n|}}, \text{ --- абсолютно сходится} \\
    |z - z_0| > \frac{1}{\lim_{n \rightarrow \infty}\sqrt[n]{|a_n|}}, \text{ --- расходится}
\end{cases}\]

Проблема лишь в том, что предел иногда может и не существовать, поэтому возьмём верхний предел, и всё получится. Эта формула (радиуса сходимости) называется \textit{формулой Коши-Адамара}:

\[R = \frac{1}{ \overline{\lim}_{n \rightarrow \infty}\sqrt[n]{|a_n|}}\]

\newpage

\subsection{Определения}

\subsubsection{Поточечная сходимость последовательности функций на множестве}

\begin{itemize}
    \item $(f_n): \mathbb{N} \rightarrow \mathbb{F}$
    \item $f_n: E \subset \underbrace{X}_{\text{мн-во}} \rightarrow \mathbb{R}$
\end{itemize}

Если $\exists f(x): \forall x \in E$:

\[\forall \varepsilon > 0 \dbl \exists N: \forall \dbl n > N \quad |f(x) - f_n(x)| < \varepsilon\]

--- тогда $f_n$ поточечно сходится к $f$ на $E$.

\subsubsection{Формулировка критерия Больцано--Коши для равномерной сходимости}

\[f_n \rsh{X} f \Leftrightarrow \forall \varepsilon > 0 \dbl \exists n \dbl \forall p \dbl \forall x \in X \dbl \left|f_{n + p}(x) - f_n(x)\right| < \varepsilon \]

\subsubsection{Равномерная сходимость функционального ряда}

$f_n: E \subset X \rightarrow \mathbb{R}, \sum_{k = 1}^{\infty} f_k(x)$ --- функциональный ряд

$S_N(x) = \sum_{k = 1}^N f_k(x)$ --- частичная сумма

\[\sum_{k = 1}^{\infty} f_k(x) \rsh{E} S(x) \Leftrightarrow S_N(x) \rshe S(x)\]

\[\forall \varepsilon > 0 \dbl \exists N \dbl \forall n > N \dbl \forall x \in X \dbl \left|S_N(x) - S(x)\right| < \varepsilon\]

\textit{Замечания:}

\begin{enumerate}
    \item равномерная сходимость $\Rightarrow$ поточечная
    \item $R_N(x) = \sum_{k = N + 1}^{\infty} f_k(x)$ --- остаток ряда. $S_N(x) + R_N(x) = S(x)$. $\Rightarrow \sum f_n \rshe \Leftrightarrow R_N(x) \rshe 0$
    \item $\sum f_n \rshe \Rightarrow f_n \rshe, f_n = R_{N} - R_{N - 1}$
\end{enumerate}

\subsubsection{Формулировка критерия Больцано--Коши для равномерной сходимости рядов}

\[\sum_{k = 1}^{\infty} f_k \rshe \Leftrightarrow \forall \varepsilon > 0 \dbl \exists n \dbl \forall p \dbl \forall x \in E \dbl \left|\sum_{k = n}^{n + p} f_k(x)\right| < \varepsilon \]

\subsubsection{Признак Абеля равномерной сходимости функционального ряда}

\begin{itemize}
    \item $a_n, b_n: X \rightarrow \mathbb{R}$
    \item $\sum a_n(x)$ --- равномерно ограничена ($\exists C_a > 0 \dbl \forall x \in X \dbl \forall n \in \mathbb{N} \dbl |a_n(x)| \le C_a$) и монотонна по $n$ при любом $x$
    \item $\sum b_n \rsh{X}$
\end{itemize}

Тогда $\sum a_n(x) b_n(x) \rsh{X}$

\subsubsection{Преамбула к суммам расходящися рядов}

В списке определений как-то подозрительно мало определений, и некоторые темы вообще непокрыты, хотя теоремы на них есть. Поэтому я решил расписать самую базу в формате ``преамбулы'', чтобы иметь общее представление о происходящем.

Смотрите, введём иерархию множеств всевозможных рядов: $\textfrak{R}$ --- вообще все ряды, $\textfrak{Q} \subset \textfrak{R}$ --- множество сходящихся рядов.

Теперь зададим сумму ряда как $F: \textfrak{R} \rightarrow \mathbb{R}$ --- и будем называть это суммой в смысле $F: \dbl \sum a_n \eqby{F} S \in \mathbb{R}$

Посмотрим на примеры: 

\begin{enumerate}
    \item $S: \textfrak{Q} \rightarrow \mathbb{R} = \lim_{n \rightarrow \infty} S_N$ --- предел частичных сумм, наша самая простая и приятная версия, но она задана только на множестве сходящихся рядов. Сейчас хотелось бы создать такие суммы, чтобы они считались на большем множестве рядов, но при этом нормально считали бы и обычные ряды. Поэтому давайте дальше считать множество $\textfrak{J}$ таким: $\textfrak{Q} \subset \textfrak{J} \subset \textfrak{R}$ (это моя ремарка, но, кмк, так гораздо понятнее)
    \item $\sum a_n x^n \eqby{AP} S_{AP}$ --- сумма в смысле Абеля--Пуассона (раньше это просто называлось методом Абеля суммирования рядов, но тут КПК стал называть это методом Абеля--Пуассона)
    \item $\sigma_n = \frac{S_1 + S_2 + \ldots + S_n}{n + 1}, \lim_{n \rightarrow \infty} \sigma_n \eqby{\text{с. а.}} S_{\text{с. а.}}$ --- сумма в смысле среднего арифметического (сумма по Чезаро). Определяется через предел частичных срадних арифметических частичных сумм.
\end{enumerate}

Вот все эти новые суммы нужны, чтобы получить возможность считать суммы большего количества рядов. Отсюда же берутся фокусы в стиле ``сумма натуральных чисел равна $-\frac{1}{12}$'' (через дзета-функцию Римана $\zeta(s) = \sum \frac{1}{n^s}$)

\subsubsection{Преамбула к асимптотическим рядам}

Пусть $\varphi_i: \langle a, b \rangle \rightarrow \mathbb{R}$, или $: \langle b, a \rangle \rightarrow \mathbb{R} (a \in \overline{\mathbb{R}})$, или $: \langle c, d \rangle \rightarrow \mathbb{R}$ при $a \in (c, d)$ --- шкала арифметического ряда. Причём: 

\[\exists \dot{U}(a) \forall k \forall x \in \dot{U}(a) \cap \langle a, b \rangle \quad \varphi_{k}(x) \neq 0, \quad \varphi_{k + 1}(x) \underset{x \rightarrow a}{=} o(\varphi_{k}(x))\]

Тогда, если существует последовательность равенств $\forall n (f: \langle a, b \rangle \rightarrow \mathbb{R})$:

\[f(x) = c_{0}\varphi_{0}(x) + o(\varphi_{0}(x))\]
\[f(x) = c_{0}\varphi_{0}(x) + c_{1}\varphi_{1}(x) + o(\varphi_{1}(x))\]
\[\vdots\]
\[f(x) = c_{0}\varphi_{0}(x) + c_{1}\varphi_{1}(x) + \ldots + c_{n}\varphi_n(x) + o(\varphi_{n}(x))\]
\[\vdots\]

То можно сказать, что 

\[f(x) \sim \sum_{k = 0}^{\infty} c_{k}\varphi_{k}(x)\]


При этом, если оказалось, что у $f(x)$ и $g(x)$ оказались одинаковые асимптотические разложения, то: 

\[\forall k \quad f(x) - g(x) \sim o(\varphi_k(x))\]

Пример (ряд Тейлора): 

\[f \in C^{\infty}(x_0 - \varepsilon, x_0 + \varepsilon), \quad f(x) = \sum_{k = 0}^{\infty} \frac{f^{(k)}(x_0)}{k!}(x - x_0)^{n}\]


Ещё нужно отметить, что асимптотические ряды выдерживают конечные линейные преобразования.

Интеграл Лапласа: 

\[\int_a^b f(x)e^{A\varphi(x)}dx\]

\newpage

\subsection{Важные теоремы}

\subsubsection{Теорема Стокса--Зайдля о непрерывности предельной функций. Следствие для рядов}

\textit{Формулировка (последовательности):}

\begin{itemize}
    \item $f_n, f: \underbrace{X}_{\text{метрическое пространство}} \rightarrow \mathbb{R}$
    \item $c \in X: f_n$ --- непрерывно в $c$  
    \item $f_n \rsh{X} f$
\end{itemize}

Тогда: $f$ --- непрерывно в $c$

\textit{Следствие:}

$f_n \in C(X), f_n \rsh{X} f \Rightarrow f \in C(X)$ --- доказательства не требует, просто по всем точкам пробегаемся

\textit{Доказательство:}

Зафиксируем $\varepsilon$ из определения равномерной сходимости и распишем гига-неравенство треугольника:

\[|f(x) - f(c)| \le \underbrace{|f(x) - f_n(x)|}_{(1)} + \underbrace{|f_n(x) - f_n(c)|}_{(2)} + \underbrace{|f_n(c) - f(c)|}_{(3)}\]

Оно верно при всех $n$. Но нам дали равномерную сходимость, из чего мы достаём $\sup_{x \in X} |f_n(x) - f(x)| < \varepsilon$. Это обстоятельство с ходу говорит нам, что существует большое $n$, при котором $(1)$ и $(3)$ (то, что они $< \varepsilon$). С другой стороны, раз так, мы можем считать, что в $(2)$ стоит вполне конкретная функция, непрерывная в $c \Leftrightarrow \forall x \in U(c): (2) < \varepsilon$. Ну и всё, получается, что наше неравенство целиком меньше $3 \varepsilon$:

\[|f(x) - f(c)| \le |f(x) - f_n(x)| + |f_n(x) - f_n(c)| + |f_n(c) - f(c)| < 3\varepsilon\]

Ну и вот, по китайской методике в определении непрерывности всё работает.

ч. т. д.

\textit{Бонус:}

\Smiley 

Доказательство работает и в топологических пространствах без единой правки, потому что мы разговариваем на языке окрестностей и метрику не трогаем!

\textit{Формулировка (ряды):}

\begin{itemize}
    \item $u_n: \underbrace{X}_{\text{метрическое пространство}} \rightarrow \underbrace{Y}_{\text{нормированное пространство}}$
    \item $c \in X: u_n$ --- непрерывно в $c$ 
    \item $S_n(x) := \sum^n u_n(x)$
    \item $S(x) := \sum u_n(x) \rsh{X}$
\end{itemize}

Тогда $S(x)$ --- непрерывно в $c$

\textit{Доказательство:}

По предыдущей теореме $S_n(x) \rsh{X} S(x), S_n(c)$ --- непрерывно в $c \Rightarrow S(x)$ --- непрерывно в $c$

ч. т. д.

\subsubsection{Признак Вейерштрасса равномерной сходимости функционального ряда}
\textit{Формулировка:}

\begin{itemize}
    \item $\sum u_n(x)$ --- функциональный ряд
    \item $u_n: \underbrace{X}_{\text{мн-во}} \rightarrow \mathbb{R}$
    \item $\exists (c_n)$ --- вещественная последовательность, причём $\sum c_n$ --- сходится
    \item $\forall n \in \mathbb{N} \dbl x \in X: |u_n(x)| \le c_n$
\end{itemize}

Тогда $\sum u_n \rsh{X}$

\textit{Доказательство:}

Доказательство более-менее тривиально. Распишем определение равномерной сходимости:

\[n \rightarrow \infty: \quad \sup_{x \in X} \left| \sum_{k = n + 1}^{\infty} u_n(x) \right| \le \sup_{x \in X} \sum_{k = n + 1}^{\infty} \left| u_n(x) \right| \le \sum_{k = n + 1}^{\infty} c_n \longrightarrow 0\]

ч. т. д.

\subsubsection{Признак Дирихле равномерной сходимости функционального ряда}
\textit{Формулировка:}

\begin{itemize}
    \item $\sum a_n(x) b_n(x), \quad a_n, b_n: X \rightarrow \mathbb{R}$
    \item $\sum a_n$ --- равномерно ограничена ($\exists C_a \dbl \forall n \in \mathbb{N} \dbl \forall x \in X \dbl \left|\sum_{k = 1}^{n}(x)\right| \le C_a$)
    \item $\sum b_n(x) \rsh{X}$
    \item $b_n$ --- монотонны по $n \dbl \forall x \in X$
\end{itemize}

Тогда $\sum a_n(x) b_n(x) \rsh{X}$

\textit{Доказательство:}

Пусть $A_n = \sum_{k = 1}^n a_k$. Рассмотрим такую сумму (опустим $(x)$, но они там есть): 

\[
    \sum_{N \le K \le M} {a_K b_K} = A_Mb_M - A_{N - 1}b_{N - 1} + \sum_{N \le K \le M - 1} {(b_K - b_{K - 1})A_K}
\]

Если взять всё под модуль и применить неравенство треугольника, то получится выдержка из критерия Коши: 

\[
    \left|\sum_{N \le K \le M} {a_K b_K}\right| \le |A_Mb_M| - |A_{N - 1}b_{N - 1}| + \sum_{N \le K \le M - 1} {\left|b_K - b_{K - 1}\right| \cdot \left|A_K\right|} \qquad (*)
\]

Вспоминаем, что $b_n$ монотонна, поэтому можно раскрыть модуль внутри суммы и домножить всю сумму на ``знак монотонности'' (1, если возрастающая и -1, если убывающая). Тогда у нас раскрывается телескопическая сумма, и остаётся 2 члена из правой суммы. Ну и ещё оценим все $A_i$-шки константой из условия (у нас есть равномерная ограниченность):

\[
    (*) \le C_a \left( |b_M| + |b_{N - 1}| + |b_{M - 1}| + |b_{N
    }|\right) \qquad (**)    
\]

И ещё вспоминаем, что у нас ряд из $b_n$ равномерно сходится, что значит (с небольшой китайской бухгалтерией): 

\[ \forall \varepsilon > 0 \dbl \exists k \dbl \forall n > k \dbl \sup_{x \in X} |b_n(x)| < \frac{\varepsilon}{4 \cdot C_a} \]

Тогда при достаточно больших $N, M$:

\[(**) \dbl C_a \cdot \left(\frac{\varepsilon}{4 \cdot C_a} + \frac{\varepsilon}{4 \cdot C_a} + \frac{\varepsilon}{4 \cdot C_a} + \frac{\varepsilon}{4 \cdot C_a} \right) < \varepsilon\]

Критерий выполнен, всё хорошо.

ч. т. д.

\newpage

\subsection{Теоремы}

\subsubsection{Метрика в пространстве непрерывных функций на компакте, его полнота}
\textit{Формулировка:}

\begin{itemize}
    \item $\rho(f, g) = \sup_{x \in X} |f(x) - g(x)|$ --- метрика (это доказывалось на лекции, хз, надо ли тут, но там вроде всё тривиально: аксиомы тождества, симметрии и правило треугольника) 
    \item $X$ --- компактное метрическое пространство
\end{itemize}

Тогда $\left(C(X), \rho \right)$ --- полное метрическое пространство

\textit{Доказательство:}

Полное метрическое пространство --- это такое, в котором у любой фундаментальной последовательности есть предел:

\[\forall \varepsilon > 0 \dbl \exists N \dbl \forall n, m > N \dbl \rho(f_n, f_m) = \sup_{x \in X}|f_n(x) - f_m(x)| < \varepsilon\]

Давайте возьмём какой-нибудь $x_0 \in X$ и заметим, что $|f_n(x_0) - f_m(x_0)| < \varepsilon$ (очев, раз супремум меньше, то и отдельный $x_0$ меньше). Значит $n \mapsto f_n(x_0)$ --- фундаментальная \textbf{вещественная} последовательность (просто подставить в определение выше)! Ну а $\mathbb{R}$ --- полное, поэтому у такой последовательности сущесвтует предел: $\lim_{n \rightarrow \infty} f_n(x_0) = f(x_0)$ (к какой-то $f$). Получается, что пототочечно всё норм сходится. Но нам то надо равномерную (в силу того, какую метрику мы выбрали). Давайте немного перепишем определение фундаментальной последовательности:

\[\forall \varepsilon > 0 \dbl \exists N \dbl \forall n, m > N \dbl \forall x \in X |f_n(x) - f_m(x)| < \varepsilon\]

Сделаем предельный переход по $m \rightarrow \infty$ и подставим найдённую предельную функцию: 

\[\forall \varepsilon > 0 \dbl \exists N \dbl \forall n > N \dbl \forall x \in X |f_n(x) - f(x)| \le \varepsilon\]

А это --- определение $f_n \rsh{X} f$. Ну всё, супер, значит фундаментальная последовательность сходится.

А непрерывность приходит из теоремы Стокса-Зайдля. Значит наша фундаментальная последовательность имеет предел, и этот предел лежит в $C(X)$.

ч. т. д. 

\subsubsection{Теорема о предельном переходе под знаком интеграла. Следствие для рядов}
\textit{Формулировка (последовательности):}

\begin{itemize}
    \item $f_n \in C[a, b]$
    \item $f_n \rsh{[a, b]} f$
\end{itemize}

Тогда:

\[\int_a^b{f_n(x)dx} \underset{n \rightarrow \infty}{\longrightarrow} \int_a^b{f(x)dx}\]

\textit{Доказательство:}

Тривиалити (скажем, что их разность стремится к 0, т. к. есть равномерная сходимость): 

\begin{align*}
    \left|\int_a^b{f_n(x)dx} - \int_a^b{f(x)dx}\right|
 &=  \left|\int_a^b{f_n(x) - f(x)dx}\right| \\
  &\le \int_a^b \left|f_n(x) - f(x) dx\right| \\ 
  &\underset{\text{по метрике}}{\le} {\sup_{x \in X} \left|f_n(x) - f(x)\right|}(b - a) \rightarrow 0
\end{align*}

ч. т. д.

\textit{Формулировка (ряды):}

\begin{itemize}
    \item $u_n: C[a, b] \rightarrow \mathbb{R}$
    \item $\sum u_n(x) \rsh{[a, b]} S(x)$
\end{itemize}

Тогда $\int_a^bS(x)dx =\sum \int_a^b u_n(x) dx$, причём интегрировать можно, т. к. $S(x)$ --- непрерывна по Стоксу-Зайдлю

\textit{Доказательство:}

По теореме для последовательностей: 

\[\int_a^b S_n(x) dx \underset{n \rightarrow \infty}{\longrightarrow} \int_a^b S(x) dx\]

С другой стороны:

\[\int_a^b S_n(x) dx = \int_a^b \sum_{i = 1}^n u_n(x) dx \underset{\text{линейность интеграла}}{=} \sum_{i = 1}^n \int_a^b u_n(x) dx \underset{n \rightarrow \infty}{\longrightarrow} = \sum_{i = 1}^{\infty} \int_a^b u_n(x) dx\]

Ну и вот, у нас интеграл частичных сумм в пределе стремится одновременно к интергалу предельной суммы и ряду интегралов. Всё хорошо.

\subsubsection{Правило Лейбница дифференцирования интеграла по параметру}
\textit{Формулировка:}

\begin{itemize}
    \item $f: [a, b] \times [c, d] \rightarrow \mathbb{R}, f(x, y)$
    \item $f, f_y'$ --- непрерывны на $[a, b] \times [c, d]$
    \item $\Phi(y) = \int_a^b f(x, y) dx$
\end{itemize}

Тогда $\Phi(y)$ --- дифференцируема и $\Phi(y) = \int_a^b f'_y(x, y) dx$

\textit{Доказательство:}

Ну, давайте попробуем подифференцировать. Возьмём какую-то $t_n \rightarrow 0$ и напишем а-ля определение дифференцируемости и применим теорему Лагранжа (привет, НТР!):

\[\frac{\Phi(y + t_n) - \Phi(y)}{t_n} = \int_a^b \frac{f(x, y + t_n) - f(x, y)}{t_n} = \int_a^b f_y'(x, y + \Theta_xt_n) = \]

\[=\Phi'(y + \Theta_x t_n) = \int_a^b f_y'(x, y + \Theta_x t_n) dx \underset{?}{\longrightarrow} \int_a^b f_y'(x, y) dx\]

Ну и вот, мы теперь хотим понять, а действительно ли оно стремится? Применим ``тяжёлую артиллерию'': теорема Кантора о равномерной непрерывности:

\[f \text{ --- непр. } \in C(K) \text{ (компакт)} \Rightarrow f\text{ --- равномерно непрерывна}\]

\[\forall \varepsilon > 0 \dbl \exists \delta > 0: \rho(x_1, x_2) < \delta \dbl |f(x_1) - f(x_1)| < \varepsilon\]

У нас непрерывная функция на компакте, поэтому давайте подгоним под Кантора наше условие:

\[\exists N \dbl \forall n > N \dbl |t_n| < \delta, \dbl \rho((x, y + \Theta_x t_n), (x, y)) < \delta, \dbl
\left|f_y'(x, y + \Theta_x t_n) - {f_y'(x, y)}\right| < \varepsilon\]

Тогда: 

\[\left|\int_a^b{f_y'(x, y + \Theta_x t_n) dx} - \int_a^b{f_y'(x, y) dx}\right| \le \varepsilon (b - a)\]

Следовательно, разность между ними меньше $\varepsilon$, тогда всё действительно стремится.

ч. т. д.

\subsubsection{Теорема о предельном переходе под знаком производной. Дифференцирование функционального ряда}
\textit{Формулировка (последовательности):}

\begin{itemize}
    \item $f_n \in C^1\langle a, b \rangle$
    \item $f_n \rightarrow f_0$ --- поточечно
    \item $f_n' \rsh{\langle a, b \rangle} \varphi$
\end{itemize}

Тогда $f_0 \in C^1\langle a, b \rangle$ и $f_0' = \varphi$ на $\langle a, b \rangle$

\textit{Доказательство:}

Давайте (не умаляя общности) возьмём какой-то подотрезок $[x_0, x_1] \subset \langle a, b \rangle$. Тогда по предыдущей теореме (у нас непрерыно равномерно сходится по условию):

\[\int_{x_0}^{x_1} f_n' \longrightarrow \int_{x_0}^{x_1} \varphi \]

Интегрируем:

\[\int_{x_0}^{x_1} f_n' = f_n(x_1) - f_n(x_0) \underset{n \rightarrow \infty}{=} f_0(x_1) - f_0(x_0) \underset{n \rightarrow \infty}{=} \int_{x_0}^{x_1} \varphi \]

Получается, что $f_0$ --- первообразная $\varphi$. Причём, по Стоксу-Зайдлю, $\varphi$ --- непрерывна, значит и её первообразная тоже непрерывна ($f_0' = \varphi$).

ч. т. д. 

\textit{Формулировка (ряды):}

\begin{itemize}
    \item $u_n \in C^1\langle a, b \rangle$
    \item $\sum u_n(x) = S(x)$ --- поточечно
    \item $\sum u_n'(x) \rsh{\langle a, b \rangle} = \varphi(x)$
\end{itemize}

Тогда $S(x) \in C^1\langle a, b \rangle$ и $S'(x) = \varphi (x)$ на $\langle a, b \rangle$. То есть $\left(\sum u_n(x)\right)' = \sum u_n'(x)$

\textit{Доказательство:}

Запускаем теорему для последовательностей с вводными: $f_n = S_n, f_0 = S, f_n' = \sum_{k = 1}^n u_k'$

ч. т. д.

\subsubsection{Теорема о предельном переходе в суммах.}
\textit{Формулировка:}

\begin{itemize}
    \item $u_n: E \subset \underbrace{X}_{\text{метрическое пространство}} \rightarrow \mathbb{R}$
    \item $x_0 \in X$ --- предельная точка $E$
    \item $\forall n \dbl \exists$ конечный $\lim_{x \rightarrow x_0} u_n(x) = a_n$
    \item $\sum u_n(x) \rshe$
\end{itemize}

Тогда:

\begin{enumerate}
    \item $\sum a_n$ --- сходится
    \item $\sum a_n = \lim_{x \rightarrow x_0} \left(\sum_{n = 1}^{\infty} u_n(x) \right)$
\end{enumerate}

\[\lim_{x \rightarrow x_0} \sum_{n = 1}^{\infty} u_n(x) = \sum_{n = 1}^{\infty} \lim_{x \rightarrow x_0} u_n(x)\]

\textit{Доказательство:}

Нам, честно говоря, не за что зацепиться, поэтому давайте попробуем проверить, что $a_n$ --- фундаментальная последовательность, тогда у неё точно будет предел.

Пусть $S_n(x) = \sum_{k = 1}^n u_n(x), S^a_n = \sum_{k = 1}^n a_n$

Опять распишем гига-неравенство трегугольника: 

\[|S^a_{n + p} - S^a_n| \le \underbrace{|S^a_{n + p} - S_{n + p}(x)|}_{(1)} + \underbrace{|S_{n + p}(x) - S_n(x)|}_{(2)} + \underbrace{|S_n(x) - S^a_n|}_{(3)}\]

По критерию Больцано-Коши равномерной сходимости ряда:

\[\forall \varepsilon > 0 \dbl \exists N \dbl \forall n > N \dbl \forall p: \sup_{x \in E} \left|S_{n + p}(x) - S_n(x) \right| < \varepsilon\]

Сейчас мы получили, что при достаточно большом $n \dbl (2) < \frac{\varepsilon}{3}$ при любых $x \in E$. Теперь заметим, что мы доказываем фундаментальность числовой последовательности, следовательно никаких ограничений на $x$ изначально не наложено. (???) Поэтому давайте возьмём такой $x$, близкий к $x_0$, чтобы $(1)$ и $(3)$ были $\frac{\varepsilon}{3}$. Итого:

\[|S^a_{n + p} - S^a_n| < \frac{\varepsilon}{3} + \frac{\varepsilon}{3} + \frac{\varepsilon}{3} = \varepsilon\]

Мы взяли частичные суммы $a_n$, и проверили, что разности рядом лежащих сумм сколь угодно малы.

Ура, у $a_n$ есть предел! А чему же он равен? Давайте заведём дополнительную функцию: 

\[\widetilde{u}_n(x) := \begin{cases}
    u_n(x), x \neq x_0\\
    a_n, x = x_0
\end{cases}\]

Такая сложность необходима для обеспечения непрерывности в $x_0$ (если $x_0$ лежит в $E$, то мы просто подменили значение, а если не лежала, то дополнили). Теперь эта функция непрерывна на $x_0$, ряд $\sum \widetilde{u}_n \rsh{E \cup \{x_0\}}$ (*) $ \Rightarrow$ по Стоксу-Зайдлю $S_{\widetilde{u}_n}(x)$ непрерывна в $x_0$. А поэтому:

\[\lim_{x \rightarrow x_0} \left(\sum_{n = 1}^{\infty} u_n(x) \right) = \lim_{x \rightarrow x_0} \left(\sum_{n = 1}^{\infty} \widetilde{u}_n(x) \right) = \sum_{n = 1}^{\infty} \widetilde{u}_n(x) = \sum a_n\]

А равномерная сходимость ряда $\sum \widetilde{u}_n$ доказывается так:

\[(*): \sup_{x \in E \cup \{x_\}} \left|\sum_{k = n + 1}^\infty \widetilde{u}_k \right| = \max\{\sup_{x \in E} \left|\sum_{k = n + 1}^\infty \widetilde{u}_k \right|, \sum_{k = n + 1}^\infty a_n\} \le \underbrace{\sup_{x \in E} \left|\sum_{k = n + 1}^\infty \widetilde{u}_k \right|}_{\rightarrow 0} + \underbrace{\sum_{k = n + 1}^\infty a_n}_{\rightarrow 0} \longrightarrow 0\]

ч. т. д.

\subsubsection{Теорема о перестановке двух предельных переходов}
\textit{Формулировка:}

\begin{itemize}
    \item $f_n: E \subset X \rightarrow \mathbb{R}$
    \item $x_0 \in X$ --- предельная точка $E$
    \item $\forall n \in \mathbb{N}: \exists$ конечный $\lim_{x \rightarrow x_0} f_n(x) = A_n$
    \item $f_n(x) \rshe S(x)$ при $n \rightarrow \infty$
\end{itemize}

Тогда: 

\begin{enumerate}
    \item $\exists \lim_{n \rightarrow \infty} A_n \in \mathbb{R}$
    \item $S(x) \underset{x \rightarrow x_0}{\longrightarrow} A$
\end{enumerate}

\[\lim_{x \rightarrow x_0} \underbrace{\lim_{n \rightarrow \infty}f_n(x)}_{\text{равномерный, } S(x)} =  \lim_{n \rightarrow \infty} \underbrace{\lim_{x \rightarrow x_0} f_n(x)}_{A_n}\]

\textit{Доказательство:}

Это такая попытка сделать двойной предел и для функций. Подгоняем под предыдущую теорему: $u_1 = f_1, u_n = f_n - f_{n - 1}, \quad a_1 = A_1, a_n = A_n - A_{n - 1}$. $\sum_{k = 1}^n u_k = f_n \rshe S(x)$, то есть $\sum u_n \rshe$. Супер, предыдущая теорема запущена. Пожинаем плоды ($\sum a_n$ сходится):

\[\sum_{k = 1}^n A_n \text{ --- имеет конечный предел}\]

\[\lim_{x \rightarrow x_0} \sum_{k = 1}^{\infty} u_k(x) = \lim_{x \rightarrow x_0} S(x) = A\]

ч. т. д.

\subsubsection{Теорема о круге сходимости степенного ряда}
\textit{Формулировка:}

$\sum a_n (z - z_0)^n$ --- степенной ряд

Тогда верно одно из этого:

\begin{enumerate}
    \item ряд сходится при всех $z \in \mathbb{C}$
    \item ряд сходится только при $z = z_0$
    \item $\exists R \in (0, \infty):$ \begin{enumerate}
        \item $|x - x_0| < R$ --- ряд абсолютно сходится
        \item $|x - x_0| > R$ --- ряд расходится
    \end{enumerate}
\end{enumerate}

\textit{Доказательство:}

Вспомним преамбулу из определения формулы Коши-Адамара. Ну и всё, вот берём признак Коши, и если предел равен нулю, то $(a)$ работает. Если бесконечности, то $(b)$ работает. А иначе, просто берём за радиус формклу Коши-Адамара.

ч. т. д.


\subsubsection{Теорема о непрерывности степенного ряда}
\textit{Формулировка:}

\begin{itemize}
    \item $\sum a_n(z - z_0)^n$ --- степенной ряд
    \item $0 < R \le \infty$ --- радиус сходимости
\end{itemize}

Тогда:

\begin{enumerate}
    \item $\forall r:  0 < r < R$ ряд равномерно сходится в $\overline{B(z_0, r)}$
    \item $f(z) = \sum a_n (z - z_0)^n$ --- непрерывно на $B(z_0, R)$
\end{enumerate}

\textit{Доказательство:}

\textbf{(1)}

Давайте докажем по признаку Вейерштрасса, оценим каждый член по модулю:

\[\left|a_n \cdot (z - z_0)^n\right| \le |a_n| \cdot r^n = c_n\]

А почему же $\sum c_n$ сходится? Да всё очень просто. $\sum a_n (z - z_0)^n|_{z = z_0 + r} = \sum a_n r^n$ --- сходится абсолютно на $r \Rightarrow$ сходится и всё работает.

\textbf{(2)}

Вспоминаем нашу любимую теорему Стокса-Зайдля, и там у нас доказательство строилось на гига-неравенстве треугольника (всё по модулю). Поэтому наш переход к комплексным числам вообще не мещает, оцениваем то мы уже вещественные. Поэтому давайте для каждого $z$ возьмём $r$ чуть больший, чем $|z - z_0|$, но в пределе круга сходимости: $|z - z_0| < r < R$: 

\images{0.3}{st_r_nepr.png}

А на нём по \textbf{(1)} у нас есть равномерная сходимость, поэтому по Стоксу-Зайдлю $f(z)$ --- непрерывна.

ч. т. д. 


\subsubsection{Теорема о дифференцировании степенного ряда. Следствие об интегрировании. Пример.}

\textit{Предисловие о комплексном дифференцировании:} 

Чтобы определить комплексное дифференцирование, необходимо ввести какое-то определение а-ля дифференцируемости: 

\[f'(z) = f(z_0) + A(z - z_0) + o(|z - z_0|)\]

Но здесь $A$ --- комплексный производный оператор. Можно рассмотреть комплексное число $z = u + iv$ как вектор в $\mathbb{R}^2 \Leftrightarrow f(u, v)$. Тогда производный оператор будет матрицей $2 \times 2$, подчиняющейся условию Коши-Римана: 

\[A = \begin{pmatrix}
    u_x' & u_y' \\
    v_x' & v_y'
\end{pmatrix}, \quad \begin{cases}
    u_x' = v_y' \\
    v_x' = -u_y'
\end{cases}\]

(его выводили на лекции, не уверен, что тут это нужно. Будет время, допишу)

Вместе с этим, можем доказать неравенство(оно же --- дифференцируемость $f(z) = z^n$ по ``школьному'' определению):

\[ \lim_{z \rightarrow z_0}{\frac{z^n - z^n_0}{z - z_0}} = \lim_{z \rightarrow z_0}\frac{(z - z_0)(z^{n - 1} + z^{n - 2}z_0 + z^{n - 3}z^2_0 + \ldots + z^{n - 1}_0)}{z - z_0} = nz^{n - 1}\]

Типа у нас там в скобочке $n$ слагаемых, и они все стремятся к $z^{n - 1}$. И ещё мини-лемма:

\textit{Формулировка (лемма):}

$w, w_0 \in \mathbb{C}, |w| \le r, |w_0| \le r$

Тогда: $|w^n - w^n_0| \le nr^{n - 1} |w - w_0|$

\textit{Доказательство:}

По тому же принципу, что и выше: 

\[|w^n - w^n_0| = |w - w_0| \cdot |w^{n - 1} + w^{n - 2}w_0 + \ldots + w^{n - 1}_0| \le nr^{n - 1}|w - w_0|\]

ч. т. д. 

\textit{Формулировка:}

\begin{enumerate}
    \item $f(z) = \sum a_n (z - z_0)^n = f(z)$ --- степенной ряд с радиусом сходимости $0 < R \le \infty$, равномерно сходится на нём
    \item $\sum_{k = 1}^\infty na_n(z - z_0)^{n - 1}$
\end{enumerate}

Тогда:

\begin{enumerate}
    \item $(2)$ имеет такой же радиус сходимости, что и $(1)$
    \item $\forall z \in B(z_0, R): \dbl f'(z) = \sum_{k = 1}^\infty na_n(z - z_0)^{n - 1}$
\end{enumerate}

\textit{Доказательство:}

\textbf{(1)}

По формуле Коши-Адамара (для ряда $(z - z_0) \cdot (1) = \sum na_n(z - z_0)^n$, просто домножили на скобку, по идее нам ничего это не ломает, т. к. мы можем рассмотреть предел частичных сумм и там всё будет хорошо): 

\[R_{(1)} = \frac{1}{\overline{\lim}_{n \rightarrow \infty} \sqrt[n]{n|a_n|}} \underset{\sqrt[n]{n} \underset{n \rightarrow \infty}{\longrightarrow} 1}{=} R\]

\textbf{(2)}

Найдём производную в произвольной точке $a \in B(z_0, r), 0 < r < R$: 

\[f'(a) = \lim_{z \rightarrow a} \frac{f(z) - f(a)}{z - a}\]

Заведём $w = z - z_0, w_0 = a - z_0$ и заменим в пределе функции на суммы: 

\[\lim_{w \rightarrow w_0} \sum a_n \frac{(z - z_0)^n - (a - z_0)^n}{w - w_0} = \lim_{w \rightarrow w_0} \sum a_n \frac{w^n - w^n_0}{w - w_0}\]

Мы хотим перейти к сумме пределов, но для этого, по теореме, которую мы доказывали ранее, нам надо, чтобы под переделом была равномерная сходимость. Давайте оценим по модулю по лемме: 

\[|a_n|\frac{|w^n - w^n_0|}{|w - w_0|} \le |a_n|nr^{n - 1}\]

\images{0.3}{st_r_diff.png}

Возьмём $r$ из определения шара выше ($w, w_0$ по модулю меньше $r$, см. картинку) и заметим, что ряд $(1)|_{z = z_0 + r}$ сходится. К чему бы всё это? Да к тому, что вся сумма под пределом сходится равномерно по Вейерштрассу (мажорирующая вещественаная последовательность сходится)! Значит можем поменять местами предел и сумму, и по дифференцируемости $z^n$:

\[\sum a_n \lim_{w \rightarrow w_0} \frac{w^n - w^n_0}{w - w_0} = \sum n a_n w^{n - 1} = \sum n a_n (z - z_0)^{n - 1} = (1)\]

Ну и всё, раз мы произвольно выбирали $r$, то для любых $z$ из круга сходимости всё сошлось.

ч. т. д.

\textit{Следствия:}

\begin{itemize}
    \item $f(z) = \sum a_n(z - z_0)^n, R > 0 \Rightarrow f(z) \in C^{\infty}(B(z_0, R))$ (просто дифференцируем бесконечное число раз)
    \item Рассмотрим вещественный ряд $f(x) = \sum a_n (x - x_0)^n$. Тогда  $F = C + \sum \frac{a_n}{n + 1}(x - x_0)^{n + 1}$ --- первообразная $f(x)$, и у него такой же радиус сходимости, как и у $f(x)$.
    
    \textit{Замечание:}

    Если считать интеграл по какому-то промежутку, то константа сократится и $\int_{x_0}^{x} f(x) dx = \sum \int_{x_0}^{x}a_n(x - x_0)^n dx$
\end{itemize}

\textit{Пример:}

Рассмотрим $f(x) = \arctan(x)$. Хотим разложить в ряд. Сложновато. А в производной что? $f'(x) = -\frac{1}{a + x^2} = -1 + x^2 - x^4 + \ldots$ при $x \in (-1, 1)$ --- степенной ряд. Супер, давайте поинтегрируем: 

\[\arctan(x) = C - x + \frac{x^3}{3} - \frac{x^5}{5} + \ldots\]

Надо найти константу. КПК way: подствим $x = 0: \arctan(0) = \frac{\pi}{2} \Rightarrow$

\[\arctan(x) = \frac{\pi}{2} - x + \frac{x^3}{3} - \frac{x^5}{5} + \ldots\]

ч. т. д.

\subsubsection{Свойства экспоненты}
\textit{Формулировка:}

\[z \in \mathbb{C}, \quad \exp(z) = \sum_{n = 0}^{\infty} \frac{z^n}{n!}, \quad R = \frac{1}{\overline{\lim}_{n \rightarrow \infty}\sqrt[n]{\left|\frac{1}{n}\right|}} = \infty\]

\begin{enumerate}
    \item $\exp(0) = 1$
    \item $\exp'(z) = \exp(z)$
    \item $\overline{\exp(z)} = \exp(\overline{z})$, где $\overline{z}$ --- сопряжённое к $z$
    \item $\exp(z + w) = \exp(z) \cdot \exp(w)$
\end{enumerate}

\textit{Доказательство:}

\textbf{(1)}

\[\exp(0) = \sum_{n = 0}^{\infty} \frac{z^n}{n!} = \frac{0^0}{0!} = 1\]

\textbf{(2)}

\[\exp'(z) = \sum_{n = 1}^{\infty} \frac{z^{n - 1}}{(n - 1)!} = \sum_{n = 0}^{\infty} \frac{z^n}{n!} = \exp(z)\]

\textbf{(3)}

\[\overline{\exp(z)} = \overline{\sum_{n = 0}^{\infty} \frac{z^n}{n!}} = \sum_{n = 0}^{\infty} \frac{\overline{z^n}}{n!} = \exp(\overline{z})\]

\textbf{(4)}

Вспоминаем правило прямого перемножения рядов:

\[c_n = \left(\sum_{m = 1}^{n} a_m \right) \cdot \left(\sum_{k = 1}^{n} b_k \right) = (a_1b_1 + a_1b_2 + \ldots + a_nb_n)\]

Ну и по нему: 

\[\exp(z) \cdot \exp(w) = \left(\sum_{k = 1}^{\infty} \frac{z^k}{k!} \right) \cdot \left(\sum_{n = 1}^{\infty} \frac{w^n}{n!} \right) = \left(\sum_{m = 1}^{\infty} \frac{z^m}{m!}\frac{w^0}{0!} + \frac{z^{m - 1}}{(m - 1)!}\frac{w^1}{1!} + \ldots + \frac{z^0}{0!}\frac{w^m}{m!}\right) =\]

\[= \sum_{m = 1}^{\infty} \left(\frac{1}{m!} \sum_{k = 1}^{m} \frac{z^{m - k} \cdot w^k \cdot m!}{(m - k)! \cdot k!}\right) = \sum_{m = 1}^{\infty}\frac{(z + w)^m}{m!} = \exp(z + w)\]

ч. т. д.

\textit{P. S.}

Там КПК рассказывал ещё весёлые выоды тригонометрических формул через экспоненту, но, КМК, оно тут не нужно

\subsubsection{Метод Абеля суммирования рядов. Следствие}
\textit{Формулировка:}

\begin{itemize}
    \item $\sum c_n$ --- сходящийся вещественный ряд
    \item Пусть $f(x) = \sum c_nx^n$ при $x \in (-1, 1)$
\end{itemize}

Тогда: $\lim_{x \rightarrow 1 - 0} f(x) = \sum c_n$

\textit{Доказательство:}

По признаку Абеля $f(x)$ сходится равномерно на $E = [0, 1]$ ($x^n$ равномерно ограничено $1$, а $c_n$ --- сходится $\Rightarrow$ равномерно сходится). Всё это вместе даёт нам непрерывность в $1$ (на интервале нам её даёт теорема Стокса-Зайдля, но тут нам надо именно в $1$, поэтому приходится использовать тяжёлую артиллерию).

ч. т. д.

\textit{Следствие:}

\begin{itemize}
    \item $\sum a_n = A, \sum b_n = B$
    \item $c_n = \left(\sum_{m = 1}^{n} a_m \right) \cdot \left(\sum_{k = 1}^{n} b_k \right) = (a_1b_1 + a_1b_2 + \ldots + a_nb_n)$
    \item $\sum c_n = C$
\end{itemize}

Тогда: $\sum c_n = A \cdot B$

\textit{Доказательство:}

Пусть $f(x) = \sum a_nx^n, \quad g(x) = \sum b_nx^n$ и $h(x) = \sum c_nx^n$ при $x \in [0, 1]$. Тогда при $x < 1: h(x) = f(x)g(x)$ (очевидно, просто перемножить по  условию ивсё будет норм). При $x = 1$ это неочевидно, но мы применяем теорему, демаем предельный переход при $x \rightarrow 1 - 0: C = A \cdot B$.

ч. т. д.

\subsubsection{Единственность разложения функции в ряд (Тейлора)}
\textit{Формулировка:}

Ряд Тейлора: $(a_n) \dbl \exists U(x_0): \dbl f(x) = \sum a_n (x - x_0)^n$

Бесплатно: $f(x) \in C^{\infty}(U(x_0))$

Если существует разложение функции в ряд Тейлора, то оно единственно.

\textit{Доказательство:}

Давайте посмотрим на $k$-ю производную функции:

\[f^{(k)}(x) = \sum_{n = k}^{\infty}\frac{n!}{(n - k)!}a_n(x - x_0)^{n - k}, \quad f^{(k)}(x_0) = a_kk!\]

Тогда $a_k$ однозначно можно определить: 

\[a_k = \frac{f^{(k)}(x_0)}{k!}\]

Ничего не напоминает?

ч. т. д.


\subsubsection{Разложение бинома в ряд Тейлора}
\textit{Формулировка:}

$\forall \sigma \in \mathbb{R} \quad (1 + x)^{\sigma} = S(x) = 1 + \sigma x + \frac{\sigma (\sigma - 1)}{2!}x^2 + \ldots$

Заметим, что если $\sigma \in \mathbb{N}$, то ряд в один момент обрубится и у нас просто будет бином.

\textit{Доказательство:}

Для начала выясним радиус сходимости. В выводе формулы Коши-Адамара мы оценивали ряд абсолютно с помощью признака Коши. Давайте сделаем так же, но с помощью признака Даламбера:

\[\lim_{n \rightarrow \infty}\left|\frac{a_{n + 1}}{a_{n}}\right| = \lim_{n \rightarrow \infty} \left| \frac{\frac{\sigma(\sigma - 1)\ldots(\sigma - n)}{(n + 1)!}}{\frac{\sigma(\sigma - 1)\ldots(\sigma - n - 1)}{n!}} \right| = \lim_{n \rightarrow \infty}{\frac{|\sigma - n|}{n}} = 1\]

Получается, что $|x| < 1$. Теперь докажем, что $S'(x)(1 - x) = \sigma S(x)$. Для этого достаточно показать, что $coef(x_k)$ у обоих частей равенства одинаковый (так как мы взяли первую производную и тут же домножили на $(1 - x)$, тем самым изменились только коэффициенты при членах ряды).

\[coef_{left}(x_k) = coef(x_k) + coef(x_{k - 1}) =\]

Можно представить себе как $S'(x)x + S'(x)$ (это $x_k$ и $x_{k - 1}$)
 
\[= \frac{\sigma(\sigma - 1) \ldots (\sigma - k)}{k!} + \frac{\sigma(\sigma - 1) \ldots (\sigma - k + 1)k}{k!} =\]

Тут суммируем последнюю скобку $(\sigma - k)$:

\[ = \frac{\sigma(\sigma - 1) \ldots (\sigma - k + 1)}{k!}\sigma \]

Ну а справа:

\[coef_{right}(x_k) = \frac{\sigma(\sigma - 1) \ldots (\sigma - k + 1)}{k!}\sigma\]

Всё равно. Супер. При этом --- это диффур. То есть мы предполагаем, что:

\[S(x) = C \cdot (1 + x)^{\sigma} \Leftrightarrow \frac{S(x)}{(1 + x)^{\sigma}} = C\]

Продифференцируем с обоих сторон и удостоверимся, что это правда:

\[\left(\frac{S(x)}{(1 + x)^{\sigma}}\right)' = \frac{S'(x)(1 + x)^{\sigma} - S(x)\sigma(1 + x)^{\sigma - 1}}{(1 + x)^{2\sigma}} = \frac{S'(x)(1 + x)^{\sigma} - S'(x)(1 + x)^{\sigma}}{(1 + x)^{2\sigma}} = 0 = C'\]

Ну всё, функция действительно верно найдена. Поэтому осталось найти константу: 

\[S(0) = 1 = C \cdot (1 + 0)^{\sigma} \Rightarrow C = 1\]

ч. т. д.

\textit{Следствия:}

\begin{enumerate}
    \item $\arcsin x = \sum \frac{(2k - 1)!!}{(2k)!!} \frac{x^{2k + 1}}{2k + 1}, |x| < 1$ --- доказывается дифференцированием и интегрированием
    \item $\sum n(n-1)\ldots(n - m + 1)t^{n - m} = \frac{m!}{(1 - t)^{m + 1}}, |t| < 1$ --- выводится из формулы для геометрической прогессии дифференцированием
\end{enumerate}

\subsubsection{Теорема о разложимости функции в ряд Тейлора}
\textit{Формулировка:}

\begin{itemize}
    \item $f \in C^{\infty}(x_0 - h, x_0 + h)$
\end{itemize}

Тогда:

\[f\text{ --- разложима в } U(x_0) \dbl \Leftrightarrow \dbl \exists \delta, C, A > 0 \dbl \forall n: |f^{(n)}(x)| \le C \cdot A^n \cdot n!, \quad |x - x_0| < \delta\]

\textit{Доказательство:}

\textbf{$\Leftarrow$}

Запишем формулу Тейлора с остатком в форме Лагранжа: 

\[f(x) = f(x_0) + \frac{f'(x_0)}{1!}(x - x_0)^{1} + \ldots + \frac{f^{(n)}(x_0)}{n!}(x - x_0)^n + \underbrace{\frac{f^{(n + 1)}(\overline{x})}{(n + 1)!}(x - x_0)^{n + 1}}_{R_n \underset{n \rightarrow \infty}{\longrightarrow} 0, \overline{x} \in (x_0, x_0 + x)}\]

Оценим остаток: 

\[|R_n| \le \frac{\left|f^{(n + 1)}(\overline{x})\right|}{(n + 1)!}|x - x_0|^{n + 1} \le C \cdot \left(A \cdot |x - x_0| \right)^{n + 1} \underset{|x - x_0| < \frac{1}{A}, |x - x_0| < \delta, n \rightarrow \infty}{\longrightarrow} 0\]

Оценили и доказали.

\textbf{$\Rightarrow$}

Раз функция раскладывается, значит существует $U(x_0)$, в которой ряд Тейлора сходится. Давайте возьмём $x_1 \in U(x_0) \setminus \{x_0\}$ и оценим $n$-й член ряда Тейлора, который стремится к 0 при $n \rightarrow \infty$, раз ряд сходится, и, следовательно, ограничен: 

\[\left|\frac{f^{(n)}(x_0)}{n!}(x_1 - x_0)^n\right| \le C\]

Оценим $|f^{(n)}(x_0)|$:

\[\left|f^{(n)}(x_0)\right| \le \frac{C k!}{|x_1 - x_0|^n} \le\]

Пусть $B = \frac{1}{|x_1 - x_0|}$:

\[\le CB^kk!\]

Теперь рассмотрим ряд в произвольной точке из окрестности, оценим её $m$-тую производную:

\[\left|f^{(m)}(x)\right| \le \left|\sum_{n = m}^{\infty}\frac{f^{(k)}(x_0)}{k!}n(n - 1)\ldots(n - m + 1)(x - x_0)^{k - m}\right| \le \]

\[\le \sum \frac{\left|f^{(k)}(x_0)\right|}{k!}n(n - 1)\ldots(n - m + 1)|x - x_0|^{k - m} \le\]

Оценим сверху новейшими достижениями: 

\[\le CB^m\sum n(n - 1)\ldots(n - m + 1)B^{k - m}|x - x_0|^{k - m} = \]

Тогда по следствию из предыдущей теоремы заменяем: 

\[ = CB^m\frac{m!}{(1 - (B|x - x_0|))^{m + 1}} \le\]

Заметим, что это равенство существует только при $|x - x_0| < \frac{1}{2B}$ --- чтобы под знаменателем не было нуля, и мы не вышли за радиус сходимости нашей формулы, по которой заменяли. Далее (оцениваем этим): 

\[\le CB^mm!2^{m + 1} = \underbrace{2C}_{C}(\underbrace{2B}_{A})^m\underbrace{m!}_{k!}\]

Ну всё, супер, осталось только аккуратно подобрать $\delta = \min\{\frac{1}{2B}, \text{радиус } U(x_0)\}$ --- чтобы точно всё было хорошо.

ч. т. д. 

\subsubsection{Теорема Таубера о совпадении суммы ряда с суммой в смысле метода Абеля}

Перед началом, советую прочитать ``Преамбулу к сумме расходящихся рядов''

\textit{Формулировка:}

$a_n = o(\frac{1}{n}), \sum a_n \eqby{AP} A \Rightarrow \sum a_n = A$

\textit{Доказательство:}

Заведём $\delta_n = \max_{k \ge n} |ka_k|$ ($ka_k$ тоже, поэтому всё хорошо, максимум есть). Эта последовательность также стремится к 0 монотонно.

Давайте рассмотрим такую разность частичныой суммы и найденного предела:

\[\sum_{n = 0}^{N} a_n - A = \left(\sum_{n = 0}^{N} a_n - \sum_{n = 0}^{N} a_n x^n\right) - \sum_{n = N + 1}^{\infty} a_n x^n + \left(\sum_{n = 0}^{\infty} a_n x^n - A\right)\]

И ``бесцеремонно'' оцениваем его модулями (в т. ч. модули под суммой): 

\[\left|\sum_{n = 0}^{N} a_n - A\right| \le \underbrace{\sum_{n = 0}^{N} |a_n|(1 - x^n)}_{(1)} + \underbrace{\sum_{n = N + 1}^{\infty} \frac{|n|a_n x^n}{n}}_{(2)} + \underbrace{\left|\sum_{n = 0}^{\infty} a_n x^n - A\right|}_{(3)} \quad (*)\]

$(1)$ --- вынесли $a_n$, и по определению метьода суммы Абеля $|x| < 1$, поэтому можно снять модуль. $(2)$ --- просто записали так для более удобной дальнейшей оценки. 

Давайте запустим стандартное $\varepsilon$ определение: берём $\varepsilon > 0$, и вычисляем по нему такое $N$, чтобы: 

\[\begin{cases}
    (3) < \varepsilon\\
    \delta_{N + 1} < \varepsilon^2
\end{cases}\]

Причём $x$ будем выбирать согласованно с $N$ по формуле $(1 - x)N = \varepsilon$. Ещё вспомним неравнство Бернулли: $(1 + x^n) \le n(x + 1)$ (это выводится из обычного неравенства Бернулли $(1 - x)^n \le 1 - nx$ заменой $t = 1 - x$). Погнали оценивать: 

\[(1) \le \sum |a_n|n(1 - x) = (1 - x)\sum_{n = 0}^{N}|na_n| \le (1 - x)N\delta_1\]

Оценили наибольшим членом, умноженным на количество членов.

\[(2) \le \frac{\delta_{N + 1}}{N + 1}\sum_{n = N + 1}^{\infty} x_n < \frac{\delta_{N + 1}}{N + 1}\sum_{n = 0}^{\infty} x_n = \frac{\delta_{N + 1}}{(N + 1)(1 - x)} < \frac{\delta_{N + 1}}{N(1 - x)} < \frac{\varepsilon^2}{\varepsilon}= \varepsilon\]

Сначала мы оценили наибольшим членом (это валидно, так как $\delta_n$ монотонно стремится к $0$, потом оценили ряд из $x^n$ рядом из геометрической прогрессии (т. к. $|x| < 1$), ну а потом подогнали под условия выбора $N$). $(3) < \varepsilon$ по тем же причинам. Итого: 

\[(*) < \varepsilon\delta_1 + \varepsilon + \varepsilon = \varepsilon(\delta_1 + 2)\]

Ну всё, супер, разница между частичной суммой и ответом сколь угодно мала (на $\delta_1$ не сильно смотрим, всё супер).

ч. т. д.

\subsubsection{Теорема Коши о перманентности метода средних арифметических}
\textit{Формулировка:}

$\sum a_n = S \in \mathbb{R} \Rightarrow \sum a_n \eqby{\text{с. а.}} S$

\textit{Доказательство:}

Для начала, запишем определение сходимости обычной суммы:

\[\forall \varepsilon > 0 \dbl \exists N_{1} \dbl \forall n > N \dbl |S_n - S| < \varepsilon\]

А теперь попробуем оценить разность частичной суммы и ответа:

\[\left|\sigma_n - S\right| = \left|\frac{1}{n + 1}\sum_{k = 0}^{n}S_k - \underbrace{S}_{\frac{(n + 1)S}{n + 1}}\right| = \frac{1}{n + 1} \left|\sum_{k = 0}^{n} \left(S_k - S\right)\right| \le \frac{1}{n + 1}\sum_{k = 0}^{n}\left|S_k - S\right| \le\]

По определению выше, мы берём $\varepsilon$, по нему вычисляем большой $N_1$, и говорим, что $n > N_{1}$. ``Расчекрыживаем'' сумму:

\[\le \underbrace{\frac{1}{n + 1}\sum_{k = 1}^{N_1} |S_k - S|}_{(1)} + \underbrace{\frac{1}{n + 1}\sum_{k = N_1 + 1}^{n} |S_k - S|}_{(2)} \le \]

$(2)$ уже сразу $< \varepsilon$, т. к. $n - (N_1 - 1)$ очевидно меньше, чем $n + 1$, так что по определению это работает. С другой стороны, мы можем управлять $n$, поэтому давайте выберем его таким, чтобы $(1)$ было $< \varepsilon$. Вуаля: 

\[< \varepsilon + \varepsilon = 2 \varepsilon\]

ч. т. д. 

\subsubsection{Преобразование Абеля степенного ряда}
\textit{Формулировка:}

\begin{itemize}
    \item $A_n = (a_1 + a_2 + \ldots + a_n)$
    \item $\sum_{n = 0}^{N} a_n x^n = \sum_{k = 0}^{N - 1} A_k(x^k - x^{k + 1}) + A_N x^N$
\end{itemize}

Тогда при $N \rightarrow \infty$ эту сумму можно заменить на

\[\sum_{n = 0}^{\infty} a_n x^n = (1 - x)\sum_{n = 0}^{\infty} A_n x^n\]

при $|x| < \min \{1, R_{\text{сходимости}}\}$

\textit{Доказательство:}

Сначала докажем, что радиус сходимости у правого ряда не изменился относительно оригинального ряда. Возьмём рандомные $r, r_1$ так, чтобы $r < r_1 < R$. Теперь сужаем наш степенной ряд на $x = r$:

\[\left(\sum_{n = 0}^{\infty} a_n x^n\right)_{x = x_0 + r}, \forall n: |a_n|x^n \le |a_n|r^n_{1} \underset{\text{на радиусе сходимости}}{\longrightarrow} 0\]

Значит, $a_n = o(\frac{1}{r^n_{1}}) \Rightarrow A_n = o(\frac{n}{r^n_1}) \Rightarrow \frac{A_nr^n_{1}}{n} \rightarrow 0$

Теперь оценим ряд после преобразования:

\[\sum_{n = 0}^{\infty} |A_n| x^n \le \sum_{n = 0}^{\infty} |A_n| r^n = \sum_{n = 0}^{\infty} |A_n|r^n_{1} \left(\frac{r}{r_{1}}\right)^n \le\]

Утверждается, что раз $|A_n|r^n_1$ растёт незначительно (порядка) $o(k)$, тогда можно оценить сверху константой (???): 

\[\le C \sum_{n = 0}^{\infty}k \left(\frac{r}{r_1}\right)^{n}\]

И тут дробь, меньшая единицы в степени, против линейной функции, так что ряд сходится. Ну супер, тогда для любых $r, r_1$ ряд сходится и значит круг сходимости не изменился. Теперь гораздо интереснее то, почему второй член исходной суммы исчез. Давайте его оценим. Опять берём $|x| < r < \min\{1, R_{\text{сходимости}}\}$. Вот тут уже будет иметь значение то, что $r < 1$. И заметим, что раз в круге сходимости у нас ряд сходится, то $|a_nx^n| \le |a_nr^n| \rightarrow 0$, значит эта последовательность ограничена:

\[\exists L > 0: \forall n \dbl |a_n|r^n < L\]

И погнали оценивать ($|a_n| < \frac{L}{r^n}$):

\[|A_Nx^N| \le L|x|^N\left(1 + \frac{1}{r} + \ldots + \frac{1}{r^n}\right) \le\]

Это --- геометрическая прогрессия, заменяем по формуле: 

\[\le L|x|^N\frac{\frac{1}{r^{N + 1}} - 1}{1 - \frac{1}{r}} \le\]

Снизу у нас $r < 1$, поэтому оцениваем сверху, меняя знак, и раскрываем верхнюю скобку: 

\[\le L |x|^N\frac{\frac{1}{r^{N + 1}} - 1}{\frac{1}{r} - 1} =  L |x|^N\frac{(\frac{1}{r^{N + 1}} - 1)r}{r - 1} = \underbrace{\frac{L}{r - 1}\left(\frac{|x|}{r}\right)^{N}}_{(1)} + \underbrace{\frac{Lr|x|^n}{r - 1}}_{(2)}\]

Ну и всё, в $(1)$ у нас дробь, меньшая единицы, в степени, поэтому при $N \rightarrow \infty$ стремится к $0$, а $(2)$ сам $|x| < 1$ в степени, поэтому тоже стремится к $0$.

ч. т. д. 

\subsubsection{Теорема о связи суммируемости по Чезаро и по Абелю}

\textit{Рассуждения:}

Давайте оценим всякие суммы в смысле средних арифметических: 

\[\sigma_n = \frac{1}{n + 1}(S_1 + S_2 + \ldots + S_n)\]

Если сходится, то:

\[\sigma_n \underset{n \Longrightarrow \infty}{\longrightarrow} 0 \quad \Rightarrow \quad \frac{n}{n + 1} \cdot \sigma_{n - 1} \underset{n \rightarrow \infty}{\longrightarrow} 0\]

Тогда давайте оценим частиыную сумму в пределе:

\[\frac{S_n}{n} = \frac{\sigma_n - \frac{n}{n + 1}\sigma_{n - 1}}{n} \underset{n \rightarrow \infty}{\longrightarrow} 0 \quad \Longrightarrow \quad S_n = o(n)\]

И сами члены:

\[\frac{a_n}{n} = \frac{S_{n + 1} - S_{n}}{n} \underset{n \rightarrow \infty}{\longrightarrow} 0 \quad \Longrightarrow \quad a_n = o(n)\]

\textit{Формулировка:}

\begin{itemize}
    \item $\sum a_n$ --- числовой ряд
    \item $\sum a_n \eqby{\text{с. а.}} A$
\end{itemize}

Тогда $\exists \sum a_n \eqby{AP} A$

\textit{Доказательство:}

Раз существует сумма в смысле средних арифметических, то $a_n = o(n) \Rightarrow f(x) = \sum a_n x^n$ при $x \in (0, 1)$. Тогда по преобразованию Абеля (оно работает, т. к. $|x| < 1$) (применим его 2 раза, это тоже законно, т. к. $a_n = o(n) \Rightarrow A_n = o(n^2)$ и всё равно сидим в круге сходимости $1$, тогда ещё вспоминаем, что $(n + 1)\sigma_n = \sum A_k$): $f(x) = (1 - x)\sum A_n x^n = (1 - x)^2\sum (n + 1)\sigma_n x^n$. 

С другой стороны, нам сообщают секретную формулу: $1 = (1 - x)^2\sum(n + 1)x^n$ (доказывается дифференцированием $\frac{1}{1 - x}$ и соответствующего ряда ($\frac{1}{(1 - x)^2} = \sum (n + 1)x^n$)). Преобразуем её, домножив на $A: A = (1 - x)^2\sum (n + 1)A x_n$ и начинаем оценивать $A - f(x)$, фиксируя $\varepsilon > 0$:

\[A - f(x) = (1 - x)^2\sum (n + 1)A x_n - (1 - x)^2\sum (n + 1)\sigma_n x^n = (1 - x)^2\sum_{n = 0}^{\infty} (n + 1) x^n (A - \sigma_n) = \]

Разобьём сумму каким-то $N$, пока не понятно каким, но мы это потом придумаем: 

\[= (1 + x)^2 \underbrace{\sum_{n = 0}^{N} (n + 1) x^n (A - \sigma_n)}_{(3)} + \underbrace{(1 - x)^2 \sum_{n = N + 1}^{\infty} (n + 1) x^n \underbrace{(A - \sigma_n)}_{(1)}}_{(2)} \le \]

Заметим, что $(1)$ уже очень маленькое ($< \varepsilon$) при больших $N$ (по определению сходимости в смысле среднего арифметического). А сумму $(2)$ мы умеем считать по секретной формуле (там, конечно, с нуля, а тут с $N + 1$, но мы то оцениваем сверху). Поэтому $(2) < (1 - x)^2\frac{\varepsilon}{(1 - x)^2 } = \varepsilon$. Ну так давайте после выбора $N$ выберем такое $x$, близкое к $1$, чтобы $(3)$ было меньше $\varepsilon$. Ну и всё, супер: 

\[< \varepsilon + \varepsilon = 2\varepsilon\]

ч. т. д. 

\subsubsection{Две леммы об интегрировании асимптотических равенств}
\textit{Формулировка (лемма 1):}

\begin{itemize}
    \item $f, g: C[a, b) \rightarrow \mathbb{R}$
    \item $g \ge 0, \int_a^b g(x) dx$ --- расходящийся (и несобственный)
    \item $F(x) = \int_a^x f(t) dt, \quad G(x) = \int_a^x g(t) dt$
\end{itemize}

Тогда:

\begin{enumerate}
    \item $f = O(g) \Rightarrow F = O(G)$
    \item $f = o(g) \Rightarrow F = o(G)$
    \item $f \sim g \Rightarrow F \sim G$
\end{enumerate}

\textit{Доказательство:}

\textbf{(1)}

По определнию ``О''-большого: 

\[\exists M > 0 \dbl \exists x_0 \in [a, b) \dbl \forall x \in [x_0, b): |f(x)| < Mg(x) \]

Теперь давайте рассмотрим нормальный (не-несобственный) интеграл $\int_a^{x_0} |f(x)| dx = C_1 \ge 0$ и выберем $x_1 > x_0$ и вот такой интеграл $\int_{x_0}^{x_1} g(x) dx = \alpha > 0$. Оцениваем $F(x)$ при любом $x > x_0$:

\[|F(x)| = \left|\int_a^x f(x) dx\right| \le \int_a^x |f(x)| dx = \int_a^{x_0} + \int_{x_0}^{x} \le C_1 + M\int_{x_0}^x g(x) dx \le \]

Оценили константой и по определению ``О''-большого. Теперь ``присобачим'' $\alpha$:

\[\le \frac{C_1}{\alpha} \int_{x_0}^{x_1}g(x) dx + M\int_{x_0}^x g(x) dx \le \]

А теперь цинично запихиваем всё под один интеграл $\int_a^x$ (только увеличиваем сумму):

\[\le \underbrace{\left(\frac{C_1}{\alpha} + M\right)}_{M'} \int_{a}^x g(x) dx = M'G(x)\]

По определению доказано.

\textbf{(2)}

По определнию ``о''-маленького: 

\[\forall \varepsilon > 0 \dbl \exists x_0 \in [a, b) \dbl \forall x \in [x_0, b): |f(x)| < \frac{\varepsilon}{2}g(x) \]

Теперь давайте рассмотрим нормальный (не-несобственный) интеграл $\int_a^{x_0} |f(x)| dx = C_1 \ge 0$ и выберем $x_1 > x_0$ и вот такой интеграл $\int_{x_0}^{x_1} g(x) dx = \alpha > 0$ так, чтобы $\alpha >> C_1$ (вспоминаем, что исходный интеграл расходится, тогда мы можем выбрать $x_1$ так, чтобы $\frac{C_1}{\alpha} < \frac{\varepsilon}{2}$). Ну всё, запускаем рассуждения из прошлой теоремы при ($M \leftrightarrow \frac{\varepsilon}{2}$) и получаем:

\[|F(x)| < \varepsilon G(x)\]

\textbf{(3)}

По определению $f \sim g \Leftrightarrow \lim_{x \rightarrow b - 0} \frac{f(x)}{g(x)} = 1$

Для интегралов (вспомним, что они расходящиеся ($g$ --- по определению, $f$ --- по эквивалентности), поэтому можем применить правило Лопиталя для бексонечностей,, плюс у нас $f, g$ --- непрерывны):

\[\lim_{x \rightarrow b} \frac{\int_a^x f(t) dt}{\int_a^x g(t) dt} = \lim_{x \rightarrow b} \frac{\infty}{\infty} = \lim_{x \rightarrow b}\frac{f(x)}{g(x)} = 1\]

По определению доказано.

ч. т. д. 

\textit{Формулировка (лемма 2):}

\begin{enumerate}
    \item 
        \begin{itemize}
            \item $\varphi_i \in C[a, b)$ --- шкала при $x \rightarrow a$
            \item $\int_a^b \varphi_i(x) dx$ --- сходится для любого $i$
            \item $\Phi_i(x) = \int_x^b \varphi_i(x) dx$
        \end{itemize}

        Тогда $\Phi_i$ --- тоже шкала

    \item     
        \begin{itemize}
            \item $f\in C[a, b)$
            \item $\int_a^b f(x) dx$ --- сходится для любого $i$
            \item Пусть $F(x) = \int_x^b f(x) dx$
        \end{itemize}

        Тогда если $f \sim \sum c_n \varphi_n$, то $F \sim \sum c_n \Phi_n$
\end{enumerate}

\textit{Доказательство:}

\textbf{(1)}

\[\lim_{x \rightarrow b}\frac{\Phi_{i + 1}(x)}{\Phi_{i}(x)} \eqby{\text{Лопиталь } \frac{0}{0} + \text{непр. }\varphi_i} \lim_{x \rightarrow b}\frac{\varphi_{i + 1}(x)}{\varphi_{i}(x)} = 0\]

Шкала --- по определению.

\textbf{(2)}

Проверим, что $F - \sum_{k = 0}^{n} c_k \Phi_{k} = o(\Phi_n)$.

\[\lim_{x \rightarrow b}\frac{F - \sum_{k = 0}^{n} c_{k} \Phi_{k}}{\Phi_{n}(x)} \eqby{\text{Лопиталь } \frac{0}{0} + \text{непр. }\varphi_i} \lim_{x \rightarrow b}\frac{-(f - \sum_{k = 0}^{n} c_{k} \varphi_{k})}{-\varphi_{n}(x)} = 0\]

Минусы вылазят из-за дифференцирования интегралов с переменным верхним пределом снизу.

ч. т. д. 


\subsubsection{Лемма о локализации для интегралов Лапласа}

\textit{Я с ума сойду техать этот раздел...}

\imageh{loc_lapl_1.jpg}

\imageh{loc_lapl_2.jpg}

\subsubsection{Метод Лапласа}

\imageh{met_lapl_1.jpg}

\imageh{met_lapl_2.jpg}

\imageh{met_lapl_3.jpg}

\subsubsection{Формула Стирлинга для гамма-функции}

\imageh{stirl_gamm.jpg}

\subsubsection{Теорема Вейерштрасса о приближении непрерывной функции многочленами}

\imageh{veiersht_lap.jpg}

\subsubsection{Лемма о каноническом виде функции в окрестности стационарной точки}

\imageh{kanon_vd.jpg}

\subsubsection{Лемма Ватсона}

\imageh{watson_l.jpg}

\newpage

\section{Период Кайнозойский}
\subsection{Важные определения}

\subsubsection{Полукольцо, алгебра, сигма-алгебра}

Дизъюнктный набор множеств $A_1, A_2, \ldots, A_n$ такой, что $\bigcap_{1 < i, j < n} A_i A_j = \O$. Значит, что дизъюнктное объединение это: $\bigsqcup_{1 < i < n} A_i$

Это такие системы множеств над $X$ с разными прикольными свойствами: 

\begin{enumerate}
    \item $\mathcal{P} \subset 2^X$ --- полукольцо, если: 
        \begin{enumerate}
            \item $\O \in \mathcal{P}$
            \item $\forall A, B \in \mathcal{P}: A \cup B \in \mathcal{P}$
            \item $\forall A, B \in \mathcal{P}: \exists D_1, D_2, \ldots, D_n$ --- дизъюнктные, такие что: $A \setminus B = \bigsqcup_{i} D_i$
        \end{enumerate}
        
        Свойства: 
        \begin{enumerate}
            \item $A \in \mathcal{P} \nRightarrow A^c \in \mathcal{P}$
            \item $A, B \in \mathcal{P} \nRightarrow (A \cup B | A \vartriangle B | A \setminus B) \in \mathcal{P}$
            \item Аксиома $3^M$: $\forall A, B_1, B_2, \ldots, B_k \in \mathcal{P} \exists D_1, D_2, \ldots D_n$ --- дизъюнктные. Тогда $A \setminus \left(\bigcup_{i = 1}^{k} B_i\right) = \bigsqcup_{i = 1}^{n} D_i$. Доказывается индукцией.
        \end{enumerate}

    \item $\mathcal{A} \subset 2^X$ --- алгебра, если: 
        \begin{enumerate}
            \item $X \in \mathcal{A}$
            \item $A, B \in \mathcal{A} \quad A \setminus B \in \mathcal{A}$
        \end{enumerate}

        Свойства: 
        \begin{enumerate}
            \item $\O = X \setminus X \in \mathcal{A}$
            \item $A, B \in \mathcal{A} \quad A \cap B = A \setminus (A \setminus B) \in \mathcal{A}$
            \item $A^C = X \setminus A \in \mathcal{A}$
            \item $A \cup B = (A^C \cap B^C)^C$
            \item $A_1, A_2, \ldots, A_{n} \in \mathcal{A}, \bigcup_{i = 1}^{n} A_i \in \mathcal{A}, \bigcap_{i = 1}^{n} A_i \in \mathcal{A}$ --- ну тоже тривиально
            \item Всякая алгебра есть полукольцо
        \end{enumerate}
    
    \item $\textfrak{A} \subset 2^X$ --- $\sigma$-алгебра --- это просто алгебра, с аксиомой 3: $\forall A_1, A_2, \ldots \in \textfrak{A}$ --- счётный набор, тогда $\bigcup A_i \in \textfrak{A}$
\end{enumerate}

\subsubsection{Объем}

$\mu: \mathcal{P} \rightarrow \overline{\mathbb{R}}$ --- конечно-аддитивна, если: 

\begin{enumerate}
    \item не принимает одновременно $+$ и $- \infty$ (это защита для физиков, там можно рассматривать кусочек поля и заряды разнознаковые)
    \item $\mu(\O) = 0$
    \item $\forall  A_1, A_2, \ldots, A_{n} \in \mathcal{P}$ --- дизъюнктные. Если $\bigsqcup_{i = 1}^{n} A_i = A \in \mathcal{P}$, то $\mu A = \sum_{i = 1}^{n} \mu A_i$
\end{enumerate}

Если $\mu: \mathcal{P} \rightarrow \overline{\mathbb{R}}$, аддитивна, $\mu \ge 0$, тогда $\mu$ --- \textit{объём}. Если при этом $X \in \mathcal{P}, \mu X < + \infty$, то это --- \textit{конечный объём}.

\textit{Замечание:}

Если $\mathcal{P}$ --- алгебра, то аксиома $3 \Leftrightarrow 3': \forall A, B \in \mathcal{A}$ (дизъюнктные) $\mu (A \sqcup B) = \mu A + \mu B$

\subsubsection{Ячейка}

Это такой параллелепипед: $[a, b) \subset \mathbb{R}^m = \{x \dbl | \dbl a_i \le x_i < b_i\}$

Кубическая ячейка: $a \in \mathbb{R}^m, r \in \mathbb{R}, \quad Q(a, r) = [a_1 - r, a_1 + r) \times [a_2 - r, a_2 + r) \times \ldots \times [a_m - r, a_m + r]$

\subsubsection{Мера, пространство с мерой}

Если $\mu: \mathcal{P} \rightarrow \overline{\mathbb{R}}$ --- объём, счётно-аддитивна (в том смысле, что аддитивная не только для конечного, но и счётного множества подмножеств $\forall A, \underbrace{A_1, A_2, \ldots}_{\text{дизъюнктны, НЧСЧ}}, \bigsqcup_{i = 1}^{\infty} A_i \Rightarrow \mu A = \sum_{i = 1}^{\infty} \mu A_i$), то $\mu$ --- мера.

$(\underbrace{X}_{\text{множество}}, \underbrace{\textfrak{A}}_{\sigma-\text{алгебра}}, \underbrace{\mu}_{\text{мера на } \textfrak{A}})$ --- пространство с мерой

\subsubsection{Сигма-конечная мера}

Если $\mathcal{P}$ --- полукольцо, $\mu$ --- мера на $\mathcal{P}$: 

\[\exists P_1, P_2, \ldots \in \mathcal{P} \quad \forall k \dbl \mu P_k < + \infty \quad \bigcup P_k = X\]

то $\mu$ --- $\sigma$-конечная

\subsubsection{Мера Лебега, измеримое по Лебегу множество}

\textit{Мера Лебега} --- лебеговское продолжение классического объёма. 

$\mathfrak{M}^m$ --- $\sigma$-алгебра измеримых по Лебегу множеств размерности $m$.

$\lambda, \lambda_m$ --- мера Лебега. 


\textit{Свойства: }

\begin{enumerate}
    \item $\mathfrak{M}^m$ содержит все ячейки, их пересечения и объединения. В том числе и точки, которые имеют меру $0$ (оцениваются с помощью одной кубичской ячейки радиуса $\frac{1}{n}$). Ещё, если $A_n \in \mathfrak{M}^m, \lambda A_n = 0 \dbl \bigcup A_n \Rightarrow \lambda \bigcup A_n = 0$. 
    \item Содержит все открытые и (следовательно) замкнутые множества
    \item Канторово множество несчётно
    \item Существуют неизмеримые множества
    \item $A$ --- ограничено и измеримо $\Rightarrow \lambda A < + \infty$
    \item $A$ --- открыто, $\lambda A > 0$
    \item $E$ --- измеримо, $\lambda E = 0 \Rightarrow$ у $E$ нет внутренних точек
    \item $A \in \mathfrak{M}^m, \forall \varepsilon > 0$:
        \begin{enumerate}
            \item $\exists$ открытое $G_{\varepsilon}: A \subset G_{\varepsilon}: \lambda (G_{\varepsilon} \setminus A) < \varepsilon$
            \item $\exists$ замкнутое $F_{\varepsilon}: G_{\varepsilon} \subset A: \lambda (A \setminus F_{\varepsilon}) < \varepsilon$
        \end{enumerate}

    \item Парадокс Хаусфорда-Банаха-Тарского --- нельзя создать меру уже для $\mathbb{R}^3$, покрывающую все подмножества и устойчивую к преобразованиям (конгруэнтным?).
\end{enumerate}

\subsubsection{Измеримая функция}

$f: E \subset X \rightarrow \overline{\mathbb{R}}, a \in \mathbb{R}$

\textit{Лебеговы множества} функции $f$ (*):

\begin{itemize}
    \item $E(f < a) = \{x \in E \dbl | \dbl f(x) < a\}$
    \item $E(f \le a) = \{x \in E \dbl | \dbl f(x) \le a\}$
    \item $E(f > a) = \{x \in E \dbl | \dbl f(x) > a\}$
    \item $E(f \ge a) = \{x \in E \dbl | \dbl f(x) \ge a\}$
\end{itemize}

\textit{Замечани: }

\begin{itemize}
    \item \[E(f > a) = (E(f \le a))^{C} \quad \text{дополнение в }E\]
    \item \[E(f \le a) = \bigcap_{n \in \mathbb{N}} E(f < a + \frac{1}{n})\]
\end{itemize}

Тогда: 

\begin{itemize}
    \item $(X, \textfrak{A}, \mu)$ --- пространство с мерой
    \item $f: E \subset X \rightarrow \overline{\mathbb{R}}, E \in \textfrak{E}$
\end{itemize}

$f$ --- \textit{измерима на множестве} $E$, если $\forall a \in \mathbb{R} \quad E(f < a)$ --- измерима

\textit{Пример:}

$X = \mathbb{R}^m, \textfrak{A} = \mathfrak{M}^m, f: \mathbb{R}^m \rightarrow \mathbb{R}$ --- непрерывная на $X \Rightarrow f$ --- измерима

\textit{Замечания: }

\begin{itemize}
    \item $f$ --- измерима = $f$ --- измерима на $X$
    \item $X = \mathbb{R}^m, \textfrak{A} = \mathfrak{M}^m, f$ --- измерима по Лебегу, если она измерима на $X = \mathbb{R}^m$
    \item Эквивалентны все 4 (*) определения измеримых множеств по Лебегу для $\forall a \in \mathbb{R}$ (следует из предыдущего замечания)
    \item $f$ --- измерима $\Rightarrow \forall a \in \mathbb{R} \quad E(f = a) = E(f \ge a) \cap E(f \le a)$ --- измеримо (обратное неверно)
    \item $f$ --- измерима $\Rightarrow (-)$ --- измеримо   
        \begin{itemize}
            \item $\alpha f (\alpha \in \mathbb{R}), -f$
            \item на $(E_k)$, то и на $\bigcup_{k} E_k$
            \item на $E$, $E' \subset E$, то и на $E'$
            \item при $f \neq 0$, то и $\frac{1}{f}$
            \item при $f \ge 0$, на $E$, то и на $\forall \alpha \in \mathbb{R} \dbl f^{\alpha}$
        \end{itemize}
\end{itemize}


\subsubsection{Сходимость почти везде}

Даны $f_n, f: E \rightarrow \overline{\mathbb{R}}$, тогда $f_n \rightarrow f$ \textit{почти везде (п. в.)}, если $\exists e \subset E, \mu e =0, \forall x \in E \setminus e \quad f_n \rightarrow f$ всюду.

\textit{Пример: }

$x^n \rightarrow 0$ почти везде на $[0, 1]$

\textit{Свойства: }

\begin{enumerate}
    \item $f_n, f: X \rightarrow \overline{\mathbb{R}}, f_n$ --- измерима, $f_n \rightarrow f$ почти всюду. Тогда $f$ --- измерима. Доказывается отбрсыванием плохих точек меры 0.
    \item $g, f: X \rightarrow \overline{\mathbb{R}}, f = g$ почти всюду, тогда $g$ --- измерима на $X$
    \item $\mu$ --- необязательно полная. $f: X \setminus e \rightarrow \overline{\mathbb{R}}, \mu e = 0, f$ --- измерима на $X \setminus e$. Тогда $\exists g: X \rightarrow \overline{\mathbb{R}}$, измеримая на $X$, такая что $f = g$ почти всюду
    \item $f \sim g$, если $f = g$ почти всюду
\end{enumerate}


\subsubsection{Интеграл неотрицательной измеримой функции}

$f \ge 0$ --- измерима

\[\int_{X} f d \mu = \sup \{\int_{X} g d \mu, g\text{ --- ступенчатые}, 0 \le g \le f\}\]

\textit{Замечания:}

\begin{enumerate}
    \item если $f$ --- ступенчатая, то интеграл равен интегралу ступенчатой
    \item $0 \le \int_{X} f d \mu \le + \infty$
    \item ступенчатая $g \le f \Rightarrow \int_{X} g d \mu \le \int_{X} f d \mu$
\end{enumerate}

\subsubsection{Суммируемая функция}

$E \subset X$ --- измеримо, $f$ --- измерима на $X, f$ --- суммируема на $E$, если:

\[\int_{X} f^{+} \chi_{E} < + \infty, \int_{X} f^{-} \chi_{E} < + \infty\]


\newpage

\subsection{Определения}

\subsubsection{Классический объем в $\mathbb R^m$}

Это из серии примеров, было на лекции, будет время, допишу и остальное.

\[\mu [a, b) \subset \mathbb{R}^m = \prod_{i = 1}^{m} |b_i - a_i|\]

Про аддитивность там помахали руками, сказали дробить на мелкие кусочки и всё получится.

\subsubsection{Формулировка теоремы о непрерывности меры снизу}

\textit{Формулировка:}

$\textfrak{A}$ --- алгебра, $\mu: \textfrak{A} \rightarrow \overline{\mathbb{R}}$ --- объём. Тогда следующие утверждения эквивалентны: 

\begin{itemize}
    \item $\mu$ --- мера (то есть счётно-аддитивна)
    \item $\mu$ --- непрерывна снизу, то есть: 
    \[\forall A_1, A_2, \ldots \in \textfrak{A}, \quad A_1 \subset A_2 \subset \ldots, \quad A = \bigcup_{i = 1}^{\infty} A_i \Leftrightarrow A = \lim_{i \rightarrow \infty} A_i\]
\end{itemize}

\subsubsection{Полная мера}

Если $\textfrak{A}$ --- $\sigma$-алгебра, $\mu$ --- мера на $\textfrak{A}$. Тогда $\mu$ --- полная, если $\forall A \subset B, B \in \textfrak{A}, \mu B = 0 \Rightarrow A \in \textfrak{A}, \mu A = 0$


\subsubsection{Формулировка теоремы о лебеговском продолжении меры}

$X$ --- множество, $\mathcal{P}$ --- полукольцо подмнжеств в $X$, $\mu_0$ --- $\sigma$-конечная мера на $\mathcal{P}$.

Тогда существуют $\textfrak{A}$ --- $\sigma$-алгебра и $\mu$ --- мера на $\textfrak{A}$, такие что:

\begin{enumerate}
    \item $\mathcal{P} \subset \textfrak{A}, \mu|_{\mathcal{P}} = \mu_0$
    \item $\mu$ --- полная мера
    \item Если $\mu_1$ --- мера на $\sigma$-алгебре $\textfrak{A}_1 \supset \mathcal{P}$, полная, $\mu_1|_{\mathcal{P}} = \mu_0$, то $\textfrak{A}_1 \supset \textfrak{A}$ и $\mu_1|_{\textfrak{A}} = \mu$
    \item Если $\mu_2$ --- мера на алгебре $\textfrak{A}_2 \subset \textfrak{A}$, $\mathcal{P} \subset \textfrak{A}_2$, $\mu_2|_{\mathcal{P}} = \mu_0$, то $\mu|_{\textfrak{A}_2} = \mu_2$
    \item $\forall A \in \textfrak{A}: \quad \mu A = \inf\left(\sum \mu P_k, A \subset \bigcup_{k = 1}^{\infty} P_k, P_k \in \mathcal{P}\right)$
\end{enumerate}

\subsubsection{Болерелевская $\sigma$-алгебра}

$\textfrak{B}$ --- борелевская $\sigma$-алгбера в $\mathbb{R}^m$ --- минимальная $\sigma$-алгебра, содержащая все открытые множества (ячейки).

\textit{Следствия:}

\begin{enumerate}
    \item $\forall A \subset \mathfrak{M}^m \dbl \exists B, C$ --- борелевские, такие что $B \subset A \subset C, \lambda_m(C \setminus A) = \lambda_m(A \setminus B) = 0$. Доказывается по 8 свойству меры Лебега.
    \item $\forall A \in \mathfrak{M}^m$ представимо в виде $A = B \cup \mathcal{N}$, где $B$ --- борелевское, а $\mu \mathcal{N} = 0$
    \item Регулярность меры Лебега
\end{enumerate}

\subsubsection{Теорема о мерах, инвариантных относительно сдвигов}

\begin{itemize}
    \item $\mu$ --- мера на $\mathfrak{M}^m$
    \item $\mu$ --- инвариантна относительно сдвигов 
    
    \[\forall a \in \mathbb{R}^m \dbl \forall E \in \mathfrak{M}^m \quad \mu(a + E) = \mu(E)\]

    \item $\mu$ --- любого ограниченного множества конечна
\end{itemize}

Тогда:

\[\exists k \in [0, +\infty), \quad \mu = k \cdot \lambda \quad \Leftrightarrow \quad \forall E: \mu E = k \cdot \lambda E\]

\subsubsection{Ступенчатая функция}

$f: X \rightarrow \mathbb{R}$ называется \textit{ступенчатой}, если существует \textit{разбиение} (конечное), такое что $X = \bigsqcup_{i = 1}^{n} e_i, \quad \forall k \dbl f|_{e_k} = \const_k$

\textit{Пример:}

Характеристическая функция множества $e_k$: 
\[\chi_{e_k}(x) = \begin{cases}
    1, x \in e_k \\
    0, x \notin e_k
\end{cases}\]

\textit{Общий вид ступенчатой функции:}

\[f(x) = \sum_{i = 1}^{n} c_i \chi_{e_i}(x)\]

\textit{Замечание:}

Если $\exists f, g$ --- ступенчатые, то можно разбиения подразбить на конечное число кусочков, чтобы оно было общим для обоих функций ($\exists$ разбиение $X = \bigsqcup_{i = 1}^{n} e_i$, такое что $f|_{e_i} = \const, g|_{e_i} = \const$)

\subsubsection{Разбиение, допустимое для ступенчатой функции}

См. выше

\subsubsection{Свойство, выполняющееся почти везде}

$W(x)$ --- высказывание (true или false), вычисляемое относительно точки

Говорят, что утверждение $W(x)$ \textit{выполнено почти всюду (почти везде) на $E$}, если $\exists e \subset E, \mu E = 0$, и для $\forall x \in E \setminus e \dbl W(x) == true$

\textit{Важный принцип:}

Если есть последовательность высказываний $W_n(x)$, тогда утверждение ``$\forall n \dbl W_n(x)$ --- истинно'' выполнено при почти всех $x$.

\subsubsection{Сходимость по мере}

$f_n, f: X \rightarrow \overline{\mathbb{R}}$ почти везде конечны, измеримы. Тогда $f_n \underset{\mu}{\Longrightarrow} f$ сходится по мере (при $n \rightarrow \infty$), если: 

\[\forall \varepsilon > 0 \quad \mu X(|f_n - f| > \varepsilon) \underset{n \rightarrow \infty}{\longrightarrow} 0\]

\subsubsection{Теорема Егорова о сходимости почти везде и почти равномерной сходимости}

\begin{itemize}
    \item $\mu X < + \infty$
    \item $f_n, f$ --- измеримы, почти везде конечны
    \item $f_n \rightarrow f$ почти везде
\end{itemize}

Тогда:

\[\forall \varepsilon > 0 \dbl \exists e \subset X, \dbl \mu e < \varepsilon \quad f_n \rsh{X \setminus e} f\]

\subsubsection{Интеграл ступенчатой функции}

$f \ge 0, f = \sum_{i = 1}^{n} \lambda_i \chi_{E_i}(x)$ --- ступенчатая, $\bigsqcup_{i = 1}^{k} E_{i} = X$ --- допустимое разбиение

\[\int_{X} f d \mu = \sum_{i = 1}^{n} \lambda_i \mu E_i \] 

\textit{Свойства: }

\begin{enumerate}
    \item Не зависит от представления $f$ (можно передробить кусочки и получить то же самое)
    \item Монотонен $f \le g \Rightarrow \int_{X} f d \mu \le \int_{X} g d \mu$
\end{enumerate}


\subsubsection{Интеграл суммируемой функции (интеграл Лебега)}

Пусть $f$ --- измеримая функция, и при этом хотя бы один из этих интегралов конечен $\int_{X} f^{+}, \int_{X} f^{-}$, тогда \textit{интеграл Лебега это}:

\[\int_{X} f d \mu := \int_{X} f^{+} d \mu - \int_{X} f^{-} d \mu\]

Если оказалось, что оба интеграла конечны, то функция называется \textit{суммируемой}.

\textit{Замечания: }
\begin{enumerate}
    \item $f \ge 0$ --- измерима, тогда интеграл Лебега равен интегралу неотрицательной измеримой функции
    \item \[\int_{E} f d \mu := \int_{E} f \chi_{E} d \mu\]
    \item \[f = \sum \lambda_{k} \chi_{E_{k}}, \quad \int_{E} f = \sum \lambda_k \mu(E_{k} \cap E)\]
    \item $f \ge 0$ --- измерима
        \[\int_{E} f d \mu = \sup \{\int_{E} g d \mu, 0 \le g \le f, g\text{ --- ступенчатые}\}\]
    \item $\int_{E} f d \mu$ --- не зависит от значений вне $E$
\end{enumerate}

\newpage

\subsection{Важные теоремы}

\subsubsection{Регулярность меры Лебега}

\images{0.95}{reg_mer_leb.jpg}

\subsubsection{Теорема о преобразовании меры Лебега при линейном отображении}

\images{0.95}{iz_lin_otob.jpg}

\subsubsection{Характеризация измеримых функций с помощью ступенчатых. Следствия}

\images{0.95}{harakt_iz_st_1.jpg}

\images{1}{harakt_iz_st_2.jpg}

\subsubsection{Теорема Леви}

\images{0.95}{t_levi_1.jpg}

\images{1}{t_levi_2.jpg}


\newpage

\subsection{Теоремы}

\subsubsection{Свойства объема: усиленная монотонность, конечная полуаддитивность}

\textit{Формулировка:}

$\mu: \mathcal{P} \rightarrow \overline{\mathbb{R}}$ --- объём. Тогда: 

\begin{enumerate}
    \item $\forall A, \underbrace{A_1, A_2, \ldots, A_n}_{\text{дизъюнктные}} \in \mathcal{P}, \bigsqcup_{i = 1}^{n} \subset A$ выполняется $\sum \mu A_i \le \mu A$ (усиленная монотонность)
    \item $\forall A, A_1, A_2, \ldots, A_n \in \mathcal{P}, A \subset \bigcup_{i = 1}^{n} A_i$, тогда $\mu A \le \sum \mu A_i$ (конечная аддитивность)
    \item $A, B, A \setminus B \in \mathcal{P}, \mu B < +\infty$. Тогда $\mu(A \setminus B) \ge \mu A - \mu B$
\end{enumerate}

\textit{Замечание:}

В $(1)$ и $(2)$ мы не предполагаем, что $\bigcup A_i \in \mathcal{P}$

\textit{Доказательство:}

\textbf{(1)}

По 3 аксиоме полукольца: 

\[A \setminus \left(\bigsqcup_{i = 1}^{k} A_{i}\right) = \bigsqcup_{i = 1}^{n} D_{i}\]

Тогда: 

\[A = \bigsqcup_{i = 1}^{k} A_i \sqcup \bigsqcup_{i = 1}^{n} D_i\]

Тогда по замечанию для объёма: 

\[\mu A = \sum_{i = 1}^{k} \mu A_i + \underbrace{\sum_{i = 1}^{n} \mu D_i}_{\ge 0} \ge \sum_{i = 1}^{k} \mu A_i\]

\textbf{(2)}

Давайте соорудим множества для каждого $i$, которые содержат только кусочки из $A: B_i = A \cap A_i$. Тогда $\uwave{A = \bigcup_{i = 1}^{n} B_i}$

\images{0.3}{sv_ob.png}

Теперь проблема в том, что это объединение не дизъюнктно. Исправим это: $C_1 = B_1, C_2 = B_2 \setminus C_1, \ldots, C_n = B_n \setminus (\bigcup_{i = 1}^{n - 1} B_{i})$. Тогда $\uwave{A = \bigsqcup_{i = 1}^{n} C_i}$. Осталась лишь проблема, что $C_n$ совершенно не обязаны жить внутри $\mathcal{P}$ (в свойствах написано). Но мы опять вспоминаем 3ю аксиому полукольца и то, что $B$-шки то лежат внутри $\mathcal{P}$, поэтому переопределяем: $C_k = B_k \setminus (\bigcup_{i = 1}^{k - 1} B_{i}) = \bigsqcup_{i = 1}^{n} D_i \in \mathcal{P}$. Тогда $\uwave{A = \bigsqcup_{i, j} D_{ij}}$, и по тому же замечанию для объёма $\mu A = \sum_{\text{кон.}}\mu D_{ij}$: 

\[\forall i \in [1, n]: \bigsqcup D_{ij} \subset C_i \subset B_i \subset A_i, \quad \sum \mu D_{ij} \le \mu A_i\]

Тогда: 

\[\uwave{\mu A = \sum_{i} \sum_{j} D_{ij} \le \sum_{k} \mu A_k}\]

\textbf{(3)}

Рассмотрим два случая: 

\begin{enumerate}
    \item $B \subset A: A = B \dbl \sqcup \dbl (A \setminus B)$ --- дизъюнктно. При $\mu B < + \infty $ можно перебросить в другую сторону $\mu (A \setminus B) = \mu A - \mu B$
    \item $B \nsubseteq A:$ рассмотрим $A \subset B = A \subset (A \cap B)$ --- это правда (так как когда мы вычитаем $B$, достаточно вычесть лишь пересечение. А пересечение лежит внутри $A$, поэтому для него работает пункт $(1)$):
    
    \[\mu(A \setminus B) = \mu(A \setminus (A \cap B)) = \mu A - \mu (A \cap B) \underset{\text{очевидно, уменьшаем сумму (лок. мон. по } B)}{\ge} \mu A - \mu B\]
\end{enumerate}

ч. т. д. 

\subsubsection{Теорема об эквивалентности счетной аддитивности и счетной полуаддитивности}

\textit{Формулировка:}

$\mu: \mathcal{R} \rightarrow \overline{\mathbb{R}}$ --- объём. Тогда эквиваленты два утверждения:

\begin{enumerate}
    \item $\mu$ --- счётно-аддитивна ($\mu$ --- мера)
    \item $\mu$ --- счётно-полуаддитивна (тут нет дизъюнктности): 
    
    \[\forall A_1, A_2, \ldots \in \mathcal{P}, \quad A \subset \bigcup_{i = 1}^{\infty} A_i, \quad \mu A \le \sum_{i = 1}^{\infty} \mu A_i\]
\end{enumerate}

\textit{Доказательство:}

\textbf{$(1) \Rightarrow (2)$}

Просто заменяем все \uwave{подчёркнутые строчки} в предыдущем доказательстве с конечных операций на счётные (так как у нас мера, всё работает).

\textbf{$(2) \Rightarrow (1)$}

Чтобы проверить счётную аддитивность, надо проверить следующее:

\[A = \bigsqcup_{i = 1}^{\infty} A_i \underset{?}{\Rightarrow} \mu A = \sum_{i = 1}^{\infty} \mu A_i\]

Тогда по усиленной монотонности: 

\[\sum_{i = 1}^{n} \mu A_i \le \mu A\]

Но с другой стороны, по определению счётной полуаддитивности: 

\[\mu A \le \sum_{i = 1}^{\infty} \mu A_i\]

Получается, что в конечном случае у нас мера множества больше конечной сумме мер, а в бесконечном --- меньше. Получается, это возмоожно только при: 

\[\mu A = \sum_{i = 1}^{\infty} \mu A_i\]

Посылка выполнена. 

ч. т. д.


\subsubsection{Теорема о непрерывности меры сверху}

\textit{Формулировка:}

$\textfrak{A}$ --- алгебра, $\mu: \textfrak{A} \rightarrow \mathbb{R}$ --- объём. Тогда следующие утверждения эквивалентны: 

\begin{itemize}
    \item $\mu$ --- мера (то есть счётно-аддитивна)
    \item $\mu$ --- непрерывна сверху, то есть: 
    \[\forall A_1, A_2, \ldots \in \textfrak{A}, \quad A_1 \supset A_2 \supset \ldots, \quad A = \bigcap_{i = 1}^{\infty} A_i \Leftrightarrow A = \lim_{i \rightarrow \infty} A_i\]
\end{itemize}

\textit{Доказательство:}

\textbf{$(1) \Rightarrow (2)$}

Давайте рассмотрим $B_k = A_k \setminus A_{k + 1}$: 

\images{0.3}{nepr_mera_sv.png}

Очевидно, что все такие $B_i$ дизъюнктны относительно друг друга. Тогда очевидно (из рисунка), что:

\[A_1 = \bigsqcup_{i = 1}^{\infty} B_i \sqcup A\]

Т. к. нам дана мера, давайте применим её (сработала счётная аддитивность): 

\[\mu A_1 = \sum_{i = 1}^{\infty} \mu B_i + \mu A\]

Но нам то интересно, что будет при $i \rightarrow \infty$, поэтому смотрим, что будет для $i$-го: 

\[\mu A_i = \sum_{k = i}^{\infty} \mu B_k + \mu A\]

Ну а теперь в пределе $i \rightarrow \infty$ (вспоминаем, что по условию $\mu A_1 < + \infty$ и $\sum \mu B_k$ --- сходится (т.к. на дизъюнктном объединении)):

\[\lim_{i \rightarrow \infty} \mu A_i = 0 + \mu A\]

\textbf{$(2) \Rightarrow (1)$}

Ну а тут всё по накатанной. Проверяем счётную аддитивность: $C = \bigsqcup_{i = 1}^{\infty} C_i \underset{?}{\Rightarrow} \sum_{i = 1}^{\infty} \mu C_i$. Соорудим $C_i$:

\[A_i = \bigcup_{k = i}^{\infty} C_k\]

Тогда у нас будет работать условие $A_1 \supset A_2 \supset \ldots$, однако работать с этим не супер удобно. Давайте запишем по-другому: 

\[A_i = \bigcup_{k = i}^{\infty} C_k = C \setminus \left(\bigsqcup_{k = 1}^{i} C_k\right)\]

Супер, такое мы любим, это по 3й аксиоме полукольца лежит в $\textfrak{A}$. Заметим также (так как дизъюнктное разбиение в конце ничего не останется): 

\[A_1 \supset A_2 \supset A_3 \supset \ldots A = \bigcap_{i = 1}^{\infty} A_i = \O \quad \Rightarrow \mu A_i \underset{i \rightarrow \infty}{\longrightarrow} 0 = \mu A\]

Перезапишем теперь посылку в терминах $C$ и $A_i$:

\[C = \bigsqcup_{k = 1}^{i - 1} C_k \sqcup A_i\]

И по конечной аддитивности:

\[\mu C = \sum_{k = 1}^{i - 1} \mu C_k + \mu A_i\]

Ну а в пределе $i \rightarrow \infty$:

\[\mu C = \sum_{k = 1}^{\infty} \mu C_k + 0\]

Посылка доказана. 

ч. т. д. 


\subsubsection{Счетная аддитивность классического объема}
\textit{Формулировка:}

$\mathcal{P}^m$ --- множемтво всех ячеек на $\mathbb{R}^m$, $\mu$ --- классический объём.

Тогда $\mu$ --- $\sigma$-конечная мера.

\textit{Доказательство:}

$\sigma$-конечность очевидна по определению, просто рисуем ячейки (клеточки), вот тепе и покрытие.

Гораздо интереснее, как сейчас будем разбираться со счётной аддитивностью. Изначально у нас задана мера $\mu[a, b) = \prod_{i = 1}^{m} |b_i - a_i|$. Пусть $P = [a, b), P_n = [a_n, b_n)$, и проверяем на счётную аддитивность, а точнее, на счётную полуаддитивность, ведь мы раньше доказали, что они эквивалентны! 

\[P \subset \bigcup P_n \quad ? \mu P \le \sum \mu P_n\]

Будем оценивать каждую часть неравенства. Во-первых, можно считать, что $P \neq \O$. Потом, давайте подгоинм под определение компактности (неожиданно): возьмём сначала $b'$ чуть ``меньше'' $b$ (вообще-то это векотра), чтобы $[a, b'] \subset [a, b)$. Тогда пусть это ``меньше'' описывается так:

\[\forall \varepsilon > 0 \dbl \exists b' \dbl: \mu P - \mu [a, b') < \varepsilon \quad \Leftrightarrow \quad \mu(P \setminus [a, b')) < \varepsilon\]

Это мы оценили $P$. Теперь $P_n$: берём $a_n'$ чуть ``больше'', чем $a_n$, чтобы $(a_n', b_n) \subset [a_n, b_n)$:

\[\mu (a_n', b_n) - \mu [a_n, b_n) < \frac{\varepsilon}{2^n} \quad \Leftrightarrow \quad \mu((a_n', b_n) \setminus [a_n, b_n)) < \frac{\varepsilon}{2^n}\]

Ну всё, супер: 

\[[a, b'] \subset \bigcup_{i = 1}^{\infty} (a', b)\]

По определению компакта (отрезка), существует конечное открытое подпокрытие: 

\[[a, b') \subset \underbrace{[a, b'] \subset \bigcup_{i = 1}^{N} (a_n', b_n)} \subset \bigcup_{i = 1}^{N} [a_n', b_n)\]

Класс. Значит запускаем на этом конечном подпокрытии свойство конечной полуаддитвности и оцениваем:

\[ \mu P - \varepsilon \le \underbrace{\mu [a, b') \le \sum_{i = 1}^{N} \mu [a_i', b_i)} \le \sum_{i = 1}^{N} \mu [a_i, b_i) + \frac{\varepsilon}{2^i}\]

\[\mu P - \varepsilon \le \sum_{i = 1}^{N} \mu [a_i, b_i) + \varepsilon\]

Устремляем $ N \rightarrow \infty$: 

\[\mu P - \varepsilon \le \sum_{i = 1}^{\infty} P_i + \varepsilon\]

Полуаддитивность доказана, занчит всё хорошо.

ч. т. д. 



\subsubsection{Лемма о структуре открытых множеств и множеств меры 0}

\images{0.95}{o_str_otkr_0_1.jpg}

\images{1}{o_str_otkr_0_2.jpg}


\subsubsection{Пример неизмеримого по Лебегу множества}

\images{0.95}{pr_neiz_leb_mn.jpg}


\subsubsection{Лемма о сохранении измеримости при непрерывном отображении}

\images{0.93}{soh_nepr_otob.jpg}

\subsubsection{Лемма о сохранении измеримости при гладком отображении. Инвариантность меры Лебега относительно сдвигов}

\images{0.93}{soh_gl_otob.jpg}

\subsubsection{Инвариантность меры Лебега при ортогональном преобразовании}

\images{0.95}{inv_mer_leb_ort.jpg}

\subsubsection{Лемма ``о структуре компактного оператора''}

\images{0.95}{str_komp_op.jpg}

\subsubsection{Теорема об измеримости пределов и супремумов}

\images{0.95}{iz_sup.jpg}

\subsubsection{Измеримость функции непрерывной на множестве полной меры}

\images{0.95}{iz_mn_mer_0.jpg}

\subsubsection{Теорема Лебега о сходимости почти везде и сходимости по мере}

\images{0.95}{teor_leb.jpg}

\subsubsection{Теорема Рисса о сходимости по мере и сходимости почти везде}

\images{0.95}{teor_rissa.jpg}

\subsubsection{Простейшие свойства интеграла Лебега}

\images{0.95}{pr_sv_int_leb_1.jpg}

\images{0.95}{pr_sv_int_leb_2.jpg}

\subsubsection{Счетная аддитивность интеграла (по множеству)}

\images{0.95}{sch_add_1.jpg}

\images{0.95}{sch_add_2.jpg}

\subsubsection{Линейность интеграла Лебега}

\images{0.95}{lin_int_leb.jpg}

\newpage 

\end{document}

\begin{comment}
    \subsubsection{Теорема X-N}
    \textit{Формулировка:}

    \begin{itemize}
        \item 
    \end{itemize}

    \textit{Доказательство:}
\end{comment}
